\chapter{Structural characterization of water clusters and solutions}
\section{Water clusters}\label{structure_of_clusters}
The structural parameters of the considered water clusters are shown here.
Table \ref{tab:3_nitrate_bond_0} gives the average HB lengths $r_a$ (with standard deviations)in [NO$_3\cdot$(H$_2$O)$_3$]$^-$.  
Table \ref{tab:3w_nitrate} (\ref{tab:table_geo_opt}) reports the selected distances characterizing 
[NO$_3\cdot$(H$_2$O)$_3$]$^-$ (RNO$_3$(H$_2$O)$_3$), and Table \ref{tab:table_rnitrate_3w} the selected parameters for RNO$_3$   
 (H$_2$O)$_3$ (R=Li, Na, K).
The unit for length and angle are \AA and degree ($^\circ$), respectively.
% 
\begin{table}[!h]
\centering
\caption{\label{tab:3_nitrate_bond_0}%
HB lengths $r_a$ in [NO$_3\cdot$(H$_2$O)$_3$]$^-$.} % T=300 K. 
\begin{tabular}{cc} \\\toprule
 HB bound to & \multicolumn{1}{c}{ $r_a$} \\
\hline
 w1 &2.40$\pm$0.52; 3.02$\pm$0.72 \\
 w2 &2.56$\pm$0.48; 3.20$\pm$0.41 \\
 w3 &2.29$\pm$0.47; 3.11$\pm$0.72
\end{tabular}
\end{table}
%
\begin{table}[!htbp]
\centering
\caption{\label{tab:3w_nitrate}%
Parameters of water molecules and H-bonds in [NO$_3\cdot$(H$_2$O)$_3$]$^-$.} % T=300 K.
\begin{tabular}{lccc}
water &$R_\text{OH}$ &$\angle$HOH & $r_\text{OH}$ \\
\hline
w1 &0.98$\pm$0.02 &101$\pm$4 & 2.40$\pm$0.52, 3.02$\pm$0.72 \\
w2 &0.98$\pm$0.02 &101$\pm$5 & 2.56$\pm$0.48, 3.20$\pm$0.41 \\
w3 &0.98$\pm$0.02 &101$\pm$4 & 2.29$\pm$0.47, 3.11$\pm$0.72
\end{tabular}
\end{table}
%
\begin{table}[!htbp]
\centering
\caption{\label{tab:table_geo_opt}%
  Structural parameters of RNO$_3$(H$_2$O)$_3$ from geometry optimization.} 
\begin{tabular}{l*{4}ccc}
Parameter  & LiNO$_3$(H$_2$O)$_3$& NaNO$_3$(H$_2$O)$_3$ & KNO$_3$(H$_2$O)$_3$\\
\hline
$r_\text{HB1}$& 1.67 & 1.71 & 1.82 \\
$r_\text{HB2}$& 1.91 & 1.78 & 1.92\\
$r_\text{HB3}$& 1.82 & 1.69 & 1.94\\
$r_\text{R-O(w1)}$ & 1.91 & 2.31 & 2.70\\
$r_\text{R-O(w2)}$ & 1.90 & 2.26 & 2.70\\
$r_\text{R-O(\nitrate)}$ & 1.84 & 2.29 & 2.69 \\
$\angle$HOH(w1)& 109 & 106 &107 \\
$\angle$HOH(w2)& 106 & 105&105 \\
$\angle$HOH(w3)& 108 & 107 &106
\end{tabular}
\end{table}
%
\begin{table}[H] %[!htbp]
\centering
\caption{\label{tab:table_rnitrate_3w}%
Parameters of RNO$_3$(H$_2$O)$_3$ at 300 K, obtained from the averaging during a DFTMD trajectory. 
  For RNO$_3$(H$_2$O)$_3$, $R_\text{OH}$ and $R'_\text{OH}$ 
  denote the lengths of O-H bonds in which H atoms is H-bonded and is free, respectively.
  }
\begin{tabular}{l*{4}lll}
Parameter & LiNO$_3$(H$_2$O)$_3$& NaNO$_3$(H$_2$O)$_3$ & KNO$_3$(H$_2$O)$_3$\\
\hline
$r_\text{HB1}$ & $1.83\pm0.14$ & $1.78\pm0.09$ & $1.82\pm0.13$\\
$r_\text{HB2}$ & $2.00\pm0.25$ & $1.91\pm0.24$ & $1.80\pm0.12$\\
$r_\text{HB3}$ &$1.79\pm0.16$ & $1.76\pm0.11$ & $1.89\pm0.18$\\
$R_\text{OH}$(w1) &$0.97\pm0.01$ &$0.98\pm0.04$ &$0.97\pm0.03$ \\
$R'_\text{OH}$(w1) &$1.00\pm0.02$ &$1.00\pm0.02$ & $1.00\pm0.03$ \\
$R_\text{OH} $(w2) &$0.97\pm0.01$ &$0.98\pm0.02$ &$0.97\pm0.02$ \\ 
$R'_\text{OH}$(w2) &$0.99\pm0.01$ &$1.00\pm0.02$ & $1.00\pm0.03$ \\
$R_\text{OH}$(w3) &$0.97\pm0.01$ & $0.97\pm0.02$&$0.97\pm0.03$ \\
$R'_\text{OH}$(w3) &$1.00\pm0.02$ &$1.00\pm0.02$ & $1.00\pm0.03$ \\
$r_\text{R-O(w1)}$ & $1.95\pm0.09$ & $2.34\pm0.08$ & $2.76\pm0.11$\\
$r_\text{R-O(w3)}$ & $1.92\pm0.07$ & $2.32\pm0.11$ & $2.74\pm0.13$\\
$r_\text{R-O(\nitrate)}$ & $1.91\pm0.08$ & $2.31\pm0.09$ & $2.74\pm0.12$ \\
$\angle$HOH (w1) &$107\pm4$ & $106\pm4$ &$105\pm5$ \\
$\angle$HOH (w2) &$106\pm6$ & $105\pm4$ &$106\pm4$ \\
$\angle$HOH (w3) &$108\pm5$ & $106\pm3$ &$106\pm3$ 
\end{tabular}
\end{table}
\paragraph{Structural and vibrational properties of [NO$_3\cdot$(H$_2$O)$_3$]$^-$}
% [In the main text, I deleted this paragraph, because now i am not sure on this idea.]
To find possible source of the vibrational features of water molecules in the 
cluster [NO$_3\cdot$(H$_2$O)$_3$]$^-$, we consider structural properties and the VDOS for water molecules 
in it. 
Lengths of H-bonds are shown in Table\thinspace\ref{tab:3_nitrate_bond}. 
The nitrate O (ON)--water O (OW) 
and nitrate O--water H (HW) RDFs are shown in Fig.\thinspace\ref{gdr_ON-wat--3_NO3}.

%
When $T=300$ K, the difference $r_a$ between different H atoms in one water molecule is
$\Delta{r_a}=0.69$ \AA, while $\Delta{r_a}=0.13$ \AA for $T=100$ K. It shows that vibrational 
peaks for the three water molecules are much closer than that at the higher temperature 300 K. 

%
The calculated VDOS for water molecules in the cluster at a lower temperature 100 K is given in 
Fig.\thinspace\ref{fig:vdos_LiNO3-3w_100K_w1-2-3_font35}. 
It shows that the vibrational peaks for the three water molecules 
are close to each other ($\Delta\nu <$ 10 \cm) for both vibrational and bending modes.
At the lower temperature, the three water molecules 
are more symmetric distributed bound to the central nitrate. Therefore, the difference between H-bonds in the 
symmetric isomer of [NO$_3\cdot$(H$_2$O)$_3$]$^-$ is likely a finite temperature effect,
which can be verified by the calculation of the VDOS for water molecules.

%
Both differences $\Delta\nu$ and $\Delta{d}$ decrease as the temperature decrease,
therefore, the different vibrational features are $T$-dependent effect. 

\begin{table}[H]
\centering
\caption{\label{tab:3_nitrate_bond}%
Lengths of H-bonds in [NO$_3\cdot$(H$_2$O)$_3$]$^-$. Indices of H atoms: H6, H7 in w1; 
H9, H10 in w2; and H12, H13 in w3.} 
\begin{tabular}{ccc} \\\toprule
 HB & $r_a\pm\delta$ (100 K)(\A) & \multicolumn{1}{c}{ $r_a\pm\delta$ (300 K)}(\A)\\
\hline
 H6-O2 &2.75$\pm$0.62& 2.40$\pm$0.52 \\
 H7-O4 &2.79$\pm$0.58& 3.02$\pm$0.72 \\
 H9-O3 &2.89$\pm$0.60 &2.56$\pm$0.48 \\
 H10-O4 &2.74$\pm$0.49&3.20$\pm$0.41 \\
 H12-O3 &2.46$\pm$0.45&2.29$\pm$0.47 \\
 H13-O2 &2.75$\pm$0.59 &3.11$\pm$0.72
\end{tabular}
\end{table}
%===================
\begin{figure}[H] %[htbp]
\centering
\includegraphics [width=0.5\textwidth] {./diagrams/gdr_ON-wat--3_NO3} 
\setlength{\abovecaptionskip}{0pt}
\caption{\label{gdr_ON-wat--3_NO3}
The nitrate O (ON)--water O (OW) 
and nitrate O--water H (HW) RDFs for [NO$_3\cdot$(H$_2$O)$_3$]$^-$.
Peaks for the former are 1.93, 2.95 and 3.95 \A, and for the later are 2.95 and 4.80 \A.}
\end{figure} 
%==================================================================================
%--------------------
\begin{figure}[H]%[htbp]
\centering
\centering
\includegraphics [width=0.5 \textwidth] {./diagrams/vdos_LiNO3-3w_100K_w1-2-3_font35} 
\setlength{\abovecaptionskip}{0pt}
\caption{\label{fig:vdos_LiNO3-3w_100K_w1-2-3_font35} The VDOS $g(\nu)$ for water molecules in the
cluster [NO$_3\cdot$(H$_2$O)$_3$]$^-$ at 100 K.} 
\end{figure}
%====================================================================================
%In addition, the VDOS for H atoms and water molecules in [NO$_3\cdot$(H$_2$O)$_3$]$^-$ (Fig.\thinspace\ref{fig:vdos_NO3-3w_2_H-wat}) shows 
%that H's contribution dominates that of the water molecule. 

\section{Solution/vapor interfaces}\label{SOLUTION_VAPOR_PROPERTIES}
%\paragraph{Dipole orientation distribution of water at aqueous/vapor interfaces}
%Set $\theta$ as the angle between the dipole moment of water molecules and the normal vector 
%of the interface, and $P(\theta)$ as the probability.
%The data used to statistics is the dipole tilt angle $\theta_{i}$. 
%We use the $\theta_i$ instead of $\langle\theta\rangle$ to do the statistics.
%With picking up $\theta_{l}$, $l=0, 10, 20, ...$ from the series $\theta_{i}$,
%we find that pure water's dipole moment tilt angle is smaller than that of water molecules at the salty water surface. 
%This result means that at the pure water surface, water molecules have more $p$-polarization components than those at the salty water surface. 
%However, water molecules at the salty water surface has more $s$-polarization component.
%The result is shown in Fig.\thinspace\ref{fig:prob_theta_ln_itp_256}.
%%
\paragraph{Solvation shell HB dynamics of W--W bonds at the alkali iodide/vapor interfaces}
%%%%%%%%%%%%
% !?USEFUL %
%%%%%%%%%%%%
%%
%[ALSO INTERESTING TO COMPARE THESE CURVES TO THOSE FOR THE NEAT WATER/VAPOR INTERFACE] 
\begin{figure}[H]
\centering
\includegraphics [width=\textwidth] {./diagrams/hbacf_C_sh2_2p}
\setlength{\abovecaptionskip}{0pt}
\caption{\label{fig:hbacf_C_sh2_2p}The $c^{(k)}(t)$ of W--W bonds in the solvation shell 
  of (a) cations and (b) I$^-$ at the interfaces of 0.9 M LiI, NaI and KI solutions, respectively.
  As a reference, the \CHB for the 1-\AA water/vapor interface (Paragraph\thinspace\ref{PARA_IHB} in Chapter \ref{CHAPTER_HBD}) and bulk water are also shown.
%The data for bulk water are calculated from water in the middle of the slab of the LiI solution.
}
\end{figure}
%Fecko and co-workers' study of liquid D$_2$O by IR spectroscopy reveals that the vibrational dynamics observed are dominated by underdamped displacement of the hydrogen-bond coordinate at short times ( less than 200 fs).\cite{CJF03,CJF05} 

Figures \thinspace\ref{fig:hbacf_C_sh2_2p} a and b 
report the  $c^{(k)}(t)$ of W--W bonds in the solvation shell of cations and I$^-$ at the interfaces of the alkali iodide solutions.
%[HOW TO EXPLAIN (a)?] [(WHAT DOES THIS SENTANCE FOR?)
The interface of the LiI solution contains H-bonds between water molecules similar to those in bulk water, i.e.,
water molecules participating in these H-bonds are not in the solvation shell of ions. 
This result is consistent with the observation of femtosecond mid-infrared pump-probe experiments 
on the O--H stretch vibration of water molecules in aqueous solution,
that changing the nature of the cation does not affect the dynamics of solvating water\cite{Kropman2001}.
It is also in agreement with to the following \abinitio simulation results: water molecules that directly surround the cation, the O--H groups point
away from the cation and form O--H$\cdots$O bonds with bulk water molecules\cite{Hashimoto1994,Ramaniah1998,Kropman2001}.
From Fig.\thinspace\ref{fig:hbacf_C_sh2_2p} b, we found that for all three alkali iodide solutions, $c^{(k)}(t)$ for solvation shell  
of \I decays faster than that for molecules in bulk water.
The simulation produces similar result as Omta and coworker's experiments of femtosecond pump-probe spectroscopy,
which demonstrate that anions ( $\text{SO}^{2-}_4$, $\text{ClO}^-_4$, etc) have no influence on the dynamics of bulk water, 
even at high concentration up to 6 M\cite{Omta2003, ZhangYanjie2006}. 
Here, we have found that the cations \Li and \Na do not alter the H-bonding network outside the first solvation shell of cations. 
It is concluded that no long-range structural-changing effects for alkali metal cations.

In addition, \CHB for the instantaneous interface layer of thickness $d=1$ \AA in pure water is also shown in Fig.\thinspace\ref{fig:hbacf_C_sh2_2p}.
Its ultrafast relaxation process has a shorter relaxation time, which shows that the effects of ions (alkali cations and \I) are not as obvious as that of the water/vapor interface.


\paragraph{RDFs in alkali nitrate solutions}
%\{figure}[htb]
%\centering                                          
%\includegraphics [width=0.36 \textwidth] {./diagrams/gdr_N-W_127_LiNO3} 
%\setlength{\abovecaptionskip}{0pt}
%  \caption{\label{fig:gdr_N-W_127_LiNO3} RDFs $g_{N-OW}(r)$ and $g_{N-HW}(r)$ in bulk LiNO$_3$ solution.
%Box size: 15.78 $\times$ 15.78 $\times$ 15.78 \A$^3$; $T = 300$ K.}
%\end{figure}
RDFs $g_\text{Li-OW}(r)$, $g_\text{Li-HW}(r)$, $g_\text{ON-OW}(r)$ and $g_\text{ON-HW}(r)$ in bulk LiNO$_3$ solution are shown in Fig.\thinspace\ref{fig:gdr_127_LiNO3}.
\begin{figure}[H] %[htb]
\centering                                          
\includegraphics [width=0.8 \textwidth] {./diagrams/gdr_127_LiNO3} 
\setlength{\abovecaptionskip}{0pt}
  \caption{\label{fig:gdr_127_LiNO3} RDFs $g_\text{Li-OW}(r)$, $g_\text{Li-HW}(r)$, $g_\text{ON-OW}(r)$ and $g_\text{ON-HW}(r)$ in bulk LiNO$_3$ solution.}
%Box size: 15.78 $\times$ 15.78 $\times$ 15.78 \A$^3$; $T = 300$ K.
\end{figure}
%
\paragraph{RDFs in alkali iodide solutions}
RDFs $g_\text{Li-OW}(r)$, $g_\text{Li-HW}(r)$, $g_\text{I-OW}(r)$ and $g_\text{I-HW}(r)$ for the LiI/vapor interface are shown in Fig.\thinspace\ref{fig:gdr_124_2LiI}.
\begin{figure}[H]%[htb]
\centering                                          
\includegraphics [width=0.8 \textwidth] {./diagrams/gdr_124_2LiI} 
\setlength{\abovecaptionskip}{0pt}
  \caption{\label{fig:gdr_124_2LiI} RDFs $g_\text{Li-OW}(r)$, $g_\text{Li-HW}(r)$, $g_\text{I-OW}(r)$ and $g_\text{I-HW}(r)$ for the LiI/vapor interface.}
%Box size: 15.60 $\times$ 15.60 $\times$ 31.00 \A$^3$; $T = 330$ K.
\end{figure}
The radii of the second solvation shell are: 5.0 \AA for \li, 5.38 \AA for \na,
and 6.0 \AA for \I ions, which are obtained from the RDFs.
The RDFs $g_{\text{X-O}}$ (X=\li, \na, K$^+$) for the interfaces 
of LiI (NaI,KI) solutions are shown in Fig.\thinspace\ref{fig:gdr_XO--124_2XI} a,
and the coordination numbers are in Fig.\thinspace\ref{fig:gdr_XO--124_2XI} b.
We see that the radius of the solvation shells of \Li ions, \Na ions, 
and \K ions increase sequentially, and the number of coordination molecules also increase sequentially. 
However, this order is not true for the relaxation time of HB dynamics between water molecules in the first solvation shell of the ion 
and other water molecules. 
The effects of the alkali metal ions and iodide ions are similar
(Fig.\thinspace\ref{fig:hbacf_C_sh2_2p}), 
that is, the relaxation time of HB dynamics in the outer layer is smaller than that in bulk water.
%从图中我们看出,锂离子,钠离子,钾离子的溶解壳的半径依次增大,其配位分子数也依次增大。但这种有序性对于离子第一溶解壳内的水分子与其他水分子之间的氢键的弛豫时间却不成立。因为从7.5图可以看出,R离子和碘离子的效应很类似,即,其外层的HBD的弛豫时间都比体相纯水中的更小。
%为了看清不同溶液中的氢键的差别,我们在下一段研究了I-H氢键的动力学.
\begin{figure}[H]
\centering
\includegraphics [width=0.6\textwidth]{./diagrams/gdr_XO--124_2XI}%fig.6.1 
\setlength{\abovecaptionskip}{0pt}
\caption{\label{fig:gdr_XO--124_2XI}
(a) RDFs $g_{\text{X-O}}(r)$(X=\li, \na, K$^+$) and (b) the coordination number of \Li (\na, K$^+$) ions at the interfaces of LiI (NaI, KI) solution. 
The coordination numbers are $n_{\text{Li}^+}$=4, $n_{\text{Na}^+}$=5 and $n_{\text{K}^+}$=6.} 
\end{figure} % There is a first shell exist for both \Li and \Na cations. %(NEED the NORMED VERSION)
%%
%\begin{table}[H] %[!hbtp]
%\centering
%\caption{\label{tab:table_CoordNo}%
%The coordination number for ions in LiI (NaI) solution.}
%\begin{tabular}{lccc}
%name & $r_\text{shell}$ (\AA) & coordination number \\
%\hline
%$n_\text{Li-O}$ & 3.0 & 4.0 \\
%$n_\text{I-H}$ & 3.3 & 5.5 \\
%$n_\text{Na-O}$ & 3.5 & 6.0 \\ 
%$n_\text{I-O}$ & 4.3 & 5.8 \\
%\end{tabular}
%\end{table}
%\FloatBarrier
\section{Free energy of the water separated and the contact ion pair}\label{calculate_free_energy} 
From the blue-moon ensemble method\cite{Carter1989,Sprik1998}, we have obtained the constraint force (unit: a.u.force) acting on the atoms. 
The distance between the ion pair (unit: \A) is chosen as the reaction coordinate, the formula for calculating the free energy (unit: kcal/mol) is given as follows.
%First, note that
%\begin{equation}
%  1\ \text{a.u.force} =8.2387\times 10^{-8}\ \text{N}\nonumber;
%\label{eq:au2n}
%\end{equation}
%\begin{equation}
%  1\ \text{\A} = 10^{-10}\ \text{m} \nonumber;
%\label{eq:aa}
%\end{equation}
%\begin{equation}
%  1\ \text{J} = 1.44\times 10^{20}\ \text{ kcal/mol}.\nonumber
%\label{eq:j2kcpm}
%\end{equation}
The relative free energy is given by
%\begin{equation}
%  F = \sum_{i}^{N}f_i{\Delta{r}} \ \text{ a.u.energy},
%  \label{eq:f-e}
%\end{equation}
\begin{equation}
  F = \sum_{i}^{N}f_i{\Delta{r}},\nonumber
  \label{eq:f-e}
\end{equation}
where $i$ denote a point on the one-dimensional reaction coordinate, 
$N$ is the number of the sampling points of the reaction coordinates,
and $f_i$ denotes the average force on atoms over the trajectory when $i$ is fixed. 
%Furthermore,
%\begin{equation}
%  \begin{split}
%  &F = \sum_{i}^{N}f_i{\Delta{r}}\times 8.2387\times10^{-18}\text{ J}\\
%  &= \sum_{i}^{N}f_i{\Delta{r}}\times 8.2387\times10^{-18}\times1.44\times10^{20}\text{ kcal/mol}.
%  \end{split}
%\label{eq:f-e2}
%\end{equation}
Now we estimate the error of the free energy $\delta{F}$ from the summation approximation. It reads
\begin{equation}
  \delta{F} = \frac{1}{N}\sum_{i}^{N}\delta{f_i}{\Delta{r}}.
  \label{eq:dleta_f}
\end{equation}
Usually, $\delta{f_i}\approx\delta{f}$, thus
\begin{equation}
  \delta{F} = \frac{1}{N}\delta{f}\sum_{i}^{N}{\Delta{r}}.
\label{eq:dleta_f-2}
\end{equation}
Particularly, if $\Delta{r}= 0.2$ \AA, $\delta{f}=0.0075\ \text{a.u.force}$, we get 
%\begin{equation}
%\begin{split}
%  &\delta{F} = \frac{0.0075\times\sum_{i}^{N}237}{N}\ \text{kcal/mol}\\
%  &          \approx 1.78\ \text{kcal/mol}.
%\end{split}
%\label{eq:dleta_f-3}
%\end{equation}
\begin{equation}
\begin{split}
  &\delta{F} \approx 1.78\ \text{kcal/mol}.\nonumber
\end{split}
\label{eq:dleta_f-3}
\end{equation}
%  &= 0.0075\times237\ \text{kcal/mol}\

%TODO: answer this quesntion 1
%\paragraph{Question 1}
%Is there an approximation required to obtain eq.(~\ref{eq:i1}) from Eq. \ref{eq:GDb}? 
%Since expression $\nu_{\pm}+\epsilon_{\pm}^b$ or $\frac{d (lnm_i^{\sigma})}{d (ln m_2)}+\epsilon_{\pm}^b$ rather than $1+\epsilon_{\pm}^b$ included eq.(~\ref{eq:i1}) is obtained.
%
%Since $\frac{d\mu_i^0}{dm_2} = 0$ and ln$(m_if_i) =$ln$f_i +$ln$m_i$. From eq.(~\ref{eq:GDb}),
%for bulk,  $m_i= \nu_i m_2$, then
%\begin{equation}
%\frac{d\mu_{\pm}}{dm_2} = \frac{RT}{m_2} (\nu_{\pm}+\epsilon_{\pm}^b) + z_{\pm}F\frac{d\phi^{b}}{dm_2} \approx \frac{RT}{m_2} (\nu_{\pm}+\epsilon_{\pm}^b) \nonumber,
%\label{eq:i4}
%\end{equation}
%where $\epsilon_{\pm}^b=: d($ln$f_{\pm}^b)/d($ln$m_2)$. 
%
%For surface, $m_i= m_i^{\sigma}$, then
%\begin{equation}
%\frac{d\mu_{\pm}}{dm_2}=\frac{RT}{m_2}[\frac{d(\ln{m}_i^{\sigma})}{d (\ln{m}_2)}+\epsilon_{\pm}^b] + z_{\pm}F\frac{d\phi^{b}}{dm_2} \approx\frac{RT}{m_2}[\frac{d (\ln{m}_i^{\sigma})}{d (\ln{m}_2)}+\epsilon_{\pm}^b].\nonumber
%\label{eq:i3}
%\end{equation}
%

\section{Classification of water molecules based on H-bonds}\label{classification_water}
Here we  discuss the relation between the reorientation relaxation time of water molecules and their environment. 
The method is to classify the water molecules at solution/vapor interfaces based on the types of H-bonds and the number of H-bonds. 
%For both alkali nitrate and alkali iodide solutions, these two factors are main reasons to result in different reorientation relaxation time of water molecules.  
%
\paragraph{Alkali nitrate solutions}\label{para:types_wat_alkali_nitrate}
For alkali nitrate solution/vapor interfaces, we can classify the water molecules into three types: nitrate-bound water, 
water at the water/vapor interface, and bulk water. 
Because nitrate ions have more surface propensity (Fig.\thinspace\ref{fig:prob_LiNO3-wat--256_LiNO3_double_axis}), 
and we have studied the effect of alkali cations on the reorientation relaxation of water molecules, here we consider \LiN solution/vapor interface.
We have known that nitrate-bound water are located at the solution/vapor interface (the thickness $\sim 2$ \AA, see Fig.\thinspace\ref{fig:surf_x-vs-l_x_d1-5}).
Therefore, among the three types, the first two types of water molecules are interfacial ones.
For each type of water molecules, we have chosen \emph{six} water molecules for obtaining the correlation function $C_2(t)$. 
The IMS method (Paragraph \ref{para_MS_interface}) is used to choose water molecules of each type.

\paragraph{Lithium iodide solution}
Following the definition used in Ref.\cite{TianCS08}, we use the following labels to denote water molecules in an alkali iodide solution: 
D denotes that the water molecule donates a HB, D$'$ donates a H-I bond, and A accepts a HB. 
DDAA represents a water molecule with two H-Bonds donated to water molecules and two H-bonds accepted from water molecules (Fig.\thinspace\ref{fig:Multiple_figs} a);
DD$'$AA represents a water molecule with two H-bonds donated to a water molecule and \I (Fig.\thinspace\ref{fig:Multiple_figs} b), 
and with two H-bonds accepted from other water molecules (Fig.\thinspace\ref{fig:Multiple_figs} c), 
D$'$AA represents a water molecule bonded to \I at the water/vapor interface and other H-bonds to water molecules (Fig.\thinspace\ref{fig:Multiple_figs} d).
Clearly, we found that D$'$AA molecules are of free OH stretching during the dynamics. 
% 
\begin{figure}[h]%[!htbp]
\centering
\includegraphics [width=0.7 \textwidth] {./diagrams/Multiple_figs} 
\caption{\label{fig:Multiple_figs} Four types of water molecules at the LiI/vapor interface, regarding the HB environments: (a) DDAA; (b) DDA; (c) DD$'$AA; (d) D$'$AA. The cyan balls denote \I ions. }
\end{figure} 

It is evident that $C_2(t)$ for DDAA and DD$'$AA molecules do not decay exponentially (Fig.\thinspace\ref{fig:2LiI-124w_c2_fit_single_exp_I_shell_7water_2ps_class_DDAA} 
and Table \ref{tab:fitting_c2_for_each_type_of_water}).
%[BUT Table \ref{tab:fitting_c2_for_each_type_of_water} CAN NOT GIVE THE EVIDENCE. STH. IS MISSING!] 
This result is similar to the reactive flux function $k(t)$, i.e., 
the escaping rate kinetics of H-bonds in bulk water.
The relaxation of H-bonds in water appears complicated, with no simple characterization in terms of a few relaxation rate constants. 
Most of the authors believe that the cooperation between neighbouring H-bonds\cite{Sciortino1989, Ohmine1995} or 
self evident coupling between translational diffusion and HB dynamics is the source of the complexity. 
However, for D$'$AA molecules at the LiI/vapor interface,
the $C_2(t)$ decays exponentially,
\begin{eqnarray}
  C_2(t) &=& C e^{-t/{\tau_2}},\nonumber
\end{eqnarray}
where the amplitude is $C=0.76$, and the reorientation rate is $1/\tau_2 = 7.14$ ps$^{-1}$.
The single exponential decay of $C_2(t)$ for D$'$AA molecules, indicates that each D$'$AA  molecule reorientate independently to each other. 

Furthermore, $C_2(t)$ for D$'$AA molecules decays much faster than that for DDAA or DD$'$AA molecules.
From the definitions, the D$'$AA molecule accepts two H-bonds and donates only one H-bond, 
while both DDAA and DD$'$AA molecules own \emph{four} H-bonds.
Therefore, the correlation between H-bonds around the D$'$AA molecule is weaker than those around a DDAA or DD$'$AA molecule. 
Faster decay of $C_2(t)$ for D$'$AA molecules shows that the reorientation process of D$'$AA
molecules is faster than water molecules in bulk phase, e.g., DDAA or DD$'$AA molecules.

Additionally, a D$'$AA molecule has a free OH bond, which can stretch and vibrate freely. 
This feature is not available in other types of molecules such as DDAA, DD$'$A, DD$'$AA. 
Therefore, at the LiI/vapor interface, the most closely related feature of the molecular orientation relaxation process is 
the number of H-bonds \emph{donated} by water molecules. 
The result that D$'$AA molecules have shortest relaxation time among the four types of water molecules 
implies that the factor $\langle n\rangle_\text{HB}$, the average number of H-bonds per water molecule, plays a dominate role. 
This conclusion is consistent with the one obtained from above discussion for \LiN/vapor interface.

Besides, the type (or strength) of H-bonds (water--water, or ion--water) 
also affects the orientation relaxation process, which is also consistent with the above conclusion about \LiN/vapor interface. 
However, from our calculation, the number of H-bonds \emph{accepted} by water molecules has no major effect on the orientation relaxation of the water molecules at the interface.
%D'AA 水分子有一个自由的OH化学键,它可以自由地伸缩振动。这个特征是其他类型的分子如DDAA,DD'A, DD'AA等所没有的。因此在溶液表面,与分子取向弛豫过程的快慢关系最密切的特征就是水分子贡献氢键的个数。其次氢键的类型(水与水之氢键,水与离子的氢键等)也影响水分子的取向弛豫过程。水分子所接受的氢键的个数则对界面水分子的取向弛豫不起主要作用。

Finally, D$'$AA molecules' inertial-librational motion can not be seen in Fig.\thinspace\ref{fig:2LiI-124w_c2_fit_single_exp_I_shell_7water_2ps_class_DDAA}. 
This result implies that the rotational anisotropy decay of D$'$AA molecules
are of the same time scale of the inertial libration, i.e., $\sim$ 0.2 ps. 
This conclusion can be verified from the value of $\tau_{2}$ in Table \ref{tab:fitting_c2_for_each_type_of_water}: $\tau_{2}=0.97$ for D$'$AA molecules, 
which is smaller than the $\tau_{2}$ for DDAA, DD$'$A, and DD$'$AA molecules.
%

To summarize, rotational anisotropy decay of water molecules under different local environments is calculated at the LiI/vapor interface. 
The result comes from a different HB types from the usual DDAA HB type in bulk water.
The faster anisotropy decay for D$'$AA molecules reflects the less correlation between different H-bonds for D$'$AA molecules, 
which comes from H--I bond at the interfaces and the existence of free OH stretching.
As we already known from Fig.\thinspace\ref{fig:prob_124_LiI_double_axis}, in the LiI solution, 
\I ions prefer to locate at the interface.  
Therefore, we infer that the reduction of the inter-correlations between H-bonds occurs at the interface. 
Slower rotational anisotropy decay exists for water molecules  at the alkali iodide solution/vapor interfaces, 
which is the result of a different HB types (D$'$AA) from DDAA molecules in bulk water. 
The slowing down of anisotropy decay is the effect of H--I bonds at the interface. 
Since iodide's surface propensity is high, this difference of HB structure from the water/vapor interface changed 
the Im$\chi^{(2)}$ spectrum and the total HB dynamics of the interface of alkali iodide solutions.  

%图 2LiI-124w_c2_fit_exp_I_shell_7water_2ps_class_DDAA.eps
\begin{figure}[H]%[!htbp]
\centering
\includegraphics [width=0.5 \textwidth] {./diagrams/2LiI-124w_c2_fit_single_exp_I_shell_7water_2ps_class_DDAA} 
    \caption{
Time dependence of $C_2(t)$ for DD$'$A, DD$'$AA, and D$'$AA water molecules at the LiI/vapor interface.
    \label{fig:2LiI-124w_c2_fit_single_exp_I_shell_7water_2ps_class_DDAA}%
}%
\end{figure} 
%
\begin{table}[H]
\centering
\caption{\label{tab:fitting_c2_for_each_type_of_water}%
	Exponential fitting of $C_2(t)$ for water molecules in the LiI solution. 
        The relative standard errors: $\Delta A/A \le 10^{-2}$, $\Delta \tau_{2}/\tau_{2} \le 3\times 10^{-2}$.}
% 2-ps fit.
%\begin{ruledtabular}
\begin{tabular}{lccccc}
water molecules & $A$  & $\tau_{2}$ (ps) \\
\hline
DD$'$A & 0.84 & 4.04  \\
DD$'$AA & 0.85 & 4.08  \\
DDAA & 0.87 & 3.76 \\
D$'$AA & 0.78 & 0.97 \\
\end{tabular}
%\end{ruledtabular}
\end{table}
%
%\subsection{\LiN Solution/vapor Interface}
%The anisotropy decay of OH bonds in water molecules in 0.4 M LiNO3 solution/vapor interface is shown in Fig.\space\ref{fig:c2_LiNO3_inset}.  In the model of the interface, there is one \Li and one \nitrate in the 15.6 \AA$\times$15.6 \AA$\times$31.0 \AA simulation box. 
%The larger decay rate consistent to the conclusion infered from the VDOS for the interfaces, although the concentration of \LiN is lower. This result obtained from another DFTMD trajectory consistent with the previous one, and it reflects that the \nitrate on the surface of the alkali nitrate solution weaken the H-bonds and  accelerate the anisotropy decay of water molecules at the interfaces.

%{Rotational Anisotropy Decay of Water at Aqueous/vapor Interfaces of Alkali Halide Solutions}\label{RAD}
%\paragraph{Aqueous vapor interface of LiI(NaI) solution}
%We use the following procedures to calculate the molar concentration of ions in the solutions we study.
%\begin{align}
%&n_j=N_j\times[1/(6.02\times10^{23})] {\text{ mol}} \nonumber \\
%&V_{\text{liquid}}=15.6\times15.6\times15.6 \text{\AA}^3=3796\times10^{-30}\text{ m}^3 \nonumber
%\label{eq:concen}
%\end{align}
%where $n_j$, $N_j$ and $V_{\text{liquid}}$  is the amount of substance $j$, the number of substance $j$, and the volume of the liquid part of the liquid/vapor interface.  
%For the LiI/vapor interface system. The simulation box is with the size of 31.0 \AA$ \times$15.6 \AA$ \times$15.6 \AA. Half of the volume of the simulation box is vacuum. In the liquid part of the simulation box, there are two \Li cations and two \I anions.
%Therefore, the molar concentration of the solution LiI is $c_{\text{LiI}}={n_{\text{LiI}}}/{V_\text{liquid}}=0.9\times10^3  \text{ mol}/\text{m}^3$.

%
