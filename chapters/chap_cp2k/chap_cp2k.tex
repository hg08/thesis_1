\chapter{Computational Details of the DFTMD Simulations}\label{computational_detail}
In the thesis, the DFT calculations were done with the CP2K program, which incorporates the Gaussian and Plane Waves (GPW) method\cite{GL99}. 
Core electrons were described by Goedecker-Teter-Hutter pseudopotentials\cite{MK05}, while the valence electrons were expanded as a double-zeta Gaussian basis set.
The double zete basis sets optimized for the dondesnsed phase\cite{VandeVondele2007} were used in conjunction with GTH pseudopotentials\cite{Goedecker1996} 
and a 280 Ry cutoff for the auxiliary plane wave basis.
The discretized integration time step $\Delta t$ was set to 0.5 fs. 
The Brillouin zone was sampled at the $\Gamma$-point only and, the BLYP XC functional has been employed.
\paragraph{Neat Water}\label{DETAILS_NEAT_WATER}
%In addition to the simulation in the main text of this thesis, we also use different methods for the bulk water system with the same temperature,
%volume and number of molecules. 
To simulate the bulk water for testing the algorithm of IHB, We constructed the following model. 
The number of water molecules in the simulation system is 128, 
the temperature is still $T=300$ K, and the box is a cube with a side length of 15.6404 \A. 
In this simulation, we relax the value of the target accuracy for the SCF convergence to 10$^{-6}$. 
Other settings, such as exchange correlation functional, correction of dispersion force, etc. are the same as the text. 
The size of the simulation box is obtained as follows: 
According to the experimentally obtained relations between water density,temperature and ion concentration (if ions are present), 
a nonlinear equation set is established, and then the equation set is solved to find the box size.

\paragraph{Lithium Nitrate Solutions}\label{DETAILS_LINO3}
All simulations on alkali nitrate solutions were performed at 300 K within the canonical NVT ensemble. 
The simulated interfacial system consisted of 127 water molecules and a Li$^+$--nitrate pair in a periodic
box of size $15.78 \times 15.78 \times 31.56$ \AA$^3$, which corresponds to
a density of 0.997 g/cm$^3$. 
At each MD step the corrector was applied
only once, which implies just one preconditioned gradient calculation. 
For a given molecular configuration, $\{\mathbf{r}_i (t)\}$, Eq.\thinspace\ref{eq:rho_c} can be
solved through interpolation on a spatial grid.\cite{Willard2010} 
We have taken $\{\mathbf{r}_i (t)\}$ to refer to the positions of all
atoms except hydrogen atoms in the system, and because the bulk correlation length of
liquid water is about one molecular diameter, we have used $\xi$ 
=2.4 \AA; further, we have used $\rho_0= 0.016$ \AA $^{-3}$, which is
approximately one-half the bulk density of water. 

This calculation of the $\chi^\text{(2,R)}$ is done for a model for the water/vapor interface 
where a slab of 117 water molecules containing one \Li and one \nitrate is included 
in a period simulation box of 15.60 \AA$\times$15.60 \AA$\times$31.00 \AA at 300 K.

%The keyword CUTOFF defines the plane wave cutoff (default unit is in Ry) for the finest level of the multi-grid. 
%The higher the plane wave cutoff, the finer the grid.
%Having constructed the multi-grid, QUICKSTEP then map the Gaussians onto the grids. The keyword REL$\_$CUTOFF controls which product Gaussians 
%are mapped onto which level of the multi-grid. CP2K tries to map each Gaussian onto a grid such that the number of grid points 
%covered by the Gaussian---no matter how wide or narrow---are roughly the same. REL$\_$CUTOFF defines the plane wave cutoff of a reference grid 
%covered by a Gaussian with unit standard deviation $e^{|{\bf r}|^2}$.\cite{CP2K}

%Therefore, the two most important keywords effecting the integration grid and the accuracy of a calculation are CUTOFF and REL$\_$CUTFF.
%If CUTOFF is too low, then all grids will be coarse and the calculation may become inaccurate; and if REL$\_$CUTOFF is too low, then even if one has a high CUTOFF, 
%all Gaussians will be mapped onto the coarsest level of the multi-grid, and thus the effective integration grid for the calculation may still be too coarse.

%\section{Input File for Neat Water/Vapor Interface}
%The CP2K input file to run DFTMD, for the system of water/vapor interface, is as follows. 
%% backgroundcolor=\color{lightgray},
%\begin{lstlisting}[language=C]                  ]
%&FORCE_EVAL
%  METHOD QS
%  &DFT
%    BASIS_SET_FILE_NAME ./BASIS_MOLOPT   
%    POTENTIAL_FILE_NAME ./GTH_POTENTIALS
%    &MGRID
%      CUTOFF 280    # defines the plane wave cutoff (unit: Ry) , i.e., the finest level of the multi-grid. 
%    &END MGRID
%    &QS
%      EPS_DEFAULT 1.0E-12
%      WF_INTERPOLATION PS 
%      EXTRAPOLATION_ORDER 3
%    &END QS
%    &SCF
%      &OT ON
%      &END OT
%      SCF_GUESS RESTART  # RESTART or ATOMIC
%      EPS_SCF     1.0E-6
%      MAX_SCF 300
%    &END SCF
%    &XC
%      &XC_FUNCTIONAL BLYP  
%      &END XC_FUNCTIONAL
%      &vdW_POTENTIAL    # To include vdW forces in DFT, we use DFT-D3 method  
%         DISPERSION_FUNCTIONAL PAIR_POTENTIAL
%         &PAIR_POTENTIAL
%            TYPE DFTD3    # Type of damping
%            CALCULATE_C9_TERM .TRUE.    # include 3-body term
%            REFERENCE_C9_TERM .TRUE.    
%            LONG_RANGE_CORRECTION .TRUE.
%            PARAMETER_FILE_NAME ./dftd3.dat
%            REFERENCE_FUNCTIONAL BLYP 
%            R_CUTOFF 8.    # The cutoff radius to calculate dispersion energy
%            EPS_CN 0.01
%         &END PAIR_POTENTIAL
%      &END vdW_POTENTIAL
%    &END XC
%  &END DFT
%  &SUBSYS
%    &CELL
%      ABC 15.6 15.6 31.0
%    &END CELL
%    &COORD    #input coords from an equilibrated structure 
%      @INCLUDE 'pos.inc'    #pos.inc is the coordinate file (neat water) 
%    &END COORD
%    &KIND H
%      BASIS_SET DZVP-MOLOPT-SR-GTH
%      POTENTIAL GTH-BLYP-q1
%    &END KIND
%    &KIND O
%      BASIS_SET DZVP-MOLOPT-SR-GTH
%      POTENTIAL GTH-BLYP-q6
%    &END KIND
%  &END SUBSYS
%&END FORCE_EVAL
%&GLOBAL
%  PROJECT 121wat
%  RUN_TYPE MD        
%  PRINT_LEVEL LOW
%&END GLOBAL
%
%&MOTION
%  &GEO_OPT
%    TYPE minimization
%    OPTIMIZER BFGS
%    MAX_ITER 20
%  &END GEO_OPT
%
%  &MD
%    &THERMOSTAT
%      &NOSE
%        LENGTH 3
%        YOSHIDA 3
%        TIMECON 1000.
%        MTS 2
%      &END NOSE
%    &END THERMOSTAT
%    ENSEMBLE NVT
%    STEPS 1000000
%    TIMESTEP 0.5
%    TEMPERATURE 300.0
%    TEMP_TOL 60
%  &END MD
%
%  &PRINT
%   &TRAJECTORY
%     &EACH
%       MD 1
%     &END EACH
%   &END TRAJECTORY
%   &VELOCITIES ON
%     &EACH
%       MD 1
%     &END EACH
%   &END VELOCITIES
%   &FORCES ON
%     &EACH
%       MD 1
%     &END EACH
%   &END FORCES
%
%   &RESTART_HISTORY
%     &EACH
%       MD 1000
%     &END EACH
%   &END RESTART_HISTORY
%   &RESTART
%     BACKUP_COPIES 1
%     &EACH
%       MD 1
%     &END EACH
%   &END RESTART
%  &END PRINT
%&END MOTION
% \end{lstlisting}
 
\paragraph{Interface of Alkali-Iodine Solution}
The calculation of the $\chi^\text{(2,R)}$ is done for a model for water/vapor interface where a slab of 118 
water molecules containing one \Li and one \I is included in a period simulation box of 15.60 \AA $\times$ 15.60 \AA $\times$ 31.00 \AA.

%For LiI and NaI solutions, we use the following procedures to calculate the molar concentration of ions in the solutions we study:
%%&V_{\text{liquid}}=3.796\times10^{-33}\text{ m}^3 \nonumber
%$n_j=N_j\times[1/(6.02\times10^{23})] {\text{ mol}}$, and
%$V_{\text{liquid}}=15.6\times15.6\times15.6$ \A$^3$, 
%where $n_j$, $N_j$ and $V_{\text{liquid}}$ is the amount of substance $j$, the number of substance $j$, and the 
%volume of the liquid part of the aqueous/vapor interface.  
%For the water/vapor interface of LiI solution, 
%the simulation box is with the size of $15.6 \times 15.6 \times  31.0$ \A$^3$. 
%Half of the volume of the simulation box is vacuum. 
%In the liquid part of the simulation box, there are two \Li cations and two \I anions.
%Therefore, the molar concentration of the solution LiI is $c_{\text{LiI}}={n_{\text{LiI}}}/{V_\text{liquid}}=0.9\times10^3  \text{ mol}/\text{m}^3$.
%As an sample of our input files, we list the input file to run DFTMD, for the NaI solution. 
%\begin{lstlisting}[language=C]                  ]
%&FORCE_EVAL
%  METHOD QS
%  &DFT
%    BASIS_SET_FILE_NAME ./BASIS_MOLOPT   
%    POTENTIAL_FILE_NAME ./GTH_POTENTIALS
%    &MGRID
%      CUTOFF 280
%    &END MGRID
%    &QS
%      EPS_DEFAULT 1.0E-12
%      WF_INTERPOLATION PS 
%      EXTRAPOLATION_ORDER 3
%      METHOD GAPW
%    &END QS
%    &SCF
%      &OT ON
%      &END OT
%     SCF_GUESS RESTART
%     EPS_SCF      1.0E-5
%     MAX_SCF 300
%    &END SCF
%    &XC
%      &XC_FUNCTIONAL BLYP  
%      &END XC_FUNCTIONAL
%      &vdW_POTENTIAL
%         DISPERSION_FUNCTIONAL PAIR_POTENTIAL
%         &PAIR_POTENTIAL
%            TYPE DFTD3
%            CALCULATE_C9_TERM .TRUE.
%            REFERENCE_C9_TERM .TRUE.
%            LONG_RANGE_CORRECTION .TRUE.
%            PARAMETER_FILE_NAME ./dftd3.dat
%            REFERENCE_FUNCTIONAL BLYP 
%            R_CUTOFF 8.
%            EPS_CN 0.01
%         &END PAIR_POTENTIAL
%      &END vdW_POTENTIAL
%    &END XC
%  &END DFT
%  &SUBSYS
%    &CELL
%      ABC 15.6 15.6 31.0
%    &END CELL
%    &COORD    #input coords from an equilibrated structure 
%      @INCLUDE 'pos.inc'    #pos.inc is the coordinate file (NaI solution) 
%    &END COORD
%    &KIND H
%      BASIS_SET DZVP-MOLOPT-SR-GTH
%      POTENTIAL GTH-BLYP-q1
%    &END KIND
%    &KIND O
%      BASIS_SET DZVP-MOLOPT-SR-GTH
%      POTENTIAL GTH-BLYP-q6
%    &END KIND
%    &KIND Li 
%      BASIS_SET DZVP-MOLOPT-SR-GTH
%      POTENTIAL GTH-BLYP-q3
%    &END KIND
%    &KIND Na 
%      BASIS_SET DZVP-MOLOPT-SR-GTH
%      POTENTIAL GTH-BLYP-q9
%    &END KIND
%    &KIND K 
%      BASIS_SET DZVP-MOLOPT-SR-GTH
%      POTENTIAL GTH-BLYP-q9
%    &END KIND
%    &KIND I
%      BASIS_SET DZVP-MOLOPT-SR-GTH
%      POTENTIAL GTH-BLYP-q7
%    &END KIND
%  &END SUBSYS
%&END FORCE_EVAL
%&GLOBAL
%  PROJECT 118_2NaI  
%  RUN_TYPE MD        
%  PRINT_LEVEL LOW
%&END GLOBAL
%
%&MOTION
%  &MD
%    &THERMOSTAT
%      &NOSE
%        LENGTH 3
%        YOSHIDA 3
%        TIMECON 1000.
%        MTS 2
%      &END NOSE
%    &END THERMOSTAT
%    ENSEMBLE NVT
%    STEPS 1000000
%    TIMESTEP 0.5
%    TEMPERATURE 300.0
%    TEMP_TOL 60
%  &END MD
%
%  &PRINT
%   &TRAJECTORY
%     &EACH
%       MD 1
%     &END EACH
%   &END TRAJECTORY
%   &VELOCITIES ON
%     &EACH
%       MD 1
%     &END EACH
%   &END VELOCITIES
%   &FORCES ON
%     &EACH
%       MD 1
%     &END EACH
%   &END FORCES
%
%   &RESTART_HISTORY
%     &EACH
%       MD 1000
%     &END EACH
%   &END RESTART_HISTORY
%   &RESTART
%     BACKUP_COPIES 1
%     &EACH
%       MD 1
%     &END EACH
%   &END RESTART
%  &END PRINT
%&END MOTION
%
%&EXT_RESTART
%  RESTART_FILE_NAME ./118_2NaI-1.restart
%  RESTART_COUNTERS T
%  RESTART_POS T
%  RESTART_VEL T
%  RESTART_THERMOSTAT T  
%&END EXT_RESTART
%\end{lstlisting}

