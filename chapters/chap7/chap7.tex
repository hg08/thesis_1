\chapter{Hydrogen Bond Dynamics in Solution Systems}\label{CHAPTER_HB_SOLUTIONS}
%The influence of ions propensity for the aqueous surface on the water's HB network are also of special interest to the atmospheric chemistry community.
In this chapter, we explore the effects of nitrate ions, iodide ions and alkali metal cations 
on the HB dynamics at the water/vapor interface of alkali nitrate solutions and alkali
iodine solutions, by DFTMD and we provide a microscopic interpretation of recent experimental results. \cite{HuaWei2014}
In paragraph \ref{env_effect}, we discuss the influence of ions on the hydrogen bond dynamics in the solution interface system; 
in paragraph \ref{SHB_dynamics}, we study the hydrogen bond dynamics and hydrogen bond lifetime in the dissolved shell of the ion;
and in paragraph \ref{RAD}, we study the orientation dynamics of H-bonds in the solvation shells in the interfacial system; 
Finally, we study the orientation characteristics of water molecules in the ions' solvation shell in the interfacial system.

\FloatBarrier
\section{Environment Effects on Hydrogen Bond Dynamics}\label{env_effect}
For the water/vapor interface of neat water, we focus on the reactive flux $k(t)$, 
which had been used in the study of HB dynamics of liquid water. \cite{AL96,Khaliullin2013}
The $k(t)$ calculated from the positional trajectory of water molecules in DFTMD simulations, is reported in Fig.\thinspace\ref{fig:121}. 
In the case of water/vapor interface, the $k(t)$ quickly changes from its initial value on a time scale of less than 0.2 ps 
(see the inset of Fig.\thinspace\ref{fig:121}). 
Beyond this transient period, the $k(t)$ decays to zero monotonically, and the slop of the $\ln{k(t)}$ increases monotonically with $t$ (see Fig.\thinspace\ref{fig:121}). 
These two properties have been found for bulk water using the SPC water model by Luzar and Chandler. \cite{AL96} 
This log-log plot of the $k(t)$ shows that, as in the case of liquid water, this decay behaviour does not coincide with a power-law decay for water/vapor interface of neat water.
This result is also the same as that of the classical molecular simulation of pure water. \cite{AL96b,Luzar1996}
%
\begin{figure}[htpb]
\centering
\includegraphics [width=0.42\textwidth] {./diagrams/121}
\setlength{\abovecaptionskip}{0pt}
  \caption{\label{fig:121}The time dependence of the $k(t)$ for the water/vapor interface of neat water, calculated by DFTMD simulations.
  The inset shows the log-log plot of the $k(t)$.}
\end{figure}
%
%For the water/vapor interface of alkali-iodine solutions, the $k(t)$ is also calculated.  The result for the interface of 0.9 M LiI solution is shown in Fig.\thinspace\ref{fig:hbrf_4pl} (b). The log-log plot of $k(t)$ is not a straight line, indicating that, for water/vapor interface of the LiI solution, this decay does not coincide with a power-law decay, neither.


%{As can be seen from Fig. \ref{fig:hbrf_4pl}, the fluctuations of the $k (t)$ for $d = 2$ \AA (blue solid line) are significantly larger 
%than that of other cases with larger $d$. 
%This phenomenon is due to the relatively small number of water molecules in the thin layer 
%and the insufficient sampling, resulting in large fluctuations in $k(t)$.
%For these four models, as the thickness $d$ of the interface increases, the $k(t)$ gradually converges to a function with smaller fluctuations.
%%
%This conclusion is consistent with the two conclusions we obtained earlier (see Section \ref{sfg_alkali_iodide_interface}): 
%(1) \I is a strong structure-breaking anion; %[\cite{Trevani2000}] 
%(2) compared to pure water, the OH stretching peak at the interface of a solution containing iodide ions will blue shift. [\cite{Tongraar2010}] 
%Comparing these black solid curves, we can see that the interface of the solution containing ions has lower $k(t)$.
%In other words, compared to the pure water interface, 
%the ratio of H-bonds that were initially bonded at the solution interface and broken at time $t$ is lower.
%Because the effect of iodide ions is to increase the $k(t)$ of the interface, the decrease of $k (t)$ of the interface with a larger thickness
%may only be due to the contribution of cations located under the first layer of water molecules at the interface. 
%Therefore, although the iodide ion increases the HB rupture rate at the top layer of the interface, 
%in general, the HB rupture rate of the entire solution interface is reduced due to the presence of cations under the first layer of water molecules. 
%To verify this conclusion, we calculated the $k(t)$ at the interface of NaI (Fig. \ref{fig:hbrf_4pl} (c)) and KI (Fig. \ref{fig:hbrf_4pl} (d)) aqueous solution. 
%The results for both interface systems support our conclusions above.
%}
%\stkout{ What is the differences between bulk and interface? 
%Let us examine the difference in the $k(t)$ between interface water and bulk water. 
%No matter from pure water (Fig. \ref{fig:hbrf_4pl} (a)) 
%or solution (Fig. \ref{fig:hbrf_4pl} (b), (c) or (d)), we find that when the interface thickness is thin, the fluctuation of $k(t)$ is larger.
%Because the thinner the interface, the fewer pairs of water molecules that can form hydrogen bonds. 
%In our calculations, the fewer samples are used to average, so the fluctuation of $k (t)$ is greater. 
%We can find that at the interface of pure water, when $t> 0.2$ ps, the $k(t)$ value of the interface with different thickness is almost equal 
%at any time period $\Delta t$. For example, $\Delta t$ is selected as $\sim$ 2 ps, 
%and its average value is shown in Table \ref{tab:hbrf_neat}. In each time period of 2 ps, the values of $k(t)$ for different layers are approximately equal
%($\pm 0.004$ ps$^{-1}$). Therefore, as far as the nature of HB reactive flux is concerned, the difference between interface and bulk phase of neat water is not obvious. 
%}

%To show the effect of water molecule diffusion on the HB dynamics, we can calculate the sum of the functions $c(t)$ and $n(t)$, i.e., $c(t)+n(t)$.
%Here, we take the LiI solution as an example.
%Fig.\space\ref{fig:124_2LiI_ns20_c_plus_n} shows the time dependence of the correlation functions $c(t)$, $n(t)$ and $c(t)+n(t)$ of the interface of 
%the LiI solution at a concentration of 0.9 M in the AIMD simulation.
%As can be seen, although the change in the total population, $c(t)+n(t)$, is very small in the range of 0--10 ps, it is not a constant.
%Therefore, the $n(t)$ relaxes not only by conversion back to HB \emph{on} state, 
%but is also depleted due to the diffusion process. We can estimate the time scale of water molecule diffusion at the interface of the aqueous solution by $c(t)+n(t) = 1/e$, 
%which is much larger than 10 ps. Therefore, when we analyze the HB dynamics of the solution interfaces, we do not consider the effect of water molecule diffusion.
%
%\begin{figure}[H]
%\centering
%\includegraphics [width=0.36\textwidth] {./diagrams/124_2LiI_ns20_c_plus_n}
%\setlength{\abovecaptionskip}{0pt}
%\caption{\label{fig:124_2LiI_ns20_c_plus_n} 
%The time dependence of the functions $c(t)$, $n(t)$ and $c(t)+n(t)$, where $c(t)$ represents the $C_{\text{HB}}(t)$, 
%for the interfaces of 0.9 M LiI solution.} 
%\end{figure}
%
%\begin{figure}[H]
%\centering
%\includegraphics [width=0.5\textwidth] {./diagrams/128w_bk_2delta_t_60ps_n}
%\setlength{\abovecaptionskip}{0pt}
%\caption{\label{fig:128w_bk_2delta_t_60ps_n} 
%The time dependence of the population functions $n(t)$ for bulk water, as computed from the ADH (solid line) and AHD (dashed line) criterion of H-bonds.} 
%\end{figure}
% 

It can be seen from Fig.\thinspace\ref{fig:128w_bk_itp_50ps_n_from_k_in_with_2_hb_def_type2} that the $n(t)$ of the water/vapor interface of neat water
is always greater than the value of $n(t)$ in the bulk water. This means that "the hydrogen bond between a pair of water molecules at time $t$ is broken
and the distance between them is less than 3.5 \A" in the water/vapor interface is more likely to occur than in bulk water. 
We interpret this result as the fact that at time $t$, there is a greater probability that the H-bonds on the interface are broken 
compared to the H-bonds in the bulk water.
%
\begin{figure}[H]
\centering
\includegraphics [width=0.5\textwidth] {./diagrams/128w_bk_itp_50ps_n_from_k_in_with_2_hb_def_type2}
\setlength{\abovecaptionskip}{0pt}
\caption{\label{fig:128w_bk_itp_50ps_n_from_k_in_with_2_hb_def_type2} 
The time dependence of the population functions $n(t)$ for bulk water and the water/vapor interface from (a) ADH (b) AHD criteria. ($T$=300K.)} 
\end{figure}

To study the HB dynamics after the transition phase, which is roughly at 0.1 ps (see Fig.\thinspace\ref{fig:121}) and lasts for hundreds of picoseconds, 
we set $t_1 = 1$ ps and $t_2 = 10$ ps in the fitting.
For water/vpaor interfaces of neat water and the aqueous solution interfaces, 
the optimal values of $k$ and $k'$ given by these results have been listed in Table \ref{tab:k_k_prime_pure_and_solutions}. 
These values are comparable in magnitude to those obtained by Ref.\thinspace{\cite{Khaliullin2013}}. 
It can be seen from Table \ref{tab:k_k_prime_pure_and_solutions} that the HB breaking reaction rate ($k$) at the interface of pure water is basically equivalent to 
that at the solution interface, but the HB reforming rate ($k'$) is smaller than that at the solution interface by 30\% to 50\%.
Correspondingly, we can find the HB relaxation times of the three solution interfaces are: $\tau=\frac{1}{k+k'} \sim $2.0--2.5 ps. 
For pure water interface, the relaxation time is $\tau \sim $ 3.3 ps. 
Our conclusion is that the difference between the relaxation time of H-bonds at the interface of solutions such as LiI, NaI, KI 
and the interface of pure water is mainly due to the difference in the reforming rate $k'$ of H-bonds caused by the presence of ions,
rather than the difference in the breaking rate $k$ of H-bonds.
%
\begin{table}[htbp]
\centering
\caption{\label{tab:k_k_prime_pure_and_solutions} 
    The $k$ and $k'$ for the water/vapor interface of the aqueous solution interfaces.} 
\begin{tabular}{cccc}
 Interface & $k$ (ps$^{-1}$) & $k'$ (ps$^{-1}$) & $\tau_{\text{R}}$ (ps) \\
\hline
  Neat Water & 0.10 $\pm$ 0.02 & 0.20 $\pm$ 0.02 & 11.50 \\
  LiI & 0.10 $\pm$ 0.04 & 0.30 $\pm$ 0.05 & 5.33 \\
  NaI & 0.20 $\pm$ 0.10 & 0.30 $\pm$ 0.05 & 5.77 \\
  KI  & 0.10 $\pm$ 0.04 & 0.40 $\pm$ 0.10 & 6.96 
\end{tabular}
\end{table}

As for the effect of water/vapor interface on the HB dynamics in alkali-iodine solutions,
we also calculate the survival probability for interfaces with different sizes of thickness. 
The result for the interface of the LiI solution exhibits that H-bonds at water/vapor interface decay faster than that in bulk water.
The result for the logarithm of \SHB is displayed in Fig.\space\ref{fig:2LiI-124w_S_layers} in Appendix \ref{thickness_interface}, 
in which the thickness of the alkali-iodine solutions can be determined.
Therefore, as the interface thickness increases, the \SHB converges to a curve, 
which characterizes the HB dynamics of bulk solutions. 
In particular, it gives the average continuum HB lifetime in bulk solutions.
\FloatBarrier
\paragraph{Effects of the Ion Concentration}
Effects of ions' concentration on HB dynamics have been studied extensively by Chandra. \cite{AC00}
%Pal and coworkers provided details on the structure of water around the micellar surface.\cite{SP05} 
We calculated the \CHB for the water/vapor interfaces of the alkali-iodine solutions, 
and the relaxation time $\tau_{\text{R}}$ for each of them can be determined. 
%\begin{eqnarray}
%    C_{\text{HB}}(\tau_\text{{R}})=1/e. \nonumber
%\label{eq:relaxation_time}
%\end{eqnarray}
Here, the \emph{interface} means \emph{all} the water molecules in each model. 
The $\tau_{\text{R}}$ for the water/vapor interfaces of the LiI (NaI) solutions are displayed in 
Table \ref{tab:tau_hb}. Generally, they are in the range 1--10 ps. 
The values of $\tau_{\text{R}}$ decrease as the concentration of the solutions increases.
\begin{table}[htbp]
\centering
\caption{\label{tab:tau_hb} 
  The relaxation time $\tau_{\text{R}}$ (unit: ps) of the correlation function \CHB  for the water/vapor interface of the LiI (NaI) solutions, calculated by DFTMD simulations.}
\begin{tabular}{ccc}
  concentration  & $\tau_{\text{R}}$ (LiI) & $\tau_{\text{R}}$ (NaI) \\
\hline
  0 & 11.50 & 11.50 \\
  0.9 M & 7.04 & 10.60 \\
  1.8 M & 4.40 & 1.96 
\end{tabular}
\end{table}

The concentration dependence of the HB dynamics can be also found in the \SHB. 
Fig.\space\ref{fig:124_2LiI-2NaI_hbacf_S}(a) gives the \SHB 
for the water/vapor interfaces of 0.9 M and 1.8 M LiI solutions.
The same quantity for NaI solutions is displayed in Fig.\space\ref{fig:124_2LiI-2NaI_hbacf_S}(b).
This result indicates that, for the interface of alkali-iodine solution, the continuum HB lifetime  
decrease as the concentration of LiI (or NaI) solution increase.
\begin{figure}
\centering
\includegraphics [width=0.6\textwidth, center] {./diagrams/124_2LiI-2NaI_hbacf_S} 
\setlength{\abovecaptionskip}{0pt}
  \caption{\label{fig:124_2LiI-2NaI_hbacf_S} The time dependence of the \SHB  of 
  H-bonds at the water/vapor interfaces of (a) LiI and (b) NaI solutions at 330 K.
	The insets show the plots of ln$S_{\text{HB}}(t)$.} 
\end{figure}
%%
\FloatBarrier
\paragraph{Effect of Nitrate ions}
First, let us take a look at the changes in the hydrogen bond of water by nitrate ions in the bulk solution. 
To achieve this goal, we performed a DFTMD simulation on a system containing one \Li ion, one nitrate ion 
and 127 water molecules, that is, LiNO$_3$ solution. During the simulation, the temperature of the system is 
300 K and the volume is a constant, and periodic boundary conditions are used.
The $C_{SHB}(t)$ and $\ln{S_{SHB}(t)}$ for the H-bonds between water molecules in the solvation shell of Nitrogen atoms in bulk LiNO$_3$ solution is shown in 
\ref{fig:shb_c_and_s_ln_bk_NShell_pbc}. The values of radius 4.5, 6.5, 8.0 \AA come from the first three minimum values of RDF $g_{N-OW}(r)$ 
(see Fig.\thinspace\ref{fig:gdr_N-W_127_LiNO3}). 
Both correlations $C_{\text{SHB}}(t)$ and $\ln{S_{\text{SHB}}(t)}$ show that the longer the hydrogen bond from the nitrate ion, the slower the correlation function decays. 
In other words, the closer the hydrogen bond is to the nitrate, the faster it relaxes. Therefore, in the bulk LiNO$_3$ solution, nitrate ions accelerate the dynamical 
process of H-bonds in water.
%
%\begin{figure}[htbp] % or \begin{SCfigure}
%\centering
%\includegraphics [width=0.6\textwidth] {./diagrams/shb_c_and_s_ln_bk_NShell_pbc}
%\setlength{\abovecaptionskip}{0pt}
%\caption{\label{fig:shb_c_and_s_ln_bk_NShell_pbc} The $C_{SHB}(t)$ and $S_{SHB}(t)$ of water--water H-bonds at the solvation shell 
%  of nitrate ion in the \LiN solution. These results are calculated for the temporal resolution $t_t=0.4$ ps. For the definition 
%  of $t_t$, see Appendix \ref{thickness_interface}. }
%\end{figure}


%I simulate the alkali nitrate solution/vapor interface to find how the nitrate affect the structure of the interface.
\begin{figure}[htbp] % or \begin{SCfigure}
\centering
\includegraphics [width=0.36\textwidth] {./diagrams/256_LiNO3_hbacf_sh_no3} %fig.5.10
\setlength{\abovecaptionskip}{0pt}
\caption{\label{fig:256_LiNO3_hbacf_sh_no3} The \SHB of water--water (W--W) and nitrate--water (N--W) H-bonds at the water/vapor
  interface of the \LiN solution. The inset is the plot of ln\SHB. 
  These results are calculated for the temporal resolution $t_t=1$ fs. For the definition of $t_t$, see Appendix \ref{thickness_interface}. }
\end{figure}
%
%\begin{figure}[H]
%\centering
%\includegraphics [width=0.4\textwidth] {./diagrams/256_LiNO3_hbacf_hh_all_traj_sh_no3}
%\setlength{\abovecaptionskip}{20pt}
%\caption{\label{fig:256_LiNO3_hbacf_hh_all_traj_sh_no3}The functions ln\SHB of water--water H-bonds (black) and Nitrate -water H-bonds (red) in the the \LiN solution-vapor interface at 300 K. The lifetime of H-bonds $\tau_{\text{HB}}$ is calculated by the integration of \SHB over t$\in$(0,$\infty$), which give 0.42  and 0.20 ps, for water--water H-bonds and Nitrate -water H-bonds, respectively.}
%\end{figure}
%The density profile is a indicator of a table interfacial system (see Fig.\space\ref{fig:density_4MPlus_alkali-I}).
%\begin{figure}[htbp]
%\centering
% \includegraphics [width=0.6\textwidth] {./diagrams/density_4MPlus_alkali-I} %fig5.11
%\setlength{\abovecaptionskip}{20pt}
%\caption{\label{fig:density_4MPlus_alkali-I}The density as a function of the slab coordinate \Z. The result is calculated by MD with SPC water model.}
%\end{figure}
Next, let us see what effect the nitrate ion has on the H-bonds in the interface.
As shown in Fig.\space\ref{fig:vdos_LiNO3-256w_w_near_nitrate} in chapter ~\ref{CHAPTER_SFG_Calculation}, 
from the VDOS, the water molecules bound to \nitrate have higher OH stretching frequency (55 cm$^{-1}$ larger) 
than those H-bonded to other water molecules. 
%
Now, the difference between nitrate--water and water--water H-bonds 
can be also analyzed in terms of the survival probability $S_{\text{HB}}(t)$, \cite{AKS86,JT90,AL96} 
reported in Fig.\thinspace\ref {fig:256_LiNO3_hbacf_sh_no3}.
The integration of \SHB from 0 to $t_{\max}=5.0$ ps, \cite{Steinel2004} gives the relaxation time $\tau_\text{HB}$, which can be interpreted as 
the average HB lifetime. \cite{SC02} 
The values of $\tau_{\text{HB}}$ is dependent on a temporal resolution $t_t$, during which the H-bonds that break and reform are treated as intact. \cite{AL00} 
%
Here, we choose the temporal resolution as $t_t=1$ fs. 
Then, Fig.\thinspace\ref {fig:256_LiNO3_hbacf_sh_no3} gives $\tau_\text{HB}=0.20$ ps for nitrate--water H-bonds at interfaces, and $\tau_\text{HB}=0.42$ ps for water--water H-bonds.
This result of $\tau_\text{HB}$ is consistent with the experimental result of Kropman and Bakker ($\tau_\text{HB}=0.5\pm0.2$ ps). \cite{MFK01}
The smaller value of $\tau_\text{HB}$ for nitrate--water H-bonds implies that the nitrate--water H-bonds are weaker than bonds between water molecules. 
This is also consistent with the VDOS analysis and the blue-shifted frequency of the OH stretching in the nitrate-water HB. 
%[DELETED From both the VDOS and HB dynamics calculations, we conclude that it is the weak HBs between nitrate and water make the higher surface propensity 
%of nitrate anions, and then induce the depletion of SFG intensity at 3200 \cm for the alkali nitrate salty interfaces.]

%Fig. ~\ref{fig:256_LiNO3_hbacf_Nitrate_effect} shows that the nitrate ions accelerate the HB dynamics at the vapor/water interface of alkali nitrate solution.
%\begin{figure}[H]
%\centering 
% \includegraphics [width=0.6\textwidth] {./diagrams/256_LiNO3_hbacf_Nitrate_effect} %fig5.12
%\setlength{\abovecaptionskip}{20pt}
%\caption{\label{fig:256_LiNO3_hbacf_Nitrate_effect}The functions \CHB of bulk water--water H-bonds (W-W (Bulk)) and nitrate--water H-bonds (N-W) 
%at interfaces of alkali nitrate solution  (LiNO$_3$(H$_2$O$_{256}$)  at 300 K. }
%\end{figure} 
%NOT CLEAR, TO EXPLAIN BETTER The HB relaxation time is about $2.5$ ps, which is the same as that
%for nitrate--water H-bonds at interfaces of alkali nitrate solution.
%[NOT CLEAR: For bulk water, the HB relaxation time $\tau$ is $3.7$ ps. The difference between the HB dynamics of H-bonds outside the first shell of \Li and HB dynamics for nitrate--water H-bonds at interfaces
%is not visible from the values of the HB relaxation time. They reflect the difference between HB
%dynamics between bulk water and water/vapor interfaces.]
\paragraph{Effects of Alkali Metal Ions and \I on HB Dynamics}
%\begin{figure}[!ht]
%\centering
%\includegraphics [width=\textwidth] {./diagrams/C_S_HB_124_2LiI-2NaI-2KI} %fig5.15
%\setlength{\abovecaptionskip}{0pt}
%  \caption{\label{fig:C_S_HB_124_2LiI-2NaI-2KI} The time dependence of functions (a) \CHB and (b) \SHB of water--water H-bonds at water/vapor interfaces of 0.9 M alkali-iodine solutions.} 
%\end{figure}
%[hbtp]
\begin{table}[H]
\centering
\caption{\label{tab:tau_hb_alkali_iodine} 
The continuum HB lifetime $\tau_{\text{HB}}$ (unit: ps) in the first hydration shell of I$^-$ ion and of alkali metal ion at the water/vapor interface of 0.9 M LiI (NaI, KI) solution.}
\begin{tabular}{cccc}
  &\I-shell &cation-shell& interface \\
\hline
 LiI & 0.22 & 0.24 & 0.23\\
 NaI & 0.24 & 0.28 & 0.26\\
 KI  & 0.20 & 0.23 & 0.20\\
\end{tabular}
\end{table} 
%Water/Vapor & -&-&
Table \ref{tab:tau_hb_alkali_iodine} lists the continuum HB lifetime in the first hydration shell of I$^-$ ion and of alkali metal ion
at the interfaces of the three alkali-iodine solutions. It shows that, the continuum HB lifetime $\tau_{\text{HB}}$ in the 
solvation shell of alkali metal (iodine) ions is larger (smaller) than 
that of H-bonds at the water/vapor interfaces of the same solutions, 
respectively. For LiI solution, the water molecules bound to the cation ion
\Li, on average, have a continuum HB lifetime $\tau_{\text{HB}} \sim 0.24$ ps. This
 continuum HB lifetime is longer than that of molecules bound to \I or at the interface of the LiI solution. 
%
\begin{figure}[H]
\centering
\includegraphics [width=0.6\textwidth] {./diagrams/hbacf_C_sh2_2p}
\setlength{\abovecaptionskip}{0pt}
\caption{\label{fig:hbacf_C_sh2_2p}The \CHB of water--water H-bonds in the solvation shell 
  of (a) cations and (b) I$^-$ at the interfaces of 0.9 M LiI, NaI and KI solutions, respectively.
  The dashed line shows the \CHB for the interface (the thickness $d = 8$ \A) of the LiI solution.  
  This interface contains H-bonds between water molecules similar to those in bulk water, i.e.,
  water molecules participating in these H-bonds are not in the solvation shell of ions.} 
\end{figure}
%Fecko and co-workers' study of liquid D$_2$O by IR spectroscopy reveals that the vibrational dynamics observed are dominated by underdamped displacement of the hydrogen-bond coordinate at very short times ( less than 200 fs).\cite{CJF03,CJF05} 
Fig.\thinspace\ref{fig:hbacf_C_sh2_2p} a and b show that the \CHB of H-bonds within the alkali cations and \I decay faster 
than those in bulk water and at the surface of LiI solution.
From Fig.\thinspace\ref{fig:hbacf_C_sh2_2p}b, we find that, for all three alkali-iodine solutions, the \CHB for hydration shell water molecules 
of \I decays faster than that for molecules at the water/vapor interface.
%The simulation produces similar result as Omta and coworker's experiments of femtosecond pump-probe spectroscopy, which demonstrate that anions ( $\text{SO}^{2-}_4$, $\text{ClO}^-_4$, etc) have no influence on the dynamics of bulk water, even at high concentration up to 6 M.\cite{AWO03} 
%Here, we find that the cations \Li and \Na does not alter the H-bonding network outside the first hydration shell of cations. It is concluded that no long-range structural-changing effects for alkali metal cations.
The radii of hydration shells are 5.0 \AA for \li, 5.38 \AA for \na,
5.70 \AA for \pot, and 6.0 \AA for \I ions, which are obtained from the RDFs.
The RDFs $g_{\text{ion-O}}$ (ion=\li, \na) for the interfaces 
of LiI (NaI) solutions are shown in Fig.\thinspace\ref{fig:124_2NaI-2LiI_gdr_Li-O_Na-O_1501}(a),
and the coordination numbers are in Fig.\thinspace\ref{fig:124_2NaI-2LiI_gdr_Li-O_Na-O_1501}(b).
\begin{figure}[H]
\centering
\includegraphics [width=0.42\textwidth]{./diagrams/124_2NaI-2LiI_gdr_Li-O_Na-O_1501}%fig.6.1 
\setlength{\abovecaptionskip}{0pt}
\caption{\label{fig:124_2NaI-2LiI_gdr_Li-O_Na-O_1501}
  (a) The RDF $g_{\text{ion-O}}(r)$(ion=\li, \na) and (b) the coordination number of \Li (\na) ions at the interfaces of LiI (NaI) solution. 
  For \Na, the coordination number $n_\text{Na}$=5; while for \Li, $n_\text{Li}$=4.} 
\end{figure} % There is a first shell exist for both \Li and \Na cations.
%\section{Hydrogen Bond Dynamics by Classical Molecular Dynamics Simulations}
%\begin{figure}[H]
%\centering
% \includegraphics [width=0.5\textwidth] {./diagrams/4MPlus-alkali-I_hbacf_C1603}
%\setlength{\abovecaptionskip}{20pt}
%\caption{\label{fig:4MPlus-alkali-I_hbacf_C1603}The function \CHB of water--water H-bonds at interfaces with different alkali metal ions in 4.0 M water solution at 300 K.}
%\end{figure}
%The HB dynamics obtained from classical MD simulations can not catch the fast HB relaxation, and it give a totally different HB dynamics for the water molecules in these alkali halide solution/vapor interfaces.
%
\section{Solvation Shell Hydrogen Bond Dynamics} \label{SHB_dynamics}
\paragraph{Solvation Shells as Special interface}
We will extend the IHB dynamics to H-bonds around certain ions, so that we can more easily study the H-bonds in aqueous solutions. 
Similar to the determination of the instantaneous surface, we can define an solvation shell for molecules in aqueous solution systems. 
Below we will combine the new "interface" and Luzar-Chandler's HB population operator \cite{AL96} to select the H-bonds 
in the solvation shells of ions. The dynamics of these H-bonds will vary with the shell radius $r_{shell}$.
From the characteristics of HB dynamics in the solvation shells, we can obtain the effects of various ions on structure and dynamics of aqueous solutions. 

\paragraph{Solvation Shell Hydrogen Bond Population}
After one have determined the surface (solvation shell) ${\mathbf k}(t)={\mathbf k}(\{{\mathbf r}_i(t)\})$, we can define Solvation shell H-Bonds (SHBs).
We use the parameter $r_{shell}$ to denote the radius of the solvation shell.
Now we define the solvation shell HB population operator $h^{k}[{\mathbf r}(t)]$ as follows:
It has a value 1 when the the particular tagged molecular pair are H-bonded \emph{and} one of the molecules are inside the solvation shell
with a thickness $r_{shell}$, and zero otherwise. 
The definition of  $h^{k}[{\mathbf r}(t)]$ is very similar to $h^{s}[{\mathbf r}(t)]$, which is defined in \ref{IHBP} for studying the interfacial H-bonds.
Similarly, $h^{k}[{\mathbf r}(t)]$ can help us to efficiently obtain the dynamic characteristics of H-bonds in solvation shells of any radius. 
The definition of HB here can be based on water molecule pairs or O-H pairs. 
Like in the IHB case, in this paragraph, we also just discuss H-bonds based on water molecule pairs. 
For the hydrogen bond defined based on the O-H pairs, one can do a similar analysis.

Similar to the correlation function $C^s_\text{HB}(t)$ for the H-bonds in instantaneous interfaces,
we define the correlation function $C^{k,X}_\text{HB}(t)$ that describes the fluctuation of the solvation shell H-bonds for ion $X$: 
\begin{eqnarray}
C^{k,X}_{\text{HB}}(t)=\langle h^{k,X}(0)h^{k,X}(t) \rangle/\langle h^{k,X}\rangle
\label{eq:C_k_HB}.
\end{eqnarray}
When not considering specific ions, we denote it $C^{k}_\text{HB}(t)$ for short.
%
Similarly, we can define correlation functions 
\begin{eqnarray}
n^{k,X}(t)=\langle h^{k,X}(0)[1-h^{k,X}(t)]h^{(d),k,X} \rangle/\langle h^{k,X}\rangle
\label{eq:n_k_HB},
\end{eqnarray}
and 
\begin{eqnarray}
k^\text{k,X}(t)= -\frac{dC_\text{HB}^\text{k,X}}{dt}
\label{eq:k_k_HB}.
\end{eqnarray}
Then, using these correlation functions, we can determine the reaction rate constant of breaking and reforming and the lifetimes of solvation shell H-bonding.
%
\paragraph{$C^{k}_\text{HB}(t)$ as function of $r_\text{shell}$}
\begin{figure}[h]
\centering
\includegraphics [width=0.60\textwidth] {./diagrams/shb_c_ln_bk_NShell_pbc}
\setlength{\abovecaptionskip}{0pt}
\caption{\label{fig:shb_c_ln_bk_NShell_pbc}
The correlation function $C^\text{k,N}_\text{HB}(t)$ for the solvation shell H-bonds with differnt radius ($r_\text{shell}$), based on water-water 
pair HB population operator $h^\text{k}(t)$, as computed from the (a) ADH and (b) AHD criteria of H-bonds.} 
\end{figure}
%
\begin{figure}[h]
\centering
\includegraphics [width=0.60\textwidth] {./diagrams/shb_c_ln_bk_LiShell_pbc}
\setlength{\abovecaptionskip}{0pt}
\caption{\label{fig:shb_c_ln_bk_LiShell_pbc}
The correlation function $C^\text{k,Li}_\text{HB}(t)$ for the solvation shell H-bonds with differnt radius ($r_\text{shell}$), based on water-water 
pair HB population operator $h^\text{k}(t)$, as computed from the (a) ADH and (b) AHD criteria of H-bonds.} 
\end{figure}
Aso for \Li ion. The $S^\text{k,Li}_\text{HB}$.
\begin{figure}[h]
\centering
\includegraphics [width=0.60\textwidth] {./diagrams/shb_log_s_lii_bk_new_LiShell_pbc}
\setlength{\abovecaptionskip}{0pt}
\caption{\label{fig:shb_log_s_lii_bk_new_LiShell_pbc}
The logarithm of the correlation function $S^\text{k,Li}_\text{HB}(t)$ for the solvation shell H-bonds with differnt radius ($r_\text{shell}$), based on water-water 
pair HB population operator $h^\text{k}(t)$, as computed from the (a) ADH and (b) AHD criteria of H-bonds.} 
\end{figure}

In order to make the results clearer, we only added one cation and one anion to the simulated aqueous system. 
We have done DFTMD simulations for LiI, NaI, KI bulk system and interface system respectively.
\begin{figure}[h]
\centering
\includegraphics [width=0.60\textwidth] {./diagrams/shb_c_lii_itp_LiShell_pbc}
\setlength{\abovecaptionskip}{0pt}
\caption{\label{fig:shb_c_lii_itp_LiShell_pbc}
The correlation function $C^{k,Li}_{HB}(t)$ for the solvation shell H-bonds (in water/vapor interface) with differnt radius ($r_\text{shell}$), based on water-water 
pair HB population operator $h^{k}(t)$, as computed from the (a) ADH and (b) AHD criteria of H-bonds. 
The existence of the interface makes the relationship between the correlation function and $r_\text{shell}$ more complicated.} 
\end{figure}
%
\begin{figure}[h]
\centering
\includegraphics [width=0.60\textwidth] {./diagrams/shb_c_li_bk_IShell_pbc}
\setlength{\abovecaptionskip}{0pt}
\caption{\label{fig:shb_c_li_bk_IShell_pbc}
The correlation function $C^\text{k,I}_\text{HB}(t)$ (version 1) for the solvation shell H-bonds with differnt radius ($r_\text{shell}$), based on water-water 
pair HB population operator $h^\text{k}(t)$, as computed from the (a) ADH and (b) AHD criteria of H-bonds. The results are calculated from the simulated bulk LiI solution.
WHICH RESULT IS CORRECT? WE HAVE TO CHECK THE SAME CALCULATION IN INTERFACE, AND IN DIFFERENT TEMPERATURE, AND IN DIFFERENT SOLUTIONS LIKE LiI,NaI,and KI solutions,
before we obtain the conclusion.} 
\end{figure}
%
\begin{figure}[h]
\centering
\includegraphics [width=0.60\textwidth] {./diagrams/shb_c_lii_itp_IShell_pbc}
\setlength{\abovecaptionskip}{0pt}
\caption{\label{fig:shb_c_lii_itp_IShell_pbc}
The correlation function $C^\text{k,I}_\text{HB}(t)$ for the solvation shell H-bonds (in water--vapor interface) with differnt radius ($r_\text{shell}$), based on water-water 
pair HB population operator $h^\text{k}(t)$, as computed from the (a) ADH and (b) AHD criteria of H-bonds. The results are calculated from the simulated bulk LiI solution.
WHICH RESULT IS CORRECT? WE HAVE TO CHECK THE SAME CALCULATION IN INTERFACE, AND IN DIFFERENT TEMPERATURE, AND IN DIFFERENT SOLUTIONS LIKE LiI,NaI,and KI solutions,
before we obtain the conclusion.} 
\end{figure}
The $\ln S^\text{k,I}_\text{HB}(t)$ (see Fig.\thinspace\ref{fig:shb_log_s_lii_bk_new_IShell_pbc}). 
\begin{figure}[h]
\centering
\includegraphics [width=0.60\textwidth] {./diagrams/shb_log_s_lii_bk_new_IShell_pbc}
\setlength{\abovecaptionskip}{0pt}
\caption{\label{fig:shb_log_s_lii_bk_new_IShell_pbc}
The logarithm of the correlation function $S^\text{k,I}_\text{HB}(t)$ for the solvation shell H-bonds with differnt radius ($r_\text{shell}$), based on water-water 
pair HB population operator $h^\text{k}(t)$, as computed from the (a) ADH and (b) AHD criteria of H-bonds. The results are calculated from the simulated bulk LiI solution.
WHICH RESULT IS CORRECT? WE HAVE TO CHECK THE SAME CALCULATION IN INTERFACE, AND IN DIFFERENT TEMPERATURE, AND IN DIFFERENT SOLUTIONS LIKE LiI,NaI,and KI solutions,
before we obtain the conclusion.} 
\end{figure}
%
\begin{figure}[htb]
\centering
\includegraphics [width=0.60\textwidth] {./diagrams/128w_itp_pure_water_pair_c_ihb}
\setlength{\abovecaptionskip}{0pt}
\caption{\label{fig:128w_itp_pure_water_pair_c_ihb}[TODO] 
The correlation function $C^s_\text{HB}(t)$ for the instantaneous interfacial H-bonds with differnt thickness ($d$), based on water-water 
pair HB population operator $h^\text{s}(t)$, as computed from the (a) ADH and (b) AHD criteria of H-bonds.} 
\end{figure}
For the bulk LiNO$_3$, we used two geometric criteria of H-bonds to calculate the $h^\text{k}(t)$ and therefore correlation function $C^\text{k}_\text{HB}(t)$. 
The calculation results of the $C^s_\text{HB}(t)$ are shown in Fig.\thinspace\ref{fig:128w_itp_pure_water_pair_c_ihb}.
We find that the greater the thickness $d$ of the instantaneous interface is selected, 
the slower the relaxation of the interface H-bonds. When the thickness is greater than a certain thickness $d^c$ ( $\sim$ 3 \AA),
the relaxation of H-bonds at the interface hardly changes.
%
\begin{figure}[H]
\centering                                         
\includegraphics [width=0.6\textwidth] {./diagrams/128w_itp_pure_water_pair_c_ihb_scheme1}
\setlength{\abovecaptionskip}{0pt}
\caption{\label{fig:128w_itp_pure_water_pair_c_ihb_scheme1} [TODO]
The correlation functions $C_\text{HB}(t)$ for the instantaneous interfacial H-bonds with different $d$, based on water-water pair HB population operator $h(t)$, 
as computed from the (a) ADH and (b) AHD criteria of H-bonds. These results are based on selecting the water molecules in the instantaneous interface and averaging 
the correlation functions of these water molecules. The sampling is performed every 4 ps. } 
\end{figure}

For water/vapor interface,
\begin{figure}[H]
\centering                                         
\includegraphics [width=0.6\textwidth] {./diagrams/shb_n_lii_itp_IShell_pbc}
\setlength{\abovecaptionskip}{0pt}
\caption{\label{fig:shb_n_lii_itp_IShell_pbc} 
The correlation functions $n^{k,I}_\text{HB}(t)$ for the solvation shell H-bonds (in water/vapor interface system) with different $r_{shell}$, 
based on water-water pair HB population operator $h^{k,I}(t)$, 
as computed from the (a) ADH and (b) AHD criteria of H-bonds. [COMMENTS on WATER MOLECULE DIFFUSION, TODO; We'd better discuss the effect of def. of HB,
effect of interface, effect of solvation shell, separately.]} 
\end{figure}
\begin{figure}[H]
\centering                                         
\includegraphics [width=0.6\textwidth] {./diagrams/shb_n_lii_itp_LiShell_pbc}
\setlength{\abovecaptionskip}{0pt}
\caption{\label{fig:shb_n_lii_itp_LiShell_pbc} 
The correlation functions $n^{k,Li}_\text{HB}(t)$ for the solvation shell H-bonds (in water/vapor interface system) with different $r_{shell}$, 
based on water-water pair HB population operator $h^{k,Li}(t)$, 
as computed from the (a) ADH and (b) AHD criteria of H-bonds. [COMMENTS on WATER MOLECULE DIFFUSION, TODO]} 
\end{figure}

TODO:
For comparison, let us look at the HB dynamics of water molecules in the solvation shells obtained by selecting molecules 
located in instantaneous interface. (see Appendix \ref{ihb_and_selection}) 
In this algorithm, we first select the molecules in the interface at each moment and then make a statistical
average of the calculated correlation functions.
Specifically, to determine which water molecules are located in the instantaneous interface, we sample at regular intervals, and then calculate 
the correlation function $C_\text{HB}(t)$ for the water molecules located in the interface and their a statistical average.
As the thickness $d$ of the instantaneous interface changes, the $C_\text{HB}(t)$ in the interface will also change. 
Figure \ref{fig:128w_itp_pure_water_pair_c_ihb_scheme1} shows how the function $C_\text{HB}(t)$ changes with the thickness $d$.
The sub-figure (a) and (b) use HB definition criterion ADH, and AHD, respectively.
Comparing Fig.\thinspace\ref{fig:128w_itp_pure_water_pair_c_ihb} and Fig.\thinspace\ref{fig:128w_itp_pure_water_pair_c_ihb_scheme1}, we see that
when we use the method of selecting molecules in the interface, the dependence of the correlation function $C_\text{HB}(t)$  
on the interface thickness is very consistent with that of $C^s_{HB}(t)$. Moreover, regardless of the AHD definition 
or the ADH definition of the HB, this conclusion is basically valid. Beside the correlation functions $C_\text{HB}(t)$ or $C^s_\text{HB}(t)$ in the interface, we will further examine the correlation 
functions $C_\text{HB}(t)$ ($C^s_\text{HB}(t)$), $n(t)$ ($n^s(t)$), $k(t)$ ($k^s(t)$), and the rate constants $k$, $k'$ determined by them.

%[Plot the $k$ and $k'$ as functions of thickness $d$.]
\paragraph{Rate Constants $k$ and $k'$} 
[TODO]
To find the reaction rate constants $k$ and $k'$, we can start from the correlation functions $C^k_\text{HB}(t)$, $n^k(t)$ and $k^k(t)$, 
we can also first select the water molecules in the instantaneous interface at each time point $t$, and start from the corresponding 
correlation functions $C_\text{HB}(t)$, $n(t)$ and $k(t)$ of the H-bonds of these selected water molecules.
Figure \ref{fig:128w_itp_pure_water_pair_k_k_prime_ihb_both_schemes} compares the rate constants ($k$ and $k'$) 
and the lifetime $\tau_\text{HB}$ obtained by the two different methods mentioned above, i.e., IHB and molecule selection methods. 
We see that, whether it is $k$, $k'$ or $\tau_\text{HB}$, their changing \emph{trend} with the thickness $d$ of the 
instantaneous interface is only slightly affected by the calculation methods. 
To illustrate this point more clearly, we compare the $k$, $k'$ and $\tau_\text{HB}$ obtained under the two methods.
%
\begin{figure}[H]
\centering
\includegraphics [width=0.6\textwidth] {./diagrams/128w_itp_pure_water_pair_k_k_prime_ihb_both_schemes}
\setlength{\abovecaptionskip}{0pt}
\caption{\label{fig:128w_itp_pure_water_pair_k_k_prime_ihb_both_schemes} 
TODO: The dependence of (a) the rate constants $k$ and $k'$ and (b) the HB lifetime $\tau_\text{HB}$ on the interface thickness,
obtained by the Instantaneous interface Hydrogen Bond (IHB) and by selecting the water molecules in the interface, respectively.
The corresponding $k$, $k'$ and $\tau_\text{HB}$ in the bulk water are also drawn with dashed lines as a reference.
In sub-figure (a), the $k$ of bulk water is represented by a \emph{black dashed} line, and the $k'$ is represented by a \emph{blue dashed} line;
in sub-figure (b), the $\tau_\text{HB}$ of bulk water is represented by a \emph{black dashed} line.
The ADH criterion of H-bonds is used and the least square fits are carried on the time 
region 0.2 ps $< t <$ 12 ps.}
\end{figure}
TODO: As we can see from Fig.\thinspace\ref{fig:128w_itp_pure_water_pair_k_k_prime_ihb_both_schemes}, 
when the thickness is large enough ($d_0 \sim 3$ \AA), these two constants agree well quantitatively. 
This result shows that the two extreme statistical methods (see Appendix \ref{ihb_and_selection}) 
for the HB dynamics of the interface did not produce much difference for the time scale (10$^2$ ps) 
and the scale ( 10$^2$ \AA ) of the simulation box we currently use.

In summary, [TODO] 
\begin{table}[htb]
\centering
\caption{\label{tab:k_k_prime_tau_128w_pure_ihb_ADH} 
    TODO The $k$ and $k'$ for the interfacial hydrogen dynamics of the water/vapor interface (by the method of IHB and of ADH criteria). 
We carried on the longer time region 0.2 ps $< t <$ 12 ps. The unit for $k$ ($k'$) is ps$^{-1}$, and for $\tau_{\text{HB}}$ ($=1/k$) 
is ps. The parameter values and units are the same below.} 
\begin{tabular}{cccc}
 Thickness & $k$ & $k'$ & $\tau_{\text{HB}} (=1/k)$ \\
\hline
  1.0 & 0.653 & 0.080 & 1.53  \\
  2.0 & 0.261 & 0.133 & 3.83  \\
  3.0 & 0.168 & 0.104 & 5.94  \\
  4.0 & 0.148 & 0.092 & 6.76  \\
  5.0 & 0.147 & 0.087 & 6.81  \\
  6.0 & 0.139 & 0.087 & 7.17  \\
\end{tabular}
\end{table}
\begin{table}[htb]
\centering
\caption{\label{tab:k_k_prime_tau_128w_pure_ihb_AHD} 
    TODO The $k$ and $k'$ for the interfacial hydrogen dynamics of the water/vapor interface (by the method of IHB and of AHD criteria).} 
\begin{tabular}{cccc}
 Thickness & $k$ & $k'$ & $\tau_{\text{HB}} (=1/k)$ \\
\hline
  1.0 & 0.661 & 0.080 & 1.51  \\
  2.0 & 0.265 & 0.133 & 3.77  \\
  3.0 & 0.172 & 0.102 & 5.82  \\
  4.0 & 0.148 & 0.090 & 6.74  \\
  5.0 & 0.149 & 0.084 & 6.72  \\
\end{tabular}
\end{table}

\begin{table}[H]
\centering
\caption{\label{tab:k_k_prime_tau_128w_pure_ihb_scheme1_ADH} 
    TODO The $k$ and $k'$ for the interfacial hydrogen dynamics of the water/vapor interface (by the method of molecule selection and of ADH criteria).} 
\begin{tabular}{cccc}
 Thickness & $k$ & $k'$ & $\tau_{\text{HB}} (=1/k)$ \\
\hline
  1.0 & 0.526 & 0.072 & 1.90  \\
  2.0 & 0.246 & 0.158 & 4.07  \\
  3.0 & 0.160 & 0.114 & 6.26  \\
  4.0 & 0.140 & 0.097 & 7.15  \\
  5.0 & 0.138 & 0.090 & 7.24  \\
\end{tabular}
\end{table}
\begin{table}[H]
\centering
\caption{\label{tab:k_k_prime_tau_128w_pure_ihb_AHD} 
    TODO The $k$ and $k'$ for the interfacial hydrogen dynamics of the water/vapor interface (by the method of molecule selection and of AHD criteria).} 
\begin{tabular}{cccc}
 Thickness & $k$ & $k'$ & $\tau_{\text{HB}} (=1/k)$ \\
\hline
  1.0 & 0.610 & 0.083 & 1.64  \\
  2.0 & 0.235 & 0.142 & 4.62  \\
  3.0 & 0.138 & 0.102 & 7.22  \\
  4.0 & 0.141 & 0.098 & 7.07  \\
  5.0 & 0.120 & 0.078 & 8.40  \\
\end{tabular}
\end{table}

\section{Rotational Anisotropy Decay of Water at the Interface of Alkali-Iodine Solutions}\label{RAD}
Using the transition dipole auto-correlation function, 
we determined the rotational anisotropy decay and therefore the OH-stretch relaxation at water/vapor interface of alkali iodide solutions.
%The effects of ion environment on structure and dynamics of water are obtained by comparing the second-order Legendre polynomial, 
%i.e.,  $P_2(x)=\frac{1}{2}(3x^2-1)$,  orientational correlation function of the transition dipole.
The anisotropy decay can be determined from experimental signal in two different polarization configurations---parallel and perpendicular polarizations, by 
\begin{equation}
        R(t)=\frac{S_{\parallel}(t)-S_{\perp}(t)}{S_{\parallel}(t)+2S_{\perp}(t)}
\label{eq:ad}
\end{equation}
where $t$ is the time between pump and probe laser pulses.  The anisotropy decay can also be obtained by simulations, and calculated by the third-order response functions $R(t)$. \cite{Jansen10,Jansen06}
%

In the first shell with a radius 3 \A, the entropy difference betweem the \Li shell and \Na shell,
$\Delta S=k_B\text{ln}\frac{\Omega_\text{Na}}{\Omega_\text{Li}}=k_B\text{ln}\frac{n_\text{Na}/V_\text{Na}}{n_\text{Li}/V_\text{Li}} =k_B\text{ln}1.05$.

%
%\paragraph{Probability Distribution of Ions}
%The probability distribution, shown in Fig.~\ref{fig: prob_124_LiI_Sans_double_axis}, of the ions in the water/vapor interface of LiI and NaI solutions with repect to the depth of the ions in the solutions 
%indicates that the \I ions prefer to staying at the topmost layer of surface of solutions.
%(molar concentration: 0.9 M, temperature: 330 K) 
%It shows that \I ions tend to the surface of the solutions, while \Na and \Li tend to stay in the bulk. This result is consistent with the calculations from Ishiyama and Morita\cite{TI07,TI14}.
The orientational anisotropy $C_2(t)$ is given by the rotational time-correlation function 
\begin{equation}
C_2(t)=\langle P_2(\hat{u}(0)\cdot\hat{u}(t)) \rangle,
\label{eq:tcf2}
\end{equation}
where $\hat{u}(t)$ is the time dependent unit vector of the transition dipole, $P_2(x)$ is the second Legendre polynomial, and 
$\langle \rangle$ indicate equilibrium ensemble average.\cite{Corcelli05,LinYS2010} %\cite{2010Lin} % angular brackets

The anisotropy decay $C_2(t)$ for the water/vapor interface of LiI solution is shown in Fig.\space\ref{fig:c2_2LiI_16_inset}.
This function decays faster than that of neat water, indicating that H-bonds
at the interfaces of alkali-iodine solutions reorient faster than in neat water. The inset shows the first 0.4 ps of $C_2(t)$, from which we see a 
quick change during the first $\sim 0.1$ ps primarily due to librations.
%
\begin{figure}[h]
\centering
\includegraphics [width=0.36\textwidth] {./diagrams/c2_2LiI_16_inset} 
\setlength{\abovecaptionskip}{0pt}
  \caption{\label{fig:c2_2LiI_16_inset} The time dependence of the $C_2(t)$ of OH bonds at the water/vapor interfaces of 0.9 M LiI solution 
  and of neat water (dashed line) at 330 K, calculated by DFTMD simulations. The water/vapor interface of neat water is modeled with a slab 
  made of 121 water molecules in a simulation box of size $15.60 \times 15.60 \times 31.00$ \A$^3$.}
\end{figure}
%
We also calculated the $C_2(t)$ for the interface of other alkali-iodine solutions LiI and KI. 
The results of $C_2(t)$ for the water/vapor interfaces of these solutions are shown in Fig.\thinspace\ref{fig:c2_2KI_2NaI_2LiI_16}.
In all the cases $C_2(t)$ decays faster than in neat water, indicating that H-bonds
at the interfaces of the three alkali-iodine solutions are orientated faster than that of neat water.
They show that \I ions can accelerate the dynamics of molecular reorientation of water molecules at interfaces.   

%
\begin{figure}[htbp]
\centering
\includegraphics [width=0.36 \textwidth] {./diagrams/c2_2KI_2NaI_2LiI_16} 
\setlength{\abovecaptionskip}{0pt}
  \caption{\label{fig:c2_2KI_2NaI_2LiI_16} The time dependence of the $C_2(t)$ of OH bonds in water molecules at the water/vapor 
  interface of 0.9 M alkali-iodine solutions and of neat water (dashed line) at 330 K, calculated by DFTMD simulations.}
\end{figure} 

We have obtained non-single-exponential kinetics for the rotation of water molecules both at the surface 
and in bulk water (Appendix \ref{single_exp}).
%This result is true for water molecules bound to ions. 
Therefore, the rotational motion of water molecules are not simply characterized by well-defined rate constants. 
%Then the problem is to understand the kinetics.
Similar non-single-exponential kinetics is also obtained in the HB kinetics
in liquid water \cite{AL96,Dirama05} and in the time variation of the average frequency shifts of the 
remaining modes after excitation in hole burning technique \cite{Rey2002,Moller2004} and using BLYP functional. \cite{Bankura2014}
Luzar and Chandler interpreted 
the non-single-exponential kinetics as the result of an interplay between 
diffusion and HB dynamics. \cite{AL96} 
We can understand the non-single-exponential kinetics of rotational 
anisotropy decay by fitting the rotational anisotropy decay by a 
biexponential function.

To obtain the effects of diffusion and HB decay of water molecules
in solutions respectively, we assume that there are two independent 
decays in the process of an anisotropy decay. 
Therefore, the $C_2(t)$ has the form \cite{TanHS05}
\begin{equation}
C_2(t)=A_1e^{-\kappa_1 t} +A_2e^{-\kappa_2 t},
\label{eq:tcf3}
\end{equation}
where $A_i$ are constants and $\kappa_i$ are decay rates ($i=1, 2$). 
The time constants and amplitudes of the biexponentials fits for 
the $C_2(t)$ are listed in Table ~\ref{tab:2LiI_c2_biexp} and Table ~\ref{tab:2NaI_c2_biexp}.
The biexponential fit is very close to the calculated $C_2(t)$, which can be seen in Fig.\space\ref{fig:2LiI-124w_c2_fit_5ps_biexp}. 
%(compare Fig.\space\ref{fig:2LiI-124w_c2_fit_5_single-exp}).
%
\begin{table}[hbt]
\centering
\caption{\label{tab:2LiI_c2_biexp}%
	Biexponential fitting (5 ps) of the $C_2(t)$ for water molecules in 0.9 M LiI solution.}
%\begin{ruledtabular}
\begin{tabular}{lccccc}
water molecules & $A_1$  & $\kappa_1$ (THz) & $A_2$ & $\kappa_2$ (THz) \\
\hline
I$^-$-shell & 0.44 & 0.25 & 0.39 & 0.26\\
Li$^+$-shell & 0.88 & 0.07 & 0.07 & 8.24\\
bulk & 0.84 & 0.11 & 0.09 & 4.88 \\
surface & 0.73 & 0.27 & 0.22 & 13.47 \\
\end{tabular}
%\end{ruledtabular}
\end{table}
%--

\begin{table}
\centering
  \caption{\label{tab:2NaI_c2_biexp}%
	Biexponential fitting (5 ps) of the $C_2(t)$ for water molecules in 0.9 M NaI solution.}
  \begin{tabular}{lccccc}
  water molecules & $A_1$  & $\kappa_1$ (THz) & $A_2$ & $\kappa_2$ (THz) \\
  \hline
  I$^-$-shell & 0.86 & 0.14 & 0.08 &9.86 \\
  Na$^+$-shell & 0.71 & 0.06 & 0.18 &0.79 \\
  bulk & 0.81 & 0.06 & 0.10 & 1.25 \\
  surface & 0.77 & 0.11 & 0.13 & 2.31 \\
  \end{tabular}
\end{table}
%
%图
\begin{figure}[htbp]
\centering
\includegraphics [width= 0.6\textwidth] {./diagrams/2LiI-124w_c2_fit_5_biexp} 
  \caption{\label{fig:2LiI-124w_c2_fit_5ps_biexp} The time dependence of the $C_2(t)$ of OH bonds 
  in water molecules at the water/vapor interface of LiI solution.}
\end{figure} 
%
%[Notes: The 63-water-slab models is listed here as a reference. The number of water molecules is small; The data for KI/vapor and LiI/vapor interfaces come from  KI\_16 and LiI\_16 systems.  
%Water(63) &0.831$\pm(1\times10^{-4})$ &  0.08760 $\pm(2\times 10^{-5})$&0.100$\pm(2\times10^{-4})$ & 1.029 $\pm(4\times10^{-3})$  \\ ]
%
%\begin{figure}[htbp]
%\centering
%\includegraphics [width=0.4 \textwidth] {./diagrams/c2_121-pure_2KI_2LiI_16_inset_fit_biexp} 
%\setlength{\abovecaptionskip}{10pt}
%\caption{\label{fig:c2_121-pure_2KI_2LiI_16_inset_fit_biexp} The fitted and calculated anisotropy decay of OH bonds in water molecules in LiI solution/vapor interface (red), LiI solution/vapor interface (blue) and neat water/vapor interface (black). The corresponding fitted functions are denoted by dashed lines. The concentration of LiI and KI solution is 0.9 M.}
%\end{figure} 

Then we considered the effect of ion species in solutions on the anisotropy decay of water molecules.
From Table \ref{tab:2LiI_c2_biexp} and Table \ref{tab:2NaI_c2_biexp}, we find that 
for both LiI and NaI solutions, there are two decay processes in the dynamics --- amplitude $\sim 1$,
decay constant $\sim$ 0.1 THz, and for the other describe the initial fast decay 
of the anisotropy, with amplitude $\sim 0.1$, decay constant $\sim$ (1--10) THz, 
due to the inertial-librational motion preceding the orientational diffusion.
That is, two decay processes exist in the dynamics of water molecules 
at the water/vapor interfaces of alkali-iodine solutions. 
%The one describe the initial fast decay of the anisotropy, 
%with amplitude $\sim$ 0.1, decay constant $\sim$ (1--10) THz,
%results from the inertial-librational motion preceding the orientational diffusion.
%
\begin{table}[H]
\centering
\caption{\label{tab:fitting_c2_for_each_type_of_water}%
  Biexponentially fitting (2 ps) of the $C_2(t)$ for different types of water molecules at the water/vapor interface of LiI solutions.}
\begin{tabular}{lccccc}
water molecules & $A_1$  & $\kappa_1$ (THz) & $A_2$ & $\kappa_2$ (THz) \\
\hline
$DDAA$ & 0.85 & 0.25 & 0.10 & 16.0\\
$DD'AA$ & 0.89 & 0.14 & 0.06 & 14.1 \\
$D'AA$ & 0.38 & 0.99 & 0.38 & 0.99 \\
\end{tabular}
\end{table}
%
\begin{table}[H] %[!hbtp]
\centering
\caption{\label{tab:table_CoordNo}%
The coordination number of the atoms in LiI (NaI) solutions.}
\begin{tabular}{lccc}
name & radius of the first shell (\AA) & coordination number \\
\hline
$n_\text{I-H}(\text{LiI})$ & 3.3 & 5.5 \\
$n_\text{I-H}(\text{NaI)}$ & 3.3 & 5.1 \\
$n_\text{I-O}(\text{LiI)}$ & 4.3 & 5.8 \\
$n_\text{I-O}(\text{NaI)}$ & 4.3 & 6.0 \\
$n_\text{Li-O}(\text{LiI)}$ & 3.0 & 4.0 \\
$n_\text{Na-O}(\text{NaI)}$ & 3.5 & 6.0 
\end{tabular}
\end{table}

%In the first shell with a radius 3 \A, the entropy difference between the \Li shell and \Na shell,
%$\Delta S=k_B\text{ln}\frac{\Omega_\text{Na}}{\Omega_\text{Li}}=k_B\text{ln}\frac{n_\text{Na}/V_\text{Na}}{n_\text{Li}/V_\text{Li}} =k_B\text{ln}1.05$.

\paragraph{Anisotropy Decay in Hydration Shells of Ions}
In order to verify whether the water molecules in the hydration shell of different ions have different orientation dynamics. 
We calculated the anisotropy decay of water molecules in \Li and nitrate ions' hydration shells in the LiNO$_3$ interface. 
The average results of the $C_2(t)$ is shown in Fig.\thinspace\ref{fig:C2_ln_itp_pbc}.
The radius of the hydration shell of nitrate O, \Li, and water molecules are taken as 4.0, 2.8 and 3.5 \AA, respectively. 
These values come from the radial distribution functions $g_{\text{N-OW}}(r)$, $g_{\text{Li-OW}}(r)$ and $g_{\text{OW-OW}}(r)$,
for bulk LiNO$_3$ solution, as shown in Fig.\thinspace\ref{fig:gdr_127_LiNO3}a and Fig.\thinspace\ref{fig:gdr_NW_WW_127_LiNO3}.
In view of the diffusion of molecules, we only counts trajectories with a duration of 10 ps. The total number of trajectories is 6.
\begin{figure}
\centering
\includegraphics [width=0.42\textwidth] {./diagrams/C2_ln_itp_pbc} 
\setlength{\abovecaptionskip}{0pt}
\caption{\label{fig:C2_ln_itp_pbc}The $C_2(t)$ of water in the solvation shell of water, \Li and nitrate ions in the interface system of \LiN solution.} 
\end{figure}
\begin{figure}
\centering
\includegraphics [width=0.8\textwidth] {./diagrams/gdr_NW_WW_127_LiNO3} 
\setlength{\abovecaptionskip}{0pt}
\caption{\label{fig:gdr_NW_WW_127_LiNO3}The RDFs for the \LiN solution at $T=300$ K.}
\end{figure}
\paragraph{Classification of Water Molecules Based on H-Bonds}
We also studied the relation between the anisotropy decay of water molecules and their environment. 
Following the definition used in Ref.\cite{TianCS08}, we use the following labels to denote water molecules in solution: 
$D$ denotes that the water molecule donates a HB, $D'$ donates that the water donates a H-I bond, and $A$ donates that the water accepts a HB. %\cite{2008NJ} 
$DDAA$ represents a water molecule with two H-Bonds donated to water molecules and two H-Bonds accepted from water molecules (see Fig.\thinspace\ref{fig:Multiple_figs}(a));
$DD'AA$ represents a water molecule with two HBs donated to a water molecule and \I, and with two H-Bonds accepted from other water molecules 
(see Fig.\thinspace\ref{fig:Multiple_figs}(c)), 
$D'AA$ represents a water molecule bonded to \I at the water/vapor interface and other H-Bonds to water molecules (see Fig.\thinspace\ref{fig:Multiple_figs}(d)).
Clearly, we can see that $D'AA$ molecules are of free OH stretching during the dynamics. All four types of water molecules are displayed in Fig.\thinspace\ref{fig:Multiple_figs}. 
% 
\begin{figure}[ht]%[!htbp]
\centering
\includegraphics [width=0.4 \textwidth] {./diagrams/Multiple_figs} 
\caption{\label{fig:Multiple_figs} Four types of water molecules at the water/vapor interfaces of LiI solution, regarding the HB environments: (a) $DDAA$; (b) $DDA$; (c) $DD'AA$; (d) $D'AA$. The cyan balls denote \I ions. }
\end{figure} 

It is evident, from our calculations (Fig.\space\ref{fig:2LiI-124w_c2_fit_biexp_7wat_2ps_class_150324}), that the $C_2(t)$ for $DDAA$ and $DD'AA$ molecules do not decay exponentially (Table \ref{tab:fitting_c2_for_each_type_of_water}).
%[BUT Table \ref{tab:fitting_c2_for_each_type_of_water} CAN NOT GIVE THE EVIDENCE. STH. IS MISSING!] 
This result is similar to the reactive flux HB correlation function $k(t)$, i.e., 
the escaping rate kinetics of H-bonds in bulk water. \cite{Luzar1996} 
The relaxation of H-bonds in water appears complicated, with no simple characterization in terms of a few relaxation rate constants. 
Most of the authors believe that the cooperativity between neighbouring H-bonds, \cite{Sciortino1989, Ohmine1995} or 
self evident coupling between translational diffusion and HB dynamics is the source of the complexity. \cite{Luzar1996} 
However, for $D'AA$ molecules at the interface of the LiI solution,
the $C_2(t)$ decays exponentially, i.e.
\begin{eqnarray}
  C_2(t) &=& C e^{-{\kappa}t},
\label{eq:C_2_D_prime_AA}
\end{eqnarray}
where the amplitude is $C=0.76$, and the reorientation rate is $\kappa = 0.99$ ps$^{-1}$.
The single exponential decay of $C_2(t)$ for $D'AA$ molecules, indicates that each $D'AA$  molecule reorientate independently to each other. 

Furthermore, the $C_2(t)$ for $D'AA$ molecules decays much faster than that for $DDAA$ or $DD'AA$ molecules.
From the definitions, the $D'AA$ water molecule owns only three H-bonds, while both $DDAA$ and $DD'AA$ water molecules own four H-bonds.
Therefore, the correlation between H-bonds around the $D'AA$ molecule is weaker than those around the $DDAA$ or $DD'AA$ molecule. 
Faster decay of $C_2(t)$ for $D'AA$ molecules shows that the reorientation process of $D'AA$
molecules is much smaller than those water molecules in bulk phase, e.g., the $DDAA$, and $DD'AA$ molecules.

Finally, for $D'AA$ molecules, the inertial-librational motion can not be seen (Fig.\space\ref{fig:2LiI-124w_c2_fit_biexp_7wat_2ps_class_150324}). 
This result implies that the rotational anisotropy decay of $D'AA$ molecules
are of the same time scale of the inertial libration, i.e., $\sim$ 0.2 ps.

Rotational anisotropy decay of water molecules is found at the interface of LiI solution. 
The result comes from a different HB types from the usual $DDAA$ HB type in pure bulk water.
The faster anisotropy decay for $D'AA$ molecules reflects the less correlation between different H-bonds for $D'AA$ molecules, which comes from Hydrogen--Iodide bond at the interfaces, the existence of free OH stretching.
From Fig.\space\ref{fig:prob_124_LiI_double_axis}, we have known that in the LiI solution, 
\I ions prefer to locate at the water/vapor interface.  
Therefore, we infer that the reduction of the inter-correlations between H-bonds occurs at the water/vapor interfaces. 

%
In conclusion, single exponential type rotational anisotropy decay exists for water molecules at the water/vapor interface of the alkali-iodine solutions,
and this faster anisotropy decay of water molecules at the water/vapor interface is the effects of Hydrogen--Iodide (H--I) bond at the interface. 
Since the iodide's surface propensity is high, this difference of HB structure 
from neat water/vapor interface is the source of 
the HB dynamics as well as the Im$\chi^{(2)}$ spectrum of the interface of alkali-iodine solutions.  
%-------
%deleted
%\st{The effects of H--I bond on the HB dynamics at the interfaces, and the relation between the interfacial HB
%dynamics and rotational anisotropy decay can also be studied in the future.}{\color{red}[Question: This i don't understand ... is not what you have discussed so far?? Answer: It was a plan.]}
%-------
%图
\begin{figure}[H] %[!htbp]
\centering
\includegraphics [width=0.36 \textwidth] {./diagrams/2LiI-124w_c2_fit_biexp_7wat_2ps_class_150324} 
\caption{\label{fig:2LiI-124w_c2_fit_biexp_7wat_2ps_class_150324} The time dependence of the $C_2(t)$ for water molecules in different HB environments at the water/vapor interface of LiI solution.}
\end{figure}  

%\subsection{\LiN Solution/vapor Interface}
%The anisotropy decay of OH bonds in water molecules in 0.4 M LiNO3 solution/vapor interface is shown in Fig.\space\ref{fig:c2_LiNO3_inset}.  In the model of the interface, there is one \Li and one \nitrate in the 15.6 \AA$\times$15.6 \AA$\times$31.0 \AA simulation box. 
%The larger decay rate consistent to the conclusion infered from the VDOS for the interfaces, although the concentration of \LiN is lower. This result obtained from another DFTMD trajectory consistent with the previous one, and it reflects that the \nitrate on the surface of the alkali nitrate solution weaken the H-bonds and  accelerate the anisotropy decay of water molecules at the interfaces.
%\begin{figure}[htbp]
%\centering
%\includegraphics [width=0.4\textwidth] {./diagrams/c2_LiNO3_inset} 
%\setlength{\abovecaptionskip}{10pt}
%\caption{\label{fig:c2_LiNO3_inset} The anisotropy decay of OH chromophores in water molecules in LiNO3 solution/vapor interface.}
%\end{figure} 
