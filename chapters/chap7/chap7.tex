\chapter{Hydrogen Bond Dynamics in Solution Systems}\label{CHAPTER_HB_SOLUTIONS}
%The influence of ions propensity for the aqueous surface on the water's HB network are also of special interest to the atmospheric chemistry community.
In this chapter, we explore the effects of nitrate ions, iodide ions and alkali metal cations 
on the HB dynamics at the water/vapor interface of alkali nitrate solutions and alkali
iodine solutions, by DFTMD and we provide a microscopic interpretation of recent experimental results. \cite{HuaWei2014}
In paragraph \ref{env_effect}, we discuss the influence of ions on the hydrogen bond dynamics in the solution interface system; 
in paragraph \ref{SHB_dynamics}, we study the HB dynamics and HB lifetime in the solvation shells of the ions;
and in paragraph \ref{RAD}, we study the orientation dynamics of H-bonds in the solvation shells in the interfacial system; 
Finally, we study the orientation characteristics of water molecules in the ions' solvation shell in the interfacial system.

\FloatBarrier
\section{Environment Effects on Hydrogen Bond Dynamics}\label{env_effect}
For the water/vapor interface of neat water, we focus on the reactive flux $k(t)$, 
which had been used in the study of HB dynamics of liquid water. \cite{AL96,Khaliullin2013}
The $k(t)$ calculated from the positional trajectory of water molecules in DFTMD simulations, is reported in Fig.\thinspace\ref{fig:121}. 
In the case of water/vapor interface, the $k(t)$ quickly changes from its initial value on a time scale of less than 0.2 ps 
(see the inset of Fig.\thinspace\ref{fig:121}). 
Beyond this transient period, the $k(t)$ decays to zero monotonically, and the slop of the $\ln{k(t)}$ increases monotonically with $t$ (see Fig.\thinspace\ref{fig:121}). 
These two properties have been found for bulk water using the SPC water model by Luzar and Chandler. \cite{AL96} 
This log-log plot of the $k(t)$ shows that, as in the case of liquid water, this decay behaviour does not coincide with a power-law decay for water/vapor interface of neat water.
This result is also the same as that of the classical molecular simulation of pure water. \cite{AL96b,Luzar1996}
%
\begin{figure}[htpb]
\centering
\includegraphics [width=0.42\textwidth] {./diagrams/121}
\setlength{\abovecaptionskip}{0pt}
  \caption{\label{fig:121}The time dependence of the $k(t)$ for the water/vapor interface of neat water, calculated by DFTMD simulations.
  The inset shows the log-log plot of the $k(t)$.}
\end{figure}
%
%For the water/vapor interface of alkali-iodine solutions, the $k(t)$ is also calculated.  The result for the interface of 0.9 M LiI solution is shown in Fig.\thinspace\ref{fig:hbrf_4pl} (b). The log-log plot of $k(t)$ is not a straight line, indicating that, for water/vapor interface of the LiI solution, this decay does not coincide with a power-law decay, neither.


%{As can be seen from Fig. \ref{fig:hbrf_4pl}, the fluctuations of the $k (t)$ for $d = 2$ \AA (blue solid line) are significantly larger 
%than that of other cases with larger $d$. 
%This phenomenon is due to the relatively small number of water molecules in the thin layer 
%and the insufficient sampling, resulting in large fluctuations in $k(t)$.
%For these four models, as the thickness $d$ of the interface increases, the $k(t)$ gradually converges to a function with smaller fluctuations.
%%
%This conclusion is consistent with the two conclusions we obtained earlier (see Section \ref{sfg_alkali_iodide_interface}): 
%(1) \I is a strong structure-breaking anion; %[\cite{Trevani2000}] 
%(2) compared to pure water, the OH stretching peak at the interface of a solution containing iodide ions will blue shift. [\cite{Tongraar2010}] 
%Comparing these black solid curves, we can see that the interface of the solution containing ions has lower $k(t)$.
%In other words, compared to the pure water interface, 
%the ratio of H-bonds that were initially bonded at the solution interface and broken at time $t$ is lower.
%Because the effect of iodide ions is to increase the $k(t)$ of the interface, the decrease of $k (t)$ of the interface with a larger thickness
%may only be due to the contribution of cations located under the first layer of water molecules at the interface. 
%Therefore, although the iodide ion increases the HB rupture rate at the top layer of the interface, 
%in general, the HB rupture rate of the entire solution interface is reduced due to the presence of cations under the first layer of water molecules. 
%To verify this conclusion, we calculated the $k(t)$ at the interface of NaI (Fig. \ref{fig:hbrf_4pl} (c)) and KI (Fig. \ref{fig:hbrf_4pl} (d)) aqueous solution. 
%The results for both interface systems support our conclusions above.
%}
%\stkout{ What is the differences between bulk and interface? 
%Let us examine the difference in the $k(t)$ between interface water and bulk water. 
%No matter from pure water (Fig. \ref{fig:hbrf_4pl} (a)) 
%or solution (Fig. \ref{fig:hbrf_4pl} (b), (c) or (d)), we find that when the interface thickness is thin, the fluctuation of $k(t)$ is larger.
%Because the thinner the interface, the fewer pairs of water molecules that can form hydrogen bonds. 
%In our calculations, the fewer samples are used to average, so the fluctuation of $k (t)$ is greater. 
%We can find that at the interface of pure water, when $t> 0.2$ ps, the $k(t)$ value of the interface with different thickness is almost equal 
%at any time period $\Delta t$. For example, $\Delta t$ is selected as $\sim$ 2 ps, 
%and its average value is shown in Table \ref{tab:hbrf_neat}. In each time period of 2 ps, the values of $k(t)$ for different layers are approximately equal
%($\pm 0.004$ ps$^{-1}$). Therefore, as far as the nature of HB reactive flux is concerned, the difference between interface and bulk phase of neat water is not obvious. 
%}

%To show the effect of water molecule diffusion on the HB dynamics, we can calculate the sum of the functions $c(t)$ and $n(t)$, i.e., $c(t)+n(t)$.
%Here, we take the LiI solution as an example.
%Fig.\space\ref{fig:124_2LiI_ns20_c_plus_n} shows the time dependence of the correlation functions $c(t)$, $n(t)$ and $c(t)+n(t)$ of the interface of 
%the LiI solution at a concentration of 0.9 M in the AIMD simulation.
%As can be seen, although the change in the total population, $c(t)+n(t)$, is very small in the range of 0--10 ps, it is not a constant.
%Therefore, the $n(t)$ relaxes not only by conversion back to HB \emph{on} state, 
%but is also depleted due to the diffusion process. We can estimate the time scale of water molecule diffusion at the interface of the aqueous solution by $c(t)+n(t) = 1/e$, 
%which is much larger than 10 ps. Therefore, when we analyze the HB dynamics of the solution interfaces, we do not consider the effect of water molecule diffusion.
%
%\begin{figure}[H]
%\centering
%\includegraphics [width=0.36\textwidth] {./diagrams/124_2LiI_ns20_c_plus_n}
%\setlength{\abovecaptionskip}{0pt}
%\caption{\label{fig:124_2LiI_ns20_c_plus_n} 
%The time dependence of the functions $c(t)$, $n(t)$ and $c(t)+n(t)$, where $c(t)$ represents the $C_{\text{HB}}(t)$, 
%for the interfaces of 0.9 M LiI solution.} 
%\end{figure}
%
%\begin{figure}[H]
%\centering
%\includegraphics [width=0.5\textwidth] {./diagrams/128w_bk_2delta_t_60ps_n}
%\setlength{\abovecaptionskip}{0pt}
%\caption{\label{fig:128w_bk_2delta_t_60ps_n} 
%The time dependence of the population functions $n(t)$ for bulk water, as computed from the ADH (solid line) and AHD (dashed line) criterion of H-bonds.} 
%\end{figure}
% 

It can be seen from Fig.\thinspace\ref{fig:128w_bk_itp_50ps_n_from_k_in_with_2_hb_def_type2} that the $n(t)$ of the water/vapor interface of neat water
is always greater than the value of $n(t)$ in the bulk water. This means that "the hydrogen bond between a pair of water molecules at time $t$ is broken
and the distance between them is less than 3.5 \A" in the water/vapor interface is more likely to occur than in bulk water. 
We interpret this result as the fact that at time $t$, there is a greater probability that the H-bonds on the interface are broken 
compared to the H-bonds in the bulk water.
%
\begin{figure}[H]
\centering
\includegraphics [width=0.5\textwidth] {./diagrams/128w_bk_itp_50ps_n_from_k_in_with_2_hb_def_type2}
\setlength{\abovecaptionskip}{0pt}
\caption{\label{fig:128w_bk_itp_50ps_n_from_k_in_with_2_hb_def_type2} 
The time dependence of the population functions $n(t)$ for bulk water and the water/vapor interface from (a) ADH (b) AHD criteria. ($T$=300K.)} 
\end{figure}

To study the HB dynamics after the transition phase, which is roughly at 0.1 ps (see Fig.\thinspace\ref{fig:121}) and lasts for hundreds of picoseconds, 
we set $t_1 = 1$ ps and $t_2 = 10$ ps in the fitting.
For water/vpaor interfaces of neat water and the aqueous solution interfaces, 
the optimal values of $k$ and $k'$ given by these results have been listed in Table \ref{tab:k_k_prime_pure_and_solutions}. 
These values are comparable in magnitude to those obtained by Ref.\thinspace{\cite{Khaliullin2013}}. 
It can be seen from Table \ref{tab:k_k_prime_pure_and_solutions} that the HB breaking reaction rate ($k$) at the interface of pure water is basically equivalent to 
that at the solution interface, but the HB reforming rate ($k'$) is smaller than that at the solution interface by 30\% to 50\%.
Correspondingly, we can find the HB relaxation times of the three solution interfaces are: $\tau=\frac{1}{k+k'} \sim $2.0--2.5 ps. 
For pure water interface, the relaxation time is $\tau \sim $ 3.3 ps. 
Our conclusion is that the difference between the relaxation time of H-bonds at the interface of solutions such as LiI, NaI, KI 
and the interface of pure water is mainly due to the difference in the reforming rate $k'$ of H-bonds caused by the presence of ions,
rather than the difference in the breaking rate $k$ of H-bonds.
%
\begin{table}[htbp]
\centering
\caption{\label{tab:k_k_prime_pure_and_solutions} 
    The $k$ and $k'$ for the water/vapor interface of the aqueous solution interfaces.} 
\begin{tabular}{cccc}
 Interface & $k$ (ps$^{-1}$) & $k'$ (ps$^{-1}$) & $\tau_{\text{R}}$ (ps) \\
\hline
  Neat Water & 0.10 $\pm$ 0.02 & 0.20 $\pm$ 0.02 & 11.50 \\
  LiI & 0.10 $\pm$ 0.04 & 0.30 $\pm$ 0.05 & 5.33 \\
  NaI & 0.20 $\pm$ 0.10 & 0.30 $\pm$ 0.05 & 5.77 \\
  KI  & 0.10 $\pm$ 0.04 & 0.40 $\pm$ 0.10 & 6.96 
\end{tabular}
\end{table}

As for the effect of water/vapor interface on the HB dynamics in alkali-iodine solutions,
we also calculate the survival probability for interfaces with different sizes of thickness. 
The result for the interface of the LiI solution exhibits that H-bonds at water/vapor interface decay faster than that in bulk water.
The result for the logarithm of \SHB is given in Fig.\space\ref{fig:2LiI-124w_S_layers} in Appendix \ref{thickness_interface}, 
in which the thickness of the alkali-iodine solutions can be determined.
Therefore, as the interface thickness increases, the \SHB converges to a fixed curve, 
which characterizes the HB dynamics of bulk solutions. 
In particular, it gives the average continuum HB lifetime in bulk solutions.

\FloatBarrier
\paragraph{Effect of Nitrate ions}
First, let us take a look at the changes in the hydrogen bond of water by nitrate ions in the bulk solution. 
To achieve this goal, we performed a DFTMD simulation on a system containing one \Li ion, one nitrate ion 
and 127 water molecules, that is, LiNO$_3$ solution. During the simulation, the temperature of the system is 
300 K and the volume is a constant, and periodic boundary conditions are used.
The $C_{SHB}(t)$ and $\ln{S_{SHB}(t)}$ for the H-bonds between water molecules in the solvation shell of nitrogen atoms in bulk LiNO$_3$ solution is shown in 
\ref{fig:shb_c_and_s_ln_bk_NShell_pbc}. The values of radius 4.5, 6.5, 8.0 \AA come from the first three minimum values of RDF $g_{N-OW}(r)$ 
(see Fig.\thinspace\ref{fig:gdr_N-W_127_LiNO3}). 
Both correlations $C_{\text{SHB}}(t)$ and $\ln{S_{\text{SHB}}(t)}$ show that the longer the hydrogen bond from the nitrate ion, the slower the correlation function decays. 
In other words, the closer the hydrogen bond is to the nitrate, the faster it relaxes. Therefore, in the bulk LiNO$_3$ solution, nitrate ions accelerate the dynamical 
process of H-bonds in water.
%
%\begin{figure}[htbp] % or \begin{SCfigure}
%\centering
%\includegraphics [width=0.6\textwidth] {./diagrams/shb_c_and_s_ln_bk_NShell_pbc}
%\setlength{\abovecaptionskip}{0pt}
%\caption{\label{fig:shb_c_and_s_ln_bk_NShell_pbc} The $C_{SHB}(t)$ and $S_{SHB}(t)$ of water--water H-bonds at the solvation shell 
%  of nitrate ion in the \LiN solution. These results are calculated for the temporal resolution $t_t=0.4$ ps. For the definition 
%  of $t_t$, see Appendix \ref{thickness_interface}. }
%\end{figure}

%I simulate the alkali nitrate solution/vapor interface to find how the nitrate affect the structure of the interface.
\begin{figure}[htbp] % or \begin{SCfigure}
\centering
\includegraphics [width=0.36\textwidth] {./diagrams/256_LiNO3_hbacf_sh_no3} %fig.5.10
\setlength{\abovecaptionskip}{0pt}
\caption{\label{fig:256_LiNO3_hbacf_sh_no3} The \SHB of water--water (W--W) and nitrate--water (N--W) H-bonds at the water/vapor
  interface of the \LiN solution. The inset is the plot of ln\SHB. 
  These results are calculated for the temporal resolution $t_t=1$ fs. For the definition of $t_t$, see Appendix \ref{thickness_interface}. }
\end{figure}
%
%\begin{figure}[H]
%\centering
%\includegraphics [width=0.4\textwidth] {./diagrams/256_LiNO3_hbacf_hh_all_traj_sh_no3}
%\setlength{\abovecaptionskip}{20pt}
%\caption{\label{fig:256_LiNO3_hbacf_hh_all_traj_sh_no3}The functions ln\SHB of water--water H-bonds (black) and Nitrate -water H-bonds (red) in the the \LiN solution-vapor interface at 300 K. The lifetime of H-bonds $\tau_{\text{HB}}$ is calculated by the integration of \SHB over t$\in$(0,$\infty$), which give 0.42  and 0.20 ps, for water--water H-bonds and Nitrate -water H-bonds, respectively.}
%\end{figure}
%The density profile is a indicator of a table interfacial system (see Fig.\space\ref{fig:density_4MPlus_alkali-I}).
%\begin{figure}[htbp]
%\centering
% \includegraphics [width=0.6\textwidth] {./diagrams/density_4MPlus_alkali-I} %fig5.11
%\setlength{\abovecaptionskip}{20pt}
%\caption{\label{fig:density_4MPlus_alkali-I}The density as a function of the slab coordinate \Z. The result is calculated by MD with SPC water model.}
%\end{figure}
Next, let us see what effect the nitrate ion has on the H-bonds in the interface.
As shown in Fig.\space\ref{fig:vdos_LiNO3-256w_w_near_nitrate} in chapter ~\ref{CHAPTER_SFG_Calculation}, 
from the VDOS, the water molecules bound to \nitrate have higher OH stretching frequency (55 cm$^{-1}$ larger) 
than those H-bonded to other water molecules. 
%
Now, the difference between nitrate--water and water--water H-bonds 
can be also analyzed in terms of the survival probability $S_{\text{HB}}(t)$, \cite{AKS86,JT90,AL96} 
reported in Fig.\thinspace\ref {fig:256_LiNO3_hbacf_sh_no3}.
The integration of \SHB from 0 to $t_{\max}=5.0$ ps, \cite{Steinel2004} gives the relaxation time $\tau_\text{HB}$, which can be interpreted as 
the average HB lifetime. \cite{SC02} 
The values of $\tau_{\text{HB}}$ is dependent on a temporal resolution $t_t$, during which the H-bonds that break and reform are treated as intact. \cite{AL00} 
%
Here, we choose the temporal resolution as $t_t=1$ fs. 
Then, Fig.\thinspace\ref {fig:256_LiNO3_hbacf_sh_no3} gives $\tau_\text{HB}=0.20$ ps for nitrate--water H-bonds at interfaces, and $\tau_\text{HB}=0.42$ ps for water--water H-bonds.
This result of $\tau_\text{HB}$ is consistent with the experimental result of Kropman and Bakker ($\tau_\text{HB}=0.5\pm0.2$ ps). \cite{MFK01}
The smaller value of $\tau_\text{HB}$ for nitrate--water H-bonds implies that the nitrate--water H-bonds are weaker than bonds between water molecules. 
This is also consistent with the VDOS analysis and the blue-shifted frequency of the OH stretching in the nitrate-water HB. 
%[DELETED From both the VDOS and HB dynamics calculations, we conclude that it is the weak HBs between nitrate and water make the higher surface propensity 
%of nitrate anions, and then induce the depletion of SFG intensity at 3200 \cm for the alkali nitrate salty interfaces.]

%Fig. ~\ref{fig:256_LiNO3_hbacf_Nitrate_effect} shows that the nitrate ions accelerate the HB dynamics at the vapor/water interface of alkali nitrate solution.
%\begin{figure}[H]
%\centering 
% \includegraphics [width=0.6\textwidth] {./diagrams/256_LiNO3_hbacf_Nitrate_effect} %fig5.12
%\setlength{\abovecaptionskip}{20pt}
%\caption{\label{fig:256_LiNO3_hbacf_Nitrate_effect}The functions \CHB of bulk water--water H-bonds (W-W (Bulk)) and nitrate--water H-bonds (N-W) 
%at interfaces of alkali nitrate solution  (LiNO$_3$(H$_2$O$_{256}$)  at 300 K. }
%\end{figure} 
%NOT CLEAR, TO EXPLAIN BETTER The HB relaxation time is about $2.5$ ps, which is the same as that
%for nitrate--water H-bonds at interfaces of alkali nitrate solution.
%[NOT CLEAR: For bulk water, the HB relaxation time $\tau$ is $3.7$ ps. The difference between the HB dynamics of H-bonds outside the first shell of \Li and HB dynamics for nitrate--water H-bonds at interfaces
%is not visible from the values of the HB relaxation time. They reflect the difference between HB
%dynamics between bulk water and water/vapor interfaces.]

\subsection{Effects of Alkali Metal Ions and \I on HB Dynamics}
%\begin{figure}[!ht]
%\centering
%\includegraphics [width=\textwidth] {./diagrams/C_S_HB_124_2LiI-2NaI-2KI} %fig5.15
%\setlength{\abovecaptionskip}{0pt}
%  \caption{\label{fig:C_S_HB_124_2LiI-2NaI-2KI} The time dependence of functions (a) \CHB and (b) \SHB of water--water H-bonds at water/vapor interfaces of 0.9 M alkali-iodine solutions.} 
%\end{figure}
%[hbtp]
\begin{table}[H]
\centering
\caption{\label{tab:tau_hb_alkali_iodine} 
The continuum HB lifetime $\tau_{\text{HB}}$ (unit: ps) in the first hydration shell of I$^-$ ion and of alkali metal ion at the water/vapor interface of 0.9 M LiI (NaI, KI) solution.}
\begin{tabular}{cccc}
  &\I-shell &cation-shell& interface \\
\hline
 LiI & 0.22 & 0.24 & 0.23\\
 NaI & 0.24 & 0.28 & 0.26\\
 KI  & 0.20 & 0.23 & 0.20\\
\end{tabular}
\end{table} 
%Water/Vapor & -&-&
Table \ref{tab:tau_hb_alkali_iodine} lists the continuum HB lifetime in the first hydration shell of I$^-$ ion and of alkali metal ion
at the interfaces of the three alkali-iodine solutions. It shows that, the continuum HB lifetime $\tau_{\text{HB}}$ in the 
solvation shell of alkali metal (iodine) ions is larger (smaller) than 
that of H-bonds at the water/vapor interfaces of the same solutions, 
respectively. For LiI solution, the water molecules bound to the cation ion
\Li, on average, have a continuum HB lifetime $\tau_{\text{HB}} \sim 0.24$ ps. This
 continuum HB lifetime is longer than that of molecules bound to \I or at the interface of the LiI solution. 
%
\begin{figure}[H]
\centering
\includegraphics [width=0.6\textwidth] {./diagrams/hbacf_C_sh2_2p}
\setlength{\abovecaptionskip}{0pt}
\caption{\label{fig:hbacf_C_sh2_2p}The \CHB of water--water H-bonds in the solvation shell 
  of (a) cations and (b) I$^-$ at the interfaces of 0.9 M LiI, NaI and KI solutions, respectively.
  The dashed line shows the \CHB for the interface (the thickness $d = 8$ \A) of the LiI solution.  
  This interface contains H-bonds between water molecules similar to those in bulk water, i.e.,
  water molecules participating in these H-bonds are not in the solvation shell of ions.} 
\end{figure}
%Fecko and co-workers' study of liquid D$_2$O by IR spectroscopy reveals that the vibrational dynamics observed are dominated by underdamped displacement of the hydrogen-bond coordinate at very short times ( less than 200 fs).\cite{CJF03,CJF05} 
Fig.\thinspace\ref{fig:hbacf_C_sh2_2p} a and b show that the \CHB of H-bonds within the alkali cations and \I decay faster 
than those in bulk water and at the surface of LiI solution.
From Fig.\thinspace\ref{fig:hbacf_C_sh2_2p}b, we find that, for all three alkali-iodine solutions, the \CHB for hydration shell water molecules 
of \I decays faster than that for molecules at the water/vapor interface.
%The simulation produces similar result as Omta and coworker's experiments of femtosecond pump-probe spectroscopy, which demonstrate that anions ( $\text{SO}^{2-}_4$, $\text{ClO}^-_4$, etc) have no influence on the dynamics of bulk water, even at high concentration up to 6 M.\cite{AWO03} 
%Here, we find that the cations \Li and \Na does not alter the H-bonding network outside the first hydration shell of cations. It is concluded that no long-range structural-changing effects for alkali metal cations.
The radii of hydration shells are 5.0 \AA for \li, 5.38 \AA for \na,
5.70 \AA for \pot, and 6.0 \AA for \I ions, which are obtained from the RDFs.
The RDFs $g_{\text{ion-O}}$ (ion=\li, \na) for the interfaces 
of LiI (NaI) solutions are shown in Fig.\thinspace\ref{fig:124_2NaI-2LiI_gdr_Li-O_Na-O_1501}a,
and the coordination numbers are in Fig.\thinspace\ref{fig:124_2NaI-2LiI_gdr_Li-O_Na-O_1501}b.
\begin{figure}[H]
\centering
\includegraphics [width=0.42\textwidth]{./diagrams/124_2NaI-2LiI_gdr_Li-O_Na-O_1501}%fig.6.1 
\setlength{\abovecaptionskip}{0pt}
\caption{\label{fig:124_2NaI-2LiI_gdr_Li-O_Na-O_1501}
  (a) The RDF $g_{\text{ion-O}}(r)$(ion=\li, \na) and (b) the coordination number of \Li (\na) ions at the interfaces of LiI (NaI) solution. 
  For \Na, the coordination number $n_\text{Na}$=5; while for \Li, $n_\text{Li}$=4.} 
\end{figure} % There is a first shell exist for both \Li and \Na cations.
\FloatBarrier
\paragraph{Effects of the Ion Concentration}
Effects of ions' concentration on HB dynamics have been studied extensively by Chandra. \cite{AC00}
%Pal and coworkers provided details on the structure of water around the micellar surface.\cite{SP05} 
We calculated the \CHB for the water/vapor interfaces of the alkali-iodine solutions, 
and the relaxation time $\tau_{\text{R}}$ for each of them can be determined. 
%\begin{eqnarray}
%    C_{\text{HB}}(\tau_\text{{R}})=1/e. \nonumber
%\label{eq:relaxation_time}
%\end{eqnarray}
Here, the \emph{interface} means \emph{all} the water molecules in each model. 
The $\tau_{\text{R}}$ for the water/vapor interfaces of the LiI (NaI) solutions are given in 
Table \ref{tab:tau_hb}. Generally, they are in the range 1--10 ps. 
The values of $\tau_{\text{R}}$ decrease as the concentration of the solutions increases.
\begin{table}[htbp]
\centering
\caption{\label{tab:tau_hb} 
  The relaxation time $\tau_{\text{R}}$ (unit: ps) of the correlation function \CHB  for the water/vapor interface of the LiI (NaI) solutions, calculated by DFTMD simulations.}
\begin{tabular}{ccc}
  concentration  & $\tau_{\text{R}}$ (LiI) & $\tau_{\text{R}}$ (NaI) \\
\hline
  0 & 11.50 & 11.50 \\
  0.9 M & 7.04 & 10.60 \\
  1.8 M & 4.40 & 1.96 
\end{tabular}
\end{table}

The concentration dependence of the HB dynamics can be also found in the \SHB. 
Figure \thinspace\ref{fig:124_2LiI-2NaI_hbacf_S}a gives the \SHB 
for the water/vapor interfaces of 0.9 M and 1.8 M LiI solutions.
The same quantity for NaI solutions is given in Fig.\thinspace\ref{fig:124_2LiI-2NaI_hbacf_S}b.
This result indicates that, for the interface of alkali-iodine solution, the continuum HB lifetime  
decrease as the concentration of LiI (or NaI) solution increase.
\begin{figure}
\centering
\includegraphics [width=0.6\textwidth, center] {./diagrams/124_2LiI-2NaI_hbacf_S} 
\setlength{\abovecaptionskip}{0pt}
  \caption{\label{fig:124_2LiI-2NaI_hbacf_S} The time dependence of the \SHB  of 
  H-bonds at the water/vapor interfaces of (a) LiI and (b) NaI solutions at 330 K.
	The insets show the plots of ln$S_{\text{HB}}(t)$.} 
\end{figure}
\FloatBarrier
\section{Solvation Shell Hydrogen Bond Dynamics} \label{SHB_dynamics}
\paragraph{Solvation Shells as Special interface}
We will extend the IHB dynamics to H-bonds around certain ions, so that we can study the H-bonds in aqueous solutions. 
Similar to the determination of the instantaneous surface, we can define an interface for each molecules in aqueous solutions, that is, 
the first olvation shell of the molecule. Here, we use the word molecule to denote ions, and water molecules.  
Below we will combine the interface defined by the first solvation shell of ions, and Luzar-Chandler's HB population operator \cite{AL96} to calculate the HB
dynamics for the H-bonds between water molecules in the solvation shell of ion and those outside the shell. 
The dynamics of these H-bonds may vary with the shell radius $r_{shell}$ we set.
From the characteristics of HB dynamics in the solvation shells, we can obtain the effects of various ions on structure and dynamics of aqueous solutions. 

\paragraph{Solvation Shell Hydrogen Bond Population}
After one has determined the solvation shell ${\mathbf k}(t)={\mathbf k}(\{{\mathbf r}_i(t)\})$, we can define Solvation shell H-Bonds (SHBs).
We use the parameter $r_{shell}$ to denote the radius of the solvation shell.
Now we define the solvation shell HB population operator $h^{k}(t) = h^{k}[{\mathbf r}(t)]$ as follows:
It has a value 1 when the the particular tagged molecular pair are H-bonded \emph{and} one of the molecules are inside the solvation shell
with a radius $r_{shell}$, and zero otherwise. 
The definition of $h^{k}(t)$ is very similar to $h^{s}[{\mathbf r}(t)]$, which is defined in \ref{IHBP} for studying the interfacial H-bonds.
Similarly, $h^{k}(t)$ can help us to efficiently obtain the dynamic characteristics of H-bonds in solvation shells of any radius. 
The definition of HB here can be based on water molecule pairs or O-H pairs. 
Like in the IHB case, in this paragraph, we also just discuss H-bonds based on water molecule pairs. 
For the hydrogen bond defined based on the O-H pairs, one can do a similar analysis.

Similar to the correlation function $C^s_\text{HB}(t)$ for the H-bonds in instantaneous interfaces,
we define the correlation function $C^{k,X}_\text{HB}(t)$ that describes the fluctuation of the solvation shell H-bonds for ion $X$: 
\begin{eqnarray}
C^{k,X}_{\text{HB}}(t)=\langle h^{k,X}(0)h^{k,X}(t) \rangle/\langle h^{k,X}\rangle
\label{eq:C_k_HB}.
\end{eqnarray}
When not considering specific ions, we denote it $C^{k}_\text{HB}(t)$ for short.
%
Similarly, we can define correlation functions 
\begin{eqnarray}
n^{k,X}(t)=\langle h^{k,X}(0)[1-h^{k,X}(t)]h^{(d),k,X} \rangle/\langle h^{k,X}\rangle
\label{eq:n_k_HB},
\end{eqnarray}
and 
\begin{eqnarray}
k^\text{k,X}(t)= -\frac{dC_\text{HB}^\text{k,X}}{dt}
\label{eq:k_k_HB}.
\end{eqnarray}
Then, using these correlation functions, we can determine the reaction rate constant of breaking and reforming and the lifetimes of solvation shell H-bonding.
%
\FloatBarrier
\paragraph{$C^{k}_\text{HB}(t)$}
\begin{figure}[H] 
\centering
\includegraphics [width=0.60\textwidth] {./diagrams/shb_c_ln_bk_Shell_pbc}
\setlength{\abovecaptionskip}{0pt}
\caption{\label{fig:shb_c_ln_bk_Shell_pbc}
The correlation function $C^\text{k}_\text{HB}(t)$ for the H-bonds in the solvation shells, based on water-water 
pair HB population operator $h^\text{k}(t)$, as computed from the (a) ADH and (b) AHD criteria of H-bonds.} 
\end{figure}
%The results are obtained from the DFTMD simulation for the bulk LiNO$_3$ solution at $T=300$ K. 
We calculated the HB dynamics of the H-bonds between water molecules in the first solvation sphere and other water molecules outside the sphere. 
The choice of the shell radius comes from the RDFs (see Fig.\ref{fig:gdr_NW_WW_127_LiNO3}). Comparing Fig.\ref{fig:256_LiNO3_hbacf_sh_no3} 
and Fig.\thinspace\ref{fig:shb_c_ln_bk_Shell_pbc}, we conclude that: Although the HB between nitrate and water molecule is significantly 
weaker than the bond between \Li ion and water molecule under the same conditions,
the HB strength between water molecules in the ions' hydration shells and those outside the shells are not affected by the nature of the ions evidently.
Specificly, Fig.\thinspace\ref{fig:shb_c_ln_bk_Shell_pbc} shows that the relaxation process of the H-bonds between the water molecules in NO$^-_3$-shell and 
those outside the shell is \emph{not} faster than the relaxation process of H-bonds between the water molecules in the \Li-shell and those outside the \Li-shell. 

Like $C^s_\text{HB}(t)$, $C^k_\text{HB}(t)$ also accelerates the hydrogen bond kinetics to a certain extent. 
Therefore, the two are not comparable. For example, it can be seen from Fig.\thinspace\ref{fig:shb_c_ln_bk_Shell_bulk_wat_pbc_r5} 
that $C^{k}_\text{HB}(t)$ decays faster than $C_\text{HB}(t)$ in pure water.
\begin{figure}[H] 
\centering
\includegraphics [width=0.60\textwidth] {./diagrams/shb_c_ln_bk_Shell_bulk_wat_pbc_r5}
\setlength{\abovecaptionskip}{0pt}
\caption{\label{fig:shb_c_ln_bk_Shell_bulk_wat_pbc_r5}
The correlation function $C^\text{k}_\text{HB}(t)$ for the H-bonds in the solvation shell of NO$^-_3$ ion, based on water-water 
pair HB population operator $h^\text{k}(t)$, as computed from the (a) ADH and (b) AHD criteria of H-bonds.
The correlation functions $C_\text{HB}(t)$ (dashed line) in the bulk water defined by two different HB criterion 
ADH and AHD are also plotted in sub-figure a and b respectively. } 
\end{figure}
%
%\begin{figure}[h]
%\centering
%\includegraphics [width=0.60\textwidth] {./diagrams/shb_log_s_lii_bk_new_LiShell_pbc}
%\setlength{\abovecaptionskip}{0pt}
%\caption{\label{fig:shb_log_s_lii_bk_new_LiShell_pbc}
%The logarithm of the correlation function $S^\text{k,Li}_\text{HB}(t)$ for the solvation shell H-bonds with differnt radius ($r_\text{shell}$), based on water-water 
%pair HB population operator $h^\text{k}(t)$, as computed from the (a) ADH and (b) AHD criteria of H-bonds.} 
%\end{figure}
%%
%\section{Hydrogen Bond Dynamics by Classical Molecular Dynamics Simulations}
%\begin{figure}[H]
%\centering
% \includegraphics [width=0.5\textwidth] {./diagrams/4MPlus-alkali-I_hbacf_C1603}
%\setlength{\abovecaptionskip}{20pt}
%\caption{\label{fig:4MPlus-alkali-I_hbacf_C1603}The function \CHB of water--water H-bonds at interfaces with different alkali metal ions in 4.0 M water solution at 300 K.}
%\end{figure}
%The HB dynamics obtained from classical MD simulations can not catch the fast HB relaxation, and it give a totally different HB dynamics for the water molecules in these alkali halide solution/vapor interfaces.
%
\FloatBarrier
\paragraph{$C^k_\text{HB}(t)$}
To make the results clearer, we only added one cation and one anion to the simulated aqueous system. 
We have done DFTMD simulations for LiI, NaI, KI bulk system and interface system respectively.
%\begin{figure}[h]
%\centering
%\includegraphics [width=0.60\textwidth] {./diagrams/shb_c_lii_itp_Shell_pbc}
%\setlength{\abovecaptionskip}{0pt}
%\caption{\label{fig:shb_c_lii_itp_Shell_pbc}
%The correlation function $C^{k}_{HB}(t)$ for the solvation shell H-bonds (in water/vapor interface) , based on water-water 
%pair HB population operator $h^{k}(t)$, as computed from the (a) ADH and (b) AHD criteria of H-bonds.} 
%\end{figure}
% 
Further, we choose LiI solution to calculate its HB correlation function $C^k_\text{HB}(t)$. 
The calculation results are shown in Fig.\ref{fig:shb_c_lii_bk_new_Shell_pbc_r5}. Like the LiNO$_3$ solution, the relaxation functions 
$C^\text{k,Li}_\text{HB}(t)$ and  $C^\text{k,I}_\text{HB}(t)$ are very close to each other, 
and this result shows that the presence of ions has no significant effect 
on the relaxation of H-bonds outside the first solvation shell.
\begin{figure}[h]
\centering
\includegraphics [width=0.60\textwidth] {./diagrams/shb_c_lii_bk_new_Shell_pbc_r5}
\setlength{\abovecaptionskip}{0pt}
\caption{\label{fig:shb_c_lii_bk_new_Shell_pbc_r5} 
The correlation function $C^\text{k}_\text{HB}(t)$ for the solvation shell H-bonds with radius ($r_\text{shell}=5.0$ \AA), based on water-water 
pair HB population operator $h^\text{k}(t)$, as computed from the (a) ADH and (b) AHD criteria of H-bonds. }
%The results are calculated from the simulated bulk LiI solution (new) at $T=300$ K.
\end{figure}
%
\begin{figure}[h]
\centering
\includegraphics [width=0.60\textwidth] {./diagrams/shb_s_lii_itp_Shell_pbc}
\setlength{\abovecaptionskip}{0pt}
\caption{\label{fig:shb_s_lii_itp_Shell_pbc}
The correlation function $\ln S^\text{k}_\text{HB}(t)$ for H-bonds in the solvation shells of \Li and \I ions (in water--vapor interface)  based on water-water 
pair HB population operator $h^\text{k}(t)$, as computed from the (a) ADH and (b) AHD criteria of H-bonds.} 
%The results are calculated from the simulated interface system of LiI solution at $T=300$ K.
\end{figure}
%
%The $\ln S^\text{k,I}_\text{HB}(t)$ (see Fig.\thinspace\ref{fig:shb_log_s_lii_bk_new_IShell_pbc}). 
%\begin{figure}[h]
%\centering
%\includegraphics [width=0.60\textwidth] {./diagrams/shb_log_s_lii_bk_new_IShell_pbc}
%\setlength{\abovecaptionskip}{0pt}
%\caption{\label{fig:shb_log_s_lii_bk_new_IShell_pbc}
%The logarithm of the correlation function $S^\text{k,I}_\text{HB}(t)$ for the solvation shell H-bonds with differnt radius ($r_\text{shell}$), based on water-water 
%pair HB population operator $h^\text{k}(t)$, as computed from the (a) ADH and (b) AHD criteria of H-bonds. The results are calculated from the simulated bulk LiI solution.
%WHICH RESULT IS CORRECT? WE HAVE TO CHECK THE SAME CALCULATION IN INTERFACE, AND IN DIFFERENT TEMPERATURE, AND IN DIFFERENT SOLUTIONS LIKE LiI,NaI,and KI solutions,
%before we obtain the conclusion.} 
%\end{figure}
%
\section{Rotational Anisotropy Decay of Water at the Interface}\label{RAD}
Using the transition dipole auto-correlation function, 
we determined the rotational anisotropy decay and therefore the OH-stretch relaxation at water/vapor interface of alkali iodide solutions.
The effects of ion environment on structure and dynamics of water are obtained by comparing the second-order Legendre polynomial, 
i.e.,  $P_2(x)=\frac{1}{2}(3x^2-1)$,  orientational correlation function of the transition dipole.
The anisotropy decay can be determined from experimental signal in two different polarization configurations---parallel and perpendicular polarizations, by 
\begin{equation}
        R(t)=\frac{S_{\parallel}(t)-S_{\perp}(t)}{S_{\parallel}(t)+2S_{\perp}(t)}
\label{eq:ad}
\end{equation}
where $t$ is the time between pump and probe laser pulses.  The anisotropy decay can also be obtained by simulations, and calculated by the third-order response functions $R(t)$. \cite{Jansen10,Jansen06}

%In the first shell with a radius 3 \A, the entropy difference betweem the \Li shell and \Na shell,
%$\Delta S=k_B\text{ln}\frac{\Omega_\text{Na}}{\Omega_\text{Li}}=k_B\text{ln}\frac{n_\text{Na}/V_\text{Na}}{n_\text{Li}/V_\text{Li}} =k_B\text{ln}1.05$.

%
%\paragraph{Probability Distribution of Ions}
%The probability distribution, shown in Fig.~\ref{fig: prob_124_LiI_Sans_double_axis}, of the ions in the water/vapor interface of LiI and NaI solutions with repect to the depth of the ions in the solutions 
%indicates that the \I ions prefer to staying at the topmost layer of surface of solutions.
%(molar concentration: 0.9 M, temperature: 330 K) 
%It shows that \I ions tend to the surface of the solutions, while \Na and \Li tend to stay in the bulk. This result is consistent with the calculations from Ishiyama and Morita\cite{TI07,TI14}.
The orientational anisotropy $C_2(t)$ is given by the rotational time-correlation function 
\begin{equation}
C_2(t)=\langle P_2(\hat{u}(0)\cdot\hat{u}(t)) \rangle,
\label{eq:tcf2}
\end{equation}
where $\hat{u}(t)$ is the time dependent unit vector of the transition dipole, $P_2(x)$ is the second Legendre polynomial, and 
$\langle \rangle$ indicate equilibrium ensemble average.\cite{Corcelli05,LinYS2010} %\cite{2010Lin} % angular brackets

The anisotropy decay $C_2(t)$ for the water/vapor interface of LiI solution is shown in Fig.\space\ref{fig:c2_2LiI_16_inset}.
This function decays faster than that of neat water, indicating that H-bonds
at the interfaces of alkali-iodine solutions reorient faster than in neat water. The inset shows the first 0.4 ps of $C_2(t)$, from which we see a 
quick change during the first $\sim 0.1$ ps primarily due to librations.
%
\begin{figure}[h]
\centering
\includegraphics [width=0.36\textwidth] {./diagrams/c2_2LiI_16_inset} 
\setlength{\abovecaptionskip}{0pt}
  \caption{\label{fig:c2_2LiI_16_inset} The time dependence of the $C_2(t)$ of OH bonds at the water/vapor interfaces of 0.9 M LiI solution 
  and of neat water (dashed line) at 330 K, calculated by DFTMD simulations. The water/vapor interface of neat water is modeled with a slab 
  made of 121 water molecules in a simulation box of size $15.60 \times 15.60 \times 31.00$ \A$^3$.}
\end{figure}
%
We also calculated the $C_2(t)$ for the interface of other alkali-iodine solutions LiI and KI. 
The results of $C_2(t)$ for the water/vapor interfaces of these solutions are shown in Fig.\thinspace\ref{fig:c2_2KI_2NaI_2LiI_16}.
In all the cases $C_2(t)$ decays faster than in neat water, indicating that H-bonds
at the interfaces of the three alkali-iodine solutions are orientated faster than that of neat water.
They show that \I ions can accelerate the dynamics of molecular reorientation of water molecules at interfaces.   

%
\begin{figure}[htbp]
\centering
\includegraphics [width=0.36 \textwidth] {./diagrams/c2_2KI_2NaI_2LiI_16} 
\setlength{\abovecaptionskip}{0pt}
  \caption{\label{fig:c2_2KI_2NaI_2LiI_16} The time dependence of the $C_2(t)$ of OH bonds in water molecules at the water/vapor 
  interface of 0.9 M alkali-iodine solutions and of neat water (dashed line) at 330 K, calculated by DFTMD simulations.}
\end{figure} 

We have obtained non-single-exponential kinetics for the rotation of water molecules both at the surface 
and in bulk water (Appendix \ref{single_exp}).
%This result is true for water molecules bound to ions. 
Therefore, the rotational motion of water molecules are not simply characterized by well-defined rate constants. 
%Then the problem is to understand the kinetics.
Similar non-single-exponential kinetics is also obtained in the HB kinetics
in liquid water \cite{AL96,Dirama05} and in the time variation of the average frequency shifts of the 
remaining modes after excitation in hole burning technique \cite{Rey2002,Moller2004} and using BLYP functional. \cite{Bankura2014}
Luzar and Chandler interpreted 
the non-single-exponential kinetics as the result of an interplay between 
diffusion and HB dynamics. \cite{AL96} 
We can understand the non-single-exponential kinetics of rotational 
anisotropy decay by fitting the rotational anisotropy decay by a 
biexponential function.
\subsection{Alkali-Iodine Solutions}
To obtain the effects of diffusion and HB decay of water molecules
in solutions respectively, we assume that there are two independent 
decays in the process of an anisotropy decay. 
Therefore, the $C_2(t)$ has the form \cite{TanHS05}
\begin{equation}
C_2(t)=A_1e^{-\kappa_1 t} +A_2e^{-\kappa_2 t},
\label{eq:tcf3}
\end{equation}
where $A_i$ are constants and $\kappa_i$ are decay rates ($i=1, 2$). 
The time constants and amplitudes of the biexponentials fits for 
the $C_2(t)$ are listed in Table ~\ref{tab:2LiI_c2_biexp} and Table ~\ref{tab:2NaI_c2_biexp}.
The biexponential fit is very close to the calculated $C_2(t)$, which can be seen in Fig.\space\ref{fig:2LiI-124w_c2_fit_5ps_biexp}. 
%(compare Fig.\space\ref{fig:2LiI-124w_c2_fit_5_single-exp}).
%
\begin{table}[hbt]
\centering
\caption{\label{tab:2LiI_c2_biexp}%
	Biexponential fitting (5 ps) of the $C_2(t)$ for water molecules in 0.9 M LiI solution.}
%\begin{ruledtabular}
\begin{tabular}{lccccc}
water molecules & $A_1$  & $\kappa_1$ (THz) & $A_2$ & $\kappa_2$ (THz) \\
\hline
I$^-$-shell & 0.44 & 0.25 & 0.39 & 0.26\\
Li$^+$-shell & 0.88 & 0.07 & 0.07 & 8.24\\
bulk & 0.84 & 0.11 & 0.09 & 4.88 \\
surface & 0.73 & 0.27 & 0.22 & 13.47 \\
\end{tabular}
%\end{ruledtabular}
\end{table}
%--

\begin{table}
\centering
  \caption{\label{tab:2NaI_c2_biexp}%
	Biexponential fitting (5 ps) of the $C_2(t)$ for water molecules in 0.9 M NaI solution.}
  \begin{tabular}{lccccc}
  water molecules & $A_1$  & $\kappa_1$ (THz) & $A_2$ & $\kappa_2$ (THz) \\
  \hline
  I$^-$-shell & 0.86 & 0.14 & 0.08 &9.86 \\
  Na$^+$-shell & 0.71 & 0.06 & 0.18 &0.79 \\
  bulk & 0.81 & 0.06 & 0.10 & 1.25 \\
  surface & 0.77 & 0.11 & 0.13 & 2.31 \\
  \end{tabular}
\end{table}
%
%图
\begin{figure}[htbp]
\centering
\includegraphics [width= 0.6\textwidth] {./diagrams/2LiI-124w_c2_fit_5_biexp} 
  \caption{\label{fig:2LiI-124w_c2_fit_5ps_biexp} The time dependence of the $C_2(t)$ of OH bonds 
  in water molecules at the water/vapor interface of LiI solution.}
\end{figure} 
%
%[Notes: The 63-water-slab models is listed here as a reference. The number of water molecules is small; The data for KI/vapor and LiI/vapor interfaces come from  KI\_16 and LiI\_16 systems.  
%Water(63) &0.831$\pm(1\times10^{-4})$ &  0.08760 $\pm(2\times 10^{-5})$&0.100$\pm(2\times10^{-4})$ & 1.029 $\pm(4\times10^{-3})$  \\ ]
%
%\begin{figure}[htbp]
%\centering
%\includegraphics [width=0.4 \textwidth] {./diagrams/c2_121-pure_2KI_2LiI_16_inset_fit_biexp} 
%\setlength{\abovecaptionskip}{10pt}
%\caption{\label{fig:c2_121-pure_2KI_2LiI_16_inset_fit_biexp} The fitted and calculated anisotropy decay of OH bonds in water molecules in LiI solution/vapor interface (red), LiI solution/vapor interface (blue) and neat water/vapor interface (black). The corresponding fitted functions are denoted by dashed lines. The concentration of LiI and KI solution is 0.9 M.}
%\end{figure} 

Then we considered the effect of ion species in solutions on the anisotropy decay of water molecules.
From Table \ref{tab:2LiI_c2_biexp} and Table \ref{tab:2NaI_c2_biexp}, we find that 
for both LiI and NaI solutions, there are two decay processes in the dynamics --- amplitude $\sim 1$,
decay constant $\sim$ 0.1 THz, and for the other describe the initial fast decay 
of the anisotropy, with amplitude $\sim 0.1$, decay constant $\sim$ (1--10) THz, 
due to the inertial-librational motion preceding the orientational diffusion.
That is, two decay processes exist in the dynamics of water molecules 
at the water/vapor interfaces of alkali-iodine solutions. 
%The one describe the initial fast decay of the anisotropy, 
%with amplitude $\sim$ 0.1, decay constant $\sim$ (1--10) THz,
%results from the inertial-librational motion preceding the orientational diffusion.
%
\begin{table}[H]
\centering
\caption{\label{tab:fitting_c2_for_each_type_of_water}%
  Biexponentially fitting (2 ps) of the $C_2(t)$ for different types of water molecules at the water/vapor interface of LiI solutions.}
\begin{tabular}{lccccc}
water molecules & $A_1$  & $\kappa_1$ (THz) & $A_2$ & $\kappa_2$ (THz) \\
\hline
$DDAA$ & 0.85 & 0.25 & 0.10 & 16.0\\
$DD'AA$ & 0.89 & 0.14 & 0.06 & 14.1 \\
$D'AA$ & 0.38 & 0.99 & 0.38 & 0.99 \\
\end{tabular}
\end{table}
%
\begin{table}[H] %[!hbtp]
\centering
\caption{\label{tab:table_CoordNo}%
The coordination number of the atoms in LiI (NaI) solutions.}
\begin{tabular}{lccc}
name & radius of the first shell (\AA) & coordination number \\
\hline
$n_\text{I-H}(\text{LiI})$ & 3.3 & 5.5 \\
$n_\text{I-H}(\text{NaI)}$ & 3.3 & 5.1 \\
$n_\text{I-O}(\text{LiI)}$ & 4.3 & 5.8 \\
$n_\text{I-O}(\text{NaI)}$ & 4.3 & 6.0 \\
$n_\text{Li-O}(\text{LiI)}$ & 3.0 & 4.0 \\
$n_\text{Na-O}(\text{NaI)}$ & 3.5 & 6.0 
\end{tabular}
\end{table}

%In the first shell with a radius 3 \A, the entropy difference between the \Li shell and \Na shell,
%$\Delta S=k_B\text{ln}\frac{\Omega_\text{Na}}{\Omega_\text{Li}}=k_B\text{ln}\frac{n_\text{Na}/V_\text{Na}}{n_\text{Li}/V_\text{Li}} =k_B\text{ln}1.05$.

\FloatBarrier
\paragraph{Anisotropy Decay in Hydration Shells of Ions}
To verify whether the water molecules in the hydration shell of different ions have different orientation dynamics. 
We calculated the anisotropy decay of water molecules in \Li and nitrate ions' hydration shells in the LiNO$_3$ interface. 
The average results of the $C_2(t)$ is shown in Fig.\thinspace\ref{fig:C2_ln_itp_pbc}.
The radius of the hydration shell of nitrate O, \Li, and water molecules are taken as 4.0, 2.8 and 3.5 \AA, respectively. 
These values come from the radial distribution functions $g_{\text{N-OW}}(r)$, $g_{\text{Li-OW}}(r)$ and $g_{\text{OW-OW}}(r)$,
for bulk alkali nitrate solution, as shown in Fig.\thinspace\ref{fig:gdr_127_LiNO3}a and Fig.\thinspace\ref{fig:gdr_NW_WW_127_LiNO3}.
In view of the diffusion of molecules, we only counts trajectories with a duration of 10 ps. The total number of trajectories is 6.

\begin{figure}[H]
\centering
\includegraphics [width=0.42\textwidth] {./diagrams/C2_ln_itp_pbc} 
\setlength{\abovecaptionskip}{0pt}
\caption{\label{fig:C2_ln_itp_pbc}The $C_2(t)$ for water in the solvation shell of water, \Li and nitrate ions in the interface system of \LiN solution at $T=300$ K.
The size of periodic box is $(a,b,c)=(15.7787, 15.7787, 31.5574)$ (\AA). } 
\end{figure}
\begin{figure}[H]
\centering
\includegraphics [width=0.8\textwidth] {./diagrams/gdr_NW_WW_127_LiNO3} 
\setlength{\abovecaptionskip}{0pt}
\caption{\label{fig:gdr_NW_WW_127_LiNO3}The RDFs for the \LiN solution at $T=300$ K.}
\end{figure}

In the interface of the NaNO$_3$ solution, the average value of the $C_2(t)$ functions are shown in 
Fig.\thinspace\ref{fig:C2_nn_itp_pbc}. 
The radius of the hydration shell of nitrate O, \Na, and water molecules are taken as 4.0, 3.2 and 3.5 \AA, respectively. 
(These values come from the radial distribution functions $g_{\text{N-OW}}(r)$, $g_{\text{Na-OW}}(r)$ and $g_{\text{OW-OW}}(r)$,
for bulk alkali nitrate solution, as shown in Fig.\thinspace\ref{fig:gdr_127_LiNO3}a and Fig.\thinspace\ref{fig:gdr_NW_WW_127_NaNO3}.)
\begin{figure}[H]
\centering
\includegraphics [width=0.42\textwidth] {./diagrams/C2_nn_itp_pbc} 
\setlength{\abovecaptionskip}{0pt}
\caption{\label{fig:C2_nn_itp_pbc}The $C_2(t)$ for water in the solvation shell of water, Na$^+$ and nitrate ions in the interface system of NaNO$_3$ solution at $T=300$ K.
The size of periodic box is $(a,b,c)=(15.6530, 15.6530, 31.3060)$ (\AA). } 
\end{figure}
\begin{figure}[H]
\centering
\includegraphics [width=0.8\textwidth] {./diagrams/gdr_NW_WW_127_NaNO3} 
\setlength{\abovecaptionskip}{0pt}
\caption{\label{fig:gdr_NW_WW_127_NaNO3}The RDFs for the NaNO$_3$ solution at $T=300$ K.}
\end{figure}

For the interface of the KNO$_3$ solution, we also did the same calculation of $C_2(t)$. 
The average value of the $C_2(t)$ functions are shown in Fig.\thinspace\ref{fig:C2_kn_itp_pbc}.
The radius of the hydration shell of nitrate O, K$^+$, and water molecules are taken as 4.0, 3.6 and 3.5 \AA, respectively. 
(These values come from the radial distribution functions $g_{\text{N-OW}}(r)$, $g_{\text{K-OW}}(r)$ and $g_{\text{OW-OW}}(r)$,
for bulk alkali nitrate solution, as shown in Fig.\thinspace\ref{fig:gdr_127_LiNO3}a and Fig.\thinspace\ref{fig:gdr_NW_WW_127_KNO3}.)
\begin{figure}[H]
\centering
\includegraphics [width=0.42\textwidth] {./diagrams/C2_kn_itp_pbc} 
\setlength{\abovecaptionskip}{0pt}
\caption{\label{fig:C2_kn_itp_pbc}The $C_2(t)$ for water in the solvation shell of water, K$^+$ and nitrate ions in the interface system of KNO$_3$ solution at $T=300$ K.
The size of periodic box is $(a,b,c)=(15.7160, 15.7160, 31.4320)$ (\AA). } 
\end{figure}
\begin{figure}[H]
\centering
\includegraphics [width=0.8\textwidth] {./diagrams/gdr_NW_WW_127_KNO3} 
\setlength{\abovecaptionskip}{0pt}
\caption{\label{fig:gdr_NW_WW_127_KNO3}The RDFs for the KNO$_3$ solution at $T=300$ K.}
\end{figure}

It can be found that in the interfacial systems alkali metal nitrate solution, 
nitrate ions always accelerate the reorientation dynamics of water molecules in the hydration shells of nitrate ions.
However, the reorientation dynamics of water molecules in the hydration shells of alkali metal ions may slowed down 
(for the cases of LiNO$_3$ and NaNO$_3$ solutions, respectively) or accelerated (for the case of KNO$_3$ solution), 
due to the presence of alkali metal ions. 

We find an interesting relation between the reorientation relaxation time and the radius of hydration shell.
For the interface systems of the above three nitrate solutions, the decay rate of the orientation correlation function 
is positively correlated with the radius of the particle's dissolved sphere. To see this dependence, 
we use an exponential decay function $C_2(t) = e^{-t/\tau_2}$ to fit the water molecule Correlation function 
to calculate the $\tau_2$ value corresponding to the dissolved shell of different particles. 
We list the calculated values of $\tau_2$ in Table \ref{tab:relaxation_tau_vs_radius_ln}--\ref{tab:relaxation_tau_vs_radius_kn}. 
It can be seen that the reorientaion relaxation time $\tau_2$ of the water molecules in the hydration shell increases with 
the increase of the radius of the hydration shell, and their relationship is shown in Fig.\thinspace\ref{fig:ln_nn_kn_tau2_vs_shell_radius}.
\begin{table}[H]
\centering
\caption{\label{tab:relaxation_tau_vs_radius_ln} 
    The radius $r$ of hydration shells and corresponding relaxation times $\tau_2$ in the interface system of LiNO$_3$ solution at $T=300$ K.} 
\begin{tabular}{ccc}
 ion (molecule) & $r$ (\AA) & $\tau_2$ (ps)  \\
\hline
  Water & 3.5 & 6.48 $\pm$ 0.02  \\
  \Li & 2.8 & 9.26 $\pm$ 0.02 \\
  NO$^-_3$ & 4.0 & 5.30 $\pm$ 0.02 \\
\end{tabular}
\end{table}
\begin{table}[H]
\centering
\caption{\label{tab:relaxation_tau_vs_radius_nn} 
    The radius $r$ of hydration shells and corresponding relaxation times $\tau_2$ in the interface system of NaNO$_3$ solution at $T=300$ K.} 
\begin{tabular}{ccc}
 ion (molecule) & $r$ (\AA) & $\tau_2$ (ps)  \\
\hline
  Water & 3.5 & 7.12 $\pm$ 0.02  \\
  \Na & 3.2 & 9.24 $\pm$ 0.02 \\
  NO$^-_3$ & 4.0 & 5.66 $\pm$ 0.01 \\
\end{tabular}
\end{table}
\begin{table}[H]
\centering
\caption{\label{tab:relaxation_tau_vs_radius_kn} 
    The radius $r$ of hydration shells and corresponding relaxation times $\tau_2$ in the interface system of KNO$_3$ solution at $T=300$ K.} 
\begin{tabular}{ccc}
 ion (molecule) & $r$ (\AA) & $\tau_2$ (ps)  \\
\hline
  Water & 3.5 & 7.04 $\pm$ 0.02  \\
  \K & 3.6 & 6.49 $\pm$ 0.02 \\
  NO$^-_3$ & 4.0 & 5.30 $\pm$ 0.01 \\
\end{tabular}
\end{table}
\begin{figure}[H]
\centering
\includegraphics [width=0.42\textwidth] {./diagrams/ln_nn_kn_tau2_vs_shell_radius} 
\setlength{\abovecaptionskip}{0pt}
\caption{\label{fig:ln_nn_kn_tau2_vs_shell_radius}The dependence of $\tau_2(t)$ on the radius of the solvation shell of molecules 
(water,\Li,\Na, \K and nitrate ions) in the interface system of alkali nitrate solutions at $T=300$ K.}
\end{figure}
  
%{Rotational Anisotropy Decay of Water at Aqueous/vapor Interfaces of Alkali Halide Solutions}\label{AD}
\FloatBarrier
\paragraph{$C_2(t)$: Aqueous vapor interface of LiI(NaI) solution}
Here shows the simulated SFG spectrum of the aqueous-vapor interface of LiI solutions at 330 K.
The spectrum are calculated from the average of two LiI solution-vapor interfaces, which are under the same condition.
 We use the following procedures to calculate the molar concentration of ions in the solutions we study.
\begin{align}
&n_j=N_j\times[1/(6.02\times10^{23})] {\text{ mol}} \nonumber \\
&V_{\text{liquid}}=15.6\times15.6\times15.6 \text{\AA}^3=3796\times10^{-30}\text{ m}^3 \nonumber
\label{eq:concen}
\end{align}
where $n_j$, $N_j$ and $V_{\text{liquid}}$  is the amount of substance $j$, the number of substance $j$, and the volume of the liquid part of the liquid/vapor interface.  
For the LiI solution/vapor interface system. The simulation box is with the size of 31.0 \AA$ \times$15.6 \AA$ \times$15.6 \AA. Half of the volume of the simulation box is vacuum. In the liquid part of the simulation box, there are two \Li cations and two \I anions.
Therefore, the molar concentration of the solution LiI is $c_{\text{LiI}}={n_{\text{LiI}}}/{V_\text{liquid}}=0.9\times10^3  \text{ mol}/\text{m}^3$.

\FloatBarrier
\paragraph{Probability Distribution of Ions}
The probability distribution of the ions in the aqueous-vapor interface of  LiI and NaI solutions with repect to the depth of the ions in the solutions (molar concentration: 0.9 M, temperature: 330 K). The distribution indicates that the \I ions prefer to staying at the topmost layer of surface of solutions.
 The probability distribution shows that \I ions tend to the surface of the solutions, while \Na and \Li tend to stay in the bulk. This result is consistent with the calculations from Ishiyama and Morita\cite{TI07,TI14}.

\FloatBarrier
\paragraph{The Anisotropy Decay}
The orientational anisotropy $r(t)$ is given by the rotational time-correlation function 
\begin{equation}
C_2(t)=\langle P_2(\hat{u}(0)\cdot\hat{u}(t)) \rangle,
\label{eq:tcf2}
\end{equation}
where $\hat{u}(t)$ is the time dependent unit vector of the transition dipole, $P_2(x)$ is the second Legendre polynomial, and the angular brackets indicate equilibrium ensemble average.\cite{SAC05}%\cite{2010Lin}

First, we assume that the anisotropy decay is a single exponential given by 
\begin{equation}
C_2(t)=e^{-\kappa t},
\label{eq:tcf2}
\end{equation}
where $\kappa$ is a rate constant for the anisotropy decay.

The time correlation functions for water molecules bound to specific ions, for selected frequency windows up to 1.5 ps, are shown in Fig.~\ref{fig:2NaI-124w_0-25ps_c2_150224}.

\begin{figure}
\centering
\includegraphics [width=0.4 \textwidth] {./diagrams/2NaI-124w_c2_fit_150223} 
\setlength{\abovecaptionskip}{10pt}
\caption{\label{fig:2NaI-124w_c2_fit_150223} The aniostropy decay of OH chromophores in water molecules in NaI water solutions. Water molecules bound to \I anions decay fastest,
while those bound to \Na slowest. (time range: 0-10 ps).}
\end{figure} 

\begin{figure}
\centering
\includegraphics [width=0.4 \textwidth] {./diagrams/2LiI-124w_0-25ps_c2_150222b} 
\setlength{\abovecaptionskip}{10pt}
\caption{\label{fig:2LiI-124w_0-25ps_c2_150222b} The aniostropy decay of OH chromophores in water molecules in NaI and LiI water solutions.}
\end{figure} 

We obtain non-exponential kinetics for the rotation of water molecules in both surface and bulk water (and this is true for water molecules bound to ions).
Therefore, the rotational motion of water molecules are not simpley characterized by well-defined rate constants. Then the problem is how to understand the kinetics.
The similar non-exponential kinetics is also obtain in the HB kinetcs in liquid water.\cite{AL96,TEDirama}  Luzar and Chandler interpreted the non-exponential kinetics as the 
result of an interplay between diffusion and HB dynamics \cite{AL96}. (We can undertand the non-exponential kinetics of rotational anistropy decay by the following model:
%-------------------
\begin{table}
\caption{\label{tab:table_alkali_halide_1}%
The fitted parameters of aniostropy decay of water moelcules in LiI (NaI) solutions. The decay rate $\kappa$ comes from fitting. }
%\begin{ruledtabular}
\begin{tabular}{lccc}
Central ions& Period $T$ (ps) & Decay rate $\kappa$ (THz) \\
\hline
\I (LiI solution) & 3.5 & 0.29 \\
\Li (LiI solution) & 12.3 & 0.08 \\
\I (NaI solution) &10.7 & 0.09 \\
\Na (NaI solution) & 14.8 & 0.07 \\
\end{tabular}
%\end{ruledtabular}
\end{table}
%--------------
%-------------------
\begin{table}
\caption{\label{tab:table_alkali_halide_2}%
The fitted parameters of aniostropy decay of water moelcules in LiI (NaI) solutions. The decay rate $\kappa$ comes from fitting. }
%\begin{ruledtabular}
\begin{tabular}{lccc}
Water molecules& Period $T$ (ps) & Decay rate $\kappa$ (THz) \\
\hline
bulk (LiI solution, or LiI) & 6.9 & 0.14 \\
surface (LiI) & 2.3 & 0.44 \\
bulk (NaI solution, or NaI) & 11.8 & 0.08 \\
surface (NaI) & 5.6 & 0.18 \\
\end{tabular}
%\end{ruledtabular}
\end{table}
%--------------------
\begin{table}[h!]
\centering
\caption{\label{tab:table_expfit}%
The fitted parameters of aniostropy decay of water moelcules in LiI (NaI) solutions. The decay rate $\kappa$ comes from fitting ([0,10] ps).}
%\begin{ruledtabular}
\begin{tabular}{lccc}
Water molecules & $c$  &  $\kappa$ (THz) &$T=1/\kappa$ (ps) \\
\hline
bound to I (LiI) & 0.82 & 0.24 & 4.2 \\
bound to Li (LiI) & 0.88 & 0.07 & 14.3 \\
in bulk (LiI) & 0.85 & 0.12 & 8.4\\
in surface (LiI) & 0.81 & 0.35 & 2.9  \\
bound to I (NaI) & 0.86 & 0.14 & 7.1 \\
bound to Na (NaI) & 0.79 & 0.07 & 14.3 \\
in bulk (NaI) & 0.83 & 0.06 & 16.7 \\
in surface (NaI) & 0.78 & 0.12 & 8.4 \\
\end{tabular}
%\end{ruledtabular}
\end{table}
%
\begin{figure}
\centering
\includegraphics [width=0.4 \textwidth] {./diagrams/c2_121-pure_2KI_2LiI_16_inset} 
\setlength{\abovecaptionskip}{10pt}
\caption{\label{fig:c2_121-pure_2KI_2LiI_16_inset} The aniostropy decay of OH chromophores in water molecules in 0.8M LiI (KI) solution /vapor interface (red and blue) and in Neat water/vapor interface (black). The water/vapor interface is modeled with a slab made of 121 water molecules in a simulation box of size 15.6\AA$\times$15.6\AA$\times$31.0\A. The inset plots the same correlation function $C_2(t)$ for the interval [0,1]ps.}
\end{figure} 
Fig.~\ref{fig:c2_121-pure_2KI_2LiI_16_inset} shows that the solution with \I ions affect the rotational anisotropy decay of water molecules at interfaces.

\begin{table}[h!]
\centering
\caption{\label{tab:table_2KI_2LiI_anisotropy_decay}%
The fitted parameters of aniostropy decay of water moelcules in LiI (KI) solution/vapor interfaces and neat water/vapor interface. The constants and amplitudes comes from fitting ([0,20]ps). Notes: The 63-water-slab models is listed here as a reference. The number of water molecules is small; The data for KI/vapor and LiI/vapor interfaces come from  KI\_16 and LiI\_16 systems. } 
%\begin{ruledtabular}
\begin{tabular}{lccccc}
Interfaces & $A_1$  & $\kappa_1$ (THz) & $A_2$ & $\kappa_2$ (THz) \\
\hline
Water(63) &0.831$\pm(1\times10^{-4})$ &  0.08760 $\pm(2\times 10^{-5})$&0.100$\pm(2\times10^{-4})$ & 1.029 $\pm(4\times10^{-3})$  \\
Water &0.595$\pm(1\times10^{-3})$ &  0.07434 $\pm(9\times 10^{-5})$&0.310$\pm(9\times10^{-4})$ & 0.290 $\pm(8\times10^{-4})$  \\
KI/Vapor &0.797$\pm(4\times10^{-4})$ & 0.16530 $\pm(7\times 10^{-5})$ &0.122$\pm(4\times10^{-4})$ & 0.884 $\pm(6\times10^{-3})$ \\ 
LiI/Vapor &0.836$\pm(2\times10^{-4})$ & 0.20463$\pm(4\times 10^{-5})$ &0.091$\pm(2\times10^{-4})$ & 1.752 $\pm(9\times10^{-3})$ \\
\end{tabular}
\label{biexponential}
%\end{ruledtabular}
\end{table}

\begin{figure}
\centering
\includegraphics [width=0.4 \textwidth] {./diagrams/c2_121-pure_2KI_2LiI_16_inset_fit_biexp} 
\setlength{\abovecaptionskip}{10pt}
\caption{\label{fig:c2_121-pure_2KI_2LiI_16_inset_fit_biexp} The fitted and calculated aniostropy decay of OH chromophores in water molecules in LiI solution/vapor interface (red), LiI solution /vapor interface (blue) and neat water/vapor interface (black). The corresponding fitted functions are denoted by dashed lines. The concentration of LiI and KI solution is 0.9 M.}
\end{figure} 

\FloatBarrier
\paragraph{Fitting by a biexponential}
To obtain the effects of diffusion and HB decay of water molecules in solutions respectively, we assume that there are two independent decays in the process of an anisotropy decay. 
Therefore, $C_2(t)$ has the form\cite{TanHS05}
\begin{equation}
C_2(t)=A_1e^{-\kappa_1 t} +A_2e^{-\kappa_2 t},
\label{eq:tcf3}
\end{equation}
where $A_i$ are constants and $\kappa_i$ are decay rates ($i=1,2$). 
The time constants and amplitudes of the biexponentials fits for the $C_2(t)$ are listed in table ~\ref{tab:table8}
\begin{table}  % or [!htbp]
\caption{\label{tab:table8}%
The fitted parameters of aniostropy decay of water moelcules in LiI (NaI) solutions. The constants and amplitudes comes from fitting ([0,10] ps).}
%\begin{ruledtabular}
\begin{tabular}{lccccc}
Water molecules & $A_1$  & $\kappa_1$ (THz) & $A_2$ & $\kappa_2$ (THz) \\
\hline
bonded to I (LiI solution)  &0.45 & 0.31  & 0.45 & 0.31\\
bonded to Li (LiI solution) & 0.56 & 0.12 & 0.33 &0.02  \\
bulk (LiI solution) &0.43  & 0.11& 0.43 & 0.12 \\
surface (LiI solution) & 0.41 & 0.35 & 0.40  & 0.36 \\
bonded to I (NaI solution) &0.86 & 0.14 & 0.08 &9.86 \\
bonded to Na (NaI solution) & 0.71 & 0.06 & 0.185 &0.79 \\
bulk (NaI solution) & 0.81 & 0.06 & 0.099  &1.25  \\
surface (NaI solution) & 0.77 & 0.11& 0.126 & 2.31 \\
\end{tabular}
\label{biexponential}
%\end{ruledtabular}
\end{table}

Let us consider the effect of ion species in solutions on the aniostropy decay of water molecules. From table ~\ref{biexponential}, we find that $\kappa_1\sim\kappa_2$ for LiI solution, 
while for NaI solution, $\kappa_1<<\kappa_2$.
The parameters can have the following explanation: $\kappa_1$ and $\kappa_2$ describe the time scale of Hydrogen bond dynamics and O---Li (O---Na) bond dynamics, respectively. 
In LiI solution, O---Li bond dynamics and Hydrogen bond dynamics nearly are of the same time scale ($\sim$ 5.0 ps), 
while in NaI solutin,time scale of O---Na bond dynamics ($\sim 0.3$ps) is much shorter than that of HB dynamics ($\sim 11.1$ ps). 
%-----------------
\begin{figure}
\centering
\includegraphics [width=0.4 \textwidth] {./diagrams/2NaI-124w_c2_fit_biexp_150310} 
\setlength{\abovecaptionskip}{10pt}
\caption{\label{fig:2NaI-124w_c2_fit_biexp_150310} The aniostropy decay of OH groups in water molecules in NaI water solutions can be fit (dashed lines) very well by a biexponential in [0, 10] ps).
The biexponential fits better than single exponential does (see fig.~\ref{fig:2NaI-124w_c2_fit_150223}). }
\end{figure} 

\begin{table}
\centering
\caption{\label{tab:table9}%
The fitted parameters of aniostropy decay of different types of water moelcules in LiI solutions ([0,2] ps).}
%\begin{ruledtabular}
\begin{tabular}{lccccc}
Water molecules & $A_1$  & $\kappa_1$ (THz) & $A_2$ & $\kappa_2$ (THz) \\
\hline
$DDAA$ &0.85 &0.25   & 0.10 & 16.0\\
$DD'AA$ &0.89 &0.14  & 0.06 & 14.1 \\
$D'AA$ &0.38 & 0.99 &0.38 & 0.99 \\
\end{tabular}
%\end{ruledtabular}
\end{table}
%--------------
In conclusion, two decay processes exist in the dynamics of water molecules in the alkali halide solution/vapor interface. The amplitude $\sim$ 1, decay constant $\sim$ 0.1 THz, and for the other describe the initial fast decay of the anisotropy, with amplitude $\sim$ 0.1, decay constant $\sim$ (1 $\sim$ 10) THz, due to the inertial-librational motion preceding the orientational diffusion.
%
\section{$C_2(t)$ of water molecules in dydration shell of ions}
\begin{figure}
\centering
\includegraphics [width=0.4 \textwidth] {./diagrams/c2_hal_sh1_s} 
\setlength{\abovecaptionskip}{10pt}
\caption{\label{fig:c2_hal_sh1_s} The aniostropy decay of OH chromophores in water molecules localized in the first hydration shell of \I ions in LiI water solutions (0.9 M; 20 ps). }
\end{figure} 

\begin{figure}
\centering
\includegraphics [width=0.4 \textwidth] {./diagrams/124_2NaI_c2_comp_na_bulk} 
\setlength{\abovecaptionskip}{10pt}
\caption{\label{fig:124_2NaI_c2_comp_na_bulk} The aniostropy decay of OH chromophores in water molecules localized in the first hydration shell of \Na ions in NaI water solutions (0.9 M); The slower correlation time indicates that the water molecules are held more rigidly within the 1st hydration shell of \Na ions.}
\end{figure}


\FloatBarrier
\paragraph{Aqueous/Vapor Interface of LiI(NaI) Solution}
Here shows the simulated SFG spectrum of the aqueous-vapor interface of LiI solutions at 330 K.
The spectrum are calculated from the average of two LiI solution-vapor interfaces, which are under the same condition.
The simulation box is with the size of 15.6 \AA$ \times$15.6 \AA$ \times$31.0 \A. Half of the volume of the simulation box is vacuum. 
There are two \Li cations and two \I anions in the solution part of the simulation box.
The molar concentration of the solution is $c_{\text{LiI}}=0.9\times10^3 \text{ mol}/\text{m}^3$.
For both LiI and NaI solutions, the Im$\chi^{(2)}$ spectrum shows a sharp peek around 3700 cm$^{-1}$, which aassociated the free OH stretch (or dangling OH stretch).

%\begin{figure}
%\includegraphics [width=1.0 \textwidth] {./diagrams/2LiI-2NaI-124w_gdr_OH_150122} % Here is how to import EPS art
%\caption{\label{fig:rdf_OH_different_layers} The RDF $g_{\text{O-H}}(r)$ in the surfaces of the aqueous-vapor interfaces of solutions (molar concentration: 0.9 M) at 330 K. Solid and dashed lines
%are corresponding to the two interface in our simulation, respectively. The simulation time is 22.5ps. 
%The result for two interface consistent to each other when the interface is large enough, i.e., the thickness is 6 \A.
%(a) LiI solution; (b) NaI solution. }
%\end{figure} 
%Fig.~\ref{fig:rdf_OH_different_layers} shows that when the thickness of the layer is 8 \A, the structure of the surface does not change. 
%This result is the same for both aqueous-vapor interface of NaI solutions (molar concentration: 0.9 M) at 330 K. 
%As the thickness increases, the coordination number of water O atoms is increasing from about 3 to 5.
%
\begin{table}
\caption{\label{tab:table_CoordNo}%
The coordination number of the atoms in LiI (NaI) solutions.}
%\begin{ruledtabular}
\begin{tabular}{lccc}
Name & Radius of the fisrt shell(\AA) & Coordination number \\
\hline
$n_\text{I-H}(\text{LiI})$ & 3.3 & 5.5 \\
$n_\text{I-H}(\text{NaI)}$ & 3.3 & 5.1 \\
$n_\text{I-O}(\text{LiI)}$ & 4.3 & 5.8 \\
$n_\text{I-O}(\text{NaI)}$ & 4.3 & 6.0 \\
$n_\text{Li-O}(\text{LiI)}$ & 3.0 & 4.0 \\
$n_\text{Na-O}(\text{NaI)}$ & 3.5 & 6.0 
\end{tabular}
%\end{ruledtabular}
\end{table}

In the first shell with a radius 3 \A, the entropy difference betweem the \Li shell and \Na shell,
$\Delta S=k_B\text{ln}\frac{\Omega_\text{Na}}{\Omega_\text{Li}}=k_B\text{ln}\frac{n_\text{Na}/V_\text{Na}}{n_\text{Li}/V_\text{Li}} =k_B\text{ln}1.05$.


\FloatBarrier
\paragraph{The anisotropy decay}
\begin{figure}
\includegraphics [width=0.6 \textwidth] {./diagrams/2LiI-124w_c2_fit_150223} 
\caption{\label{fig:2LiI-124w_c2_fit_150223} The aniostropy decay of OH chromophores in water molecules in LiI water solutions (time range: 0-10ps)(fit in [0, 10] ps).}
\end{figure} 

\begin{figure}
\includegraphics [width=0.6 \textwidth] {./diagrams/2LiI-124w_0-25ps_c2_150222b} 
\caption{\label{fig:2LiI-124w_0-25ps_c2_150222b} The aniostropy decay of OH chromophores in water molecules in NaI and LiI water solutions.}
\end{figure} 

We obtain non-exponential kinetics for the rotation of water molecules in both surface and bulk water (and this is true for water molecules bonded to ions).
Therefore, the rotational motion of water molecules are not simpley characterized by well-defined rate constants. Then the problem is how to understand the kinetics.
The similar non-exponential kinetics is also obtain in the HB kinetcs in liquid water.\cite{AL96,AL96b}  Luzar and Chandler interpreted the non-exponential kinetics as the result of an interplay between diffusion and HB dynamics \cite{AL96}. 
%\cite{2008TEDirama}
\begin{table}
\caption{\label{tab:table6}%
The fitted parameters of aniostropy decay of water moelcules in LiI (NaI) solutions. The decay rate $\kappa$ comes from fitting.}
%\begin{ruledtabular}
\begin{tabular}{lccc}
Water molecules& Period $T$ (ps) & Decay rate $\kappa$ (THz) \\
\hline
bonded to I (LiI solution) & 3.5 & 0.29 \\
bonded to Li (LiI solution) & 12.3 & 0.08 \\
bulk (LiI solution) & 6.9 & 0.14 \\
surface (LiI solution) & 2.3 & 0.44 \\
bonded to I (NaI solution) &10.7 & 0.09 \\
bonded to Na (NaI solution) & 14.8 & 0.07 \\
bulk (NaI solution) & 11.8 & 0.08 \\
surface (NaI solution) & 5.6 & 0.18 \\
\end{tabular}
%\end{ruledtabular}
\end{table}
%---------------------------------------------------
%\paragraph{Fitting by a biexponential}
\begin{table}
\caption{\label{tab:table_center_ion}%
The fitted parameters of aniostropy decay of water moelcules in LiI (NaI) solutions. The constants and amplitudes comes from fitting ([0,10] ps).}
%\begin{ruledtabular}
\begin{tabular}{lccccc}
centeral ions & $A_1$  & $\kappa_1$ (THz) & $A_2$ & $\kappa_2$ (THz) \\
\hline
\I (LiI solution) &0.45 &0.31   & 0.45 & 0.31\\
\Li (LiI solution) & 0.56 & 0.12 &0.33 &0.02  \\
\I (NaI solution) &0.86 & 0.14 &0.08 &9.86 \\
\Na (NaI solution) & 0.71 &0.06 & 0.185 &0.79 \\
\end{tabular}
\label{biexponential}
%\end{ruledtabular}
\end{table}
%--------------------
\begin{table}
\caption{\label{tab:table_surf-bulk}%
The fitted parameters of aniostropy decay of bulk and surfacal water moelcules in LiI (NaI) solutions. The constants and amplitudes comes from fitting ([0,10]ps).}
%\begin{ruledtabular}
\begin{tabular}{lccccc}
Water molecules & $A_1$  & $\kappa_1$ (THz) & $A_2$ & $\kappa_2$ (THz) \\
\hline
bulk (LiI solution) & 0.43 & 0.11 & 0.43 & 0.12 \\
surface (LiI solution) & 0.41 & 0.35 & 0.40 & 0.36 \\
bulk (NaI solution) & 0.81 & 0.06 & 0.099 & 1.25 \\
surface (NaI solution) & 0.77 & 0.11 & 0.126 & 2.31 \\
\end{tabular}
\label{biexponential}
%\end{ruledtabular}
\end{table}

\begin{figure}%[!htbp]
\includegraphics [width=0.6 \textwidth] {./diagrams/2NaI-124w_c2_fit_biexp_150310} 
\caption{\label{fig:2NaI-124w_c2_fit_biexp_150310} The aniostropy decay of OH groups in water molecules in NaI water solutions can be fit (dashed lines) very well by a biexponential in [0, 10]ps).
The biexponential fits better than single exponential does (see fig.~\ref{fig:2NaI-124w_c2_fit_150223}). Two decay processes exist in the dynamics: amplitude $\sim$ 1,
decay constant $\sim$ 0.1 THz, and for the other describe the initial fast decay of the anisotropy, with amplitude $\sim$ 0.1, decay constant $\sim$ (1 $\sim$ 10) THz, 
due to the inertial-librational motion preceding the orientational diffusion.}
\end{figure} 

\begin{table}
\caption{\label{tab:table9}%
The fitted parameters of aniostropy decay of different types of water moelcules in LiI solutions ([0,2] ps).}
%\begin{ruledtabular}
\begin{tabular}{lccccc}
Water molecules & $A_1$  & $\kappa_1$ (THz) & $A_2$ & $\kappa_2$ (THz) \\
\hline
$DDAA$ &0.85 &0.25   & 0.10 & 16.0\\
$DD'AA$ &0.89 &0.14  & 0.06 & 14.1 \\
$D'AA$ &0.38 & 0.99 &0.38 & 0.99 \\
\end{tabular}
%\end{ruledtabular}
\end{table}

\FloatBarrier
\paragraph{ Classification of Hydrogen Bonds}
$D$ denotes that the water molecule donates a H-Bond, $D'$ donates a H-I bond, and $A$ accepts a H-Bond. \cite{TianCS08} %\cite{2008NJi} 
$DDAA$: water molecules with two H-Bonds donated to water molecules and two H-Bonds accepted from water molecules (usually, water molecules in pure bulk water);
$DD'AA$: water molecules in bulk and with on H-Bond bonded to \I and other H-Bonds to water molecules, 
$D'AA$: water molecules bonded to \I (at the water/vapor interface) and other H-Bonds to water molecules.
Therefore, water molecules with $D'AA$ bond type are of free O-H stretching during the dynamics. 
For water molecules with $D'AA$-type H-Bond, the inertial-librational motion can not be seen, which means the rotational anisotropy decay of these water molecules are with the same time scale of the inertial libration. See fig.~\ref{fig:Multiple_figs}.

%\begin{figure}%[!htbp]
%\includegraphics [width=0.6 \textwidth] {./diagrams/2LiI-124w_c2_fit_biexp_7wat_2ps_class_150715} 
%\caption{\label{fig:2LiI-124w_c2_fit_biexp_7wat_2ps_class_150715} The $C_2(t)$ function for different HB environments in LiI solution/vapor interface.}
%\end{figure} 

[Conclusion]
Slower rotational anisotropy decay of water molecules is found at these water/vapor interfaces. The result comes from a different H-Bond types from the usual $DDAA$ Hydrogen bond type in pure bulk water.%\cite{2008NJi}
This slower anisotropy decay reflects the effect of Hydrogen-Iodide bond at the interfaces.

In conclusion, the ultrafast anisotropy decay, is dominated by population transfer.
Slower rotational anisotropy decay exists for water molecules  at the water/vapor interface of the alkali-iodide solutions, which is the result of a different H-Bond types ($D'AA$) from the usual H-Bond type ($DDAA$) in pure bulk water. This slowing down of anisotropy decay is the effects of Hydrogen-Iodide bond at the interface. Since the iodide's surface perpensity is high, this difference of H-Bond structure from pure water/vapor interface changed the Im$\chi^{(2)}$ spectrum and the total H-Bond dynamics of the interface of alkali-iodide solutions.  The effects of Hydrogen-Iodide bond on the H-Bond dynamics at the interfaces, and the relation between the interfacial H-Bond dynamics and rotational anisotropy decay can also be studied in the future.

%\paragraph{Power Spectra}
%The power spcetra of dipole correlation fuction.
%\begin{figure}
%\centering
%\includegraphics [width=0.4 \textwidth] {./diagrams/2LiI-124w_sp_power_relax} 
%\setlength{\abovecaptionskip}{10pt}
%\caption{\label{fig:2LiI-124w_sp_power_relax} The FT of aniostropy decay of OH chromophores in water molecules in 0.9 M LiI water solutions at 330 K.}
%\end{figure} 

\FloatBarrier
\section{Dipole Orientation of Water at Aqueous/Vapor Interfaces}
Set $\theta$ as the angle between the diple moment of water molecules and the normal vector 
of the interface, and $P(\theta)$ as the probability.

The data used to statistics is the value of the dipole tilt angle $\theta_{i}$, i.e., 
We now use the $\theta_i$ instead of $\langle\theta\rangle$ to do the statistics.
We pick up $\theta_{l}$, $l=0, 10, 20, ...$  ($\text{mod}_{10}(l)=0$, $l<n$) from the series $\theta_{i}$.
With this method, we find that generally, pure water's dipole moment tilt angle is smaller than  that of water molecules at the salty water surface. 
This means that in the pure water surface, water molecules has more $p$-polarization components than those in the salty water surface. 
But the water molecules at the salty water surface has more $s$-polarization component.
The result is shown in Fig.\ref{fig:salty-pure_7ps}.
%----------------
\begin{figure}
\centering
\includegraphics [width=1.0\textwidth]{./diagrams/dipole_orien_comparison} % Here is how to import EPS art
\setlength{\abovecaptionskip}{20pt}
\caption{\label{fig:salty-pure_7ps} The diple orientation for water molecules at pure water surface (red) and salty water surface (green).}
\end{figure}

\FloatBarrier
\section{\LiN Solution/Vapor Interface}
The aniostropy decay of OH chromophores in water molecules in 0.4 M LiNO3 solution/vapor interface is shown in Fig.~\ref{fig:c2_LiNO3_inset}.  
In the model of the interface, there is one \Li and one \nitrate in the 15.6 \AA$\times$15.6 \AA$\times$31.0 \AA simulation box. 
The larger decay rate consistent to the conclusion infered from the VDOS of the interfaces, although the concentration of \LiN is lower. 
This result obtained from another DFTMD trajectory consistent with the previous one, and it reflects that the \nitrate on the surface of the alkali nitrate solution weaken the H-bonds and  accelerate the anisotropy decay of water molecules at the interfaces.
\begin{figure}
\centering
\includegraphics [width=0.4\textwidth] {./diagrams/c2_LiNO3_inset} 
\setlength{\abovecaptionskip}{10pt}
\caption{\label{fig:c2_LiNO3_inset} The aniostropy decay of OH chromophores in water molecules in LiNO3 solution/vapor interface.}
\end{figure} 

\FloatBarrier
\paragraph{Classification of Water Molecules Based on H-Bonds}
We also studied the relation between the anisotropy decay of water molecules and their environment. 
Following the definition used in Ref.\cite{TianCS08}, we use the following labels to denote water molecules in solution: 
$D$ denotes that the water molecule donates a HB, $D'$ donates that the water donates a H-I bond, and $A$ donates that the water accepts a HB. %\cite{2008NJ} 
$DDAA$ represents a water molecule with two H-Bonds donated to water molecules and two H-Bonds accepted from water molecules (see Fig.\thinspace\ref{fig:Multiple_figs}a);
$DD'AA$ represents a water molecule with two HBs donated to a water molecule and \I, and with two H-Bonds accepted from other water molecules 
(see Fig.\thinspace\ref{fig:Multiple_figs}c), 
$D'AA$ represents a water molecule bonded to \I at the water/vapor interface and other H-Bonds to water molecules (see Fig.\thinspace\ref{fig:Multiple_figs}(d)).
Clearly, we can see that $D'AA$ molecules are of free OH stretching during the dynamics. All four types of water molecules are displayed in Fig.\thinspace\ref{fig:Multiple_figs}. 
% 
\begin{figure}[ht]%[!htbp]
\centering
\includegraphics [width=0.4 \textwidth] {./diagrams/Multiple_figs} 
\caption{\label{fig:Multiple_figs} Four types of water molecules at the water/vapor interfaces of LiI solution, regarding the HB environments: (a) $DDAA$; (b) $DDA$; (c) $DD'AA$; (d) $D'AA$. The cyan balls denote \I ions. }
\end{figure} 

It is evident, from our calculations (Fig.\space\ref{fig:2LiI-124w_c2_fit_biexp_7wat_2ps_class_150324}), that the $C_2(t)$ for $DDAA$ and $DD'AA$ molecules do not decay exponentially (Table \ref{tab:fitting_c2_for_each_type_of_water}).
%[BUT Table \ref{tab:fitting_c2_for_each_type_of_water} CAN NOT GIVE THE EVIDENCE. STH. IS MISSING!] 
This result is similar to the reactive flux HB correlation function $k(t)$, i.e., 
the escaping rate kinetics of H-bonds in bulk water. \cite{Luzar1996} 
The relaxation of H-bonds in water appears complicated, with no simple characterization in terms of a few relaxation rate constants. 
Most of the authors believe that the cooperativity between neighbouring H-bonds, \cite{Sciortino1989, Ohmine1995} or 
self evident coupling between translational diffusion and HB dynamics is the source of the complexity. \cite{Luzar1996} 
However, for $D'AA$ molecules at the interface of the LiI solution,
the $C_2(t)$ decays exponentially, i.e.
\begin{eqnarray}
  C_2(t) &=& C e^{-{\kappa}t},
\label{eq:C_2_D_prime_AA}
\end{eqnarray}
where the amplitude is $C=0.76$, and the reorientation rate is $\kappa = 0.99$ ps$^{-1}$.
The single exponential decay of $C_2(t)$ for $D'AA$ molecules, indicates that each $D'AA$  molecule reorientate independently to each other. 

Furthermore, the $C_2(t)$ for $D'AA$ molecules decays much faster than that for $DDAA$ or $DD'AA$ molecules.
From the definitions, the $D'AA$ water molecule owns only three H-bonds, while both $DDAA$ and $DD'AA$ water molecules own four H-bonds.
Therefore, the correlation between H-bonds around the $D'AA$ molecule is weaker than those around the $DDAA$ or $DD'AA$ molecule. 
Faster decay of $C_2(t)$ for $D'AA$ molecules shows that the reorientation process of $D'AA$
molecules is much smaller than those water molecules in bulk phase, e.g., the $DDAA$, and $DD'AA$ molecules.

Finally, for $D'AA$ molecules, the inertial-librational motion can not be seen (Fig.\space\ref{fig:2LiI-124w_c2_fit_biexp_7wat_2ps_class_150324}). 
This result implies that the rotational anisotropy decay of $D'AA$ molecules
are of the same time scale of the inertial libration, i.e., $\sim$ 0.2 ps.

Rotational anisotropy decay of water molecules is found at the interface of LiI solution. 
The result comes from a different HB types from the usual $DDAA$ HB type in pure bulk water.
The faster anisotropy decay for $D'AA$ molecules reflects the less correlation between different H-bonds for $D'AA$ molecules, which comes from Hydrogen--Iodide bond at the interfaces, the existence of free OH stretching.
From Fig.\space\ref{fig:prob_124_LiI_double_axis}, we have known that in the LiI solution, 
\I ions prefer to locate at the water/vapor interface.  
Therefore, we infer that the reduction of the inter-correlations between H-bonds occurs at the water/vapor interfaces. 

%
In conclusion, single exponential type rotational anisotropy decay exists for water molecules at the water/vapor interface of the alkali-iodine solutions,
and this faster anisotropy decay of water molecules at the water/vapor interface is the effects of Hydrogen--Iodide (H--I) bond at the interface. 
Since the iodide's surface propensity is high, this difference of HB structure 
from neat water/vapor interface is the source of 
the HB dynamics as well as the Im$\chi^{(2)}$ spectrum of the interface of alkali-iodine solutions.  
%-------
%deleted
%\st{The effects of H--I bond on the HB dynamics at the interfaces, and the relation between the interfacial HB
%dynamics and rotational anisotropy decay can also be studied in the future.}{\color{red}[Question: This i don't understand ... is not what you have discussed so far?? Answer: It was a plan.]}
%-------
%图
\begin{figure}[H] %[!htbp]
\centering
\includegraphics [width=0.36 \textwidth] {./diagrams/2LiI-124w_c2_fit_biexp_7wat_2ps_class_150324} 
\caption{\label{fig:2LiI-124w_c2_fit_biexp_7wat_2ps_class_150324} The time dependence of the $C_2(t)$ for water molecules in different HB environments at the water/vapor interface of LiI solution.}
\end{figure}  

%\subsection{\LiN Solution/vapor Interface}
%The anisotropy decay of OH bonds in water molecules in 0.4 M LiNO3 solution/vapor interface is shown in Fig.\space\ref{fig:c2_LiNO3_inset}.  In the model of the interface, there is one \Li and one \nitrate in the 15.6 \AA$\times$15.6 \AA$\times$31.0 \AA simulation box. 
%The larger decay rate consistent to the conclusion infered from the VDOS for the interfaces, although the concentration of \LiN is lower. This result obtained from another DFTMD trajectory consistent with the previous one, and it reflects that the \nitrate on the surface of the alkali nitrate solution weaken the H-bonds and  accelerate the anisotropy decay of water molecules at the interfaces.
%\begin{figure}[htbp]
%\centering
%\includegraphics [width=0.4\textwidth] {./diagrams/c2_LiNO3_inset} 
%\setlength{\abovecaptionskip}{10pt}
%\caption{\label{fig:c2_LiNO3_inset} The anisotropy decay of OH chromophores in water molecules in LiNO3 solution/vapor interface.}
%\end{figure} 
