\chapter{Hydrogen bond dynamics in electrolyte solutions}\label{CHAPTER_HB_SOLUTIONS}
%We have combined the IHB and MS methods and determined the reaction rate constants of the H-bonds at the water-vapor interface. 
%Below we will make some changes to the way we have used and apply it to the solution systems.
In this chapter, we explore the effects of nitrate ions, iodide ions and alkali metal cations 
on the HB dynamics and water reorientation dynamics at the interface of alkali nitrate solutions and alkali
iodine solutions through DFTMD simulations. %We will also provide microscopic interpretations of recent experimental results. \cite{HuaWei2014}
In paragraphs \ref{HBD_ITP} and \ref{PARA_ION-WAT}, we discuss the HB dynamics in the whole interface system and the ion--water hydrogen bonds. 
Then in paragraph \ref{PARA_SHBD}, we study the HB dynamics and HB lifetime in the solvation shells of the ions.
In paragraphs \ref{RAD} and \ref{RAD_SHELL}, we focus on the reorientation dynamics of water molecules at the interfaces
and in the solvation shells of ions in the solutions.

%[all the data on the simulation]
All simulations in this chapter were performed at 300 K within the canonical NVT ensemble.
The length of each trajectory is 60 ps.
%The definition of $h(t)$ is based on specific H--O bond, instead of water-water pairs.
For the same solution, we have done DFTMD simulations for the interface and the bulk system respectively. 
For the detailed parameter settings of different systems, see the appendix.

\section{HB dynamics at electrolyte solution/vapor interface}\label{HBD_ITP}
\paragraph{Lithium nitrate solutions} \label{PARAGRAPH_LINO3}
%Fig.\thinspace\ref{fig:lino3_interface_all_add_z_trimed}
%illustrates the obtained instantaneous interfaces for one configuration of a slab of 
%LiNO$_3$ solution at 300 K. 
%\egin{figure}[H]
%\centering
%\includegraphics [width=0.32\textwidth] {./diagrams/lino3_interface_all_add_z_trimed}
%\setlength{\abovecaptionskip}{0pt}
%\caption{\label{fig:lino3_interface_all_add_z_trimed}
%A slab of LiNO$_{3}$ solution with the instantaneous
%interface $\mathbf{s}$ represented as a blue mesh on the upper and lower phase
%boundary. The initial position of each ion in the solution is random.
%The slab is periodically replicated in the horizontal directions.} 
%\end{figure}
%
We investigate the HB dynamics in the interface of lithium nitrate solutions. 
%A DFTMD simulation of a bulk phase of LiNO$_3$ solution is performed. 
%The simulated system consisted of 127 water molecules and a Li$^+$--NO$_3^-$ ion pair
%in a periodic cubic box of length $L=15.79$ \AA, which corresponds to a density of 0.997 g cm$^{-3}$. 
The probability distribution of O and H atoms in the model of LiNO$_3$ interface is showed in Fig.\thinspace\ref{fig:prob_wat--ln_itp}. 
\begin{figure}[H]
\centering
\includegraphics [width=0.36 \textwidth] {./diagrams/prob_wat--ln_itp}
\setlength{\abovecaptionskip}{0pt}
\caption{\label{fig:prob_wat--ln_itp}Probability distributions $P(z)$, along the normal direction,
 of O and H atoms at the interface of the LiNO$_3$ solution.} %, through the trajectory of 20 ps.
%$15.7787 \times 15.7787 \times 31.5574$ \AA$^3$
\end{figure}
%

The correlation functions $C_\text{HB}(t)$ and $S_\text{HB}(t)$ for both nitrate--water (N--W) and water--water (W--W) H-bonds 
are shown in Fig.\thinspace\ref{fig:c_and_s_ln_bk_pbc} a and b, respectively.
For both HB definitions, we found that the decay of the $C_\text{HB}(t)$ and $S_\text{HB}(t)$ for nitrate--water hydrogen bonds is much faster 
than that for water--water hydrogen bonds. From the relation \ref{eq:calculate_hb_lifetime_from_s}, the faster relaxation of $S_\text{HB}(t)$ implies that 
the nitrate--water hydrogen bonds have shorter lifetime than the water--water hydrogen bonds in the bulk phase. 
The calculation results obtained from the simulations are in good agreement with numerous experimental and simulation results. \cite{PS03,Vrbka2004,Tongraar2006,Otten2007} 
It means that nitrate has structure-breaking ability, or compared with W--W interaction, N--W HB interaction is weaker.
%Actual value of the length for LiNO3 solution at 300 K : 15.7797 \AA
%Project location: /home/gang/Proj--LiNO3_Interface/Prepare_LiNO3_2020
\begin{figure}[H]
\centering
\includegraphics [width=0.6\textwidth] {./diagrams/c_and_s_ln_bk_pbc}
\setlength{\abovecaptionskip}{0pt}
\caption{\label{fig:c_and_s_ln_bk_pbc} 
Time dependence of (a) $C_\text{HB}(t)$ and (b) ln$S_\text{HB}(t)$ of all water--water and nitrate--water hydrogen bonds
for the slab of LiNO$_3$ solution, as computed from the ADH (solid line) and AHD (dashed line) criterion of H-bonds. 
The definition of the correlation function is based on the specific O-H pairs between molecules.} 
\end{figure}
%
\begin{figure}[H]%
    \centering
    \subfloat[]{{\includegraphics[width=4.0cm]{./diagrams/distance_ions2surf_lino3_trimed} }}
    \qquad
    \subfloat[]{{\includegraphics[width=5.2cm]{./diagrams/prob_dist_li_surf_no3_surf} }}
    \caption{
Distribution of ions at the interface of LiNO$_3$ solution. 
(A) 
Distances between ions and one of the instantaneous surfaces (blue meshes) for a slab of aqueous LiNO$_3$ solution. 
(B)
Density distribution the Li$^+$--surface and \nitrate--surface distances at the interface of LiNO$_3$ solution. 
The horizontal axis represents the distance between the ion and the instantaneous surface, which is defined in 
Eq.\thinspace\ref{eq:distance_particle2surf_1}. The \emph{distance} refers specifically to $d_{\text{X},1}$, the distance between the ion 
X and one of the instantaneous surfaces. Zero distance denotes the instantaneous surface of the interfacial system of LiNO$_3$ solution.
}%
    \label{fig:prob_dist_li_surf_no3_surf}%
\end{figure}


For the DFTMD trajectory of a solution interface, we can now get instantaneous surface (see Fig.\thinspace\ref{fig:prob_dist_li_surf_no3_surf} A).
For any molecule or ion X in such a solution interface system, we can get its distance from the instantaneous surface.
We assume that the $z$-axis is the normal direction, then the distances between the particle and the instantaneous surfaces are:
%
\begin{eqnarray}
    d_{\text{X},1}(t)=  z^\text{surf}_{\text{X},1}(t) - z_{\text{X}}(t),\label{eq:distance_particle2surf_1}\\
    d_{\text{X},2}(t)= z_{\text{X}}(t) - z^\text{surf}_{\text{X},2}(t), 
\label{eq:distance_particle2surf_2}
\end{eqnarray}
%
where $z_{\text{X}}$ is the coordinate of the particle X in the normal direction at time $t$, 
$z^\text{surf}_{\text{X},i}(t)$ is the $z$ coordinate of the surface position corresponding to particle X at time $t$, 
and the subscripts $i=1$ and 2 respectively identify the lower and upper instantaneous surfaces.
As an example, Fig.\thinspace\ref{fig:prob_dist_li_surf_no3_surf} A shows the distance, 
$d_{\text{Li}^+,1}$ ($d_{\text{NO}_3^-,1}$), between \Li (\nitrate) ions and one of the instantaneous surfaces at a certain moment.

Figure \ref{fig:prob_dist_li_surf_no3_surf} B shows the probability density of the distance between the anion (cation) 
and the instantaneous surface in the simulated water/vapor interface of LiNO$_3$ solution. We found that when the system reaches an equilibrium state, 
the \Li ion is stable within a few angstroms below the interface, while the NO$^-_3$ ion resides near the surface. 
When the system is in equilibrium ($t>10$ ps), the distance $d_{\text{NO}_3^-,1}$ is around 2 \A, 
which indicates that the nitrate ion is in the top layer the instantaneous interface. 

From the calculation of the VSFG spectrum of the LiNO$_3$ solution interface and the VDOS of water molecules in the water clusters NO$^-_3$(H$_2$O)$_3$,
we have known that (1) Compared with the water/vapor interface, the VSFG spectrum of the LiNO$_3$ solution interface has a blue-shifted HB band 
(see Fig.\thinspace\ref{fig:sfg_LiNO3_7A_20ps_gauss150});
(2) The vibration frequency of water molecules from the LiNO$_3$ interface is higher than the vibration frequency of water molecules in pure water 
(see Fig.\thinspace\ref{fig:surf_x-vs-l_x_d1-5});
From the above two conclusions indicate that nitrate ions have the surface propensity. 
Here, the probability distribution of the nitrate ion in the slab shows that the average position of the nitrate ion relative to the solution/air interface is 2 \AA,
which is consistent with the conclusion from the VSFG spectra and VDOS calculations: Nitrate ions have a significant propensity for the water/vapor interface.
%
\paragraph{Effects of alkali metal ions and \I on HB dynamics}

The radii of hydration shells are 5.0 \AA for \li, 5.38 \AA for \na,
and 6.0 \AA for \I ions, which are obtained from the RDFs.
The RDFs $g_{\text{X-O}}$ (X=\li, \na, K$^+$) for the interfaces 
of LiI (NaI,KI) solutions are shown in Fig.\thinspace\ref{fig:gdr_XO--124_2XI} a,
and the coordination numbers are in Fig.\thinspace\ref{fig:gdr_XO--124_2XI} b.

From Fig.\thinspace\ref{fig:gdr_XO--124_2XI}, we see that the radius of the solvation shells of \Li ions, \Na ions, 
and \K ions increase sequentially, and the number of coordination molecules also increase sequentially. 
However, this order is not true for the relaxation time of HB dynamics between water molecules in the first solvation shell of the ion 
and other water molecules. As can be seen later (Fig.\thinspace\ref{fig:hbacf_C_sh2_2p}), the effects of the alkali metal ions and iodide ions are very similar, 
that is, the relaxation time of the HB dynamics in the outer layer is smaller than that in bulk water.
To see the characteristics of H-bonds in different solutions, we studied the dynamics of I$^-$-H hydrogen bonds in the next paragraph.
%从图中我们看出,锂离子,钠离子,钾离子的溶解壳的半径依次增大,其配位分子数也依次增大。但这种有序性对于离子第一溶解壳内的水分子与其他水分子之间的氢键的弛豫时间却不成立。因为从7.5图可以看出,R离子和碘离子的效应很类似,即,其外层的HBD的弛豫时间都比体相纯水中的更小。
%为了看清不同溶液中的氢键的差别,我们在下一段研究了I-H氢键的动力学.
\begin{figure}[H]
\centering
\includegraphics [width=0.42\textwidth]{./diagrams/gdr_XO--124_2XI}%fig.6.1 
\setlength{\abovecaptionskip}{0pt}
\caption{\label{fig:gdr_XO--124_2XI}
 (a) RDFs $g_{\text{X-O}}(r)$(X=\li, \na, K$^+$) and (b) the coordination number of \Li (\na, K$^+$) ions at the interfaces of LiI (NaI, KI) solution. 
 The coordination number $n_{\text{Li}^+}$=4, $n_{\text{Na}^+}$=5 and $n_{\text{K}^+}$=6.} 
\end{figure} % There is a first shell exist for both \Li and \Na cations. %(NEED the NORMED VERSION)

Figure \ref{fig:prob_dist_Li_surf_I_surf}--\ref{fig:prob_dist_K_surf_I_surf} show the density of the distance between the anion (cation) 
and the instantaneous surface in the simulated water/vapor interface of LiI, NaI and KI solution, respectively. We found that, 
the \I ion resides near the surface, while the alkali metal ions' density distribution does not have such a strong tendency toward the surface. 
\begin{figure}[H]%
    \centering
    \subfloat[]{{\includegraphics[width=5.2cm]{./diagrams/prob_dist_Li_surf_I_surf} }}
    \qquad
    \subfloat[]{{\includegraphics[width=5.2cm]{./diagrams/prob_dist_Li_surf_I_surf} }}
    \caption{
Distribution of ions at the interface of LiI  and NaI solution.   
(A) 
Density distribution the Li$^+$--surface and I$^-$--surface distances at the interface of LiI solution. 
(B)
(TO ADD)Density distribution the Na$^+$--surface and I$^-$--surface distances at the interface of NaI solution. 
%The horizontal axis represents the distance between the ion and the instantaneous surface, which is defined in 
%Eq.\thinspace\ref{eq:distance_particle2surf_1}. 
%The \emph{distance} refers specifically to $d_{\text{X},1}$, the distance between the ion 
%X and one of the instantaneous surfaces. Zero distance denotes the instantaneous surface of the solution.
}%
    \label{fig:prob_dist_Li_surf_I_surf}%
\end{figure}
\begin{figure}[H]%
    \centering
    \subfloat[]{{\includegraphics[width=4.0cm]{./diagrams/KI_interface_add_arrows_distances_axis_trimed} }}
    \qquad
    \subfloat[]{{\includegraphics[width=5.2cm]{./diagrams/prob_dist_K_surf_I_surf} }}
    \caption{
Distribution of ions at the interface of KI solution.   
(A) 
Distances between ions and one of the instantaneous surfaces (blue meshes) for a slab of aqueous KI solution. 
(B)
Density distribution the K$^+$--surface and I$^-$--surface distances at the interface of KI solution. 
%The horizontal axis represents the distance between the ion and the instantaneous surface, which is defined in 
%Eq.\thinspace\ref{eq:distance_particle2surf_1}. 
%The \emph{distance} refers specifically to $d_{\text{X},1}$, the distance between the ion 
%X and one of the instantaneous surfaces. Zero distance denotes the instantaneous surface of the interfacial system of KI solution.
}%
    \label{fig:prob_dist_K_surf_I_surf}%
\end{figure}

The feature can also be seen from the change of the distance of the ion relative to the interface over time. 
For example, Figure \thinspace\ref{fig:dist_K_surf1_I_surf1} shows the change of the distance between each ion in the KI solution 
and the interface over time. It illustrates that the fluctuation of the distance of \I ions from the interface 
is smaller than the fluctuation of the distance of \K ions from the interface.

%
%\begin{figure}[!ht]
%\centering
%\includegraphics [width=\textwidth] {./diagrams/C_S_HB_124_2LiI-2NaI-2KI} %fig5.15
%\setlength{\abovecaptionskip}{0pt}
%  \caption{\label{fig:C_S_HB_124_2LiI-2NaI-2KI} The time dependence of functions (a) \CHB and (b) \SHB of water--water hydrogen bonds at water/vapor interfaces of 0.9 M alkali-iodine solutions.} 
%\end{figure}
%[hbtp]
\begin{table}[H]
\centering
\caption{\label{tab:tau_hb_alkali_iodine} 
The average of the continuum HB lifetimes $\langle\tau_{\text{a}}\rangle=\int_0^\infty S_\text{HB}(t) dt$ (unit: ps) in the first hydration shell of I$^-$ ion 
and of alkali metal ion at the interface of the three 0.9 M alkali-iodine solutions.}
\begin{tabular}{cccc}
  &\I-shell &cation-shell& interface \\
\hline
 LiI & 0.22 & 0.24 & 0.23\\
 NaI & 0.24 & 0.28 & 0.26\\
 KI  & 0.20 & 0.23 & 0.20\\
\end{tabular}
\end{table} 
%Water/Vapor & -&-&

To investigate ions' effects, we compared the HB population correlation functions between the water/vapor interface and aqueous electrolyte interfaces.
We consider the \CHB and \SHB, which is related to the HB relaxation time $\tau_\text{R}$ and the HB lifetime $\tau_\text{a}$, respectively.

Table \ref{tab:tau_hb_alkali_iodine} lists the continuum HB lifetime in the first hydration shell of I$^-$ ion and of alkali metal ion
at the interfaces of three alkali-iodine solutions: LiI, NaI and KI. It shows that, the continuum HB lifetime $\tau_{\text{a}}$ in the 
solvation shell of alkali metal (iodine) ions is larger (smaller) than 
that of H-bonds at the water/vapor interfaces of the same solutions, 
respectively. For LiI solution, the water molecules bound to the cation ion
\Li, on average, have a continuum HB lifetime $\tau_{\text{HB}} \sim 0.24$ ps,
which is longer than that of molecules bound to \I or at the interface of the LiI solution. 
There are similar HB lifetime $\tau_\text{HB}$ for NaI and KI solutions.
%
\begin{figure}[H]
\centering
\includegraphics [width=0.6\textwidth] {./diagrams/hbacf_C_sh2_2p}
\setlength{\abovecaptionskip}{0pt}
\caption{\label{fig:hbacf_C_sh2_2p}The \CHB of water--water hydrogen bonds in the solvation shell 
  of (a) cations and (b) I$^-$ at the interfaces of 0.9 M LiI, NaI and KI solutions, respectively.
  The dashed line is for the slab of the LiI solution.}  
\end{figure}
%Fecko and co-workers' study of liquid D$_2$O by IR spectroscopy reveals that the vibrational dynamics observed are dominated by underdamped displacement of the hydrogen-bond coordinate at very short times ( less than 200 fs).\cite{CJF03,CJF05} 
Fig.\thinspace\ref{fig:hbacf_C_sh2_2p} a and b show that the \CHB of H-bonds within the alkali cations and \I decay faster 
than those in bulk water and at the surface of LiI solution. 
%[HOW TO EXPLAIN (a)?] [(WHAT DOES THIS SENTANCE FOR?)
The interface of LiI solution contains H-bonds between water molecules similar to those in bulk water, i.e.,
water molecules participating in these H-bonds are not in the solvation shell of ions. 
This result is consistent with the independence of observations of femtosecond midinfrared pump-prob experiments 
on the O--H stretch vibration of water molecules in aqueous solution,
which found that changing the nature of the cation does not affect the dynamics of solvating water.\cite{Kropman2001}
It also conforms to the following \emph{ab initio} simulation results: water molecules that directly surround the cation, the O--H groups point
away from the cation and form O--H$\cdots$O hydrogen bonds with bulk water molecules.\cite{Hashimoto1994,Ramaniah1998,Kropman2001}
From Fig.\thinspace\ref{fig:hbacf_C_sh2_2p} b, we found that for all three alkali-iodine solutions, the \CHB for hydration shell water molecules 
of \I decays faster than that for molecules at the water/vapor interface.
The simulation produces similar result as Omta and coworker's experiments of femtosecond pump-probe spectroscopy,
which demonstrate that anions ( $\text{SO}^{2-}_4$, $\text{ClO}^-_4$, etc) have no influence on the dynamics of bulk water, 
even at high concentration up to 6 M.\cite{Omta2003, ZhangYanjie2006} 
Here, we find that the cations \Li and \Na do not alter the H-bonding network outside the first hydration shell of cations. 
It is concluded that no long-range structural-changing effects for alkali metal cations.

\FloatBarrier
\paragraph{Effects of the ion concentration}
To investigate the effect of the ion concentration we considered one additional system with higer concentration, namely 1.8 M.

We calculated the \CHB for the interfaces of the alkali-iodine solutions, 
and the relaxation time $\tau_{\text{R}}$ for each of them can be determined according to formula \ref{eq:tau_relaxation}. 
Here, the \emph{interface} means \emph{all} the water molecules in each model. 
The $\tau_{\text{R}}$ for the interfaces of the LiI (NaI) solutions are given in 
Table \ref{tab:tau_hb}. Generally, they are in the range 1--10 ps. 
The values of $\tau_{\text{R}}$ decrease as the concentration of the solutions increases.
\begin{table}[htbp]
\centering
\caption{\label{tab:tau_hb} 
  The relaxation time $\tau_{\text{R}}$ (unit: ps) of the \CHB  for the interface of the LiI (NaI) solutions.}
\begin{tabular}{ccc}
  concentration  & $\tau_{\text{R}}$ (LiI) & $\tau_{\text{R}}$ (NaI) \\
\hline
  0 & 11.50 & 11.50 \\
  0.9 M & 7.04 & 10.60 \\
  1.8 M & 4.40 & 1.96 
\end{tabular}
\end{table}

The concentration dependence of the \SHB can also be calculated. 
Figure \thinspace\ref{fig:124_2LiI-2NaI_hbacf_S} a (\ref{fig:124_2LiI-2NaI_hbacf_S} b) gives the \SHB 
for the interfaces of 0.9 M and 1.8 M LiI (NaI) solutions.
This result indicates that, for the interface of alkali-iodine solution, the continuum HB lifetime  
decrease as the concentration of LiI (or NaI) solution increase.
\begin{figure}[H]
\centering
\includegraphics [width=0.6\textwidth, center] {./diagrams/124_2LiI-2NaI_hbacf_S} 
\setlength{\abovecaptionskip}{0pt}
  \caption{\label{fig:124_2LiI-2NaI_hbacf_S} Time dependence of the \SHB  of 
  H-bonds at the interfaces of (a) LiI and (b) NaI solutions at 330 K.
	The insets show the plots of ln$S_{\text{HB}}(t)$.} 
\end{figure}

To summarize, we have investigated the effect of alkali-nitrate and alkali-iodide on the HB dynamics
of water molecules in interfaces, obtained from \emph{ab inito} simulations. 
%我们视气液界面中的所有水分子作为一个整体,来计算其氢键动力学。涉及到离子溶解壳层内的氢键动力学时,我们通过在采样时间点水分子是否处于离子溶解壳内来确定需要考虑的水分子。
N-W's \CHB and \SHB decay faster than W-W's, 
which proves again that the N-W hydrogen bond is weaker than W-W HB. 
The result we obtained from VSFG spectra in Chapter \label{CHAPTER_SFG_Calculation} is that the blue shift of the HB band in the Im$\chi^{(2)}$, 
or, the nitrate ion in the solution tends to be distributed on the surface, 
which is also consistent with the smaller relaxation time given by the HB population correlation functions \CHB and \SHB.

Compared with the water/vapor interface, the characteristic relaxation time of the H-bonds between the water molecules 
in the interface of the alkali metal salt solution as a whole becomes smaller and increases with the increase of the solute concentration.
%与纯水界面相比,作为一个整体的卤化碱金属盐溶液界面中的水分子之间的氢键特征弛豫时间变小,且随着溶质浓度的增加而增大。

%Additional notes (TODO)
%Effects of ions' concentration on HB dynamics have been studied extensively by Chandra. \cite{Chandra2000}
%Pal and coworkers provided details on the structure of water around the micellar surface.\cite{SP05} 

\section{Ion--water hydrogen bond dynamics}\label{PARA_ION-WAT}
%For the water/vapor interface of alkali-iodine solutions, the $k(t)$ is also calculated.  The result for the interface of 0.9 M LiI solution is shown in Fig.\thinspace\ref{fig:hbrf_4pl} (b). The log-log plot of $k(t)$ is not a straight line, indicating that, for water/vapor interface of the LiI solution, this decay does not coincide with a power-law decay, neither.

%{As can be seen from Fig. \ref{fig:hbrf_4pl}, the fluctuations of the $k (t)$ for $d = 2$ \AA (blue solid line) are significantly larger 
%than that of other cases with larger $d$. 
%This phenomenon is due to the relatively small number of water molecules in the thin layer 
%and the insufficient sampling, resulting in large fluctuations in $k(t)$.
%For these four models, as the thickness $d$ of the interface increases, the $k(t)$ gradually converges to a function with smaller fluctuations.
%%
%This conclusion is consistent with the two conclusions we obtained earlier (see Section \ref{sfg_alkali_iodide_interface}): 
%(1) \I is a strong structure-breaking anion; %[\cite{Trevani2000}] 
%(2) compared to pure water, the OH stretching peak at the interface of a solution containing iodide ions will blue shift. [\cite{Tongraar2010}] 
%Comparing these black solid curves, we can see that the interface of the solution containing ions has lower $k(t)$.
%In other words, compared to the water/vapor interface, 
%the ratio of H-bonds that were initially bonded at the solution interface and broken at time $t$ is lower.
%Because the effect of iodide ions is to increase the $k(t)$ of the interface, the decrease of $k (t)$ of the interface with a larger thickness
%may only be due to the contribution of cations located under the first layer of water molecules at the interface. 
%Therefore, although the iodide ion increases the HB rupture rate at the top layer of the interface, 
%in general, the HB rupture rate of the entire solution interface is reduced due to the presence of cations under the first layer of water molecules. 
%To verify this conclusion, we calculated the $k(t)$ at the interface of NaI (Fig. \ref{fig:hbrf_4pl} (c)) and KI (Fig. \ref{fig:hbrf_4pl} (d)) aqueous solution. 
%The results for both interface systems support our conclusions above.
%}
%\stkout{ What is the differences between bulk and interface? 
%Let us examine the difference in the $k(t)$ between interface water and bulk water. 
%No matter from pure water (Fig. \ref{fig:hbrf_4pl} (a)) 
%or solution (Fig. \ref{fig:hbrf_4pl} (b), (c) or (d)), we find that when the interface thickness is thin, the fluctuation of $k(t)$ is larger.
%Because the thinner the interface, the fewer pairs of water molecules that can form hydrogen bonds. 
%In our calculations, the fewer samples are used to average, so the fluctuation of $k (t)$ is greater. 
%We can find that at the water/vapor interface, when $t> 0.2$ ps, the $k(t)$ value of the interface with different thickness is almost equal 
%at any time period $\Delta t$. For example, $\Delta t$ is selected as $\sim$ 2 ps, 
%and its average value is shown in Table \ref{tab:hbrf_neat}. In each time period of 2 ps, the values of $k(t)$ for different layers are approximately equal
%($\pm 0.004$ ps$^{-1}$). Therefore, as far as the nature of HB reactive flux is concerned, the difference between interface and bulk phase of neat water is not obvious. 
%}

%To show the effect of water molecule diffusion on the HB dynamics, we can calculate the sum of the functions $c(t)$ and $n(t)$, i.e., $c(t)+n(t)$.
%Here, we take the LiI solution as an example.
%Fig.\space\ref{fig:124_2LiI_ns20_c_plus_n} shows the time dependence of the correlation functions $c(t)$, $n(t)$ and $c(t)+n(t)$ of the interface of 
%the LiI solution at a concentration of 0.9 M in the AIMD simulation.
%As can be seen, although the change in the total population, $c(t)+n(t)$, is very small in the range of 0--10 ps, it is not a constant.
%Therefore, the $n(t)$ relaxes not only by conversion back to HB \emph{on} state, 
%but is also depleted due to the diffusion process. We can estimate the time scale of water molecule diffusion at the interface of the aqueous solution by $c(t)+n(t) = 1/e$, 
%which is much larger than 10 ps. Therefore, when we analyze the HB dynamics of the solution interfaces, we do not consider the effect of water molecule diffusion.
%
%\begin{figure}[H]
%\centering
%\includegraphics [width=0.36\textwidth] {./diagrams/124_2LiI_ns20_c_plus_n}
%\setlength{\abovecaptionskip}{0pt}
%\caption{\label{fig:124_2LiI_ns20_c_plus_n} 
%The time dependence of the functions $c(t)$, $n(t)$ and $c(t)+n(t)$, where $c(t)$ represents the $C_{\text{HB}}(t)$, 
%for the interfaces of 0.9 M LiI solution.} 
%\end{figure}
%
%\begin{figure}[H]
%\centering
%\includegraphics [width=0.5\textwidth] {./diagrams/128w_bk_2delta_t_60ps_n}
%\setlength{\abovecaptionskip}{0pt}
%\caption{\label{fig:128w_bk_2delta_t_60ps_n} 
%The time dependence of the population functions $n(t)$ for bulk water, as computed from the ADH (solid line) and AHD (dashed line) criterion of H-bonds.} 
%\end{figure}
% 
%

%To study the HB dynamics after the transition phase, which is roughly at 0.1 ps (see Fig.\thinspace\ref{fig:121}) and lasts for hundreds of picoseconds, 
%we set $t_1 = 1$ ps and $t_2 = 10$ ps in the fitting.
For the water/vapor interface and the aqueous electrolyte solution interfaces, 
we have performed the same analysis described in chapter \ref{CHAPTER_HB} where the $k(t)=-dc(t)/dt$ is decomposed into two terms: $kc(t)$ and $-k'n(t)$. 
The optimal values of coefficients $k$ and $k'$ given for these interfaces have been listed in Table \ref{tab:k_k_prime_pure_and_solutions}. 
These values are comparable in magnitude to those obtained by Ref.\thinspace{\cite{Khaliullin2013}}. 
It can be seen from Table \ref{tab:k_k_prime_pure_and_solutions} that the HB breaking reaction rate ($k$) at the water/vapor interface is basically equivalent to 
that at the solution interface, but the HB reforming rate constant($k'$) is smaller than that at the solution interface by 30\% to 50\%.
Correspondingly, we find the HB relaxation times for the three interfaces are: $\tau={1}/{(k+k')} \sim $2.0--2.5 ps. 
For water/vapor interface, the relaxation time is $\tau \sim $ 3.3 ps. 
Our conclusion is that the difference between the relaxation time of H-bonds at the interface of solutions such as LiI, NaI, KI 
and the water/vapor interface is mainly due to the difference in the reforming rate $k'$ of H-bonds caused by the presence of ions,
rather than the difference in the breaking rate $k$ of H-bonds.
%
\begin{table}[htbp]
\centering
\caption{\label{tab:k_k_prime_pure_and_solutions} 
    The $k$ and $k'$ for the water/vapor interface of the aqueous solution interfaces.} 
\begin{tabular}{cccc}
 Interface & $k$ (ps$^{-1}$) & $k'$ (ps$^{-1}$) & $\tau_{\text{R}}$ (ps) \\
\hline
  water/vapor & 0.10 $\pm$ 0.02 & 0.20 $\pm$ 0.02 & 11.50 \\
  LiI & 0.10 $\pm$ 0.04 & 0.30 $\pm$ 0.05 & 5.33 \\
  NaI & 0.20 $\pm$ 0.10 & 0.30 $\pm$ 0.05 & 5.77 \\
  KI  & 0.10 $\pm$ 0.04 & 0.40 $\pm$ 0.10 & 6.96 
\end{tabular}
\end{table}
As for the effect of the interface on the HB dynamics in alkali-iodine solutions,
we also calculate the survival probability for interfaces with different sizes of thickness. 
The result for the interface of the LiI solution shows that H-bonds at water/vapor interface decay faster than that in bulk water.
The logarithm of \SHB is given in Fig.\thinspace\ref{fig:2LiI-124w_S_layers} in Appendix \ref{thickness_interface}, 
in which the thickness of the alkali-iodine solutions can be determined.
Therefore, as the interface thickness increases, the \SHB converges to a fixed curve, 
which characterizes the HB dynamics of the interface of the bulk solutions. 
In particular, it gives the average continuum HB lifetime in bulk solutions.
\paragraph{Nitrate--water hydrogen bonds}
First, let us take a look at the changes in the H-bonds of water molecules by nitrate ions in the bulk solution. 
The $\ln{S_\text{HB}(t)}$ for the water--water hydrogen bonds and nitrate--water hydrogen bonds at the interface of the \LiN solution is shown in 
Fig.\thinspace\ref{fig:256_LiNO3_hbacf_sh_no3}. 
The $\ln{S_{\text{HB}}(t)}$ shows that
the nitrate--water bonds are weaker than water--water bonds, i.e.,nitrate ions accelerate the dynamics of H-bonds in water.
%
%\begin{figure}[htbp] % or \begin{SCfigure}
%\centering
%\includegraphics [width=0.6\textwidth] {./diagrams/shb_c_and_s_ln_bk_NShell_pbc}
%\setlength{\abovecaptionskip}{0pt}
%\caption{\label{fig:shb_c_and_s_ln_bk_NShell_pbc} The $C_{SHB}(t)$ and $S_{SHB}(t)$ of water--water hydrogen bonds at the solvation shell 
%  of nitrate ion in the \LiN solution. These results are calculated for the temporal resolution $t_t=0.4$ ps. For the definition 
%  of $t_t$, see Appendix \ref{thickness_interface}. }
%\end{figure}
%
%I simulate the alkali nitrate solution/vapor interface to find how the nitrate affect the structure of the interface.
\begin{figure}[htbp] % or \begin{SCfigure}
\centering
\includegraphics [width=0.36\textwidth] {./diagrams/256_LiNO3_hbacf_sh_no3} %fig.5.10
\setlength{\abovecaptionskip}{0pt}
\caption{\label{fig:256_LiNO3_hbacf_sh_no3} The \SHB of water--water (W--W) and nitrate--water (N--W) H-bonds at the 
  interface of the \LiN solution. The inset is the plot of ln\SHB. 
  These results are calculated for the temporal resolution $t_\text{t}=1$ fs. For the definition of $t_\text{t}$, see Appendix \ref{thickness_interface}. }
\end{figure}
%
%\begin{figure}[H]
%\centering
%\includegraphics [width=0.4\textwidth] {./diagrams/256_LiNO3_hbacf_hh_all_traj_sh_no3}
%\setlength{\abovecaptionskip}{20pt}
%\caption{\label{fig:256_LiNO3_hbacf_hh_all_traj_sh_no3}The functions ln\SHB of water--water hydrogen bonds (black) and Nitrate -water hydrogen bonds (red) in the the \LiN solution-vapor interface at 300 K. The lifetime of H-bonds $\tau_{\text{HB}}$ is calculated by the integration of \SHB over t$\in$(0,$\infty$), which give 0.42  and 0.20 ps, for water--water hydrogen bonds and Nitrate -water hydrogen bonds, respectively.}
%\end{figure}
%The density profile is a indicator of a table interfacial system (see Fig.\space\ref{fig:density_4MPlus_alkali-I}).
%\begin{figure}[htbp]
%\centering
% \includegraphics [width=0.6\textwidth] {./diagrams/density_4MPlus_alkali-I} %fig5.11
%\setlength{\abovecaptionskip}{20pt}
%\caption{\label{fig:density_4MPlus_alkali-I}The density as a function of the slab coordinate \Z. The result is calculated by MD with SPC water model.}
%\end{figure}


%
The difference between nitrate--water and water--water hydrogen bonds 
can be also analyzed in terms of the survival probability $S_{\text{HB}}(t)$, \cite{AKS86,JT90,AL96} 
reported in Fig.\thinspace\ref {fig:256_LiNO3_hbacf_sh_no3}.
The integration of \SHB from 0 to $t_{\max}=5.0$ ps, \cite{Steinel2004} gives the approximate lifetime $\tau_\text{a}$. \cite{SC02} 
The values of $\tau_{\text{a}}$ depend on the temporal resolution $t_t$, during which the H-bonds that break and reform are treated as intact. \cite{AL00} 
%
Here, we choose the temporal resolution as $t_t=1$ fs. 
Then, Fig.\thinspace\ref {fig:256_LiNO3_hbacf_sh_no3} gives $\tau_\text{a}=0.20$ ps for nitrate--water hydrogen bonds at interfaces, and $\tau_\text{a}=0.42$ ps for water--water hydrogen bonds.
This result of $\tau_\text{a}$ is consistent with the experimental result of Kropman and Bakker ($\tau_\text{a}=0.5\pm0.2$ ps). \cite{MFK01}
The smaller value of $\tau_\text{a}$ for nitrate--water hydrogen bonds implies that the nitrate--water hydrogen bonds are weaker than bonds between water molecules. 
This is also consistent with the VDOS analysis and the blue-shifted frequency (of 55 \cm towards the blue) of the OH stretching in the nitrate-water HB 
(see Fig.\thinspace\ref{fig:vdos_LiNO3-256w_w_near_nitrate}). 
%[DELETED From both the VDOS and HB dynamics calculations, we conclude that it is the weak HBs between nitrate and water make the higher surface propensity 
%of nitrate anions, and then induce the depletion of SFG intensity at 3200 \cm for the alkali nitrate salty interfaces.]

%Fig. ~\ref{fig:256_LiNO3_hbacf_Nitrate_effect} shows that the nitrate ions accelerate the HB dynamics at the vapor/water interface of alkali nitrate solution.
%\begin{figure}[H]
%\centering 
% \includegraphics [width=0.6\textwidth] {./diagrams/256_LiNO3_hbacf_Nitrate_effect} %fig5.12
%\setlength{\abovecaptionskip}{20pt}
%\caption{\label{fig:256_LiNO3_hbacf_Nitrate_effect}The functions \CHB of bulk water--water hydrogen bonds (W-W (Bulk)) and nitrate--water hydrogen bonds (N-W) 
%at interfaces of alkali nitrate solution  (LiNO$_3$(H$_2$O$_{256}$)  at 300 K. }
%\end{figure} 
%NOT CLEAR, TO EXPLAIN BETTER The HB relaxation time is about $2.5$ ps, which is the same as that
%for nitrate--water hydrogen bonds at interfaces of alkali nitrate solution.
%[NOT CLEAR: For bulk water, the HB relaxation time $\tau$ is $3.7$ ps. The difference between the HB dynamics of H-bonds outside the first shell of \Li and HB dynamics for nitrate--water hydrogen bonds at interfaces
%is not visible from the values of the HB relaxation time. They reflect the difference between HB
%dynamics between bulk water and water/vapor interfaces.]

\FloatBarrier

\paragraph{I$^-$--water hydrogen bond dynamics}\label{PARAGRAPH_I--W}
Hydrogen bonds between water molecules and other species also play decisive role in chemical and biological systems. 
For this type of H-bond, some results obtained by molecular simulations have been published. For example,
the HB dynamics of surfactant--water and water--water hydrogen bonds at the air/water interface has been analyzed by Chanda 
and Bandyopadhyay.\cite{Chanda2006} 
Similar analyse for nitrate--water is also done by Yadav, Choudhary and Chandra by first-principles MD simulations.\cite{Yadav2017} 
In the case of water--water hydrogen bonds, the cutoff radius $r_\text{OO}^{\text{c}}=3.5$ \AA is the position of the first minimum of the oxygen--oxygen RDF.

\begin{figure}[H]
\centering
\includegraphics [width=0.6 \textwidth] {./diagrams/X-O_c_lii_xlogscale} 
\setlength{\abovecaptionskip}{0pt}
  \caption{\label{fig:X-O_c_lii_xlogscale}Time dependence of the intermittent correlation functions \CHB of I$^-$--water (I$^-$--W) and water--water hydrogen bonds. 
A base-10 log scale is used for the $x$-axis.
}
\end{figure} %(300 K)
\begin{figure}[H]
\centering
\includegraphics [width=0.6 \textwidth] {./diagrams/wat-wat_s_lii} 
\setlength{\abovecaptionskip}{0pt}
  \caption{\label{fig:wat-wat_s_lii}Time dependence of the continuous correlation functions \SHB of I$^-$--water (I$^-$--W) and water--water hydrogen bonds.}
\end{figure} % 300 K
For the anion--oxygen H-bonds, we can use the similar criteria. The cutoff values for X--oxygen distance are obtained from the positions of the first
minimum of the X--oxygen RDF, i.e., $R_\text{XO}^\text{c}$=3.7 and 4.1 \A, for X = nitrate O (see Fig.\thinspace\ref{fig:gdr_127_LiNO3}) 
and X = I$^-$ (see Fig.\thinspace\ref{fig:gdr_124_LiI}). We have used $\phi^\text{c} = 30^{\circ}$ for the angular cutoff.\cite{Chowdhuri2006}
The function \CHB of I$^-$--water and water--water hydrogen bonds describes the structural relaxation of these H- bonds. 
The $C_\text{HB}(t)$ are shown in Fig.\thinspace\ref{fig:X-O_c_lii_xlogscale}, and
the results of the continuous correlation functions for both definitions (ADH and AHD criterions) for the H-bonds are shown in Fig.\thinspace\ref{fig:wat-wat_s_lii} 
for I$^-$--water hydrogen bonds. The results of water--water hydrogen bonds are also included for comparison in both Fig.\thinspace\ref{fig:X-O_c_lii_xlogscale} 
and \thinspace\ref{fig:wat-wat_s_lii}.
For both ADH and AHD definitions of the H-bonds, it is found that the I$^-$--water hydrogen bonds show faster dynamics than water--water hydrogen bonds 
(consistent to the previous MD results by Chowdhuri and Chandra).\cite{Chowdhuri2006} 
%\begin{figure}[htbp]
%\centering
%\includegraphics [width=0.6 \textwidth] {./diagrams/X-O_c_lii} 
%\setlength{\abovecaptionskip}{0pt}
%  \caption{\label{fig:X-O_c_lii}Time dependence of the intermittent correlation functions $C_\text{HB}(t)$ of I$^-$--water (I$^-$--W) and water--water hydrogen bonds at 300 K.}
%\end{figure}


The time scales of the relaxation of the I$^-$--water hydrogen bonds are obtained for both definitions. 
In Table \ref{tab:properties_anion-water_hbs}, we have included the average lifetimes $\langle\tau_\text{a}\rangle$ for I$^-$--water and nitrate--water hydrogen bonds. 
We have performed the fitting in the time region 0.2 ps < $t$ < 12 ps to calculate the forward and backward rate constants for HB reactive flux.
No matter from ADH or AHD criteria, the average lifetime $\langle\tau_\text{a}\rangle$ of I$^-$--water hydrogen bonds is shorter than that of NO$_3$--water hydrogen bonds.
In addition, based on the HB population operator for ion--molecule pairs, we also calculated the HB lifetime $1/k$ for these two hydrogen bonds. 
The results show that the lifetime of the I$^-$--water bonds is only half of lifetime of NO$_3^-$--water bonds. Therefore, from the perspective of HB dynamics,
we can draw the following conclusion: I$^-$--water hydrogen bonds and NO$_3^-$--water hydrogen bonds are both weaker than water--water hydrogen bonds. In particular, 
the I$^-$--water bonds is weaker than the NO$_3^-$--water ones.
\begin{table}[htbp]
\centering
\caption{ 
    The dynamical properties of I$^-$--water and nitrate--water hydrogen bonds within ADH (AHD) criterion.} 
\begin{tabular}{ccc}
\label{tab:properties_anion-water_hbs}
 Quantities  & I$^-$--water & NO$_3^-$--water \\
\hline
  $\langle\tau_a\rangle$ (ps) & 0.10 (0.11) & 4.35 (7.91) \\
  $1/k$ (ps) & 2.80 (2.40) & 4.15(6.02) \\
\end{tabular} % 300 K
% Data from: LiI solution:  gang@XPS /home/gang/Github/water_pair_HB_dynamics/src/least_square_fit
% Data from: LiNO3 solution:  /home/gang/Github/water_pair_HB_dynamics/src/least_square_fit/LiNO3 
\end{table}
%
%\begin{figure}[H] % htbp
%\centering
%\includegraphics [width=0.6 \textwidth] {./diagrams/I--wat_n_lii} 
%\setlength{\abovecaptionskip}{0pt}
%  \caption{\label{fig:I--wat_n_lii} Time dependence of the correlation functions $n_\text{HB}(t)$ of I$^-$--water (I$^-$--W) and water--water hydrogen bonds at 300 K.}
%\end{figure}
%
%\begin{figure}[H]
%\centering
%\includegraphics [width=0.6 \textwidth] {./diagrams/I--wat_n_lii_xlogscale} 
%\setlength{\abovecaptionskip}{0pt}
%  \caption{\label{fig:I--wat_n_lii_xlogscale} Time dependence of the correlation functions $n_\text{HB}(t)$ of I$^-$--water (I$^-$--W) and water--water hydrogen bonds at 300 K.
%A base-10 log scale is used for the $x$-axis.}
%\end{figure}

\FloatBarrier
\section{Solvation shell HB dynamics} \label{PARA_SHBD}
We will extend the IHB dynamics to H-bonds around certain ions in aqueous solutions. 
Similar to the determination of the instantaneous surface, we can define an interface for each molecule or ion in aqueous solutions, i.e., 
the first solvation shell of them. 
Below we will combine the interface defined by the first solvation shell of the ion, 
and Luzar-Chandler's HB population \cite{AL96} to calculate the HB
dynamics for the H-bonds between water molecules in the solvation shell of the ion and those outside the shell. 
From the characteristics of HB dynamics in the solvation shells, we can obtain the effects of various ions on structure and dynamics of aqueous solutions. 

\paragraph{Solvation shell HB population}
Given the solvation shell ${\mathbf k}(t)={\mathbf k}(\{{\mathbf r}_i(t)\})$, we can define the Solvation shell H-Bonds (SHBs).
We use the parameter $r_\text{shell}$ to denote the radius of the solvation shell.
We define the solvation shell HB population operator $h^{(\text{k})}(t) = h^{(\text{k})}[{r}(t)]$ as follows:
It has a value 1 when the particular tagged molecular pair are H-bonded \emph{and} one of the water molecules are inside the solvation shell
with a radius $r_\text{shell}$, and zero otherwise. 
The definition of $h^{(\text{k})}(t)$ is similar to $h^{(\text{s})}[{r}(t)]$, which is defined in \ref{IHBP} for studying the interfacial H-bonds.
Similarly, $h^{(\text{k})}(t)$ can be used to obtain the dynamic characteristics of H-bonds in the solvation shell with radius $r_\text{shell}$. 
%The definition of HB here can be based on water molecule pairs or O-H pairs. 
Like in the IHB case, in this paragraph, we just discuss water molecule pair-based H-bonds. 

Similar to the \CSHB for the H-bonds in instantaneous interfaces, for a given $r_\text{shell}$,
we define the correlation function $C^\text{(k,X)}_\text{HB}(t)$ that describes the fluctuation of the solvation shell H-bonds for ion X: 
\begin{eqnarray}
C^\text{(k,X)}_{\text{HB}}(t)=\langle h^\text{(k,X)}(0)h^\text{(k,X)}(t) \rangle/\langle h^\text{(k,X)}\rangle
\label{eq:C_k_HB}.
\end{eqnarray}
When not considering specific ions, we denote $C^\text{(k,X)}_{\text{HB}}(t)$ as $C^{\text{(k)}}_\text{HB}(t)$ for short.
%
Similarly, we define a correlation function 
\begin{eqnarray}
n^\text{(k,X)}(t)=\langle h^\text{(k,X)}(0)[1-h^\text{(k,X)}(t)]h^\text{(d,k,X)} \rangle/\langle h^\text{(k,X)}\rangle
\label{eq:n_k_HB},
\end{eqnarray}
and a reactive flux function
\begin{eqnarray}
k^\text{(k,X)}(t)= -\frac{dC_\text{HB}^\text{(k,X)}}{dt}
\label{eq:k_k_HB}.
\end{eqnarray}
Using these functions, we can determine the reaction rate constants of breaking and reforming and the lifetimes for the solvation shell H-bonding.
The results are given in the next paragraphs.
%
\paragraph{Alkali-nitrate solution}
%{Probability Distribution of Ions}

\begin{figure}[H] 
\centering
\includegraphics [width=0.6\textwidth] {./diagrams/shb_c_ln_bk_Shell_pbc}
\setlength{\abovecaptionskip}{0pt}
\caption{\label{fig:shb_c_ln_bk_Shell_pbc}
Correlation function $C^\text{(k)}_\text{HB}(t)$ for the H-bonds in the solvation shells, based on %(water-water pair based) 
the HB population $h^\text{(k)}(t)$, as computed from the (a) ADH and (b) AHD criteria of H-bonds.} 
\end{figure}
%The results are obtained from the DFTMD simulation for the bulk LiNO$_3$ solution at $T=300$ K. 
We calculated the HB dynamics of the H-bonds between water molecules in the first solvation shell and molecules outside the first solvation shell. 
The choice of the shell radius comes from the RDFs (see Fig.\ref{fig:gdr_NW_WW_127_LiNO3}). Comparing Fig.\ref{fig:256_LiNO3_hbacf_sh_no3} 
and Fig.\thinspace\ref{fig:shb_c_ln_bk_Shell_pbc}, we conclude that: In bulk system, although the HB between nitrate and water molecule is significantly 
weaker than the bond between \Li ion and water molecule under the same conditions,
the HB strength between water molecules in the ions' hydration shells and those outside the shells are not affected by the nature of the ions evidently.
Specifically, Fig.\thinspace\ref{fig:shb_c_ln_bk_Shell_pbc} shows that the relaxation process of the H-bonds between the water molecules in NO$^-_3$-shell and 
those outside the shell is \emph{not} faster than the relaxation process of H-bonds between the water molecules in and outside the \Li-shell. 

Like \CSHB, $C^\text{(k)}_\text{HB}(t)$ shows that the HB dynamics is accelerated to a certain extent. 
%Therefore, the two are not comparable. 
For example, it can be seen from Fig.\thinspace\ref{fig:shb_c_ln_bk_Shell_bulk_wat_pbc_r5} 
that for NO$^-_3$ ion, the $C^\text{(k)}_\text{HB}(t)$ decays much faster than \CHB in pure water.
\begin{figure}[H] 
\centering
\includegraphics [width=0.60\textwidth] {./diagrams/shb_c_ln_bk_Shell_bulk_wat_pbc_r5}
\setlength{\abovecaptionskip}{0pt}
\caption{\label{fig:shb_c_ln_bk_Shell_bulk_wat_pbc_r5}
The $C^\text{k}_\text{HB}(t)$ for the H-bonds in the solvation shell of NO$^-_3$ ion, 
as computed from the (a) ADH and (b) AHD criteria of H-bonds.
The \CHB (dashed line) in the bulk water is also plotted in sub-figure a and b respectively.} 
\end{figure}
%
%\begin{figure}[h]
%\centering
%\includegraphics [width=0.60\textwidth] {./diagrams/shb_log_s_lii_bk_new_LiShell_pbc}
%\setlength{\abovecaptionskip}{0pt}
%\caption{\label{fig:shb_log_s_lii_bk_new_LiShell_pbc}
%The logarithm of the correlation function $S^\text{k,Li}_\text{HB}(t)$ for the solvation shell H-bonds with differnt radius ($r_\text{shell}$), based on water-water 
%pair HB population operator $h^\text{k}(t)$, as computed from the (a) ADH and (b) AHD criteria of H-bonds.} 
%\end{figure}
%%
%\section{Hydrogen Bond Dynamics by Classical Molecular Dynamics Simulations}
%\begin{figure}[H]
%\centering
% \includegraphics [width=0.5\textwidth] {./diagrams/4MPlus-alkali-I_hbacf_C1603}
%\setlength{\abovecaptionskip}{20pt}
%\caption{\label{fig:4MPlus-alkali-I_hbacf_C1603}The function \CHB of water--water hydrogen bonds at interfaces with different alkali metal ions in 4.0 M water solution at 300 K.}
%\end{figure}
%The HB dynamics obtained from classical MD simulations can not catch the fast HB relaxation, and it give a totally different HB dynamics for the water molecules in these alkali halide solution/vapor interfaces.
%
\FloatBarrier
\paragraph{Alkali-iodine solutions}
%To make the results clearer, we only added one cation and one anion to the simulated aqueous system. 
We have done DFTMD simulations for LiI, NaI, KI bulk system and interface system respectively.
%\begin{figure}[h]
%\centering
%\includegraphics [width=0.60\textwidth] {./diagrams/shb_c_lii_itp_Shell_pbc}
%\setlength{\abovecaptionskip}{0pt}
%\caption{\label{fig:shb_c_lii_itp_Shell_pbc}
%The correlation function $C^{k}_{HB}(t)$ for the solvation shell H-bonds (in water/vapor interface) , based on water-water 
%pair HB population operator $h^{k}(t)$, as computed from the (a) ADH and (b) AHD criteria of H-bonds.} 
%\end{figure}
% 
The probability distribution of the ions in the aqueous-vapor interface of LiI and NaI solutions with repect to the depth 
of the ions in the solutions (molar concentration: 0.9 M, temperature: 330 K) is calculated here. 
The distribution indicates that the \I ions prefer to staying at the topmost layer of surface of solutions.
 The probability distribution shows that \I ions tend to the surface of the solutions, while \Na and \Li tend to stay in the bulk. 
This result is consistent with the calculations from Ishiyama and Morita\cite{TI07,Ishiyama2014}.
We choose LiI solution to calculate HB correlation function $C^\text{k}_\text{HB}(t)$,
and the results are shown in Fig.\ref{fig:shb_c_lii_bk_new_Shell_pbc_r5}. 
Like the LiNO$_3$ solution, the relaxation functions $C^\text{(k,Li)}_\text{HB}(t)$ and 
$C^\text{(k,I)}_\text{HB}(t)$ are very close to each other. 
This result shows that the presence of ions has no significant effect on the relaxation of H-bonds outside the first solvation shell.
Because of the definition of $C^\text{(k)}_\text{HB}(t)$, 
we know that some artificial speed-up in the HB dynamics is included in this concepts.


Moreover, the $S^\text{(k,Li)}_\text{HB}(t)$ and $S^\text{(k,I)}_\text{HB}(t)$ have no significant difference. 
This result also implies that \Li ions have not significantly affected the relaxation process of the H-bonds 
between the water molecules in the first solvation shells and those in the outer layer. 
Comparing the results in Fig.\thinspace\ref{fig:vdos_LiNO3-3-5w} in Chapter \ref{CHAPTER_results_clusters},
we can summarize the influence of lithium ions on the dynamics of water molecules in the cluster (solution) as follows:
Although in the water molecule clusters, the vibration frequency of the water molecules directly connected to the \Li ion has a redshift, 
the HB relaxation between the water molecules in the first solvation shell of the ion and those outside the shell 
is almost no longer significantly affected by the ion. 


Also, there are still some differences in the HB kinetics outside the solvation shell of different ions, 
but we cannot infer more qualitative conclusions from this small difference.
 
\begin{figure}[h]
\centering
\includegraphics [width=0.60\textwidth] {./diagrams/shb_c_lii_bk_new_Shell_pbc_r5}
\setlength{\abovecaptionskip}{0pt}
\caption{\label{fig:shb_c_lii_bk_new_Shell_pbc_r5} 
The $C^\text{(k)}_\text{HB}(t)$ for the solvation shell H-bonds with radius ($r_\text{shell}=5.0$ \AA),
as computed from the (a) ADH and (b) AHD criteria of H-bonds. }
% based on water-water pair HB population operator $h^\text{(k)}(t)$, 
%The results are calculated from the simulated bulk LiI solution (new) at $T=300$ K.
\end{figure}
%
\begin{figure}[h]
\centering
\includegraphics [width=0.60\textwidth] {./diagrams/shb_s_lii_itp_Shell_pbc}
\setlength{\abovecaptionskip}{0pt}
\caption{\label{fig:shb_s_lii_itp_Shell_pbc}
The $\ln S^\text{(k)}_\text{HB}(t)$ for H-bonds in the solvation shells of \Li and \I ions (in the water/vapor interface) 
as computed from the (a) ADH and (b) AHD criteria of H-bonds.} 
% based on water-water pair HB population operator $h^\text{(k)}(t)$, 
%The results are calculated from the simulated interface system of LiI solution at $T=300$ K.
\end{figure}
%


%The $\ln S^\text{k,I}_\text{HB}(t)$ (see Fig.\thinspace\ref{fig:shb_log_s_lii_bk_new_IShell_pbc}). 
%\begin{figure}[h]
%\centering
%\includegraphics [width=0.60\textwidth] {./diagrams/shb_log_s_lii_bk_new_IShell_pbc}
%\setlength{\abovecaptionskip}{0pt}
%\caption{\label{fig:shb_log_s_lii_bk_new_IShell_pbc}
%The logarithm of the correlation function $S^\text{k,I}_\text{HB}(t)$ for the solvation shell H-bonds with differnt radius ($r_\text{shell}$), based on water-water 
%pair HB population operator $h^\text{k}(t)$, as computed from the (a) ADH and (b) AHD criteria of H-bonds. The results are calculated from the simulated bulk LiI solution.
%WHICH RESULT IS CORRECT? WE HAVE TO CHECK THE SAME CALCULATION IN INTERFACE, AND IN DIFFERENT TEMPERATURE, AND IN DIFFERENT SOLUTIONS LIKE LiI,NaI,and KI solutions,
%before we obtain the conclusion.} 
%\end{figure}
%
\section{Rotational anisotropy decay of water at the water/vapor interface}\label{RAD}
The effect of ions on the dynamics of water molecules can also be characterized by rotational anisotropy decay of water molecules.
The pump-prob polarization anisotropy monitors electronic alignment.\cite{Jonas96,Farrow08} 
The anisotropy decay can be determined from experimental signal in two different polarization configurations---parallel and perpendicular polarizations, by\cite{Fleming86} 
\begin{equation}
        R(t)=\frac{S_{\parallel}(t)-S_{\perp}(t)}{S_{\parallel}(t)+2S_{\perp}(t)}
\label{eq:ad}
\end{equation}
where $t$ is the time between pump and probe laser pulses. 
Using the transition dipole auto-correlation function, 
we determined the rotational anisotropy decay and therefore the OH-stretch relaxation at the interface of alkali iodide solutions.
The effects of ion environment on structure and dynamics of water are obtained by comparing the second-order Legendre polynomial, 
i.e.,  $P_2(x)=\frac{1}{2}(3x^2-1)$,  orientational correlation function of the transition dipole.
The anisotropy decay can also be obtained by simulations, and calculated by the third-order response functions. \cite{Jansen10,Jansen06}
The orientational anisotropy $C_2(t)$ is given by the rotational time-correlation function 
\begin{equation}
C_2(t)=\langle P_2(\hat{u}(0)\cdot\hat{u}(t)) \rangle,
\label{eq:tcf2}
\end{equation}
where $\hat{u}(t)$ is the time dependent unit vector of the transition dipole, $P_2(x)$ is the second Legendre polynomial, and 
$\langle \rangle$ indicate equilibrium ensemble average.\cite{Corcelli05,LinYS2010} %\cite{2010Lin} % angular brackets
The anisotropy decay $C_2(t)$ for the water/vapor interface is shown in Fig.\space\ref{fig:c2_pure_water_inset}.
The inset shows $C_2(t)$ in the first 1 ps , from which we see a quick change during the first $\sim 0.1$ ps primarily due to libration.
\begin{figure}[h]
\centering
\includegraphics [width=0.36\textwidth] {./diagrams/c2_pure_water_inset} 
\setlength{\abovecaptionskip}{0pt}
  \caption{\label{fig:c2_pure_water_inset}Time dependence of $C_2(t)$ of OH bonds at the water/vapor interface.}
% at 330 K. 
\end{figure}

\paragraph{\LiN interface}
%In the model of the interface, there is one \Li and one \nitrate in the 15.60 \AA$\times$15.60 \AA$\times$31.00 \AA simulation box. 
The anisotropy decay of OH bonds in water molecules in 0.4 M LiNO3 solution/vapor interface is shown in Fig.\thinspace\ref{fig:c2_LiNO3_inset}.  
The larger decay rate consistent obtained from the VDOS of the interfaces, although the concentration of \LiN is lower. 
%This result obtained from another DFTMD trajectory consistent with the previous one, and it 
This result implies that the \nitrate on the surface of the alkali nitrate solution weaken the H-bonds and  accelerate the anisotropy decay of water molecules at the interface.
\begin{figure}[H]
\centering
\includegraphics [width=0.36\textwidth] {./diagrams/c2_LiNO3_inset} 
\setlength{\abovecaptionskip}{10pt}
\caption{\label{fig:c2_LiNO3_inset} Anisotropy decay of OH bonds in water molecules in the interface of the LiNO$_3$ solution.}
\end{figure} 


\paragraph{Alkali-iodine solutions}
The anisotropy decay $C_2(t)$ for the whole slab of the LiI aqueous solution is given in Fig.\thinspace\ref{fig:c2_2LiI_16_inset}.
This function decays faster than that of water/vapor interface, indicating that H-bonds
at the interfaces of alkali-iodine solutions reorient faster than in the water/vapor interface. 
%
\begin{figure}[h]
\centering
\includegraphics [width=0.36\textwidth] {./diagrams/c2_2LiI_16_inset} 
\setlength{\abovecaptionskip}{0pt}
  \caption{\label{fig:c2_2LiI_16_inset}Time dependence of $C_2(t)$ of OH bonds at the interfaces of the LiI solution 
and the water/vapor interface (dashed line).} 
% 0.9 M LiI solution; at 330 K.
\end{figure}
%

We also calculated $C_2(t)$ for the interface of other alkali-iodine solutions NaI and KI. 
The results of $C_2(t)$ for the interfaces of these solutions are shown in Fig.\thinspace\ref{fig:c2_2KI_2NaI_2LiI_16}.
In all the cases $C_2(t)$ decays faster than in neat water, indicating that H-bonds
at the interfaces of the three alkali-iodine solutions are orientated faster than that of the water/vapor interface.
They show that \I ions accelerate the dynamics of molecular reorientation of water molecules at the interface. 
Here, we have considered the water molecules in the entire slab, 
and the calculation results are relatively rough. 
To verify the above conclusions, we need to investigate the dynamic behavior of water molecules located in the solvation shells of ions.   
%Just from Fig.\thinspace\ref{fig:c2_2KI_2NaI_2LiI_16}, we can not infer that the solution with \I ions 
%affect the rotational anisotropy decay of water molecules at interfaces.
\begin{figure}[H] %[htbp]
\centering
\includegraphics [width=0.4 \textwidth] {./diagrams/c2_2KI_2NaI_2LiI_16} 
\setlength{\abovecaptionskip}{0pt}
  \caption{\label{fig:c2_2KI_2NaI_2LiI_16} Time dependence of $C_2(t)$ for OH bonds in water molecules at the interface of the alkali-iodine solutions and of neat water (dashed line).
% at 330 K; 0.9 M
}
%The water/vapor interface is modeled with a slab made of 121 water molecules in a simulation box of size 15.60\AA$\times$15.60\AA$\times$31.00\A. 
\end{figure} 
%

We have obtained non-single-exponential kinetics for the rotation of water molecules both at the surface 
and in bulk water (Appendix \ref{single_exp}).
Therefore, the rotational motion of water molecules are not simply characterized by a well-defined rate constant. 
Similar non-single-exponential kinetics is also obtained in the HB kinetics
in liquid water \cite{AL96,Dirama2005} and in the time variation of the average frequency shifts of the 
remaining modes after excitation in hole burning technique.\cite{Rey2002,Moller2004} 
We can understand the non-single-exponential kinetics of rotational 
anisotropy decay by fitting the rotational anisotropy decay by a 
biexponential function.
Luzar and Chandler interpreted the non-single-exponential kinetics as the result of an interplay between 
diffusion and HB dynamics. \cite{AL96} 
To obtain the effects of diffusion and HB decay of water molecules
in solutions respectively, we assume that there are two independent 
decays in the process of an anisotropy decay. 
The correlation function $C_2(t)$ has the form \cite{TanHS05}
\begin{equation}
C_2(t)=A_1e^{-t/\tau_{2,1}} +A_2e^{-t/\tau_{2,2}},
\label{eq:tcf3}
\end{equation}
where $A_i$ are amplitudes and $1/\tau_{2,i}$ are decay rates ($i=1, 2$) of $C_2(t)$. 
Then, the $\tau_{2,i}$ represent the relaxation times of the $C_2(t)$. 
For alkali-iodine solutions (LiI and NaI), the $A_i$ and $\tau_{2,i}$ of the biexponentials fits for 
$C_2(t)$ are in Table ~\ref{tab:2LiI_c2_biexp}.
The biexponential fit is very close to the calculated $C_2(t)$. 
As an example, for the interface of LiI solution, we made a comparison between $C_2(t)$ 
and the function fitted according to Eq.\thinspace\ref{eq:tcf3},
as shown in Fig.\thinspace\ref{fig:2LiI-124w_c2_fit_5ps_biexp}. 
We found that Eq.\thinspace\ref{eq:tcf3}, the biexponential kinetics is sufficiently accurate to 
describe the attenuation characteristics of $C_2(t)$.                        
%(compare Fig.\space\ref{fig:2LiI-124w_c2_fit_5_single-exp}).
\begin{table}[H]%[hbt]
\centering
\caption{\label{tab:2LiI_c2_biexp}%
	Biexponential fitting (5 ps) of $C_2(t)$ for water molecules in the LiI (NaI) solution.}
\begin{tabular}{lccccc}
water molecules & $A_1$  & $\tau_{2,1}$ (ps) & $A_2$ & $\tau_{2,2}$ (ps) \\
\hline
%I$^-$-shell & 0.44 & 4.00 & 0.39 & 3.85\\
%bulk & 0.84 & 9.09 & 0.09 & 0.20 \\
%surface & 0.73 & 3.70 & 0.22 & 0.07 \\
  I$^-$-shell & 0.86 & 7.14 & 0.08 & 0.10 \\
  surface & 0.77 & 9.09 & 0.13 & 0.43 \\
  Li$^+$-shell & 0.88 & 14.29 & 0.07 & 0.12\\
  Na$^+$-shell & 0.71 & 16.67 & 0.18 & 1.27 \\
  bulk & 0.81 & 16.67 & 0.10 & 0.80 \\
\end{tabular}
\end{table}
% 0.9 M --
%

%\begin{table}[H]
%\centering
%  \caption{\label{tab:2NaI_c2_biexp}%
%	Biexponential fitting (5 ps) of the $C_2(t)$ for water molecules in 0.9 M NaI solution.}
%  \begin{tabular}{lccccc}
%  water molecules & $A_1$  & $\tau_{2,1}$ (ps) & $A_2$ & $\tau_{2,2}$ (ps) \\
%  \hline
%  I$^-$-shell & 0.86 & 7.14 & 0.08 & 0.10 \\
%  Na$^+$-shell & 0.71 & 16.67 & 0.18 & 1.27 \\
%  bulk & 0.81 & 16.67 & 0.10 & 0.80 \\
%  surface & 0.77 & 9.09 & 0.13 & 0.43 \\
%  \end{tabular}
%\end{table}
%
%
\begin{figure}[H]%[htbp]
\centering
\includegraphics [width= 0.6\textwidth] {./diagrams/2LiI-124w_c2_fit_5_biexp} 
  \caption{\label{fig:2LiI-124w_c2_fit_5ps_biexp} Time dependence of the $C_2(t)$ for OH bonds 
  in water molecules at the interface of LiI solution.}
\end{figure} 
%[Notes: The 63-water-slab models is listed here as a reference. The number of water molecules is small; The data for KI/vapor and LiI/vapor interfaces come from  KI\_16 and LiI\_16 systems.  
%Water(63) &0.831$\pm(1\times10^{-4})$ &  0.08760 $\pm(2\times 10^{-5})$&0.100$\pm(2\times10^{-4})$ & 1.029 $\pm(4\times10^{-3})$  \\ ]
%
%\begin{figure}[htbp]
%\centering
%\includegraphics [width=0.4 \textwidth] {./diagrams/c2_121-pure_2KI_2LiI_16_inset_fit_biexp} 
%\setlength{\abovecaptionskip}{10pt}
%\caption{\label{fig:c2_121-pure_2KI_2LiI_16_inset_fit_biexp} The fitted and calculated anisotropy decay of OH bonds in water molecules in LiI solution/vapor interface (red), LiI solution/vapor interface (blue) and neat water/vapor interface (black). The corresponding fitted functions are denoted by dashed lines. The concentration of LiI and KI solution is 0.9 M.}
%\end{figure} 

Then we considered the effect of ion species in solutions on the anisotropy decay of water molecules.
From Table \ref{tab:2LiI_c2_biexp}, we found that for both LiI and NaI solutions, there are two decay processes --- amplitude $\sim 1$,
decay constant $1/\tau_{2,2}\sim$ 0.1 THz, and for the other describe the initial fast decay 
of the anisotropy, with amplitude $\sim 0.1$, decay constant $1/\tau_{2,1}\sim$ 1--10 THz, 
due to the inertial--librational motion preceding the orientational diffusion.
Two decay processes can be found in the dynamics of water molecules 
at the interfaces of alkali-iodine solutions from the DFTMD simulations.
%The one describe the initial fast decay of the anisotropy, 
%with amplitude $\sim$ 0.1, decay constant $\sim$ (1--10) THz,
%results from the inertial-librational motion preceding the orientational diffusion.
%
\begin{table}[H] %[!hbtp]
\centering
\caption{\label{tab:table_CoordNo}%
The coordination number for the ions in LiI (NaI) solution.}
\begin{tabular}{lccc}
name & $r_\text{shell}$ (\AA) & coordination number \\
\hline
$n_\text{Li-O}$ & 3.0 & 4.0 \\
$n_\text{I-H}$ & 3.3 & 5.5 \\
%$n_\text{I-H}(\text{NaI)}$ & 3.3 & 5.1 \\
%$n_\text{I-O}(\text{NaI)}$ & 4.3 & 6.0 \\
$n_\text{Na-O}$ & 3.5 & 6.0 \\ 
$n_\text{I-O}$ & 4.3 & 5.8 \\
\end{tabular}
\end{table}


%{Fitting by a biexponential}
The relaxation time constants and amplitudes of the biexponentials fits (10 ps) for $C_2(t)$ are in table \ref{tab:table8}.
\begin{table}[H]  % or [!htbp]
\centering
\caption{\label{tab:table8}%
The fitted parameters of anisotropy decay of water molecules in LiI (NaI, KI) solutions. The constants and amplitudes comes from fitting ([0,10] ps).}
\begin{tabular}{lccccc}
Water molecules & $A_1$  & $\tau_{2,1}$ (ps) & $A_2$ & $\tau_{2,2}$ (ps) \\
\hline
%\I-shell (LiI)  &0.45 & 3.23  & 0.45 & 3.23\\
\Li-shell & 0.56 & 8.33 & 0.33 & 50.00  \\
%bulk (LiI) &0.43  & 9.09 & 0.43 & 8.33 \\
%surface (LiI) & 0.41 & 2.86 & 0.40  & 2.78 \\
\I-shell &0.86 & 7.14 & 0.08 & 0.10 \\
\Na-shell & 0.71 & 16.67 & 0.185 & 1.27 \\
Bulk water & 0.60 & 13.51 & 0.31  & 3.45 \\
%bulk (NaI) & 0.81 & 16.67 & 0.099  & 0.80  \\
LiI/vapor & 0.84 & 4.89 & 0.09 & 0.57 \\ 
NaI/vapor & 0.77 & 9.09 & 0.13 & 0.43 \\
KI/vapor & 0.80 & 6.06 & 0.12 & 1.13 \\
\end{tabular}
\label{tab:biexponential1}
\end{table}

In conclusion, two decay processes exist in the dynamics of water molecules in the alkali halide solution/vapor interface.
One has the amplitude $\sim$ 1, decay constant $\sim$ 0.1 THz, which describes the initial fast decay of the anisotropy, 
and the other is of amplitude $\sim$ 0.1, decay constant $\sim$ 10 THz, due to the inertial-librational motion preceding the orientational diffusion.

%\begin{table} [H] %[h!]
%\centering
%\caption{\label{tab:table_2KI_2LiI_anisotropy_decay}%
%The fitted parameters of anisotropy decay of water molecules in LiI (KI) solution/vapor interfaces and neat water/vapor interface. 
%The constants and amplitudes comes from fitting ([0,20]ps).}
%% Notes: The 63-water-slab models is listed here as a reference. The number of water molecules is small; The data for KI/vapor and LiI/vapor interfaces come from  KI\_16 and LiI\_16 systems.  
%\begin{tabular}{lcccc}
%Interfaces & $A_1$  & $\tau_{2,1}$ (ps) & $A_2$ & $\tau_{2,2}$ (ps) \\
%\hline
%Water(63) & 0.831 & 12.03 & 0.100 & 0.97 \\
%Water & 0.595 & 13.51 & 0.310  & 3.45 \\
%KI/Vapor & 0.797 & 6.06 & 0.122 & 1.13 \\
%LiI/Vapor & 0.836 & 4.89 & 0.091 & 0.57 \\ 
%\end{tabular}
%\label{tab:biexponential2}
%\end{table}

\FloatBarrier
\section{Rotational anisotropy decay in solvation shell of ions}\label{RAD_SHELL}
In this paragraph, we will answer the question that whether the water molecules in the first solvation hydration shell of different ions have different orientation dynamics. 
In addition to the calculation for the alkali-nitrate solution, we also calculated the anisotropic decay in the ion solvation shell of the alkali-iodide solution.
\paragraph{Alkali-nitrate solution}
We calculated the anisotropy decay of water molecules in \Li and nitrate ions' hydration shells in the LiNO$_3$ interface. 
The calculated average of $C_2(t)$ is shown in Fig.\thinspace\ref{fig:C2_ln_itp_pbc}. 
It is obtained by averaging the $C(t)$'s for 6 different trajectories.
The radius of the hydration shell of nitrate O, \Li, and water molecules are taken as 4.0, 2.8 and 3.5 \AA, respectively. 
These values come from the RDFs $g_{\text{N-OW}}(r)$, $g_{\text{Li-OW}}(r)$ and $g_{\text{OW-OW}}(r)$,
for bulk alkali nitrate solution, as shown in Fig.\thinspace\ref{fig:gdr_127_LiNO3} a and Fig.\thinspace\ref{fig:gdr_NW_WW_127_LiNO3}.
In view of the diffusion of molecules, we only counts trajectories with a duration of 10 ps. 
\begin{figure}[H]
\centering
\includegraphics [width=0.42\textwidth] {./diagrams/C2_ln_itp_pbc} 
\setlength{\abovecaptionskip}{0pt}
\caption{\label{fig:C2_ln_itp_pbc}The $C_2(t)$ for water in the solvation shell of water, \Li and nitrate ions in the interface system of \LiN solution.}
\end{figure} % 300 K
\begin{figure}[H]
\centering
\includegraphics [width=0.8\textwidth] {./diagrams/gdr_NW_WW_127_LiNO3} 
\setlength{\abovecaptionskip}{0pt}
\caption{\label{fig:gdr_NW_WW_127_LiNO3}RDFs for the \LiN solution.}% at $T=300$ K.
\end{figure}

For the interface of the NaNO$_3$ solution, the average values of the $C_2(t)$ are shown in 
Fig.\thinspace\ref{fig:C2_nn_itp_pbc}. 
The radius of the hydration shell of nitrate O, \Na, and water molecules are taken as 4.0, 3.2 and 3.5 \AA, respectively. 
(These values come from the radial distribution functions $g_{\text{N-OW}}(r)$, $g_{\text{Na-OW}}(r)$ and $g_{\text{OW-OW}}(r)$,
for bulk alkali nitrate solution, as shown in Fig.\thinspace\ref{fig:gdr_127_LiNO3} a and Fig.\thinspace\ref{fig:gdr_NW_WW_127_NaNO3}.)
\begin{figure}[H]
\centering
\includegraphics [width=0.42\textwidth] {./diagrams/C2_nn_itp_pbc} 
\setlength{\abovecaptionskip}{0pt}
\caption{\label{fig:C2_nn_itp_pbc}The $C_2(t)$ for water in the solvation shell of water, Na$^+$ and nitrate ions in the interface system of NaNO$_3$ solution.}
\end{figure}
\begin{figure}[H]
\centering
\includegraphics [width=0.8\textwidth] {./diagrams/gdr_NW_WW_127_NaNO3} 
\setlength{\abovecaptionskip}{0pt}
\caption{\label{fig:gdr_NW_WW_127_NaNO3}RDFs for the NaNO$_3$ solution.}
\end{figure} % 300 K

Moreover, we also did the same calculation of $C_2(t)$ for the interface of the KNO$_3$ solution. 
The average values of $C_2(t)$ are shown in Fig.\thinspace\ref{fig:C2_kn_itp_pbc}.
The radius of the hydration shell of nitrate O, K$^+$, and water molecules are taken as 4.0, 3.6 and 3.5 \AA, respectively. 
(These values come from the RDFs $g_{\text{N-OW}}(r)$, $g_{\text{K-OW}}(r)$ and $g_{\text{OW-OW}}(r)$,
for bulk alkali nitrate solution, as shown in Fig.\thinspace\ref{fig:gdr_127_LiNO3} a and Fig.\thinspace\ref{fig:gdr_NW_WW_127_KNO3}.)
\begin{figure}[H]
\centering
\includegraphics [width=0.42\textwidth] {./diagrams/C2_kn_itp_pbc} 
\setlength{\abovecaptionskip}{0pt}
\caption{\label{fig:C2_kn_itp_pbc}The $C_2(t)$ for water in the solvation shell of water, K$^+$ and nitrate ions in the interface system of KNO$_3$ solution.}
\end{figure} % 300 K
\begin{figure}[H]
\centering
\includegraphics [width=0.8\textwidth] {./diagrams/gdr_NW_WW_127_KNO3} 
\setlength{\abovecaptionskip}{0pt}
\caption{\label{fig:gdr_NW_WW_127_KNO3}RDFs for the KNO$_3$ solution.} % 300 K
\end{figure}

From the above calculation of $C_2(t)$, we found that in the interfacial systems of alkali metal nitrate solution, 
nitrate ions always accelerate the reorientation dynamics of water molecules in the hydration shells of nitrate ions.
However, the reorientation dynamics of water molecules in the hydration shells of alkali metal ions may slow down 
(for the LiNO$_3$ and NaNO$_3$ solutions, respectively) or accelerate (for the KNO$_3$ solution), 
due to the presence of alkali metal ions. 

We find a relation between the reorientation relaxation time and the radius of hydration shell.
For the interface systems of the above three nitrate solutions, the decay rate of the orientation correlation function 
is positively correlated with the radius of the particle's dissolved sphere. For simplicity, 
we assume that the anisotropy decay is a single exponential given by 
\begin{equation}
C_2(t)=e^{-t/\tau_2}\nonumber,
\label{eq:tcf2}
\end{equation}
where $\tau_2$ is a characteristic relaxation time for the anisotropy decay.
We use this exponential decay function to fit the $C_2(t)$ 
and to calculate the $\tau_2$ value corresponding to the solvation shell of different particles. 
We list the calculated values of $\tau_2$ in Table \ref{tab:relaxation_tau_vs_radius_ln}--\ref{tab:relaxation_tau_vs_radius_kn}. 
It can be seen that the reorientation relaxation time $\tau_2$ of the water molecules in the hydration shell increases with 
the increase of the radius of the hydration shell, and their relationship is shown in Fig.\thinspace\ref{fig:ln_nn_kn_tau2_vs_shell_radius}.
%We also give a linear regression function (dashed line) to fit this relation, obtained from the data from alkali nitrate solutions.
\begin{table}[H]
\centering
\caption{\label{tab:relaxation_tau_vs_radius_ln} 
    The radius $r$ of hydration shells and corresponding relaxation times $\tau_2$ at the interface of LiNO$_3$ solution.} 
\begin{tabular}{ccc}
 ion (molecule) & $r$ (\AA) & $\tau_2$ (ps)  \\
\hline
  \wat & 3.5 & 6.48 $\pm$ 0.02 \\
  \Li & 2.8 & 9.26 $\pm$ 0.02 \\
  NO$^-_3$ & 4.0 & 5.30 $\pm$ 0.02 \\
\end{tabular}
\end{table} % 300 K
\begin{table}[H]
\centering
\caption{\label{tab:relaxation_tau_vs_radius_nn} 
    The radius $r$ of hydration shells and corresponding relaxation times $\tau_2$ at the interface of NaNO$_3$ solution.} 
\begin{tabular}{ccc}
 ion (molecule) & $r$ (\AA) & $\tau_2$ (ps)  \\
\hline
  \wat & 3.5 & 7.12 $\pm$ 0.02  \\
  \Na & 3.2 & 9.24 $\pm$ 0.02 \\
  NO$^-_3$ & 4.0 & 5.66 $\pm$ 0.01 \\
\end{tabular}
\end{table} % 300 K
\begin{table}[H]
\centering
\caption{\label{tab:relaxation_tau_vs_radius_kn} 
    The radius $r$ of hydration shells and corresponding relaxation times $\tau_2$ at the interface of KNO$_3$ solution.} 
\begin{tabular}{ccc}
 ion (molecule) & $r$ (\AA) & $\tau_2$ (ps)  \\
\hline
  \wat & 3.5 & 7.04 $\pm$ 0.02  \\
  \K & 3.6 & 6.49 $\pm$ 0.02 \\
  NO$^-_3$ & 4.0 & 5.30 $\pm$ 0.01 \\
\end{tabular}
\end{table} % 300 K
\begin{figure}[H]
\centering
\includegraphics [width=0.42\textwidth] {./diagrams/ln_nn_kn_tau2_vs_shell_radius} 
\setlength{\abovecaptionskip}{0pt}
\caption{\label{fig:ln_nn_kn_tau2_vs_shell_radius}Dependence of $\tau_2(t)$ on the radius of the solvation shell of molecules 
(water, \Li, \Na, \K and nitrate ions) in the slab of alkali nitrate solutions.
%The linear regression line (dashed line) obtained by the data in Table 
%\ref{tab:relaxation_tau_vs_radius_ln}--\ref{tab:relaxation_tau_vs_radius_kn} is
% $\tau_2 = -2.15 r + 14.76$. 
}
\end{figure} % 300 K

\paragraph{Alkali-iodine solutions}
The time correlation functions for water molecules bound to specific ions, for selected frequency windows up to 2 ps, 
are shown in Fig.\thinspace\ref{fig:2LiI-124w_0-25ps_c2_150222b_s2}.
\begin{figure}[H]
\centering
\includegraphics [width=0.4 \textwidth] {./diagrams/2LiI-124w_0-25ps_c2_150222b_s2} 
\caption{\label{fig:2LiI-124w_0-25ps_c2_150222b_s2}Anisotropy decay of OH bonds in water molecules in NaI and LiI solutions.
}
\end{figure} 
It shows that water molecules bound to \I anions decay fastest,while those bound to \Na slowest. 
The larger correlation time indicates that the water molecules are held more rigidly within the 1st hydration shell of \Li ions. 
In other words, the order of the rigidity of the solvation shell is: \Li > \Na > \I.
%
%We obtain non single-exponential kinetics for the rotation of water molecules in both surface and bulk water (and this is true for water molecules bound to ions).
%Therefore, the rotational motion of water molecules are not simply characterized by a well-defined rate constant (see Fig.\thinspace\ref{fig:2NaI-124w_c2_fit_150223}).
%The similar non-exponential kinetics is also obtain in the HB dynamics in liquid water.\cite{AL96,AL96b,Dirama2005} 
%Luzar and Chandler interpreted the non-exponential kinetics as the result of an interplay between diffusion and HB dynamics \cite{AL96}. 
%Here, we try to understand the non-exponential kinetics of rotational anisotropy decay by the model in Eq.\thinspace\ref{eq:tcf3}.
%
\begin{table}[h!]
\centering
\caption{\label{tab:table_expfit}%
The fitted parameters of anisotropy decay of water molecules in LiI (NaI) solutions. The decay rate $\kappa$ comes from fitting ([0,10] ps).}
\begin{tabular}{lccc}
Water molecules & $c$  & $\tau_2$ (ps) \\
\hline
\I-shell (LiI) & 0.82 & 4.2 \\
\Li-shell (LiI) & 0.88 & 14.3 \\
%bulk (LiI) & 0.85 & 8.4\\
%surface (LiI) & 0.81 & 2.9  \\
\I-shell (NaI) & 0.86 & 7.1 \\
\Na-shell (NaI) & 0.79 & 14.3 \\
%bulk (NaI) & 0.83 & 16.7 \\
%surface (NaI) & 0.78 & 8.4 \\
\end{tabular}
\end{table}
%
%\begin{figure}
%\centering
%\includegraphics [width=0.4 \textwidth] {./diagrams/c2_hal_sh1_s} 
%\setlength{\abovecaptionskip}{10pt}
%\caption{\label{fig:c2_hal_sh1_s}
%The anisotropy decay of OH chromophores in water molecules localized in the first hydration shell of \I ions in LiI water solutions (0.9 M; 20 ps).}
%\end{figure} 
%\begin{figure}[H]
%\centering
%\includegraphics [width=0.4 \textwidth] {./diagrams/124_2NaI_c2_comp_na_bulk} 
%\setlength{\abovecaptionskip}{10pt}
%\caption{\label{fig:124_2NaI_c2_comp_na_bulk}
%The anisotropy decay of OH chromophores in water molecules localized in the first hydration shell of \Na ions in NaI water solutions (0.9 M);} 
%\end{figure}
%
%\paragraph{Aqueous/Vapor Interface of LiI(NaI) Solution}
%Here shows the simulated SFG spectrum of the aqueous-vapor interface of LiI solutions at 330 K.
%The spectrum are calculated from the average of two LiI solution-vapor interfaces, which are under the same condition.
%The simulation box is with the size of 15.6 \AA$ \times$15.6 \AA$ \times$31.0 \A. Half of the volume of the simulation box is vacuum. 
%There are two \Li cations and two \I anions in the solution part of the simulation box.
%The molar concentration of the solution is $c_{\text{LiI}}=0.9 \times 10^3 \text{ mol}/\text{m}^3$.
%For both LiI and NaI solutions, the Im$\chi^{(2)}$ spectrum shows a sharp peek around 3700 cm$^{-1}$, which associated the free OH stretch (or dangling OH stretch).
%
%\begin{figure}
%\includegraphics [width=1.0 \textwidth] {./diagrams/2LiI-2NaI-124w_gdr_OH_150122} % Here is how to import EPS art
%\caption{\label{fig:rdf_OH_different_layers} The RDF $g_{\text{O-H}}(r)$ in the surfaces of the aqueous-vapor interfaces of solutions (molar concentration: 0.9 M) at 330 K. Solid and dashed lines
%are corresponding to the two interface in our simulation, respectively. The simulation time is 22.5ps. 
%The result for two interface consistent to each other when the interface is large enough, i.e., the thickness is 6 \A.
%(a) LiI solution; (b) NaI solution. }
%\end{figure} 
%Fig.~\ref{fig:rdf_OH_different_layers} shows that when the thickness of the layer is 8 \A, the structure of the surface does not change. 
%This result is the same for both aqueous-vapor interface of NaI solutions (molar concentration: 0.9 M) at 330 K. 
%As the thickness increases, the coordination number of water O atoms is increasing from about 3 to 5.
%
%\begin{figure}
%\includegraphics [width=0.6 \textwidth] {./diagrams/2LiI-124w_c2_fit_150223} 
%\caption{\label{fig:2LiI-124w_c2_fit_150223} The anisotropy decay of OH chromophores in water molecules in LiI water solutions (time range: 0-10ps)(fit in [0, 10] ps).}
%\end{figure} 
%\begin{table}
%%\caption{\label{tab:table6}%
%\caption{\label{tab:table_alkali_halide_1}%
%The fitted parameters of anisotropy decay of water molecules in LiI (NaI) solutions. 
%The decay rate $\kappa$ comes from fitting.}
%\begin{tabular}{lccc}
%Water molecules& Period $T$ (ps) & Decay rate $\kappa$ (THz) \\
%\hline
%\I-shell (LiI) & 3.5 & 0.29 \\
%\Li-shell (LiI) & 12.3 & 0.08 \\
%\I-shell (NaI) &10.7 & 0.09 \\
%\Na-shell (NaI) & 14.8 & 0.07 \\
%\end{tabular}
%\end{table}
%---------------------------------------------------
%\paragraph{Fitting by a biexponential}
\begin{table}[H]
\centering
\caption{\label{tab:table_center_ion}%
Fitted parameters of anisotropy decay of water molecules in LiI (NaI) solutions. The constants and amplitudes comes from fitting ([0,10] ps).}
%\begin{ruledtabular}
\begin{tabular}{lccccc}
central ions & $A_1$  & $\kappa_1$ (THz) & $A_2$ & $\kappa_2$ (THz) \\
\hline
\Li & 0.56 & 0.12 & 0.33 & 0.02  \\
\Na & 0.71 & 0.06 & 0.185 & 0.79 \\
\I & 0.86 & 0.14 & 0.08 & 9.86 \\
\end{tabular}
\label{biexponential}
%\end{ruledtabular}
\end{table}
%--------------------
%\begin{table}[H]
%\centering
%\caption{\label{tab:table_surf-bulk}%
%[MAY DELETE]The fitted parameters of anisotropy decay of bulk and surfacial water molecules in LiI (NaI) solutions. The constants and amplitudes comes from fitting ([0,10]ps).}
%%\begin{ruledtabular}
%\begin{tabular}{lccccc}
%Water molecules & $A_1$  & $\kappa_1$ (THz) & $A_2$ & $\kappa_2$ (THz) \\
%\hline
%bulk (LiI) & 0.43 & 0.11 & 0.43 & 0.12 \\
%surface (LiI) & 0.41 & 0.35 & 0.40 & 0.36 \\
%bulk (NaI) & 0.81 & 0.06 & 0.099 & 1.25 \\
%surface (NaI) & 0.77 & 0.11 & 0.126 & 2.31 \\
%\end{tabular}
%\label{biexponential}
%%\end{ruledtabular}
%\end{table}
\begin{figure}[H]
\centering
\includegraphics [width=0.6 \textwidth] {./diagrams/2NaI-124w_c2_fit_150223} 
\setlength{\abovecaptionskip}{10pt}
\caption{\label{fig:2NaI-124w_c2_fit_150223}Anisotropy decay of OH bonds in water molecules in NaI solutions.}
\end{figure} 
\begin{figure}[H]%[!htbp]
\centering
\includegraphics [width=0.6 \textwidth] {./diagrams/2NaI-124w_c2_fit_biexp_150310} 
\caption{\label{fig:2NaI-124w_c2_fit_biexp_150310} The anisotropy decay of OH groups in water molecules in NaI water solutions can be fit 
(dashed lines) very well by a biexponential in [0, 10] ps).
}
\end{figure} 

From Fig.\thinspace\ref{fig:2NaI-124w_c2_fit_150223} and Fig.\thinspace\ref{fig:2NaI-124w_c2_fit_biexp_150310}, 
we found that biexponential fits better than single exponential does. 
Two decay processes exist in the dynamics: amplitude $\sim$ 1,
decay constant $\sim$ 0.1 THz, and for the other describe the initial fast decay of the anisotropy, with amplitude $\sim$ 0.1, decay constant $\sim$ (1 $\sim$ 10) THz, 
due to the inertial-librational motion preceding the orientational diffusion.

\FloatBarrier
\paragraph{Classification of water molecules based on H-bonds}
We also studied the relation between the anisotropy decay of water molecules and their environment. 
Following the definition used in Ref.\cite{TianCS08}, we use the following labels to denote water molecules in solution: 
D denotes that the water molecule donates a HB, D$'$ donates a H-I bond, and A accepts a HB. \cite{TianCS08} 
DDAA represents a water molecule with two H-Bonds donated to water molecules and two H-bonds accepted from water molecules (see Fig.\thinspace\ref{fig:Multiple_figs} a);
DD$'$AA represents a water molecule with two H-bonds donated to a water molecule and \I (see Fig.\thinspace\ref{fig:Multiple_figs} b), 
and with two H-bonds accepted from other water molecules (see Fig.\thinspace\ref{fig:Multiple_figs} c), 
D$'$AA represents a water molecule bonded to \I at the water/vapor interface and other H-bonds to water molecules (see Fig.\thinspace\ref{fig:Multiple_figs} d).
Clearly, we found that D$'$AA molecules are of free OH stretching during the dynamics. 
% 
\begin{figure}[ht]%[!htbp]
\centering
\includegraphics [width=0.4 \textwidth] {./diagrams/Multiple_figs} 
\caption{\label{fig:Multiple_figs} Four types of water molecules at the interfaces of LiI solution, regarding the HB environments: (a) DDAA; (b) DDA; (c) DD$'$AA; (d) D$'$AA. The cyan balls denote \I ions. }
\end{figure} 

It is evident that $C_2(t)$ for DDAA and DD$'$AA molecules do not decay exponentially (Fig.\thinspace\ref{fig:2LiI-124w_c2_fit_biexp_7wat_2ps_class_150324}
and Table \ref{tab:fitting_c2_for_each_type_of_water}).
%[BUT Table \ref{tab:fitting_c2_for_each_type_of_water} CAN NOT GIVE THE EVIDENCE. STH. IS MISSING!] 
This result is similar to the reactive flux function $k(t)$, i.e., 
the escaping rate kinetics of H-bonds in bulk water. \cite{Luzar1996} 
The relaxation of H-bonds in water appears complicated, with no simple characterization in terms of a few relaxation rate constants. 
Most of the authors believe that the cooperation between neighbouring H-bonds, \cite{Sciortino1989, Ohmine1995} or 
self evident coupling between translational diffusion and HB dynamics is the source of the complexity. \cite{Luzar1996} 
However, for D$'$AA molecules at the interface of the LiI solution,
the $C_2(t)$ decays exponentially, i.e.
\begin{eqnarray}
  C_2(t) &=& C e^{-t/{\tau_2}},\nonumber
\end{eqnarray}
where the amplitude is $C=0.76$, and the reorientation rate is $1/\tau_2 = 7.14$ ps$^{-1}$.
The single exponential decay of $C_2(t)$ for D$'$AA molecules, indicates that each D$'$AA  molecule reorientate independently to each other. 

Furthermore, the $C_2(t)$ for D$'$AA molecules decays much faster than that for DDAA or DD$'$AA molecules.
From the definitions, the D$'$AA water molecule accepts two H-bonds and donates only one H-bond, 
while both DDAA and DD$'$AA water molecules own four H-bonds.
Therefore, the correlation between H-bonds around the D$'$AA molecule is weaker than those around the DDAA or DD$'$AA molecule. 
Faster decay of $C_2(t)$ for D$'$AA molecules shows that the reorientation process of D$'$AA
molecules is faster than water molecules in bulk phase, e.g., the DDAA, and DD$'$AA molecules.

Additionally, D$'$AA water molecule has a free OH chemical bond, which can stretch and vibrate freely. 
This feature is not available in other types of molecules such as DDAA, DD$'$A, DD$'$AA, etc. 
Therefore, on the surface of the solution, the most closely related feature of the molecular orientation relaxation process is 
the number of H-bonds donated by water molecules. Besides, the type of HB (water--water, or ion--water HB) 
also affects the orientation relaxation process. The number of H-bonds accepted by water molecules has no major effect on the orientation relaxation 
of the water molecules at the interface.
%D'AA 水分子有一个自由的OH化学键,它可以自由地伸缩振动。这个特征是其他类型的分子如DDAA,DD'A, DD'AA等所没有的。因此在溶液表面,与分子取向弛豫过程的快慢关系最密切的特征就是水分子贡献氢键的个数。其次氢键的类型(水与水之氢键,水与离子的氢键等)也影响水分子的取向弛豫过程。水分子所接受的氢键的个数则对界面水分子的取向弛豫不起主要作用。

Finally, for D$'$AA molecules, the inertial-librational motion can not be seen (Fig.\thinspace\ref{fig:2LiI-124w_c2_fit_biexp_7wat_2ps_class_150324} 
and Fig.\thinspace\ref{fig:2LiI-124w_c2_fit_biexp_7wat_2ps_2021}). 
This result implies that the rotational anisotropy decay of D$'$AA molecules
are of the same time scale of the inertial libration, i.e., $\sim$ 0.2 ps. 
This conclusion can be verified from the value of $\tau_{2,1}$ in Table \ref{tab:fitting_c2_for_each_type_of_water}: $\tau_{2,1}=0.14$ for D$'$AA, 
which is smaller than the $\tau_{2,1}$ for DDAA, DD$'$A, and DD$'$AA.

Rotational anisotropy decay of water molecules is found at the interface of LiI solution. 
The result comes from a different HB types from the usual DDAA HB type in bulk water.
The faster anisotropy decay for D$'$AA molecules reflects the less correlation between different H-bonds for D$'$AA molecules, 
which comes from Hydrogen--Iodide bond at the interfaces, the existence of free OH stretching.
From Fig.\space\ref{fig:prob_124_LiI_double_axis}, we have known that in the LiI solution, 
\I ions prefer to locate at the interface.  
Therefore, we infer that the reduction of the inter-correlations between H-bonds occurs at the interface. 

%[RIGHT?]
Slower rotational anisotropy decay exists for water molecules  at the interface of the alkali-iodide solutions, 
which is the result of a different H-Bond types (D$'$AA) from DDAA-type molecules in bulk water. 
The slowing down of anisotropy decay is the effect of Hydrogen-Iodide bonds at the interface. 
Since the iodide's surface propensity is high, this difference of HB structure from the water/vapor interface changed 
the Im$\chi^{(2)}$ spectrum and the total HB dynamics of the interface of alkali-iodide solutions.  


%
In conclusion, single exponential type rotational anisotropy decay exists for water molecules at the interface of the alkali-iodine solutions,
and this faster anisotropy decay of water molecules at the interface is the effects of free OH stretch and Hydrogen--Iodide (H--I) bond at the interface, i.e., 
the D$'$AA molecules. 
Since the iodide's surface propensity is high, this difference of HB structure 
from the water/vapor interface is the source of 
the HB dynamics as well as the Im$\chi^{(2)}$ spectrum of the interface of alkali-iodine solutions.  
We believe that using similar classification method for water molecules, on the basis of AIMD simulations, 
the orientation correlation function $C_2(t)$ of water molecules in other kinds of solution interfaces can also be analyzed.
The effects of Hydrogen-Iodide bond on the HB dynamics at the interfaces, 
and the relation between the interfacial HB dynamics and rotational anisotropy decay can also be studied in the future.
%我们相信,利用这种对水分子的的分类方法,在AIMD模拟的基础上,也可其他种类的溶液界面中的水分子取向关联函数作分析。
%图
\begin{figure}[H] %[!htbp]
\centering
\includegraphics [width=0.36 \textwidth] {./diagrams/2LiI-124w_c2_fit_biexp_7wat_2ps_class_150324} 
\caption{\label{fig:2LiI-124w_c2_fit_biexp_7wat_2ps_class_150324} Time dependence of $C_2(t)$ for water molecules in different HB environments at the interface of LiI solution.}
\end{figure}  
\begin{figure}[H] %[!htbp]
\centering
\includegraphics [width=0.36 \textwidth] {./diagrams/2LiI-124w_c2_fit_biexp_7wat_2ps_2021} 
\caption{\label{fig:2LiI-124w_c2_fit_biexp_7wat_2ps_2021} Time dependence of the $C_2(t)$ for DD$'$A, DD$'$AA, and D$'$AA water molecules at the water/vapor interface of LiI solution.}
\end{figure}  
\begin{table}[H]
\centering
\caption{\label{tab:fitting_c2_for_each_type_of_water}%
	Biexponential fitting (2 ps) of $C_2(t)$ for water molecules in 0.9 M LiI solution. 
The relative standard errors: $\Delta A_i/A_i \le 10^{-2}$, $\Delta \tau_{2,i}/\tau_{2,i} \le 3\times 10^{-2}$.}
%\begin{ruledtabular}
\begin{tabular}{lccccc}
water molecules & $A_1$  & $\tau_{2,1}$ (ps) & $A_2$ & $\tau_{2,2}$ (ps) \\
\hline
DD$'$A & 0.90 & 16.5 & 0.12 & 0.01 \\
DD$'$AA & 0.88 & 5.09 & 0.08 & 0.07 \\
%DD$'$AA & 0.89 & 6.90 & 0.09 & 0.07 \\
%DD$'$AA & 0.88 & 6.90 & 0.06 & 0.07 \\
%DD$'$AA & 0.86 & 4.47 & 0.09 & 0.07 \\
DDAA & 0.85 & 3.98 & 0.10 & 0.06 \\
D$'$AA & 0.38 & 0.14 & 0.38 & 0.14 \\
\end{tabular}
%\end{ruledtabular}
\end{table}
%
%\subsection{\LiN Solution/vapor Interface}
%The anisotropy decay of OH bonds in water molecules in 0.4 M LiNO3 solution/vapor interface is shown in Fig.\space\ref{fig:c2_LiNO3_inset}.  In the model of the interface, there is one \Li and one \nitrate in the 15.6 \AA$\times$15.6 \AA$\times$31.0 \AA simulation box. 
%The larger decay rate consistent to the conclusion infered from the VDOS for the interfaces, although the concentration of \LiN is lower. This result obtained from another DFTMD trajectory consistent with the previous one, and it reflects that the \nitrate on the surface of the alkali nitrate solution weaken the H-bonds and  accelerate the anisotropy decay of water molecules at the interfaces.
%\begin{figure}[htbp]
%\centering
%\includegraphics [width=0.4\textwidth] {./diagrams/c2_LiNO3_inset} 
%\setlength{\abovecaptionskip}{10pt}
%\caption{\label{fig:c2_LiNO3_inset} The anisotropy decay of OH chromophores in water molecules in LiNO3 solution/vapor interface.}
%\end{figure} 

%{Rotational Anisotropy Decay of Water at Aqueous/vapor Interfaces of Alkali Halide Solutions}\label{RAD}
%\paragraph{Aqueous vapor interface of LiI(NaI) solution}
%We use the following procedures to calculate the molar concentration of ions in the solutions we study.
%\begin{align}
%&n_j=N_j\times[1/(6.02\times10^{23})] {\text{ mol}} \nonumber \\
%&V_{\text{liquid}}=15.6\times15.6\times15.6 \text{\AA}^3=3796\times10^{-30}\text{ m}^3 \nonumber
%\label{eq:concen}
%\end{align}
%where $n_j$, $N_j$ and $V_{\text{liquid}}$  is the amount of substance $j$, the number of substance $j$, and the volume of the liquid part of the liquid/vapor interface.  
%For the LiI solution/vapor interface system. The simulation box is with the size of 31.0 \AA$ \times$15.6 \AA$ \times$15.6 \AA. Half of the volume of the simulation box is vacuum. In the liquid part of the simulation box, there are two \Li cations and two \I anions.
%Therefore, the molar concentration of the solution LiI is $c_{\text{LiI}}={n_{\text{LiI}}}/{V_\text{liquid}}=0.9\times10^3  \text{ mol}/\text{m}^3$.

