\chapter{Summary}\label{CHAPTER_Summary}
Using DFTMD simulations, we have analyzed the interfacial structure and dynamics of electrolyte solutions containing alkali nitrates.
In particular we have presented a detailed analysis of the VSFG spectra, HB dynamics, and reorientation dynamics of water molecules at the solution/vapor interfaces. 
We have calculated the interface vibrational spectra in order to provide a molecular interpretation of available experimental data. 
In view of the similarity between the iodide ion and the nitrate ion in the Hofmeister sequence, 
we also did the same calculation for the electrolyte solutions of alkaline iodides. 
%鉴于碘离子与硝酸根离子在Hofmeister序列中的相似性,我们有时也对碘化碱金属盐的溶液或其界面系统做了计算。以期望对含有硝酸根和碘离子等较大的阴离子的溶液中动力学性质得到一般的认识。

We have shown that the use of simple models, such as small cluster is not suitable to reproduce the experimental spectra 
and cannot provide a microscopic interpretation of the VSFG spectra. Realistic models of the interface are required to address the 
perturbation of ions on the water surface. The elucidated mechanism is possibly more general to anions which have high 
surface propensity for the solution/vapor interface, for example the nitrate ion and the iodide ion.

As a first system we have analyzed the behaviour of a salty interface containing LiNO$_3$.
Both measured and calculated VSFG spectra shows a reduced intensity of the lower frequency portion region, 
when compared to the water/vapor interface. 
This reduction is attributed to the H-bonds established between the \nitrate and the surrounding water molecules at the interface.
This effects is only related to the presence of \nitrate at the water surface and is not affected by the presence of alkali metal ions.
Indeed we have shown that although the \Li can reside relative close to the water surface, also forming a water mediated
ion pair with \nit, its effect on the VSFG spectrum is not visible. The water molecule which mediate the interaction 
between the \nitrate and the \Li would produce a red-shifted peak in small water cluster, but its influence is not visible 
in the VSFG spectra. To verify this conclusion, we have calculated the free energy of different configurations for 
larger water clusters by the Blue-Moon method. The results give consistent results: Li$^+$--NO$_3^-$ ion pairs 
separated by a water molecule have lower free energy than the configuration in which Li$^+$ is in direct contact with NO$_3^-$. 

For bulk system, based on the Luzar-Chandler HB population operator, 
we calculated the correlation functions \CHB, $n(t)$, and $k(t)$, and obtained the reaction rate constants with the HB formation and breakage, 
and then obtained the information about the HB lifetime. 
%In order to analyze the HB dynamics on the interface, 
%we propose a statistical method which is based on the instantanous interface HB population operator for the instantaneous interface.
%Compared with traditional statistical methods, it is easier to implement and more efficient, 
%because it does not need to consider which molecules are within the instantaneous interface, 
%so there is no need to select the molecules on the interface and calculate the statistical average of physical quantities. 
%We took pure water interface and LiNO$_3$ solution as examples, and applied the above method to analyze the HB kinetics and HB lifetime on the interface. 
%This set of methods can be easily applied to general interface systems or solutions.
As we did in paragraph \ref{HBD_ITP}, we studied the population operators of nitrate ion--water hydrogen bond 
and iodide ion--water hydrogen bond and their correlation functions. 

 
Based on the DFMD simulations, the MS method % (i.e., select the water molecules in a certain thickness under the interface according to the molecular label)
can \emph{partially} gives information on the HB breaking and reforming
rates through the water/vapor interface and therefore partially shows how much the interface affects the dynamics of H-bonds in water. 
This method \emph{underestimates} the HB breaking rate constant of the water/vapor interface. 
We have provided a method based on the \emph{instantaneous} interface and the new-defined
\emph{interfacial HB population} operator to obtain the interfacial HB dynamics of instantaneous water/vapor interfaces.  
Using the correlation functions based on IHB population operator, we directly calculated the HB dynamics of water layer with a certain thickness 
under the instantaneous surface. 
The IHB method allows us to avoid choosing which molecules reside in the interfacial layer and
it also can provide partial information on the HB breaking and reforming rates at the interface. 
However, it \emph{overestimates} the HB breaking rate constant. The calculation results in these two extreme cases
indicate that the HB breaking and reforming rate constants in the water layer \emph{below} the surface tends to be uniform 
(see Fig.\ref{fig:128w_itp_pure_water_pair_k_k_prime_ihb_both_schemes} a) 
as the thickness of the water layer increases. The real HB dynamical characteristics at the water/vapor interface 
are derived from the two extreme cases. 
In particular, the result gives an estimate of the \emph{thickness} ($\sim$ 3 \AA) of the water/vapor interface.

The combination of the methods of MS and IHB is extended to the solution interface. 
It has been confirmed that different ions in the electrolyte solution interface have their own unique distributions in the normal direction of the interface. 
Therefore, the thickness of the solution is greater than the thickness of the water/vapor interface. 
In addition, we regarded the solvation shell of an ion as an interface, 
and defined the \emph{SHB population} operator, which is an extension to the IHB population.

The difference between the HB dynamics for H-bonds in the second solvation shell of the \Li and for nitrate-water H-bonds 
at the interface is not visible from the values of the HB relaxation time. They reflect the difference between HB dynamics in 
bulk water and at the water/vapor interface. For the alkaline iodide solution/vapor interfaces, we find 
that the cations does not alter the H-bonding network outside the first hydration shell. 
It is concluded that no long-range structural-changing effects for alkali metal cations.
Moreover, we use the SHB method to calculate the HB dynamics for water molecules in the second solvation shell.
As far as the types of ions (\Li, \Na, \K, NO$^-_3$ and \I ions) are concerned, 
the HB dynamics for water molecule pairs does not vary significantly with the type of ions.
%我们用SHB方法计算了离子第一溶解壳外的水分子的HBD. 结果显示,我我们所涉及到的离子种类,锂,钠,钾离子,硝酸根离子,碘离子,而言,球内的水分子与水分子之间的氢键动力学并不随着离子种类的不同而有明显差别。

From the results of nonlinear susceptibilities, which shows bonded OH-stretching peaks with higher frequencies, 
we conclude that the water molecules at the interfaces of the alkaline iodide solutions are participating 
in weaker H-bonds, compared with those at the water/vapor interface. 
This conclusion is based on the DFTMD simulations, and % (with a simulation box with a length scale $\sim$ 10\A)
the origin of the characteristics come from a unique distribution of the iodine ion and the alkali metal cations, 
which form a double layer \cite{Shultz2010} over the thickness on the order of 5 \A\ (see Appendix \ref{thickness_interface}).
%The same for alkali nitrate solution. 
Therefore, both the VSFG spectra of alkaline nitrate and iodide solutions can be explained in the same way, 
in which the anions has propensity for the surface, and the weaker anion--H bonds at the topmost interface layer contribute to the blue-shifted H-bond band,
and the relative less free OH bonds produce a lower free OH stretching band.
This conclusion is verified by the distribution of the number of H-bonds owned by a OH group at the instantaneous interface layer. 
The main difference of this distribution between alkaline nitrate (iodide) solution/vapor interfaces and the water/vapor interface is:
the $P(n)$ for $n\ne 3$. This difference can be viewed as the origin of less free OH bonds at these solution/vapor interface, 
and of weaker H-bonds at the topmost $(d \sim 1--3 \AA)$ interfacial layer.

By calculating the rotational anisotropy decay for water molecules at the insantaneous interface, 
we obtain a main result: the average ratio of free OH bonds is the most important factor that affect the decay rate of the reorientation relaxation of water molecules.
%\paragraph{Delocalization Effect in Hydrogen Bonds}
%The answer of the nature of H-bonds may lie with the electrons in the H-bonds. 
%Like all objects in nature, the electrons minimize their total energy, which includes their kinetic energy. 
%A reduced kinetic energy means a reduced momentum. According to the Heisenberg uncertainty principle, 
%the "delocalization" effect may occur for electrons in H-bonds, like in many other situations at sufficiently 
%low temperatures.\cite{Isaacs1999}

%(https://swift.cmbi.umcn.nl/teach/B2/HTML/hbonds.html)
%\paragraph{Discussion} We use DFT based molecular dynamics simulations to model alkali nitrate and alkaline iodine solutions, and calculate the SFG spectra, HB dynamics and 
%anisotropy decay of water molecules of these interfaces. The effects of alkali cations, 
%nitrate anions, and different bonding environments on these properties.

%The DFT calculations (despite taking electronic correlation into account) are not expensive,their cost is comparable with that of the Hartree–Fock method. Therefore, the same computer power allows us to explore much larger molecules than with other post-Hartree–Fock
%(correlation) methods.\cite{Piela07}

%DFT transforms the many-body problem of interacting
%electrons and nuclei into a coupled set of one-particle equations, which are
%computationally much more manageable.\cite{RMN02} 
%First-principles calculations based on the KS scheme of DFT have successfully predicted and explained a wide range of solid-state properties. However, it is true only for cohesive and structural properties. Systematicaly constructing functionals that are universally applicable is still a hard problem.
%Some examples of the failures of DFT are as follows.
%The band gaps of materials\cite{ASeidl}, the barriers of chemical reactions, 
%the energies of dissociating molecular ions, and charge transfer excitation energies are underestimated\cite{Kuehne12}. 
%The binding energies of charge transfer complexes and the response to an electric field in molecules and materials are overestimated. 
%Actually, all of these diverse issues are induced by the delocalization error of approximate functionals, due to the dominating Coulomb term that pushes electrons apart.\cite{Cohen08,Sanchez08,Cohen08b} 
%Furthermore, typical DFT calculations fail to describe degenerate or near-degenerate states, such as arise in transition metal systems, the breaking of chemical bonds, and strongly correlated materials. These problems come from another error--the static correlation error of approximate functionals, because it is difficult to describe the interaction of degenerate states by using the electron density.

%It is expected that a general understanding of the dynamical properties in solutions containing larger anions such as nitrate and iodide will be obtained.

%The delocalization error and static correlation error of commonly used approximations \cite{Cohen08} can be understood through the perspective of fractional charges and fractional spins and reducing these errors will provide wider applications of DFT.
