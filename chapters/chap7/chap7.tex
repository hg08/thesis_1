\chapter{Conclusion and Perspectives}\label{CHAPTER_Summary}
Using DFTMD simulations, we have analyzed the interfacial structure and dynamics of electrolyte solutions containing alkaline nitrates.
In particular we have presented a detailed analysis of the VSFG spectra, HB dynamics, and reorientation dynamics of water molecules at the solution/vapor interfaces. 
We have calculated the interface vibrational spectra to provide a molecular interpretation of available experimental data. 
In view of the similarity between the iodide ion and the nitrate ion in the Hofmeister sequence, 
we did the same calculations for the electrolyte solutions of alkaline iodides. 
%鉴于碘离子与硝酸根离子在Hofmeister序列中的相似性,我们有时也对碘化碱金属盐的溶液或其界面系统做了计算。以期望对含有硝酸根和碘离子等较大的阴离子的溶液中动力学性质得到一般的认识。

We have shown that the use of simple models, such as small cluster is not suitable to reproduce the experimental spectra 
and cannot provide a microscopic interpretation of the VSFG spectra. Realistic models of the interface are required to address the 
perturbation of ions on the water surface. The elucidated mechanism is possibly more general to anions which have high 
surface propensity for the solution/vapor interface, for example the nitrate ion and the iodide ion.

As a first system we have analyzed the behaviour of a salty interface containing LiNO$_3$.
Both measured and calculated VSFG spectra shows a reduced intensity of the lower frequency portion region, 
when compared to the water/vapor interface. 
This reduction is attributed to the H-bonds established between the \nitrate and the surrounding water molecules at the interface.
This effects is only related to the presence of \nitrate at the water surface and is \emph{not} affected by the presence of alkali metal ions.
We have shown that although the \Li can reside relative close to the water surface, also forming a water mediated
ion pair with \nit, its effect on the VSFG spectrum is not visible. The water molecule which mediate the interaction 
between the \nitrate and the \Li would produce a red-shifted peak in small water cluster's VDOS, but its influence is not visible 
in the VSFG spectra. This conclusion is verified by our calculation of the free energy difference between different configurations for 
larger water clusters by the blue-moon method. These results give consistent results: Li$^+$--NO$_3^-$ ion pairs 
separated by a water molecule have lower free energy than the configuration in which Li$^+$ is in direct contact with NO$_3^-$. 

Based on the instantaneous interface layer, we obtained properties for the aqueous interfaces that are consistent with the experimental results, 
such as the thickness of the water/vapor interface, the relaxation time of the H-bonds at the interface, 
and the distribution of free OH bonds at the interface that have an important contribution to the VSFG spectrum.
These results suggest that considering instantaneous interfaces is essential for understanding the microscopic structure about aqueous interfaces.

For bulk system, based on the Luzar-Chandler HB population operator, 
we calculated the correlation functions \CHB, $n(t)$, and $k(t)$, and obtained the rate constants ($k$ and $k'$) with the HB formation and breakage, 
and then obtained the information about the HB lifetime. 
%In order to analyze the HB dynamics on the interface, 
%we propose a statistical method which is based on the instantaneous interface HB population operator for the instantaneous interface.
%Compared with traditional statistical methods, it is easier to implement and more efficient, 
%because it does not need to consider which molecules are within the instantaneous interface, 
%so there is no need to select the molecules on the interface and calculate the statistical average of physical quantities. 
%We took pure water interface and LiNO$_3$ solution as examples, and applied the above method to analyze the HB kinetics and HB lifetime on the interface. 
%This set of methods can be easily applied to general interface systems or solutions.
As we did in Paragraph \ref{HBD_ITP}, the population operators of N--W 
and \I--W bonds and their correlation functions 
indicate that both \nitrate--water and \I--water bonds have shorter lifetime than water--water bonds.
The HB relaxation and the lifetime of H-bonds
at the interface are more dominated by the concentration of ions accumulated at the interface. 
Furthermore, both VSFG spectra and HB dynamics demonstrates the surface propensity of \nitrate and \I ions.
Another general conclusion is that the HB strength between water molecules in the ions' second solvation shell is independent 
of the ion--water interactions in ions' first solvation shell.

Based on the DFTMD simulations, the method of IMS can \emph{partially} gives information on the HB breaking and reforming
rates through the water/vapor interface and therefore partially shows how much the interface affects the dynamics of H-bonds in water. 
This method \emph{underestimates} the HB breaking rate constant of the water/vapor interface. 
We have provided a method based on the \emph{instantaneous} surface and the new-defined
IHB operator to obtain the interfacial HB dynamics of instantaneous water/vapor interfaces.  
Using the correlation functions based on the IHB population operator, we have calculated the HB dynamics of water layer with a certain thickness 
under the instantaneous surface. 
The IHB method allows one to avoid sampling the molecules reside in the interfacial layer and
it also provide partial information on the HB breaking and reforming rates at the interface. 
Unlike IMS method, it \emph{overestimates} the HB breaking rate constant. The calculated results in these two extreme cases indicate that,
as the thickness of the water layer increases,
the HB breaking and reforming rate constants in the water layer \emph{below} the instantaneous surface tends to be uniform 
(see Fig.\thinspace\ref{fig:128w_itp_pure_water_pair_k_k_prime_ihb_both_schemes} a in Chapter \ref{CHAPTER_HBD}). 
The real HB dynamical characteristics at the water/vapor interface 
are derived from the two extreme cases when the thickness of the interface is large enough. 
In particular, the result has given an estimate of the thickness ($\sim$ 4 \AA) of the water/vapor interface.  
The combination of the IMS method and the IHB population operator is extended to the solution/vapor interface. 
It has been confirmed that different ions in the electrolyte solution interface have their specific distributions in the normal direction of the interface. 
Therefore, the thickness of the solution/vapor interface is greater than the thickness of the water/vapor interface. 

In addition, we regarded solvation shells of an ion as interfaces, 
and defined the SHB population operator, which is an extension to the IHB population.
The difference between the HB dynamics for H-bonds in the second solvation shell of the \Li and for N--W bonds 
at the interface is not visible from the values of the HB relaxation time. They reflect the difference between HB dynamics in 
bulk water and at the water/vapor interface. For the alkaline iodide solution/vapor interfaces, we found 
that the cations does not alter the H-bonding network outside the first hydration shell. 
It is concluded that no long-range structural-changing effects for alkali metal cations.
Moreover, we used the SHB method to calculate the HB dynamics for water molecules between the first and the second solvation shell.
As far as the \Li, \Na, \K, NO$^-_3$ and \I ions are concerned, 
the HB dynamics for water molecule pairs does not vary significantly with these ions.
%我们用SHB方法计算了离子第一溶解壳外的水分子的HBD. 结果显示,我我们所涉及到的离子种类,锂,钠,钾离子,硝酸根离子,碘离子,而言,球内的水分子与水分子之间的氢键动力学并不随着离子种类的不同而有明显差别。

From the results of nonlinear susceptibilities, which show bonded OH-stretching peaks with higher frequencies, 
we conclude that the water molecules at the interfaces of the alkaline iodide solutions are participating 
in weaker H-bonds, compared with those at the water/vapor interface. 
This conclusion is based on the DFTMD simulations, and % (with a simulation box with a length scale $\sim$ 10\A)
the origin of the characteristics come from the distribution of the iodine ion and the alkali metal cations
over the thickness on the order of 5 \A\ (see Appendix \ref{thickness_interface}).
%The same for alkali nitrate solution. 
Therefore, both the VSFG spectra of alkaline nitrate and iodide solutions can be explained in the same way, 
in which the anions has propensity for the surface, and the weaker anion--H bonds at the topmost interface layer contribute to the blue-shifted H-bond band,
and the relative less free OH bonds produce a lower free OH stretching band.
This conclusion is verified by the distribution $P(n)$ of the number of H-bonds owned by a OH group at the instantaneous interface layer with thickness $d$. 
The main difference of this distribution between alkaline nitrate (iodide) solution/vapor interfaces and the water/vapor interface is:
$P(n)$ for $n= 0, 1, 2,\cdots$. This difference can be viewed as the origin of less free OH bonds at these solution/vapor interface, 
and of weaker H-bonds at the topmost ($d \sim$ 4 \AA) interfacial layer.
By calculating the rotational anisotropy decay for water molecules at the instantaneous interface, 
we obtain that the average density of free OH bonds is an important factor that affect the decay rate of the reorientation relaxation of water molecules.

%[这些是结论,不应该出现在前言]

Some methods in this thesis can be applied to other specific disciplines.
The combination of method IHB and IMS can provide a basis for determining the thickness of the interface layer in other environments. 
For example, when studying the dynamics of water molecules on proteins, the thickness of the interfacial water layer can be estimated by this combination method.
In addition, many interface systems have very obvious differences in structure and dynamics from bulk.
In recent years, there is a lot of experimental and simulation progress. For example, the diffusion acceleration effect on the glass surface\cite{ZhuL11,ZhangWei16}, 
and the aging effect of the non-equilibrium dynamics of the protein molecules on cell surface\cite{HuXiaohu16}. 
To describe these effects quantitatively, it is important to define the size of the interface more accurately. 
We believe that combining the VSFG method, IHB and IMS methods will help us more accurately identify the range and size of the interface in these systems.
This will likely promote our understanding of the nature of these effects at interfaces.

%AIMD所得之数据可以用在机器学习模型之训练中.
Finally, as we can see from the previous chapters, AIMD simulations have increased our understanding of interface properties and the interactions between atoms.
However, poor scaling renders AIMD calculations intractable for larger systems.
One possible extension of our work is to train potentials on small systems within \abinitio approaches\cite{Behler2007,Behler2011,Behler2014,Kolb2017}, 
then use these potentials to investigate large systems such as interfaces, grain boundaries and disordered systems using Monte-Carlo or MD simulations.
Therefore, the AIMD trajectories can be used as datasets for training new potentials for new interfacial systems, 
which may accelerate the computation.
%or generate distributions\cite{ZHANGPAN2021}.

%\paragraph{Delocalization Effect in Hydrogen Bonds}
%The answer of the nature of H-bonds may lie with the electrons in the H-bonds. 
%Like all objects in nature, the electrons minimize their total energy, which includes their kinetic energy. 
%A reduced kinetic energy means a reduced momentum. According to the Heisenberg uncertainty principle, 
%the "delocalization" effect may occur for electrons in H-bonds, like in many other situations at sufficiently 
%low temperatures.\cite{Isaacs1999}

%(https://swift.cmbi.umcn.nl/teach/B2/HTML/hbonds.html)
%\paragraph{Discussion} We use DFT based molecular dynamics simulations to model alkali nitrate and alkaline iodine solutions, and calculate the SFG spectra, HB dynamics and 
%anisotropy decay of water molecules of these interfaces. The effects of alkali cations, 
%nitrate anions, and different bonding environments on these properties.

%The DFT calculations (despite taking electronic correlation into account) are not expensive,their cost is comparable with that of the Hartree–Fock method. Therefore, the same computer power allows us to explore much larger molecules than with other post-Hartree–Fock
%(correlation) methods.\cite{Piela07}

%DFT transforms the many-body problem of interacting
%electrons and nuclei into a coupled set of one-particle equations, which are
%computationally much more manageable.\cite{RMN02} 
%First-principles calculations based on the KS scheme of DFT have successfully predicted and explained a wide range of solid-state properties. However, it is true only for cohesive and structural properties. Systematicaly constructing functionals that are universally applicable is still a hard problem.
%Some examples of the failures of DFT are as follows.
%The band gaps of materials\cite{ASeidl}, the barriers of chemical reactions, 
%the energies of dissociating molecular ions, and charge transfer excitation energies are underestimated\cite{Kuehne12}. 
%The binding energies of charge transfer complexes and the response to an electric field in molecules and materials are overestimated. 
%Actually, all of these diverse issues are induced by the delocalization error of approximate functionals, due to the dominating Coulomb term that pushes electrons apart.\cite{Cohen08,Sanchez08,Cohen08b} 
%Furthermore, typical DFT calculations fail to describe degenerate or near-degenerate states, such as arise in transition metal systems, the breaking of chemical bonds, and strongly correlated materials. These problems come from another error--the static correlation error of approximate functionals, because it is difficult to describe the interaction of degenerate states by using the electron density.

%It is expected that a general understanding of the dynamical properties in solutions containing larger anions such as nitrate and iodide will be obtained.

%The delocalization error and static correlation error of commonly used approximations \cite{Cohen08} can be understood through the perspective of fractional charges and fractional spins and reducing these errors will provide wider applications of DFT.
