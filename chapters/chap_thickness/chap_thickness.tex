\chapter{Thickness of the Interface of the Aqueous Solutions} \label{thickness_interface}
To determine the thickness of the water/vapor interface of alkali solutions,
we chose several different thickness values of slab of the interface and calculate 
the corresponding susceptibility for these slabs, respectively.
Take the water/vapor interface of LiNO$_3$ solution as an example. We chose seven different thickness 
values---from 2 to 8 \A, and calculate VSFG intensities $I_{SSP} \propto |\chi^{(2),\text{R}}_{SSP}|^2$ 
for the water/vapor interface with a thickness of each of these values. 
The result is given in Fig. ~\ref{fig:117_2LiNO3_30ps_2-6A_150_Im_150720} a and b.
It shows that $|\chi^{(2),\text{R}}_{SSP}|^2$ converges as the thickness increases to 8 \A. 
%
\begin{figure}[H]
%\begin{figure}[!ht]
\centering
\includegraphics [width= 0.6\textwidth] {./diagrams/117_2LiNO3_30ps_2-6A_150_Im_150720}
\setlength{\abovecaptionskip}{0pt}
\caption{\label{fig:117_2LiNO3_30ps_2-6A_150_Im_150720} The calculated $|\chi^{(2),\text{R}}_{SSP}|^2$, 
of water molecules at the aqueous/vapor interfaces with different thickness. 
This calculation is done for a model for the water/vapor interface where a slab of 117 water molecules containing one \Li and one \nitrate is included 
in a period simulation box of 15.60 \AA$\times$15.60 \AA$\times$31.00 \AA at 300 K.}
\end{figure}
%
\begin{figure}[H]
%\begin{figure}[!h]
\centering
  \includegraphics [width=0.6 \textwidth] {./diagrams/124_2LiI_30ps_2-5A_150_Im_150720}
\setlength{\abovecaptionskip}{0pt}
\caption{\label{fig:124_2LiI_30ps_2-5A_150_Im_150720} The calculated $|\chi^{(2),R}_{SSP}|^2$, 
  of water molecules at water/vapor interfaces with different thickness. This calculation is done for a model 
  for water/vapor interface where a slab of 118 water molecules containing one \Li and one \I is included 
  in a period simulation box of 15.60 \AA $\times$ 15.60 \AA $\times$ 31.00 \AA.}
\end{figure}
%
Because of the limitation of the box scale, we chose 4 different thickness values---$d=2,3,4,5$ \A, for checking. 
The VSFG intensities $I_{SSP} \propto |\chi^{(2),R}_{SSP}|^2$ for the water/vapor interface of the LiI solution 
are given in Fig. \ref{fig:124_2LiI_30ps_2-5A_150_Im_150720} a and b. The results show that $|\chi^{(2),R}_{SSP}|^2$ 
increases as the thickness increase from 2 \AA to 5 \A. 
%

The thickness of the water/vapor interface of solutions can also be seen from the calculated RDF. 
For example, the RDFs of interface of the NaI solution is shown in Fig. \ref{fig:2NaI-124w_gdr_OH_s2_150122}.
It indicates that the thickness of the water/vapor interface is $\sim$ 8 \A.
\begin{figure}[H] %[!hb]
\centering
\includegraphics [width=0.46\textwidth] {./diagrams/2NaI-124w_gdr_OH_s2_150122}
\setlength{\abovecaptionskip}{0pt}
\caption{\label{fig:2NaI-124w_gdr_OH_s2_150122} The RDF $g_{\text{OH}}(r)$ in the NaI solution/vapor interface. 
The simulation system includes one \Na ion, one \I ion, and 124 water molecules in 15.60 \AA $\times$ 15.60 \AA $\times$ 31.00 \AA 
  simulation box. The solid and dashed lines are corresponding to one of the two interfaces in our simulation, respectively. 
  The simulation time is 22.5 ps.}
\end{figure}
%
Furthermore, we also estimated the thickness of a water/vapor interface by calculating the HB dynamics for it. 
Fig. \ref{fig:2LiI-124w_S_layers} shows the dependence of the logarithm of survival probability on the thickness of the water/vapor interface 
of the LiI solution. We see that when $d = 8$ \AA the $\ln S_{\text{HB}}$ of the interface 
does not change as the thickness increases, indicating the thickness of the interface. 
\begin{figure}[H]%[!ht]
\centering
\includegraphics [width=0.38\textwidth] {./diagrams/2LiI-124w_S_layers} %fig.5.9
\setlength{\abovecaptionskip}{0pt}
\caption{\label{fig:2LiI-124w_S_layers}The logarithms of the function \SHB for water-water H-bonds at interfaces with different thickness
in 0.9 M LiI solution.}
\end{figure}

In addition, the survival probability \SHB is dependent on the temporal resolution $t_t$,
which is the time interval between two adjacent states in time used to calculate the survival probability.
As an example, the $t_t$ dependence of the $\tau_{\text{HB}}$ of the three alkali-iodine solution interfaces in our AIMD simulations is 
reported in Fig. \ref{fig:hb_lifetime_124_2LiI-2NaI-2KI}. 
It shows that, if we take the $t_t$ as small as possible, the number of times that H-bonds break and form again in this period of time ($t_t$) will be reduced.
Thus, we can obtain the continuum HB lifetime independent of the $t_t$. As shown in Fig. \ref{fig:hb_lifetime_124_2LiI-2NaI-2KI}, 
the value of $\tau_{\text{HB}}$ corresponding to the intersections of the $\tau_{\text{HB}}(t_t)$ functions and the line $t_t = 5$ fs approximately give the continuum HB lifetimes.
\begin{figure}[H]
 \centering
 \includegraphics [width=0.36\textwidth] {./diagrams/hb_lifetime_124_2LiI-2NaI-2KI} %fig5.16
 \setlength{\abovecaptionskip}{0pt}
 \caption{\label{fig:hb_lifetime_124_2LiI-2NaI-2KI} The resolution dependence of the continuum lifetime $\tau_{\text{HB}}$ of water--water H-bonds at interfaces of
    different 0.9 M alkali-iodine solutions at 330 K, calculated for six temporal resolutions ($t_t$). \cite{Ferrario1990,Mountain1995,Root1997}
    For $t_t = 5$ fs, the calculated continuum HB lifetime is 0.30 ps, 0.31 ps and 0.23 ps, for the interface of LiI, NaI and KI solution, respectively.}
\end{figure} 

