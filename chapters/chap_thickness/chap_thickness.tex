\chapter{Thickness of the interface of aqueous solutions} \label{thickness_interface}
\section{Interfacial HB dynamics} \label{ihb_and_selection}
\paragraph{Instantaneous Surfaces}
Let us consider the interfacial system of pure water.  At a certain time $t$, the instantaneous surface ${\mathbf s}^0(x(t),y(t))$ can be determined by calculating 
the coarse-graining density field: we can specify that the coarse-grained density is the reference density $\rho_\text{ref} = 0.016 $ \A$^{-3}$,
and those grid points constitute the surface ${\mathbf s}^0(x,y)$ of pure water. In our simulated pure water system, there are two such surfaces.
The code for calculating the instantaneous surfaces of interfacial systems can be found on \url{https://github.com/hg08/interface}. 
%===================
\paragraph{Griding and layering}
Wee assume that the normal is along the $z$-axis direction. 
First, we  discretize the coordinates of the $xy$ plane. 
We divide the edges along the $x$-axis direction and the edges along the $y$-axis direction of the simulated box into $N$ parts uniformly, 
and the $xy$ plane can be approximated by $N\times N$ discrete points. 
Then $N\times N$ ordinates of the surface ${\mathbf s}^0(x(t_0),y(t_0))$ at the initial time $t_0$ can be represented as components of an $N\times N$-vector ${\mathbf z}^0(t_0$). 
The surface ${\mathbf s}^0(x(t_0),y(t_0))$ is also the upper boundary of the interface.
Secondly, we define a layering strategy, or define the thickness $d$ of the interface layer. 
In this way, we then determine the lower boundary ${\mathbf s}^1(x(t_0),y(t_0))$ of the interface. 
It can be expressed as a $N\times N$-vector ${\mathbf z}^1(t_0)={\mathbf z}^0(t_0)-{\mathbf d}$, and ${\mathbf d}$ is a $N\times N$-dimensional constant 
vector in which all the entries are $d$. The superscripts 0 and 1 respectively identify the upper and lower boundaries of the interface. 
Similarly, the other interface can also be determined.

So far, our interfaces are still static, because we only considered the initial moment. 
But the instantaneous interfaces can be defined for different moments naturally. 
Therefore, according to the same method, for the molecular motion trajectory of an interface system, 
we can obtain the dynamically changing interface layer, i.e., ${\mathbf z}^1(t)={\mathbf z}^0(t)-{\mathbf d}$. 
When the interfaces are determined, we can answer the question: For an atom with coordinates $(x, y, z)$ at any time $t$, whether it is in the interface?
To answer this question, we map the atom's coordinates $(x, y)$ to an ordered integer pair $(i, j)$, $i,j \in  \{0,\cdots,N\}$, 
where $i = \text{int}(x /\Delta x), j = \text{int}(y/\Delta y)$, $\Delta x = a/N$, and $\Delta y = b/N$.
 
At a certain time $t$, the ordinate $z$ of an atom can be approximately represented as a function defined on point $(i, j)$, $z=z(i,j)$. 
Then we compare $z(i,j)$ and $z^l_{i,j}$, $l=0,1$ to determine if the atom is in the interface. 
That is, if $z^0_{i,j}(t)<z(i,j)<z^1_{ij}(t)$, then the atom at $(x,y,z)$ is located in the interface at time $t$.

\paragraph{Molecule Sampling (MS) from instantaneous interface}\label{para_MS_interface}
The method of Molecule Sampling used to extract the interfacial molecules is as follows. 
1) Determine the instantaneous surface of the water/vapor interface system;
2) Define a interfacial layer with a fixed thickness $d$ below the surface; 
3) Select the atoms which are near the instantaneous surface;
4) Calculate the HB dynamics for molecules in interface at different time $t$.
Given the thickness of layer $d$, at any sampling time $t_0$, $t_1$, $t_2$, $\cdots$, $t_n$, the set of molecules in the interface can be determined. 
Since the molecules are always in motion, generally, the set $S^0(t_i)$ of the molecules in the interface at time $t_i$ 
and the set $S^0(t_j)$ of the molecules in the interface at a different time $t_j$ is different. 
Therefore, to calculate the HB dynamics of molecules in the interface, we calculate the correlation functions of 
the HB population $h(t)$ of the molecules in the interface at different times, 
and calculate the \emph{average} of the obtained correlation functions 
over the $n$ functions, e.g., $C_{\text{HB},j}(t), j=1,\cdots,n$.
 
\paragraph{Interfacial hydrogen bond population operator \hbos} 
The IHB method to extract the interfacial molecules is as follows. 
1) Determine the instantaneous surface of a system;
2) Define a interface system with a fixed thickness below the instantaneous surface; 
3) Define the interfacial hydrogen bond population operator \hbos.
4) Calculate the autocorrelation function of \hbos and then the related observations. 

If we want to calculate dynamical characteristics of the molecules located in the interface layer, 
the MS method will have a error in select correct molecules in interface. 
In chapter \ref{CHAPTER_HB}, we had combined the recognition technology of the instantaneous liquid interfaces \cite{Willard2010} 
with the definition of H-bonds \cite{AL96b,Luzar1996} to define H-bonds that depend on the environment. 
With this method, we can have more understanding of the differences and commonalities in the kinetics of the breaking and reconstruction of the hydrogen bonds 
of the water molecules in the interface system. At the same time, we can compare the HB dynamics in the interface of different thicknesses 
with the HB dynamics in bulk water to obtain the thickness of the interface from the perspective of HB dynamics. 
In addition, for the interface system of different solutions, we can also use the layering technique to study the HB dynamics,
so that we can understand the particularity of the interface of various solutions relative to pure water. 
%These results may help us have a better understanding of the physical and chemical processes related to interfaces.

In fact, none of the above-mentioned two methods for obtaining the HB dynamics of the interface water molecules 
can give the true interface HB dynamics completely and accurately. But they respectively give an extreme case of interface HB dynamics. 
In the MS method, the formation and breaking of intermolecular H-bonds can be truly described, 
but the selection of interface water molecules is not accurate enough. Since the configuration of the molecule will change over time, 
the contribution of the H-bonds in the bulk phase will be included. 
In the IHB method, we are very accurate in choosing the interface water molecules and the H-bonds in the interface, 
but we may artificially destroy some H-bonds in the interface that were not broken. 
To a certain extent, the HB dynamics obtained by the IHB method is \emph{accelerated}. 
Therefore, the comparison of the results obtained by these two methods may give a true picture of the interface HB dynamics.
In particular, as the interface thickness increases, the two methods can achieve the same results.
%特别地,随着界面厚度的增大,两种方法可以得到相同的结果。
The code for calculating the HB dynamics for instantaneous interfaces can be found on \url{https://github.com/hg08/hb_in_interface}. 
%===================
\section{Thickness of the solution/vapor interfaces}
To determine the thickness of the water/vapor interface of alkali solutions,
we chose several different thickness values of slab of the interface and calculate 
the corresponding susceptibility for these slabs, respectively.
Take the water/vapor interface of LiNO$_3$ solution as an example. We chose several different thickness 
values---from 2 to 8 \A, and calculate VSFG intensities $I_{SSP} \propto |\chi^{(2),\text{R}}_{SSP}|^2$ 
for the water/vapor interface with a thickness of each of these values. 
The result is given in Fig.\thinspace\ref{fig:117_2LiNO3_30ps_2-6A_150_Im_150720} a and b.
It shows that $|\chi^{(2),\text{R}}_{SSP}|^2$ converges as the thickness increases. 
%
\begin{figure}[H]
%\begin{figure}[!ht]
\centering
\includegraphics [width= 0.6\textwidth] {./diagrams/117_2LiNO3_30ps_2-6A_150_Im_150720}
\setlength{\abovecaptionskip}{0pt}
\caption{\label{fig:117_2LiNO3_30ps_2-6A_150_Im_150720} The calculated $|\chi^{(2),\text{R}}_{SSP}|^2$, 
of water molecules at the aqueous/vapor interfaces with different thickness.
} 
\end{figure}
%
%\begin{figure}[H]
%\centering
%  \includegraphics [width=0.6 \textwidth] {./diagrams/124_2LiI_30ps_2-5A_150_Im_150720}
%\setlength{\abovecaptionskip}{0pt}
%\caption{\label{fig:124_2LiI_30ps_2-5A_150_Im_150720} The calculated $|\chi^{(2),R}_{SSP}|^2$, 
%  of water molecules at water/vapor interfaces (LiI solution) with different thickness. This calculation is done for a model 
%  for water/vapor interface where a slab of 118 water molecules containing one \Li and one \I is included 
%  in a period simulation box of 15.60 \AA $\times$ 15.60 \AA $\times$ 31.00 \AA.}
%\end{figure}
\begin{figure}[H]
%\begin{figure}[!h]
\centering
  \includegraphics [width=0.6 \textwidth] {./diagrams/2LiI-124w_0-30ps_2-8A_150_Im_150720}
\setlength{\abovecaptionskip}{0pt}
\caption{\label{fig:2LiI-124w_0-30ps_2-8A_150_Im_150720} The calculated $|\chi^{(2),R}_{SSP}|^2$, 
  of water molecules at the interfaces of the LiI solution with different thickness.}
\end{figure}
%
The VSFG intensities $I_{SSP} \propto |\chi^{(2),R}_{SSP}|^2$ for the water/vapor interface of the LiI solution 
are given in Fig.\thinspace\ref{fig:2LiI-124w_0-30ps_2-8A_150_Im_150720} a and b. The results show that $|\chi^{(2),R}_{SSP}|^2$ 
increases as the thickness increase from 0 to 8 \A. 
%

%The thickness of the water/vapor interface of solutions can also be seen from the calculated RDF. 
%For example, the RDFs of interface of the NaI solution is shown in Fig. \ref{fig:2NaI-124w_gdr_OH_s2_150122}.
%It indicates that the thickness of the water/vapor interface is $\sim$ 8 \A.
%\begin{figure}[H] %[!hb]
%\centering
%\includegraphics [width=0.46\textwidth] {./diagrams/2NaI-124w_gdr_OH_s2_150122}
%\setlength{\abovecaptionskip}{0pt}
%\caption{\label{fig:2NaI-124w_gdr_OH_s2_150122} The RDF $g_{\text{OH}}(r)$ in the NaI solution/vapor interface. 
%The simulation system includes one \Na ion, one \I ion, and 124 water molecules in 15.60 \AA $\times$ 15.60 \AA $\times$ 31.00 \AA 
%  simulation box. The solid and dashed lines are corresponding to one of the two interfaces in our simulation, respectively. 
%  The simulation time is 22.5 ps.}
%\end{figure}
%
Furthermore, we also estimated the thickness of a water/vapor interface by calculating the HB dynamics for it. 
Fig.\thinspace\ref{fig:2LiI-124w_S_layers} shows the dependence of the logarithm of survival probability on the thickness of the water/vapor interface 
of the LiI solution. We found that when $d = 8$ \AA the correlation function $\ln S_{\text{HB}}(t)$ of the interface 
converges as the thickness increases, indicating the thickness of the interface. 
\begin{figure}[H]%[!ht]
\centering
\includegraphics [width=0.38\textwidth] {./diagrams/2LiI-124w_S_layers} %fig.5.9
\setlength{\abovecaptionskip}{0pt}
\caption{\label{fig:2LiI-124w_S_layers}The logarithms of the function \SHB for water-water H-bonds at interfaces with different thickness
in the LiI solution.} % 0.9 M.
\end{figure}

In addition, the survival probability \SHB is dependent on the temporal resolution \rt,
which is the time interval between two adjacent states in time used to calculate the survival probability.
As an example, the \rt dependence of the $\tau_{\text{HB}}$ of the three alkali-iodine solution interfaces in AIMD simulations is 
reported in Fig.\thinspace\ref{fig:hb_lifetime_124_2LiI-2NaI-2KI}. 
It shows that, if one take the \rt as small as possible, the number of times that H-bonds break and form again in this period of time (\rt) will be reduced.
As shown in Fig.\thinspace\ref{fig:hb_lifetime_124_2LiI-2NaI-2KI}, 
The value of $\tau_{\text{HB}}$ corresponding to the intersections of the $\tau_{\text{HB}}(t_\text{t})$ functions and the line \rt = 5 fs approximately give the continuum HB lifetimes.
For \rt = 5 fs, the calculated continuum HB lifetime is 0.30, 0.31 and 0.23 ps, for the interface of LiI, NaI and KI solution, respectively.
Then, the estimated value of $\tau_\text{a}$ is $\sim 0.2$ ps.
Moreover, we can obtain the continuum HB lifetime independent of the \rt: $\tau_\text{HB} = \lim_{t_\text{t} \to 0} \tau_\text{HB}(t_\text{t})$. 

\begin{figure}[H]
 \centering
 \includegraphics [width=0.36\textwidth] {./diagrams/hb_lifetime_124_2LiI-2NaI-2KI} %fig5.16
 \setlength{\abovecaptionskip}{0pt}
 \caption{\label{fig:hb_lifetime_124_2LiI-2NaI-2KI} The resolution dependence of the continuum lifetime $\tau_{\text{HB}}$ of water--water H-bonds at interfaces of
 different alkali-iodine solutions at 330 K, calculated for six temporal resolutions (\rt). \cite{Ferrario1990,Mountain1995,Root1997}
 } % 0.9 M
\end{figure}
\paragraph{PBC's effect on $C_2(t)$}
As an example, Figure \ref{fig:c2_128w_itp_pbc_15c2-vs-21} shows the effect of PBC on $C_2(t)$. 
In our simulation, whether the PBC is considered or not will not have a substantial impact on the result 
that the orientation relaxation of OH bond depends on the thickness of the interface. 
Even without considering the PBC, we can still conclude that the orientation relaxation process of OH bonds at the water/vapor interface 
slows down obviously with the increase of interface thickness.
\begin{figure}[H]
 \centering
 \includegraphics [width=0.36\textwidth] {./diagrams/c2_128w_itp_pbc_15c2-vs-21} %fig5.16
 \setlength{\abovecaptionskip}{0pt}
 \caption{\label{fig:c2_128w_itp_pbc_15c2-vs-21} The PBC's effect on the thickness-dependence of $C_2(t)$ for water molecules at the water/vapor interfaces.
 }
\end{figure}
