\chapter{Summary}\label{CHAPTER_Summary}
Using DFTMD simulations, we have analyzed the interfacial structure and dynamics of electrolyte solutions containing alkali nitrates.
In particular we have presented a detailed analysis of the HB structure at the interface and we have calculated the interface vibrational spectra
in order to provide a molecular interpretation of available experimental data. 

As a first system we have analyzed the behaviour of a salty water/vapor interface containing LiNO$_3$.
Both the measured and calculated SFG spectra shows a reduced intensity of the lower frequency portion region, 
when compared to the pure water/vapor interface. 
This reduction is attributed to the H-bonds established between the \nitrate and the surrounding water molecules at the interface.
This effects is only related to the presence of \nitrate at the water surface and is not affected by the presence of alkali metal ions.
Indeed we have shown that although the \Li can reside relative close to the water surface, also forming a water mediated
ion pair with \nit, its effect on the SFG spectrum is not visible. The water molecule which mediate the interaction 
between the \nitrate and the \Li would produce a red-shifted peak in small water cluster, but its influence is not visible 
neither in the SFG spectra. To verify this conclusion, the free energy of different configurations was also calculated for 
larger water clusters in CP2K by the Blue-Moon method. The results still give consistent results: Li$^+$--NO$_3^-$ ion pairs 
separated by a water molecule have lower free energy than the configuration in which Li$^+$ is in direct contact with NO$_3^-$. 

We have also shown that the use of simple models, such as small cluster is not suitable to reproduce the experimental spectra 
and cannot provide a microscopic interpretation of the spectra. Realistic models of the interface are required to address the 
perturbation of the ion on the water surface. The elucidated mechanism is possibly more general to anions which have high 
propensity for the water surface, as for example other molecular ions.

The difference between the HB dynamics of H-bonds outside the first shell of the \Li and that of nitrate-water H-bonds 
at interfaces is not visible from the values of the HB relaxation time. They reflect the difference between HB dynamics in 
bulk water and that at the water/vapor interfaces. For the water/vapor interface of alkaline iodine solutions, we find 
that the cations does not alter the H-bonding network outside the first hydration shell of cations. 
It is concluded that no long-range structural-changing effects for alkali metal cations.

From the results of nonlinear susceptibilities, which shows bonded OH-stretching peaks with higher frequencies, 
we conclude that the water molecules at the water/vapor interfaces of LiI, NaI, and KI solutions are participating 
in weaker H-bonds, compared with those at the pure water surface. 
This conclusion is based on the DFTMD simulations, and % (with a simulation box with a length scale $\sim$ 10\A)
the origin of the characteristics may come from a unique distribution of \I ions and alkali metal cations, 
which form a double layer \cite{Shultz2010} over the thickness on the order of 5--10 \A\ (see Appendix \ref{thickness_interface}).

For the bulk system, starting from the data obtained from the DFTMD simulation, based on the HB population operator, 
we calculated the correlation functions $C(t)$, $n(t)$ and $k(t)$ related to it, and obtained the reaction rate constants with the HB formation and breakage in the system, 
and then obtained the information of the HB lifetime. In order to analyze the HB dynamics on the interface, 
we propose a statistical method which is based on the instantanous interface HB population operator for the instantaneous interface.
Compared with traditional statistical methods, it is easier to implement and more efficient, 
because it does not need to consider which molecules are within the instantaneous interface, 
so there is no need to select the molecules on the interface and calculate the statistical average of physical quantities. 
Except for the case where the interface thickness is less than the size of 2 to 3 water molecules, %(Except when the interface thickness is too low,) 
this method can efficiently analyze the HB dynamics and HB lifetime distribution of the instantaneous interface.
We took pure water interface and LiNO$_3$ solution as examples, and applied the above method to analyze the HB kinetics and HB lifetime on the interface. 
This set of methods can be easily applied to general interface systems or solutions.
As we did in paragraph \ref{PARAGRAPH_LINO3} and \ref{PARAGRAPH_I--W}, we studied the population operators of nitrate ion--water hydrogen bond 
and iodide ion--water hydrogen bond and their correlation functions. In general solutions, water--water H-bonds can be extended to ion--water H-bonds, 
and then similar analysis can be done to study the HB dynamics.

%Finally, calculating the rotational anisotropy decay, we obtain non-single-exponential dynamics 
%for the rotation of water molecules both at the surface and in bulk phase for alkaline 
%iodine solutions. Therefore, the rotational motion of water molecules are not simply characterized by well-defined rate constants. 
%Faster rotational anisotropy decay exists for water molecules at the interface of aqueous alkali-iodine solutions, 
%which is the result of a different HB types ($D'AA$) from the usual HB type ($DDAA$) in pure bulk water. 
%This effect on anisotropy decay is due to the H--I bond at the interface. Since the iodide's surface propensity is high, 
%this difference of HB structure from pure water/vapor interface changed the Im$\chi^{(2),\text{R}}$ spectrum and the HB dynamics 
%of the interface of alkali-iodine solution. 

%\paragraph{Delocalization Effect in Hydrogen Bonds}
%The answer of the nature of H-bonds may lie with the electrons in the H-bonds. 
%Like all objects in nature, the electrons minimize their total energy, which includes their kinetic energy. 
%A reduced kinetic energy means a reduced momentum. According to the Heisenberg uncertainty principle, 
%the "delocalization" effect may occur for electrons in H-bonds, like in many other situations at sufficiently 
%low temperatures.\cite{Isaacs1999}

%(https://swift.cmbi.umcn.nl/teach/B2/HTML/hbonds.html)
%\paragraph{Discussion} We use DFT based molecular dynamics simulations to model alkali nitrate and alkaline iodine solutions, and calculate the SFG spectra, HB dynamics and 
%anisotropy decay of water molecules of these interfaces. The effects of alkali cations, 
%nitrate anions, and different bonding environments on these properties.

%The DFT calculations (despite taking electronic correlation into account) are not expensive,their cost is comparable with that of the Hartree–Fock method. Therefore, the same computer power allows us to explore much larger molecules than with other post-Hartree–Fock
%(correlation) methods.\cite{Piela07}

%DFT transforms the many-body problem of interacting
%electrons and nuclei into a coupled set of one-particle equations, which are
%computationally much more manageable.\cite{RMN02} 
%First-principles calculations based on the KS scheme of DFT have successfully predicted and explained a wide range of solid-state properties. However, it is true only for cohesive and structural properties. Systematicaly constructing functionals that are universally applicable is still a hard problem.
%Some examples of the failures of DFT are as follows.
%The band gaps of materials\cite{ASeidl}, the barriers of chemical reactions, 
%the energies of dissociating molecular ions, and charge transfer excitation energies are underestimated\cite{Kuehne12}. 
%The binding energies of charge transfer complexes and the response to an electric field in molecules and materials are overestimated. 
%Actually, all of these diverse issues are induced by the delocalization error of approximate functionals, due to the dominating Coulomb term that pushes electrons apart.\cite{Cohen08,Sanchez08,Cohen08b} 
%Furthermore, typical DFT calculations fail to describe degenerate or near-degenerate states, such as arise in transition metal systems, the breaking of chemical bonds, and strongly correlated materials. These problems come from another error--the static correlation error of approximate functionals, because it is difficult to describe the interaction of degenerate states by using the electron density.

%The delocalization error and static correlation error of commonly used approximations \cite{Cohen08} can be understood through the perspective of fractional charges and fractional spins and reducing these errors will provide wider applications of DFT.
