\chapter{Structural Characterization of Water Clusters}\label{structure_of_clusters}
The structural parameters of the considered water clusters are shown here.
Table \ref{tab:3_nitrate_bond} gives the average HB lengths $r_a$ (with standard deviations)in [NO$_3\cdot$(H$_2$O)$_3$]$^-$.  
Table \ref{tab:3w_nitrate} (\ref{tab:table_geo_opt}) reports the selected distances characterizing 
[NO$_3\cdot$(H$_2$O)$_3$]$^-$ (RNO$_3$(H$_2$O)$_3$), and Table \ref{tab:table_rnitrate_3w} the selected parameters for RNO$_3$   
 (H$_2$O)$_3$ (R=Li, Na, K).
The unit for length and angle are \AA and degree ($^\circ$), respectively.
% 
\begin{table}[!h]
\centering
\caption{\label{tab:3_nitrate_bond}%
The HB lengths $r_a$ in [NO$_3\cdot$(H$_2$O)$_3$]$^-$ at 300 K.} 
\begin{tabular}{cc} \\\toprule
 HB bound to & \multicolumn{1}{c}{ $r_a$} \\
\hline
 w1 &2.40$\pm$0.52; 3.02$\pm$0.72 \\
 w2 &2.56$\pm$0.48; 3.20$\pm$0.41 \\
 w3 &2.29$\pm$0.47; 3.11$\pm$0.72
\end{tabular}
\end{table}
%
\begin{table}[!htbp]
\centering
\caption{\label{tab:3w_nitrate}%
The parameters of water molecules and HBs in [NO$_3\cdot$(H$_2$O)$_3$]$^-$ at 300 K.}
\begin{tabular}{lccc}
water &$R_\text{OH}$ &$\angle$HOH & $r_\text{OH}$ \\
\hline
w1 &0.98$\pm$0.02 &101$\pm$4 & 2.40$\pm$0.52, 3.02$\pm$0.72 \\
w2 &0.98$\pm$0.02 &101$\pm$5 & 2.56$\pm$0.48, 3.20$\pm$0.41 \\
w3 &0.98$\pm$0.02 &101$\pm$4 & 2.29$\pm$0.47, 3.11$\pm$0.72
\end{tabular}
\end{table}
%
\begin{table}[!htbp]
\centering
\caption{\label{tab:table_geo_opt}%
  The structural parameters of RNO$_3$(H$_2$O)$_3$ from geometry optimization.} 
\begin{tabular}{l*{4}ccc}
Parameters  & LiNO$_3$(H$_2$O)$_3$& NaNO$_3$(H$_2$O)$_3$ & KNO$_3$(H$_2$O)$_3$\\
\hline
$r_\text{HB1}$& 1.67 & 1.71 & 1.82 \\
$r_\text{HB2}$& 1.91 & 1.78 & 1.92\\
$r_\text{HB3}$& 1.82 & 1.69 & 1.94\\
$r_\text{R-O(w1)}$ & 1.91 & 2.31 & 2.70\\
$r_\text{R-O(w2)}$ & 1.90 & 2.26 & 2.70\\
$r_\text{R-O(\nitrate)}$ & 1.84 & 2.29 & 2.69 \\
$\angle$HOH(w1)& 109 & 106 &107 \\
$\angle$HOH(w2)& 106 & 105&105 \\
$\angle$HOH(w3)& 108 & 107 &106
\end{tabular}
\end{table}
%
\begin{table}[H] %[!htbp]
\centering
\caption{\label{tab:table_rnitrate_3w}%
The parameters of RNO$_3$(H$_2$O)$_3$ at 300 K, obtained from the averaging during a DFTMD trajectory. 
  For RNO$_3$(H$_2$O)$_3$, $R_\text{OH}$ and $R'_\text{OH}$ 
  denote the lengths of O-H bonds in which H atoms is H-bonded and is free, respectively.
  }
\begin{tabular}{l*{4}lll}
Parameters & LiNO$_3$(H$_2$O)$_3$& NaNO$_3$(H$_2$O)$_3$ & KNO$_3$(H$_2$O)$_3$\\
\hline
$r_\text{HB1}$ & $1.83\pm0.14$ & $1.78\pm0.09$ & $1.82\pm0.13$\\
$r_\text{HB2}$ & $2.00\pm0.25$ & $1.91\pm0.24$ & $1.80\pm0.12$\\
$r_\text{HB3}$ &$1.79\pm0.16$ & $1.76\pm0.11$ & $1.89\pm0.18$\\
$R_\text{OH}$(w1) &$0.97\pm0.01$ &$0.98\pm0.04$ &$0.97\pm0.03$ \\
$R'_\text{OH}$(w1) &$1.00\pm0.02$ &$1.00\pm0.02$ & $1.00\pm0.03$ \\
$R_\text{OH} $(w2) &$0.97\pm0.01$ &$0.98\pm0.02$ &$0.97\pm0.02$ \\ 
$R'_\text{OH}$(w2) &$0.99\pm0.01$ &$1.00\pm0.02$ & $1.00\pm0.03$ \\
$R_\text{OH}$(w3) &$0.97\pm0.01$ & $0.97\pm0.02$&$0.97\pm0.03$ \\
$R'_\text{OH}$(w3) &$1.00\pm0.02$ &$1.00\pm0.02$ & $1.00\pm0.03$ \\
$r_\text{R-O(w1)}$ & $1.95\pm0.09$ & $2.34\pm0.08$ & $2.76\pm0.11$\\
$r_\text{R-O(w3)}$ & $1.92\pm0.07$ & $2.32\pm0.11$ & $2.74\pm0.13$\\
$r_\text{R-O(\nitrate)}$ & $1.91\pm0.08$ & $2.31\pm0.09$ & $2.74\pm0.12$ \\
$\angle$HOH (w1) &$107\pm4$ & $106\pm4$ &$105\pm5$ \\
$\angle$HOH (w2) &$106\pm6$ & $105\pm4$ &$106\pm4$ \\
$\angle$HOH (w3) &$108\pm5$ & $106\pm3$ &$106\pm3$ 
\end{tabular}
\end{table}
\paragraph{Structural and Vibrational Properties of [NO$_3\cdot$(H$_2$O)$_3$]$^-$}
% [In the main text, I deleted this paragraph, because now i am not sure on this idea.]
To find the possible source of the different vibrational features of water molecules in the 
cluster [NO$_3\cdot$(H$_2$O)$_3$]$^-$, we considered the structural properties and VDOS for water molecules 
in this cluster. 
\begin{table}
\centering
\caption{\label{tab:3_nitrate_bond}%
The lengths of H-bonds in [NO$_3\cdot$(H$_2$O)$_3$]$^-$. The indices of H atoms: H6, H7 in w1; 
H9, H10 in w2 and H12, H13 in w3.} 
\begin{tabular}{ccc} \\\toprule
 HBs& $r_a\pm\delta$ (100 K)(\A) & \multicolumn{1}{c}{ $r_a\pm\delta$ (300 K)}(\A)\\
\hline
 H6-O2 &2.75$\pm$0.62& 2.40$\pm$0.52 \\
 H7-O4 &2.79$\pm$0.58& 3.02$\pm$0.72 \\
 H9-O3 &2.89$\pm$0.60 &2.56$\pm$0.48 \\
 H10-O4 &2.74$\pm$0.49&3.20$\pm$0.41 \\
 H12-O3 &2.46$\pm$0.45&2.29$\pm$0.47 \\
 H13-O2 &2.75$\pm$0.59 &3.11$\pm$0.72
\end{tabular}
\end{table}
%===================
\begin{figure}[H]
%\begin{figure}[htbp]
\centering
\includegraphics [width=0.36\textwidth] {./diagrams/gdr_ON-wat--3_NO3_Sans} 
\setlength{\abovecaptionskip}{0pt}
\caption{\label{gdr_ON-wat--3_NO3_Sans}The nitrate O ($\text{O}_\text{n}$)--water O ($\text{O}_\text{w}$) 
and nitrate O--water H ($\text{H}_\text{w}$) RDFs for [NO$_3\cdot$(H$_2$O)$_3$]$^-$.
The peaks for the former are 1.93, 2.95 and 3.95 \A, and for the later are 2.95 and 4.80 \A.}
\end{figure} 
%==================================================================================
When $T=300$ K, the difference $r_a$ between different hydrogen atoms in one water molecule is
$\Delta{r_a}=0.69$ \AA, while $\Delta{r_a}=0.13$ \AA for $T=100$ K. It shows that the vibrational 
peaks for the three water molecules are much closer than that at the higher temperature 300 K. 

The calculated VDOS for water molecules in the cluster at a lower temperature 100 K is given in 
Fig.\thinspace\ref{fig:vdos_LiNO3-3w_100K_w1-2-3_font35}. At the lower temperature, the three water molecules 
are more symmetric distributed bound to the central nitrate. Therefore, the difference between H-bonds in the 
symmetric isomer of [NO$_3\cdot$(H$_2$O)$_3$]$^-$ is likely a finite temperature effect,
which can be verified by the calculation of the VDOS for water molecules.

Both differences $\Delta\nu$ and $\Delta{d}$ decrease as the temperature decrease,
Therefore, the different vibrational features are temperature-dependent effect. 
%--------------------
\begin{figure}[H]
%\begin{figure}[htbp]
\centering
\centering
\includegraphics [width=0.36 \textwidth] {./diagrams/vdos_LiNO3-3w_100K_w1-2-3_font35} 
\setlength{\abovecaptionskip}{0pt}
\caption{\label{fig:vdos_LiNO3-3w_100K_w1-2-3_font35} The VDOS $g(\nu)$ for water molecules in the
cluster [NO$_3\cdot$(H$_2$O)$_3$]$^-$ at 100 K shows that the vibrational peaks for the three water molecules 
are very close to each other ($\Delta\nu <$ 10 \cm)for both vibrational and bending modes.}
\end{figure}
%====================================================================================
In addition, the VDOS for H atoms and water molecules in [NO$_3\cdot$(H$_2$O)$_3$]$^-$ (Fig.\thinspace\ref{fig:vdos_NO3-3w_2_H-wat}) shows 
that H's contribution dominates that of the water molecule. 
\begin{figure}[H]
%\begin{figure}[htbp]
\centering
\includegraphics [width=0.36\textwidth] {./diagrams/vdos_NO3-3w_2_H-wat}%
\setlength{\abovecaptionskip}{0pt}
\caption{\label{fig:vdos_NO3-3w_2_H-wat} The comparison between the VDOS for H and of a whole water molecule, 
for a water molecule ({w1}, Fig.\thinspace\ref{fig:clusters_4}a), in [NO$_3\cdot$(H$_2$O)$_3$]$^-$ at 300 K.}
\end{figure}  %(Calculated from the function vdos3.f and ft$\_$5s.sh)

