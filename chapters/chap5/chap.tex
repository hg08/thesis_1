\chapter{Hydrogen bond dynamics at the water/vapor interface}\label{CHAPTER_HBD}
Hydrogen bonds play a critical role in the behaviour of bulk water\cite{Eisenberg1969,Teixeira1993,Luzar1996}, 
aqueous solutions\cite{Naslund2005}, and water near interfaces\cite{Chowdhary2008}.
There are many methods to study HB dynamics in water, solutions and interfaces, 
such as molecular dynamics simulation\cite{Tongraar2006,Chanda2006,Tongraar2010,Chowdhary2008,Banerjee2016}, neutron scattering\cite{ChenSH1984,Teixeira1990}, 
IR spectroscopy\cite{Werhahn2011,Fournier2016}, 2D-IR spectroscopy\cite{Auer2007,Kim2009} and 2DSFG spectroscopy\cite{ZhangZhen2011}.
In this chapter, we will use the general concepts and methods of HB dynamics \cite{AL96,Luzar1996,DC87} introduced in Paragraph \ref{para:def_HBP} 
in Chapter \ref{CHAPTER_Methods} to analyze the structure and dynamic properties of bulk water and the water/vapor interface. 

%
\FloatBarrier
\section{Dynamical properties of H-bonds in bulk water and at the water/vapor interface}
The bulk water and the interface between pure water and vacuum, i.e., the water/vapor interface, 
are considered in this paragraph.
%For bulk water, we can compare the results of the method in this paragraph with that of previous works\cite{AL96,Kessler2015}. 
%After the validation for bulk water, we will show in this paragraph the results of HB dynamics of the water/vapor interface.
%[all the data on the simulation]
All simulations in this chapter were performed at 300 K within the canonical (NVT) ensemble.
The BLYP exchange and correlation functional\cite{Becke1988,LeeC1988}, 
and D3 dispersion corrections\cite{Grimme2010,Klimes2012} have been employed.
The electron-ion interactions are described by GTH pseudopotentials\cite{Hartwigsen1998,Lippert1999}.
The length of the trajectory is 60 ps of physical time.
%The definition of $h(t)$ is based on specific H--O bond, instead of water-water pairs.
The simulated bulk water consisted of 128 water molecules in a periodic cubic box of length $L = 15.64$ \A, which corresponds to a density of 1.00 g cm$^{-3}$.
The simulated water/vapor interface consisted of 128 water molecules in a periodic box with size 15.64 $\times$ 15.64 $\times$ 31.28 \A$^3$ (with a $\sim$15 \A 
\ separation between the periodic slabs in the $z$-direction).

\paragraph{Correlation functions $c(t)$, $n(t)$ and $k(t)$}
As we have seen in the definition of the HB population $h(t)$, the cutoff radius depends on the RDFs of water. 
To provide a basis for subsequent calculation, we calculated the basic structural properties of the simulated bulk water.
The O-O RDF is characterized by the first peak, which corresponds to the first solvation shell, followed by a minimum at 3.5 \A, 
which corresponds to the cutoff chosen in the definition of the HB criterium.
The RDFs $g_\text{OO}(r)$ and $g_\text{OH}(r)$ for bulk water system are 
shown in Fig.\thinspace\ref{fig:rdf_bk_pure_pbc}.
\begin{figure}[htb]
\centering                                          
%\includegraphics [width=0.6 \textwidth] {./diagrams/rdf_bk_pure_and_interf_pure_normed} 
\includegraphics [width=0.5 \textwidth] {./diagrams/rdf_bk_pure_pbc} 
\setlength{\abovecaptionskip}{0pt}
  \caption{\label{fig:rdf_bk_pure_pbc}Partial RDFs of the simulated bulk water.}
\end{figure}
\begin{figure}[htb]
\centering
\includegraphics [width=\textwidth] {./diagrams/pure_bk_c_n_k} 
\setlength{\abovecaptionskip}{0pt}
  \caption{\label{fig:pure_bk_c_n_k}Time dependence of (a) $n(t)$, $c(t)$ and (b) $k(t)$ 
for \emph{bulk} water.} %/home/gang/Github/hbacf/__hbacf_continuous/correlations/c/128w_bk_2delta_t_60ps_hbacf_h.dat
\end{figure}

The correlation functions \CHB from the trajectory of a bulk water simulation calculated according to Eq.\thinspace\ref{eq:C_HB} 
with the ADH (solid line) and AHD (dashed line) definition of H-bonds are 
shown in Fig.\thinspace\ref{fig:pure_bk_c_n_k} a. 
The reactive flux $k(t)$ calculated according to Eq.\thinspace\ref{eq:k} (see Fig.\thinspace\ref{fig:pure_bk_c_n_k} b) is consistent with the result in Ref.\cite{AL96b}.
For bulk water, there exists a $\sim 0.2$-ps transient period,
during which $k(t)$ quickly changes from its initial value\cite{Starr2000}.
However, at longer times, the $k(t)$ is independent of the HB definitions.
%[we performed a DFTMD simulation of bulk water system with a total time of 60 ps, and used the two different HB definitions (ADH and AHD) to calculate $k(t)$. ]
The results in Fig.\thinspace\ref{fig:pure_bk_c_n_k} b show that 
the difference in $k(t)$ caused by different HB definitions is relatively small.
Therefore, the long time decay of $k(t)$ reflects the general properties of H-bonds, and
calculating the reactive flux HB correlation functions is a more rigorous way to obtain the nature of H-bonds\cite{AL00}.

Now we discuss the result for the water/vapor interface.
At first we consider the water/vapor interface as a whole.
We reported $c(t)$ and $n(t)$
in Fig.\thinspace\ref{fig:128w_itp_c_n_k} a and the reactive flux $k(t)$ in Fig.\thinspace\ref{fig:128w_itp_c_n_k} b.
%
Also at the interface cases, $k(t)$ quickly changes from its initial value on a time scale of less than 0.2 ps. 
This can be seen from Fig.\thinspace\ref{fig:pure_bk_and_itp_k}, where $k(t)$ in Fig.s\thinspace\ref{fig:pure_bk_c_n_k} and 
\ref{fig:128w_itp_c_n_k} is reported in double logarithmic coordinates.
This log-log plot of the $k(t)$ shows that, as in bulk water, this decay behaviour cannot be described with a power-law decay for the water/vapor interface.
This result is also in good agreement with that of the classical molecular simulation of bulk water\cite{AL96b,Luzar1996}.
\begin{figure}[H] %htb
\centering
\includegraphics [width= \textwidth] {./diagrams/128w_itp_c_n_k} 
\setlength{\abovecaptionskip}{0pt}
  \caption{\label{fig:128w_itp_c_n_k}Time dependence of (a) $n(t)$, $c(t)$ and (b) $k(t)$ 
for the water/vapor \emph{interface}.}
\end{figure}


%
\begin{figure}[H]
\centering
\includegraphics [width= \textwidth] {./diagrams/pure_bk_and_itp_k} 
\setlength{\abovecaptionskip}{0pt}
  \caption{\label{fig:pure_bk_and_itp_k}Time dependence of $k(t)$ for (a) bulk water and (b) the water/vapor interface.}
\end{figure}
%
For the water/vapor interface, we focus on the reactive flux $k(t)$, 
which was used in the study of HB dynamics of liquid water\cite{AL96,Khaliullin2013}.
The $k(t)$ calculated from the trajectory of water molecules in simulations, is reported in Fig.\thinspace\ref{fig:128w_log_rf_ns40_log}. 
Beyond the 0.2-ps transient period, it decays to zero monotonically (Fig.\thinspace\ref{fig:pure_bk_and_itp_k}). 
This property has been found for bulk water using the SPC water model by Luzar and Chandler\cite{AL96}. 
%

The functions $n(t)$ calculated according to Eq.\thinspace\ref{eq:n_from_k_in} for bulk water and the water/vapor interface are shown in 
Fig.\thinspace\ref{fig:128w_bk_itp_50ps_n_from_k_in_with_2_hb_def_type2}. 
It shows that as $t$ increases $n(t)$ increases rapidly from 0, and reaches a maximum at $t \approx 10$ ps, and then gradually decreases. %[EXPLAIN THE RESULTs]
We find that the maximum of $n(t)$ for the water/vapor interface is slightly higher than that in bulk water,
for both HB definitions.
We interpret this result as the fact that at time $t$, there is a greater probability that H-bonds at the interface are broken 
compared to H-bonds in bulk water.
\begin{figure}[htpb]
\centering
\includegraphics [width=0.60\textwidth] {./diagrams/128w_log_rf_ns40_log}
\setlength{\abovecaptionskip}{0pt}
  \caption{\label{fig:128w_log_rf_ns40_log}Time dependence of $k(t)$ for the water/vapor interface, according to Eq.\thinspace\ref{eq:k}.
}
\end{figure}
\begin{figure}[H]
\centering
\includegraphics [width= \textwidth] {./diagrams/128w_bk_itp_50ps_n_from_k_in_with_2_hb_def_type2}
\setlength{\abovecaptionskip}{0pt}
\caption{\label{fig:128w_bk_itp_50ps_n_from_k_in_with_2_hb_def_type2} 
Time dependence of $n(t)$ for bulk water and the water/vapor interface from the (a) ADH and (b) AHD criteria.} 
\end{figure}

\paragraph{Reaction rate constants}
To calculate the HB relaxation times in bulk water and the water/vapor interface, we
make connection between microscopic HB dynamics and the phenomenological description of the HB breaking/reforming reaction
\begin{align}
\ce{A <=> B},
\end{align}
with $k$ and $k'$ as the forward and backward rate constants, respectively.
Here, A denotes reactants (HB \emph{on}, $\langle h\rangle$), and B denotes products (HB \emph{off}, $\langle 1-h\rangle$)\cite{Chandra2000}.
Khaliullin and K\"uhne have studied HB dynamics of bulk water using AIMD simulations\cite{Khaliullin2013}.
Based on the HB population operators $h(t)$ and $h^{(d)}(t)$, and correlation functions $n(t)$ and $k(t)$, they have used the simulation data 
to obtain the ratio $k/k'$ in bulk water, and then the lifetime and relaxation time 
of the H-bonds.  
Here, we study HB dynamics at the water/vapor interface.
We can obtain the optimal solution range of $k$ and $k'$ from the relationship between the reactive flux 
and the HB population correlation function $c(t)$ and $n(t)$, and the two rate constants $k$ and $k'$, i.e.,
\begin{eqnarray}
  k(t) = kc(t)-k'n(t).
\label{eq:fitting_k_rates}
\end{eqnarray}
%[Answer Q3]
We have found the optimal value of the rate constants, $k$ and $k'$, 
by a least squares fit of the calculated data $k(t)$, $c(t)$ and $n(t)$ beyond the transition phase.  
The function $c(t)$ is regarded as a $P$-dimensional column vector composed by $(c(1),c(2),\cdots,c(P)){\tran}$, and denoted as ${\bf c}$,
with $c(i)$ representing the value of the correlation $c(t)$ at $t=i$.
Similarly, the functions $n(t)$ and $k(t)$ are also viewed as $P$-dimensional column vectors and are denoted as ${\bf n}$ and ${\bf k}$, respectively.
Therefore, $k$ and $k'$ are determined from the matrix ${A} = \begin{bmatrix} {\bf c} & {\bf n} \end{bmatrix}$, i.e., 
\begin{equation}
\begin{bmatrix} k\\ -k' \end{bmatrix} = ({A}{\tran} {A})^{-1} {A}{\tran} {\bf k}. 
\end{equation}
For bulk water and the water/vapor interface, the optimal $k$ and $k'$ are reported in Tables 
\ref{tab:k_k_prime_128w_pure_1} and \ref{tab:k_k_prime_128w_pure_2}. 
% 
\begin{table}[htb]
\centering
\caption{\label{tab:k_k_prime_128w_pure_1} 
    The forward and backward rate constants, $k$ and $k'$, for bulk water (bulk) and the water/vapor interface (w/v). We carried on the short time region 0.2 ps $< t <$ 2 ps. 
    The unit for $k$ ($k'$) is ps$^{-1}$, and that for $\tau_{\text{HB}}$ ($=1/k$) is ps (same for Table\thinspace\ref{tab:k_k_prime_128w_pure_2}).
} 
\begin{tabular}{ccccccc}
 Criterion & $k$  (bulk) & $k'$ (bulk) & $\tau_{\text{HB}}$ (bulk) & $k$  (w/v) & $k'$ (w/v) & $\tau_{\text{HB}}$ (w/v)\\
\hline
  % With 4 digital!(Keep it)
  %ADH & 0.3345 & 0.8591 & 2.9895 & 0.3587 & 0.6730 & 2.7881  \\
  %ADH(from $k_{in}$) & 0.2959  & 0.9883 & 3.3795  & 0.3225 & 0.7652 & 3.1012 \\
  %AHD & 0.3334 & 1.0414 & 2.9991 & 0.3520  & 0.7847  &  2.8405\\
  %AHD(from $k_{in}$) & 0.2882 & 1.1490 & 3.4699 & 0.3140 & 0.8867 & 3.1836 \\
  ADH & 0.30  & 0.99 & 3.38  & 0.32 & 0.77 & 3.10 \\
  AHD & 0.29 & 1.15 & 3.47 & 0.31 & 0.89 & 3.18 \\
\end{tabular}
\end{table}
%
\begin{table}[htb]
\centering
\caption{\label{tab:k_k_prime_128w_pure_2} 
    The forward and backward rate constants, $k$ and $k'$, for bulk water (bulk) and the water/vapor interface (w/v). We carried on the long time region 2 ps $< t <$ 12 ps.
} 
\begin{tabular}{ccccccc}
 Criterion & $k$  (bulk) & $k'$ (bulk) & $\tau_{\text{HB}}$ (bulk) & $k$  (w/v) & $k'$ (w/v) & $\tau_{\text{HB}}$ (w/v)\\
\hline
  %ADH & 0.1151 & 0.0311 & 8.6872 & 0.1593 & 0.0580 & 6.2786 \\
  %ADH(from $k_{in}$) & 0.1147  & 0.0391 & 8.7184 & 0.1569  & 0.0678 & 6.3723\\
  %AHD & 0.1071 & 0.0424 & 9.3450  & 0.1572 & 0.0763 & 6.3626 \\
  %AHD(from $k_{in}$) & 0.1053  & 0.0472 & 9.4963 & 0.1545  & 0.0884 & 6.4715 \\
  ADH & 0.12  & 0.04 & 8.72 & 0.16  & 0.07 & 6.37\\
  AHD & 0.11  & 0.05 & 9.50 & 0.16  & 0.09 & 6.47 \\
\end{tabular}
\end{table}
% 

To obtain the forward and backward rate constants ($k$ and $k'$),
we performed the fitting in different time region $0.2 < t < 2$ ps and $2 < t < 12$ ps, respectively.
We note that in the larger time region, i.e., $2 < t < 12$ ps, the value of HB lifetime $\tau_\text{HB}$ is larger than that in shorter time region, $0.2 < t < 2$ ps,
no matter for bulk water or for the water/vapor interface. A larger $\tau_\text{re}$ value means that the distance between two water molecules 
stays within $r_\text{OO}^c= 3.5$ \AA for a longer time. 
For the long time region, these values of the $k$ are comparable in magnitude to that obtained by Ref.\thinspace{\cite{Khaliullin2013}}. 

%For AHD definition:The bulk water: 14.1572 ps; the water/vapor interface: 12.7806 ps.
%For the water/vapor interface of pure water, we also calculated the constants $k$ and $k'$ by least square fit. 

\section{Instantaneous interfacial HB dynamics}\label{PARA_IHB}
It can be seen from Tables\thinspace\ref{tab:k_k_prime_128w_pure_1} and \ref{tab:k_k_prime_128w_pure_2} that 
if we analyze the simulated water/vapor interface as a whole, the behavior of the water surface is masked by the behavior of the bulk contribution.
To selectively identify the properties at the interface, it is necessary to selectively identify the surface molecules.
Therefore, we define a instantaneous interface using the procedure of Willard and Chandler\cite{Willard2010} and then selectively analyze HB dynamics of the water molecules
at the water interface.

To study HB dynamics for the water/vapor interface, we first determine the instantaneous interface and then define the interfacial HB population operator. 
Based on these two definitions, we can derive the correlation functions and reaction rate constants, for interfacial layers. 
Using these quantities we can discuss the change in HB dynamics at the interface as function of the interface layer's thickness 
(for more properties related to the thickness of the interface, see Appendix \ref{thickness_more}).

% One can uncomment if remove PERCENT
%====================================
%Based on the HB definition of water molecule pairs, we can also use least squares fitting to obtain the rate constant $k$, $k'$, 
%and the average lifetime $\tau_{HB}$ of the H-bonds. The results are shown in the Table \ref{tab:k_k_prime_128w_pure_2s}  to \ref{tab:k_k_prime_128w_pure_2u}.
%\begin{table}[htb]
%\centering
%\caption{\label{tab:k_k_prime_128w_pure_2s} 
%    The $k$ and $k'$ for bulk water and the water/vapor interface. We carried on the long time region 0.2 ps $< t <$ 8 ps. 
%The unit for $k$ ($k'$) is ps$^{-1}$, and that for $\tau_{\text{HB}}$ ($=1/k$) is ps.} 
%\begin{tabular}{ccccccc}
% Criterion & $k$  (bulk) & $k'$ (bulk) & $\tau_{\text{HB}}$ (bulk) & $k$  (interf.) & $k'$ (interf.) & $\tau_{\text{HB}}$ (interf.)\\
%\hline
%  ADH & 0.14 & 0.28 & 7.16 & - & - & -  \\
%  AHD & 0.11 & 0.18 & 9.08 & - & -  &  -\\
%\end{tabular}
%\end{table}
%%
%\begin{table}[htb]
%\centering
%\caption{\label{tab:k_k_prime_128w_pure_2t} 
%    The $k$ and $k'$ for bulk water and the water/vapor interface. We carried on the longer time region 0.2 ps $< t <$ 12 ps. 
%The unit for $k$ ($k'$) is ps$^{-1}$, and that for $\tau_{\text{HB}}$ ($=1/k$) is ps.} 
%\begin{tabular}{ccccccc}
% Criterion & $k$  (bulk) & $k'$ (bulk) & $\tau_{\text{HB}}$ (bulk) & $k$  (interf.) & $k'$ (interf.) & $\tau_{\text{HB}}$ (interf.)\\
%\hline
%  ADH & 0.10 & 0.17 & 9.59 & - & - & -  \\
%  AHD & 0.09 & 0.11 & 11.62 & - & -  &  -\\
%\end{tabular}
%\end{table}
%%
%\begin{table}[htb]
%\centering
%\caption{\label{tab:k_k_prime_128w_pure_2u} 
%    The $k$ and $k'$ for bulk water and the water/vapor interface. We carried on the longer time region 1 ps $< t <$ 12 ps. 
%The unit for $k$ ($k'$) is ps$^{-1}$, and that for $\tau_{\text{HB}}$ ($=1/k$) is ps.} 
%\begin{tabular}{ccccccc}
% Criterion & $k$  (bulk) & $k'$ (bulk) & $\tau_{\text{HB}}$ (bulk) & $k$  (interf.) & $k'$ (interf.) & $\tau_{\text{HB}}$ (interf.)\\
%\hline
%  ADH & 0.06 & 0.06 & 17.96  & - & - & -  \\
%  AHD & 0.06 & 0.05 & 18.17 & - & -  & -\\
%\end{tabular}
%\end{table}
%
\FloatBarrier
\paragraph{Instantaneous Interfaces}\label{para:II}
As shown by Willard and Chandler, due to molecular motions, interfacial configurations
change with time, and the identity of molecules that lie at the interface also changes with time. 
Generally, useful procedures for identifying interfaces must take into account these motions\cite{Willard2010}. 
To determine the instantaneous interface of the system, we adopted the Willard-Chandler method based on spatial density\cite{Willard2010}.
The coarse-grained density at a space-time point $(\mathbf{r},t)$ can be expressed as polynomial
\begin{eqnarray}
\bar{\rho}(\mathbf{r}, t)=\sum_{i} \phi(|\mathbf{r}-\mathbf{r}_{i}(t)|; \xi) 
\end{eqnarray}
where ${\mathbf{r}}_i(t)$ is the position of the $i$th particle at time $t$ and the sum is over all such particles, and 
\begin{eqnarray}
\phi(\mathbf{r};\xi)=(2 \pi \xi^{2})^{-3/ 2} \exp (-r^{2} / 2 \xi^{2}) 
\label{eq:gaussian_coarse_graining}
\end{eqnarray} 
is a normalized Gaussian functions for a 3-dimensional system, where $r$ is the magnitude of ${\mathbf r}$, and $\xi$ is the coarse-graining length.
Equation \ref{eq:gaussian_coarse_graining} is introduced to improve the accuracy of the interface, such that we can extend the domain and make it a single unicom,
i.e., no cavity exists in the domain.
With the parameter $\xi$ set, the interface can be defined to be the two-dimensional manifold ${\mathbf r} = {\mathbf s}$ such that
\begin{eqnarray}
\bar\rho(\mathbf{s};t)= \rho_c, 
\label{eq:rho_c}
\end{eqnarray} 
where $\rho_c$ is a reference density. This interface depends on molecular configurations, i.e.,
${\mathbf s}(t) = {\mathbf s}(\{{\mathbf r}_i(t)\})$. 

%{Instantaneous Layering of the water/vapor interface} DELETED THE PARAGRAPH NAME
After the instantaneous interface is defined, we can define an instantaneous interface layer for any non-uniform fluid system. 
Specifically, for the simulated water/vapor interface system in the cuboid simulation box, 
we can get another two-dimensional manifold ${\mathbf s}_0(t)$ by moving the surface ${\mathbf s}(t)$ 
along the system's normal coordinate to a certain distance $d$ 
(two grey surfaces are shown in Fig.\thinspace\ref{fig:128w_itp_add_z_d_trimed_with_inner_layers}).
At any time $t$, the volume between the two surfaces 
${\mathbf s}(t)$ and ${\mathbf s}_0(t)$ is defined as an \emph{instantaneous interface layer}, or \emph{instantaneous interface}. 
In other words, these two surfaces are the two boundaries of the instantaneous interface, 
and $d$ is its thickness. 
Different values of $d$ give us different layering strategies for the interface system. 
See Fig.\thinspace\ref{fig:128w_itp_add_z_d_trimed_with_inner_layers} as an example, two instantaneous interface layers with thickness $d$ are shown.
\begin{figure}
\centering
\includegraphics [width=0.6\textwidth] {./diagrams/128w_itp_add_z_d_trimed_with_inner_layers}
\setlength{\abovecaptionskip}{0pt}
\caption{\label{fig:128w_itp_add_z_d_trimed_with_inner_layers}
A slab of water (128 water molecules are included) with the instantaneous interface represented as a blue mesh on the upper and lower phase boundary.
The normal is along the $z$-axis and the parameter $d$ is the thickness of the interfacial layer.
The grey surfaces are obtained by translating the interfaces to the interior of the slab along the $z$-axis (or the opposite direction) by $d$.
} 
%The box dimensions are $15.64 \times 15.64 \times 31.28$ \AA$^3$, and the slab is periodically replicated in the $x$, $y$ and $z$ directions. 
\end{figure}

Below we will combine the instantaneous interface and Luzar-Chandler's HB population operator\cite{AL96} to identify the H-bonds 
at the water/vapor interface. The dynamics of these H-bonds will vary with the thickness $d$ of the interface layer. 
By investigating HB dynamics for these layers, we can obtain the dynamical characteristics of the water/vapor interfaces. 
%As we will see later, this method can be extended to HB dynamics 
%in various environments, such as H-bonds around certain ions, in bulk water, etc.
%These different environments have a common feature: because the molecular configuration changes over time, the usual method first selects these molecules or molecular pairs, 
%and then determines the H-bonds in this special environment based on a HB criterion, and finally calculate the HB lifetimes or autocorrelation functions of 
%the HB population operators. 

\FloatBarrier
\paragraph{Interfacial HB population} \label{IHBP}
After we have determined the instantaneous surface ${\mathbf s}(t)={\mathbf s}(\{{r}_i(t)\})$, we can define \emph{interfacial H-bonds}.
Now we define the interface HB population operator $h^{(\text{s})}[{r}(t)]$ as follows:
It has a value 1 when the particular tagged molecular pair $i,j$ are H-bonded, \emph{and} both molecules are inside the instantaneous interface 
with a thickness $d$, and zero otherwise:
\begin{align}
   h^{(\text{s})}[{r}(t)]=\left\{
   \begin{array}{rcl}
           1       &      & {i,j\text{ are H-bonded, and}}\\
                &      & {i,j\text{ are inside the interfacial layer}} \\   \label{eqn:h_s}
           0       &      & {\text{otherwise}}
   \end{array} \right.
\end{align}
From the definition, we know that $h^{(\text{s})}(t)$ depends on the thickness $d$, therefore,
$h^{(\text{s})}(t)$ can help to efficiently obtain H-bonds' dynamic characteristics of 
the interfacial layer. %Note that the definition of HB here is based on water molecule pairs or O-H pairs. 
In this paragraph, we discuss H-bonds based on water molecule pairs for simplicity. 
%Starting from the H-bonds based on O-H pairs, the same analysis can also be done. 

Similar to \CHB in Eq.\thinspace\ref{eq:C_HB}, which describes the fluctuation of general H-bonds,
we define a correlation function \CSHB that describes the fluctuation of the interfacial H-bonds: 
\begin{eqnarray}
c^\text{(s)}(t)=\langle h^\text{(s)}(0)h^\text{(s)}(t) \rangle/\langle h^\text{(s)}\rangle
\label{eq:C_s_HB}.
\end{eqnarray}
%
Similarly, we define correlation functions 
\begin{eqnarray}
n^\text{(s)}(t)=\langle h^\text{(s)}(0)[1-h^s(t)]h^{\text{(d,s)}} \rangle/\langle h^\text{(s)}\rangle
\label{eq:n_s_HB},
\end{eqnarray}
and 
\begin{eqnarray}
k^\text{(s)}(t)= -\frac{dc^\text{(s)}}{dt}
\label{eq:k_s_HB}.
\end{eqnarray}
Using these functions, we can determine the reaction rate constant of breaking and reforming and the lifetimes of interfacial H-bonding.
We will discuss the dependence of the correlation functions \CHB, \CSHB, and the reaction rates $k$ and $k'$ on the interface thickness $d$ in the next two paragraphs.
%
\FloatBarrier
\paragraph{Depth-dependence of \CSHB}
\begin{figure}[htb]
\centering
\includegraphics [width=\textwidth] {./diagrams/128w_itp_pure_water_pair_c_ihb}
\setlength{\abovecaptionskip}{0pt}
\caption{\label{fig:128w_itp_pure_water_pair_c_ihb} 
The \CSHB for the instantaneous interfacial H-bonds with different thicknesses,
as computed from the (a) ADH and (b) AHD criteria through the IHB method.} 
\end{figure}
For the water/vapor interface, we used two geometric criteria of H-bonds to calculate \hbos and therefore \CSHB from Eq.\thinspace\ref{eq:C_s_HB}. 
The calculated results of \CSHB are shown in Fig.\thinspace\ref{fig:128w_itp_pure_water_pair_c_ihb}.
%We find that the greater the thickness $d$ of the instantaneous interface is selected, 
%the slower the relaxation of the interface H-bonds. 
%When the thickness is greater than a certain thickness $d^c$ ( $\sim$ 3 \AA),
%the relaxation of H-bonds at the interface hardly changes.
We find that HB dynamics is faster at the water/vapor interface when compared to bulk water.
As $d$ increases, HB dynamics gets slower and it recovers the bulk value. 
%When $d$ exceeds 3 \A, the interface's HB dynamics no longer changes with $d$. 
This behavior is independent of the HB definition as shown by the comparison of results in panel a and b of Fig.\thinspace\ref{fig:128w_itp_pure_water_pair_c_ihb}.
%

For comparison, we also calculate HB dynamics of water molecules at the interface obtained by interfacial molecule sampling 
(IMS, see Appendix \ref{ihb_and_selection} for details). 
In this method, we first select molecules at the interface at each sampling time and then make a statistical
average of the calculated correlation functions for those molecules.
Specifically, to determine the water molecules in the interface layer, 
we sample at regular intervals, and then calculate \CHB for these water molecules in the interface layer and then their a statistical average.
Figure \ref{fig:128w_itp_pure_water_pair_c_ihb_scheme1} shows how \CHB depends on the thickness $d$.
The panel a and b use HB definition criterion ADH, and AHD, respectively.
Comparing Fig.s\thinspace\ref{fig:128w_itp_pure_water_pair_c_ihb} and \ref{fig:128w_itp_pure_water_pair_c_ihb_scheme1}, we find that
when we use the method of IMS, the dependence of the correlation function \CHB
on the interface thickness is consistent with that of \CSHB for large $d$. 
Moreover, regardless of the ADH or AHD definition of a HB, this conclusion is valid.
 
Beside \CHB or \CSHB for the interface, we will further examine the correlation 
functions $n(t)$, $k(t)$ ($n^\text{(s)}(t)$, $k^\text{(s)}(t)$), and the rate constants $k$, $k'$.
\begin{figure}[H]
\centering                                         
\includegraphics [width=\textwidth] {./diagrams/128w_itp_pure_water_pair_c_ihb_scheme1}
\setlength{\abovecaptionskip}{0pt}
\caption{\label{fig:128w_itp_pure_water_pair_c_ihb_scheme1} 
The \CHB for the instantaneous interfacial H-bonds with different thicknesses,
as computed from (a) ADH and (b) AHD criteria. 
The IMS method is used, and the sampling rate is 1/4 per ps.} 
%These results are based on selecting the water molecules in the instantaneous interface and averaging 
%the correlation functions of these water molecules. The sampling is performed every 4 ps.
\end{figure}

%[Plot the $k$ and $k'$ as functions of thickness $d$.]
\FloatBarrier
\paragraph{Depth-dependence of reaction rate constants} 
To find the reaction rate constants $k$ and $k'$, we have two statistical methods: 
(1) Instantaneous interfacial hydrogen bond (IHB), by which we can calculate the correlation functions \CSHB, $n^\text{(s)}(t)$, and $k^\text{(s)}(t)$;
(2) IMS, by which we first identify the water molecules at the instantaneous interface at each time $t$, and start from the corresponding 
correlation functions \CHB, $n(t)$, and $k(t)$ of H-bonds of the identified water molecules.
Figure \ref{fig:128w_itp_pure_water_pair_k_k_prime_ihb_both_schemes} shows the rate constants ($k$ and $k'$) 
and the lifetime $\tau_\text{HB}$ obtained by the two methods.
We find that, for all the three parameters $k$, $k'$ and $\tau_\text{HB}$, the behavior as function of the thickness of the interface is only slightly affected 
by the calculation methods. 
%

As we can see from Fig.\thinspace\ref{fig:128w_itp_pure_water_pair_k_k_prime_ihb_both_schemes}, 
when $d$ is large enough ($d > d_0 \sim 4$ \AA), the constants $k$ an $k'$ obtained by the two methods agree quantitatively. 
This result shows that the two statistical methods (Appendix \ref{ihb_and_selection}) 
for HB dynamics of the interface do not produce much difference.

%
We also find that when we focus on molecules in the interface layer with $d < d_0$, 
the values of the reaction rate constants does depend on the method we use. 
That is, the $k$ obtained by the IHB method is slightly larger than by the IMS and $k'$ is smaller. 
Since $\tau_\text{HB} = 1/k$, a larger value of $k$ directly leads to a relatively shorter HB lifetime. 
This result is related to our definition of the IHB, and it is the same as our expectation: 
the definition of interfacial H-bonds (\hbos) makes the HB break rate 
on the interface artificially increased. At the same time, the IMS retains the original rate constant of H-bonds, 
but it may include the contribution of bulk water molecules to the rate constant. 
%That is why the IMS slightly underestimate the $k$. 

In Fig.\thinspace\ref{fig:128w_itp_pure_water_pair_k_k_prime_ihb_both_schemes}, the $k$, $k'$, and $\tau_\text{HB}$ for \emph{bulk} water 
are also reported with dashed lines as a reference.
Comparing the above-mentioned quantities for the water/vapor interface and bulk water, 
we find that when $d$ is larger than $d_0$, 
no matter which statistical method is used, the calculated reaction rate constants of the interface water is \emph{greater} than that in bulk water. 
Therefore, the HB lifetime $\tau_\text{HB}$ at the water/vapor interface, 
is smaller than in bulk water.
\begin{figure}[H]
\centering
\includegraphics [width=1.05\textwidth] {./diagrams/128w_itp_pure_water_pair_k_k_prime_ihb_both_schemes}
\setlength{\abovecaptionskip}{0pt}
\caption{\label{fig:128w_itp_pure_water_pair_k_k_prime_ihb_both_schemes}Dependence of (a) the reaction rate constants $k$ and $k'$ 
and (b) the HB lifetime $\tau_\text{HB}$ on the interface thickness $d$, obtained by the IHB and the IMS method, respectively.
The corresponding $k$, $k'$ and $\tau_\text{HB}$ in bulk water are drawn with dashed lines as references.
In panel a, the $k$ of bulk water is represented by a \emph{black dashed} line, and the $k'$ of bulk water by a \emph{blue dashed} line;
in panel b, the $\tau_\text{HB}$ of bulk water by a \emph{black dashed} line.
The ADH criterion is used and the least square fits are carried on the time 
region 0.2 ps $< t <$ 12 ps.}
\end{figure}


Furthermore, we have found from Fig.\thinspace\ref{fig:128w_itp_pure_water_pair_k_k_prime_ihb_both_schemes} that as $d$ increases, 
the values of $k$ and $k'$ also tend to bulk values at the same condition.
These results are obtained by the least squares method in the same interval (0.2--12 ps). This verifies that the IHB method 
can get as good results as the IMS method when $d$ is larger than $d_0$. 
Because the IHB method is concise to operate, it can be used to calculate HB dynamics and thus HB lifetime for the water/vapor interface 
when $d$ is larger than $d_0$. For the water/vapor interface, $d_0$ is approximately $4$ \A \ or equals to the size of $\sim$2 layers of water molecules. 
This result coincides with the previous theoretical results based on MD simulation\cite{Townsend1985,Taylor1996,Morita2000} 
and VSFG spectroscopy\cite{Tyrode2013}, which have shown that isotropic properties are already recovered 
at sub-nanometer distances from the surface of ordered hydrophobic monolayers.
For example, Stiopkin and coworkers suggested a "healing length" of about 3 \AA with the bulk-phase properties of water recovered within the top few monolayers\cite{Stiopkin2011}.

Finally, because the real HB dynamical properties of interface molecules are between the results of the above two methods, 
we can approximate the interfacial HB dynamics, by either the IHB or the IMS method if the thickness of the interface is large enough, i.e., $d>d_0$.

%In summary, if we study the dynamics of H-bonds in a thin interface, we can use the method of molecular selection, 
%because the H-bonds obtained in this way are not artificially broken, and if the interface is thick enough  
%(see Fig.\thinspace\ref{fig:128w_itp_pure_water_pair_k_k_prime_ihb_both_schemes}a), then we can use the IHB method, because it can automatically define which H-bonds come 
%from the interface without the need to select the molecules at the interface layer.

%
To illustrate this point more clearly, we compare the $k$, $k'$, and $\tau_\text{HB}$ obtained under the two methods.
We listed more detailed data in Tables\thinspace\ref{tab:k_k_prime_tau_128w_pure_ihb_ADH} to \ref{tab:k_k_prime_tau_128w_pure_ihb_AHD}.
\begin{table}[H]%[htb]
\centering
\caption{\label{tab:k_k_prime_tau_128w_pure_ihb_ADH} 
    The $k$ and $k'$ for the interfacial HB dynamics of the water/vapor interface, through the IHB method, with the ADH criteria. 
We carried on the longer time region 0.2 ps $< t <$ 12 ps (same below). 
}
%The unit for $k$ ($k'$) is ps$^{-1}$, and for $\tau_{\text{HB}}$ ($=1/k$) is ps. 
\begin{tabular}{cccc}
 $d$ (\AA) & $k$ (ps$^{-1}$) & $k'$ (ps$^{-1}$)& $\tau_{\text{HB}} (=1/k)$ (ps) \\
\hline
  1.0 & 0.653 & 0.080 & 1.53  \\
  2.0 & 0.261 & 0.133 & 3.83  \\
  3.0 & 0.168 & 0.104 & 5.94  \\
  4.0 & 0.148 & 0.092 & 6.76  \\
  5.0 & 0.147 & 0.087 & 6.81  \\
  6.0 & 0.139 & 0.087 & 7.17  \\
\end{tabular}
\end{table}
\begin{table}[htb]
\centering
\caption{\label{tab:k_k_prime_tau_128w_pure_ihb_AHD} 
    The $k$ and $k'$ for the interfacial HB dynamics of the water/vapor interface through the IHB method, with the AHD criteria.} 
\begin{tabular}{cccc}
 $d$ (\AA) & $k$ (ps$^{-1}$) & $k'$ (ps$^{-1}$) & $\tau_{\text{HB}} (=1/k)$ (ps) \\
\hline
  1.0 & 0.661 & 0.080 & 1.51  \\
  2.0 & 0.265 & 0.133 & 3.77  \\
  3.0 & 0.172 & 0.102 & 5.82  \\
  4.0 & 0.148 & 0.090 & 6.74  \\
  5.0 & 0.149 & 0.084 & 6.72  \\
  6.0 & 0.144 & 0.078 & 6.93  \\
\end{tabular}
\end{table}

\begin{table}[H]
\centering
\caption{\label{tab:k_k_prime_tau_128w_pure_ihb_scheme1_ADH} 
    The $k$ and $k'$ for the interfacial HB dynamics of the water/vapor interface through the IMS method, with the ADH criteria.} 
\begin{tabular}{cccc}
 $d$ (\AA) & $k$ (ps$^{-1}$) & $k'$ (ps$^{-1}$) & $\tau_{\text{HB}} (=1/k)$ (ps) \\
\hline
  1.0 & 0.526 & 0.072 & 1.90  \\
  2.0 & 0.246 & 0.158 & 4.07  \\
  3.0 & 0.160 & 0.114 & 6.26  \\
  4.0 & 0.140 & 0.097 & 7.15  \\
  5.0 & 0.138 & 0.090 & 7.24  \\
  6.0 & 0.133 & 0.085 & 7.49  \\
\end{tabular}
\end{table}
%  6.0 & 0.125 & 0.080 & 8.00  \\
%  7.0 & 0.133 & 0.085 & 7.49  \\
\begin{table}[H]
\centering
\caption{\label{tab:k_k_prime_tau_128w_pure_ihb_AHD} 
    The $k$ and $k'$ for the interfacial HB dynamics of the water/vapor interface through the IMS method, with the AHD criteria.} 
\begin{tabular}{cccc}
 $d$ (\AA) & $k$ (ps$^{-1}$) & $k'$ (ps$^{-1}$) & $\tau_{\text{HB}} (=1/k)$ (ps) \\
\hline
  1.0 & 0.610 & 0.083 & 1.64  \\
  2.0 & 0.235 & 0.142 & 4.62  \\
  3.0 & 0.138 & 0.102 & 7.22  \\
  4.0 & 0.141 & 0.098 & 7.07  \\
  5.0 & 0.120 & 0.078 & 8.40  \\
  6.0 & 0.119 & 0.071 & 8.39  \\
\end{tabular}
\end{table}
%  6.0 & 0.117 & 0.072 & 8.58  \\
%  7.0 & 0.119 & 0.071 & 8.39  \\

\newpage
\section{Summary}
In this chapter, the HB population operator and associated correlation functions based on it are used for studying HB dynamics. 
The HB dynamics in bulk water can be calculated from these correlation functions: 
\CHB describes the relaxation of H-bonds; \SHB gives the average lifetime $\langle \tau_a \rangle$ of the continuous H-bonds. 
Using DFTMD simulations, starting from the functions \CHB, $n(t)$, and $k(t)$, 
we have calculated the reaction rate constants $k$ and $k'$ for HB rupture and regeneration, and the average HB lifetime $\tau_{\text{HB}}$ for bulk water.

We have also studied HB dynamics at \emph{instantaneous} water/vapor interfaces using the combination 
of two statistical methods, the IMS and the newly introduced IHB.
In the IHB method, correlation functions based on the instantaneous interfacial HB population operator, 
i.e., $c^{(\text s)}(t)$, $s^{(\text s)}(t)$, $n^{(\text s)}(t)$, and $k^{(\text s)}(t)$ are defined for the water/vapor interface.
Based on the instantaneous interface, the combination of the IMS and
IHB methods provides more realistic interfacial HB dynamics of the water/vapor interface.  
The instantaneous interface describes the microscopic interface more accurately, 
and both the IHB and IMS methods provide partial information on the HB breaking and reforming reaction rates for the interface. 
As the thickness of the interface increases, comparing the calculation results obtained from the IHB and IMS methods, respectively, 
we found that HB dynamics in the instantaneous layer \emph{below} the surface recovers bulk values. 
Therefore, the real HB dynamical characteristics at the water/vapor interface, such as the HB breaking and reforming rate constants and the HB lifetime, 
can be calculated from the combination of IMS and IHB methods. 

In particular, we have found that as the thickness of the interface layer increases, 
the HB reaction rate constants tends to the ones in bulk water.
From the results for the water/vapor interface, we conclude that from the perspective of HB dynamics,
the thickness of the water/vapor interface is 4 \A. This value is smaller than that obtained from the VSFG spectra 
(ref. Paragraph \thinspace\ref{sfg_ln}) for the \LiN/vapor interface. 
This result is reasonable, because the presence of ions in solutions affects the HB network. 

The idea of IHB can be extended to electrolyte solutions, and can be used to study ions' effects on HB dynamics.
We will analyze HB dynamics in electrolyte solutions in the next chapter.

%------------------------------------------
%NO Others
%..........................................
%\paragraph{Experiments on HB dynamics}
%An important  structural characteristic of the H-bonded network is the average number of H-bonds per molecule, $\langle h_{i,j}\rangle$. \cite{Chowdhary2008} 
%For bulk water systems, we find that in the DFTMD simulations the average number of H-bonds in the bulk phase is $\sim$ 4.35 which is slightly
%(higher) than the usual estimate of 3.4 (interface system) for SPC/E water.

%TODO
%For interfacial systems of neat water, we find the average number
%of hydrogen bonds is 3.XX which is slightly
%(lower/higher) than the usual estimate of 3.4 for SPC/E water. \cite{Chowdhary2008}

%%===============================================
%\section{Rotational Anisotropy Decay of Water at the Interface of Alkali-Iodine Solutions}\label{CHAPETR_AD}
%%===============================================
%Using the transition dipole auto-correlation function, 
%we determined the rotational anisotropy decay and therefore the OH-stretch relaxation at water/vapor interface of alkali iodide solutions.
%%The effects of ion environment on structure and dynamics of water are obtained by comparing the second-order Legendre polynomial, i.e.,  $P_2(x)=\frac{1}{2}(3x^2-1)$,  orientational correlation function of the transition dipole.
%The anisotropy decay can be determined from experimental signal in two different polarization configurations---parallel and perpendicular polarizations, by 
%\begin{equation}
%        R(t)=\frac{S_{\parallel}(t)-S_{\perp}(t)}{S_{\parallel}(t)+2S_{\perp}(t)}
%\label{eq:ad}
%\end{equation}
%where $t$ is the time between pump and probe laser pulses.  The anisotropy decay can also be obtained by simulations, 
%and calculated by the third-order response functions $R(t)$. \cite{Jansen10,Jansen06}
%%
%%In the first shell with a radius 3 \A, the entropy difference between the \Li shell and \Na shell,
%%$\Delta S=k_B\text{ln}\frac{\Omega_\text{Na}}{\Omega_\text{Li}}=k_B\text{ln}\frac{n_\text{Na}/V_\text{Na}}{n_\text{Li}/V_\text{Li}} =k_B\text{ln}1.05$.
%%
%%\paragraph{Probability Distribution of Ions}
%%The probability distribution, shown in Fig.~\ref{fig: prob_124_LiI_Sans_double_axis}, of the ions in the water/vapor interface of LiI and NaI solutions with respect to the depth of the ions in the solutions 
%%indicates that the \I ions prefer to staying at the topmost layer of surface of solutions.
%%(molar concentration: 0.9 M, temperature: 330 K) 
%%It shows that \I ions tend to the surface of the solutions, while \Na and \Li tend to stay in the bulk. This result is consistent with the calculations from Ishiyama and Morita\cite{TI07,TI14}.
%The orientational anisotropy $C_2(t)$ is given by the rotational time-correlation function 
%\begin{equation}
%C_2(t)=\langle P_2(\hat{u}(0)\cdot\hat{u}(t)) \rangle,
%\label{eq:tcf2}
%\end{equation}
%where $\hat{u}(t)$ is the time dependent unit vector of the transition dipole, $P_2(x)$ is the second Legendre polynomial, and $\langle \cdots \rangle$ indicate 
%equilibrium ensemble average. \cite{Corcelli05,LinYS2010} %\cite{2010Lin} % angular brackets
%
%The anisotropy decay $C_2(t)$ for the water/vapor interface of LiI solution is shown in Fig.\thinspace\ref{fig:c2_2LiI_16_inset}.
%This function decays faster than that of neat water, indicating that H-bonds
%at the interfaces of alkali-iodine solutions reorient faster than in neat water. The inset shows the first 0.4 ps of $C_2(t)$, from which we see a 
%quick change during the first $\sim 0.1$ ps primarily due to librations.
%%
%\begin{figure}[h]
%\centering
%\includegraphics [width=0.36\textwidth] {./diagrams/c2_2LiI_16_inset} 
%\setlength{\abovecaptionskip}{0pt}
%  \caption{\label{fig:c2_2LiI_16_inset} The time dependence of the $C_2(t)$ of OH bonds at the water/vapor interfaces of 0.9 M LiI solution 
%    and of neat water (dashed line) at 330 K, calculated by DFTMD simulations.} 
%    %The water/vapor interface of neat water is modeled 
%    %with a slab made of 121 water molecules in a simulation box of size $15.6 \times 15.6 \times 31.0$ \A$^3$.
%\end{figure}
%%
%We also calculated the $C_2(t)$ for the interface of other alkali-iodine solutions LiI and KI. 
%The results of $C_2(t)$ for the water/vapor interfaces of these solutions are shown in Fig.\thinspace\ref{fig:c2_2KI_2NaI_2LiI_16}.
%In all the cases $C_2(t)$ decays faster than in neat water, indicating that H-bonds
%at the interfaces of the three alkali-iodine solutions are orientated faster than that of neat water.
%They show that \I ions can accelerate the dynamics of molecular reorientation of water molecules at interfaces.   
%
%%
%\begin{figure}[htbp]
%\centering
%\includegraphics [width=0.36\textwidth] {./diagrams/c2_2KI_2NaI_2LiI_16} 
%\setlength{\abovecaptionskip}{0pt}
%  \caption{\label{fig:c2_2KI_2NaI_2LiI_16} The time dependence of the $C_2(t)$ of OH bonds in water molecules at the water/vapor 
%  interface of 0.9 M alkali-iodine solutions and of neat water (dashed line) at 330 K, calculated by DFTMD simulations.}
%\end{figure} 
%
%We have obtained non-single-exponential kinetics for the rotation of water molecules both at the surface 
%and in bulk water (Appendix \ref{single_exp}).
%%This result is true for water molecules bound to ions. 
%Therefore, the rotational motion of water molecules are not simply characterized by well-defined rate constants. 
%%Then the problem is to understand the kinetics.
%Similar non-single-exponential kinetics is also obtained in the HB kinetics
%in liquid water \cite{AL96,Dirama05} and in the time variation of the average frequency shifts of the 
%remaining modes after excitation in hole burning technique \cite{Rey2002,Moller2004} and using BLYP functional. \cite{Bankura2014}
%Luzar and Chandler interpreted 
%the non-single-exponential kinetics as the result of an interplay between 
%diffusion and HB dynamics. \cite{AL96} 
%We can understand the non-single-exponential kinetics of rotational 
%anisotropy decay by fitting the rotational anisotropy decay by a 
%biexponential function.
%
%To obtain the effects of diffusion and HB decay of water molecules
%in solutions respectively, we assume that there are two independent 
%decays in the process of an anisotropy decay. 
%Therefore, the $C_2(t)$ has the form \cite{TanHS05}
%\begin{equation}
%C_2(t)=A_1e^{-\kappa_1 t} +A_2e^{-\kappa_2 t},
%\label{eq:tcf3}
%\end{equation}
%where $A_i$ are constants and $\kappa_i$ are decay rates ($i=1, 2$). 
%The time constants and amplitudes of the biexponentials fits for 
%the $C_2(t)$ are listed in Tables ~\ref{tab:2LiI_c2_biexp} and ~\ref{tab:2NaI_c2_biexp}.
%The biexponential fit is close to the calculated $C_2(t)$, which can be seen in Fig.\thinspace\ref{fig:2LiI-124w_c2_fit_5ps_biexp} (compare Fig.\thinspace\ref{fig:2LiI-124w_c2_fit_5_single-exp}).
%%
%\begin{table}[hbt]
%\centering
%\caption{\label{tab:2LiI_c2_biexp}%
%	Biexponential fitting (5 ps) of the $C_2(t)$ for water molecules in 0.9 M LiI solution.}
%%\begin{ruledtabular}
%\begin{tabular}{lccccc}
%water molecules & $A_1$  & $\kappa_1$ (THz) & $A_2$ & $\kappa_2$ (THz) \\
%\hline
%I$^-$-shell & 0.44 & 0.25 & 0.39 & 0.26\\
%Li$^+$-shell & 0.88 & 0.07 & 0.07 & 8.24\\
%bulk & 0.84 & 0.11 & 0.09 & 4.88 \\
%surface & 0.73 & 0.27 & 0.22 & 13.47 \\
%\end{tabular}
%%\end{ruledtabular}
%\end{table}
%%--
%
%\begin{table}
%\centering
%  \caption{\label{tab:2NaI_c2_biexp}%
%	Biexponential fitting (5 ps) of the $C_2(t)$ for water molecules in 0.9 M NaI solution.}
%  \begin{tabular}{lccccc}
%  water molecules & $A_1$  & $\kappa_1$ (THz) & $A_2$ & $\kappa_2$ (THz) \\
%  \hline
%  I$^-$-shell & 0.86 & 0.14 & 0.08 &9.86 \\
%  Na$^+$-shell & 0.71 & 0.06 & 0.18 &0.79 \\
%  bulk & 0.81 & 0.06 & 0.10 & 1.25 \\
%  surface & 0.77 & 0.11 & 0.13 & 2.31 \\
%  \end{tabular}
%\end{table}
%%
%%图
%\begin{figure}[htbp]
%\centering
%\includegraphics [width=0.60\textwidth] {./diagrams/2LiI-124w_c2_fit_5_biexp} 
%  \caption{\label{fig:2LiI-124w_c2_fit_5ps_biexp} The time dependence of the $C_2(t)$ of OH bonds 
%  in water molecules at the water/vapor interface of LiI solution.}
%\end{figure} 
%%
%%[Notes: The 63-water-slab models is listed here as a reference. The number of water molecules is small; The data for KI/vapor and LiI/vapor interfaces come from  KI\_16 and LiI\_16 systems.  
%%Water(63) &0.831$\pm(1\times10^{-4})$ &  0.08760 $\pm(2\times 10^{-5})$&0.100$\pm(2\times10^{-4})$ & 1.029 $\pm(4\times10^{-3})$  \\ ]
%%
%%\begin{figure}[htbp]
%%\centering
%%\includegraphics [width=0.4 \textwidth] {./diagrams/c2_121-pure_2KI_2LiI_16_inset_fit_biexp} 
%%\setlength{\abovecaptionskip}{10pt}
%%\caption{\label{fig:c2_121-pure_2KI_2LiI_16_inset_fit_biexp} The fitted and calculated anisotropy decay of OH bonds in water molecules in LiI solution/vapor interface (red), LiI solution/vapor interface (blue) and neat water/vapor interface (black). The corresponding fitted functions are denoted by dashed lines. The concentration of LiI and KI solution is 0.9 M.}
%%\end{figure} 
%
%Then we considered the effect of ion species in solutions on the anisotropy decay of water molecules.
%From Tables \ref{tab:2LiI_c2_biexp} and \ref{tab:2NaI_c2_biexp}, we find that 
%for both LiI and NaI solutions, there are two decay processes in the dynamics --- amplitude $\sim 1$,
%decay constant $\sim$ 0.1 THz, and for the other describe the initial fast decay 
%of the anisotropy, with amplitude $\sim 0.1$, decay constant $\sim$ (1--10) THz, 
%due to the inertial-librational motion preceding the orientational diffusion.
%That is, two decay processes exist in the dynamics of water molecules 
%at the water/vapor interfaces of alkali-iodine solutions. 
%%The one describe the initial fast decay of the anisotropy, 
%%with amplitude $\sim$ 0.1, decay constant $\sim$ (1--10) THz,
%%results from the inertial-librational motion preceding the orientational diffusion.
%%%
%%\begin{table}[H]
%%\centering
%%\caption{\label{tab:fitting_c2_for_each_type_of_water}%
%%  Biexponentially fitting (2 ps) of the $C_2(t)$ for different types of water molecules at the water/vapor interface of LiI solutions.}
%%\begin{tabular}{lccccc}
%%water molecules & $A_1$  & $\kappa_1$ (THz) & $A_2$ & $\kappa_2$ (THz) \\
%%\hline
%%$DDAA$ & 0.85 & 0.25   & 0.10 & 16.0\\
%%$DD'AA$ & 0.89 & 0.14  & 0.06 & 14.1 \\
%%$D'AA$ & 0.38 & 0.99 & 0.38 & 0.99 \\
%%\end{tabular}
%%\end{table}
%%%
%%\begin{table}[H] %[!hbtp]
%%\centering
%%\caption{\label{tab:table_CoordNo}%
%%The coordination number of the atoms in LiI (NaI) solutions.}
%%\begin{tabular}{lccc}
%%name & radius of the first shell (\AA) & coordination number \\
%%\hline
%%$n_\text{I-H}(\text{LiI})$ & 3.3 & 5.5 \\
%%$n_\text{I-H}(\text{NaI)}$ & 3.3 & 5.1 \\
%%$n_\text{I-O}(\text{LiI)}$ & 4.3 & 5.8 \\
%%$n_\text{I-O}(\text{NaI)}$ & 4.3 & 6.0 \\
%%$n_\text{Li-O}(\text{LiI)}$ & 3.0 & 4.0 \\
%%$n_\text{Na-O}(\text{NaI)}$ & 3.5 & 6.0 
%%\end{tabular}
%%\end{table}

