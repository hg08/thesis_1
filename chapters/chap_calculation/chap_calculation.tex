\chapter{Calculation of Nonlinear Optical Susceptibilities}\label{calculation_of_chi}

\paragraph{Definitions and Relations}

1. Definition of double product of a $m$-order tensor $A$ and a $n$-order tensor $B$ is a tensor with order $m+n-2$.
\begin{equation}
    A:B=A_{ij}B_{lm}\delta_{jl}\delta_{im}.
\label{tensor_double_product}
\end{equation}

2. The components of product of $AB$ is defined by 
\begin{equation}
    (AB)_{ijlm}=A_{ij}B_{lm}.
\label{tensor_product}
\end{equation}

3. For vectors $\bf{a}$, $\bf{b}$, $\bf{c}$ and $\bf{d}$, ${\bf{ab}}:{\bf{cd}}=({\bf{a}}\cdot {\bf{d}})({\bf{b}}\cdot{\bf{c}})$.

Proof:
\begin{align}
    {\bf{ab}}:{\bf{cd}}&=({\bf{ab}})_{ij}({\bf{cd}})_{lm}\delta_{jl}\delta_{im} \nonumber \\
    &=({\bf{ab}})_{ij}({\bf{cd}})_{ji} \nonumber\\
    &=(a_i b_j)(c_j d_i) \nonumber\\
    &=({\bf{b}}\cdot{\bf{c}})({\bf{a}}\cdot{\bf{d}})
\end{align}

\paragraph{Proof of Eq.\space\ref{integral_identity1a}}
Since 
\begin{equation}
  \lim_{t\to\infty} |e^{-it(a-ib)}|\le \lim_{t\to\infty} |e^{-bt}| = 0,
  \label{integral_identity0}
\end{equation}
we can obtain
\begin{equation}
  \int_0^\infty dt e^{-it(a-ib)}=\frac{1}{b+ia},
  \label{integral_identity0}
\end{equation}
i.e.,
\begin{equation}
  i\int_0^\infty dt e^{-it(a-ib)}=\frac{1}{a-ib}.
  \label{integral_identity1}
\end{equation}
Set $a = (\omega_{v'} - \omega_{v}) - \omega$, and $b = \gamma_{v'v}$,
then we have
 \begin{align}
 \int_0^\infty dt e^{-it[(\omega_{v'}-\omega_v)-\omega-i\gamma_{v'v}]}=\frac{-i}{(\omega_{v'} -\omega_v)-\omega-i\gamma_{v'v}}.\nonumber
 \end{align}
\paragraph{Derivation of Eq.\space\ref{eq:chi}}
Now we give the derivation of the expression of $\chi^{\text{(2),R}}$
\begin{align}
   \chi^{\text{(2),R}}_{\eta\xi\kappa}&=\frac{-i}{k_{\text{B}}T \omega} \int_0^\infty dt e^{i \omega t}\left\langle \dot{A}_{\eta\xi}(t) \dot{M}_{\kappa}(0)\right\rangle
\end{align}
Derivation: 
The resonant term $\chi^{(2),R}_{\eta\xi\kappa}$ is given by \cite{Morita2008}
\begin{align}
  \chi^{\text{(2),R}}_{\eta\xi\kappa}&=\frac{i\omega}{k_{\text{B}}T} \int_0^\infty dt e^{i \omega t}\left\langle {A}_{\eta\xi}(t) {M}_{\kappa}(0)\right\rangle,
\end{align}

\paragraph{Molecular Dipole Moment and Dipole Polarizibility Derivatives} \label{calculate_derivatives} 

The polarizability tensor $\boldsymbol{\alpha}$ is defined by the relation
\begin{equation}
  \delta \boldsymbol{\mu} = \boldsymbol{\alpha} \boldsymbol{\mathscr{E}}
  \label{eq:def_alpha}
\end{equation}
where $\delta \boldsymbol{\mu}$ is the electric dipole moment (a vector) induced in the molecule by
the electric field $\boldsymbol{\mathscr{E}}$, with the components $\mathscr{E}_x$,$\mathscr{E}_y$ and $\mathscr{E}_z$.
Now we describe the main algorithm to implement the parametrization of the molecular dipole moment 
derivative $\frac{\partial \mu_k}{\partial r}$ and dipole polarizibility derivative $\frac{\partial\alpha_{\eta\xi}}{\partial r}$. 
This result can be used in the velocity ACF-based method for calculating the VSFG spectroscopy intensity.

Given a DFT MD trajectory of total length $\sim 10^2$ ps for bulk water, sampled with a frequency $\sim 1$ ps$^{-1}$.
For the $j$-th water molecule in the $n$-th snapshot of the trajectory, 
we denote the two OH bonds as H$^{n,j,\epsilon=1}$ and H$^{n,j,\epsilon=-1}$. We will calculate statistical average over all time steps and all OH bonds, therefore, 
we just denote the corresponding OH bonds by ${\epsilon=1}$ and ${\epsilon=-1}$, respectively. 
The H atoms in a water molecule are denoted by H$^{\epsilon=1}$ and H$^{\epsilon=-1}$, and the O atom by O$^{0}$.
%For the bond OH$_{\epsilon}$, we alculate vector OH$_{\epsilon}$, and $|\text{OH}_{\epsilon}|$ in the lab framework,
%then the direction cosines ($\cos\alpha_{\epsilon}$, $\cos\beta_{\epsilon}$, $\cos\gamma_{\epsilon}$) of the vector OH$_{\epsilon}$ in the molecular frame MF$_{n,j}$.
%The size of the array to store the direction cosines is $ 40\times 32\times 3 \times 3$.
%Calculate the direction cosine matrix between bond framework and molecular framework. 

We used three different frameworks: the lab framework ($x^{\text{l}},y^{\text{l}},z^{\text{l}}$), the molecular framework
($x^{\text{m}},y^{\text{m}},z^{\text{m}}$) and the bond framework ($x^{\text{b}},y^{\text{b}},z^{\text{b}}$) (see Fig.\space\ref{fig:frameworks}).
In the lab framework, the $z^{\text{l}}$-axis is perpendicular to the interface.
The molecular frame will be used to decompose the signal into normal modes of water monomers.
For the $j$-th molecule, the $z^{\text m}$ axis is along the bisector of the H-O-H angle, the $x^{\text m}$ axis is in the molecular plane,
and the $y^{\text m}$ axis is out of the molecular plane. 
In the bond framework, $z^{\text{b},\epsilon}$ axis is along the bond $\epsilon$ of a molecule, $z^{\text{b},\epsilon}$
is in the molecular plane and $y^{\text{b},\epsilon}$ is out of the molecular plane.

There are two direction cosine matrices between the bond frameworks and molecular framework,
we name them as ${\bf D}^{\text{b},\epsilon=-1}$ and  ${\bf D}^{\text{b},\epsilon=1}$, \cite{Khatib2017}
or ${\bf D}^{\text{b},-1}$ and  ${\bf D}^{\text{b},1}$ for short. 
Then the direction matrix ${\bf D}^{\text{b},\epsilon} $ can be represented by direction cosines between ${\bf x}^{\text{b},\epsilon}$ and ${\bf x}^m$, 
where $\epsilon=\pm 1$ and $\theta$ denotes the H-O-H angle in the $j$-th water molecule for the $n$-step:
\begin{subequations}
\begin{align}
  &\hat x^{\text{b},\epsilon}_1 = \epsilon \cos\frac{\theta}{2} \hat x^m_1 +  \sin\frac{\theta}{2} \hat x^m_3 \\
  &\hat x^{\text{b},\epsilon}_2 = \epsilon \hat x^m_2 \\
  &\hat x^{\text{b},\epsilon}_3 = -\epsilon \sin\frac{\theta}{2} \hat x^m_1 + \cos\frac{\theta}{2} \hat x^m_3
\end{align}
\end{subequations}
i.e., 
\begin{equation}
  {\bf D}^{\text{b},\epsilon} =\left(
  \begin{matrix}
    \epsilon\text{cos}\frac{\theta}{2} &  0  & \text{sin}\frac{\theta}{2}\\
    0 & \epsilon & 0\\
    -\epsilon\text{sin}\frac{\theta}{2} & 0 & \text{cos}\frac{\theta}{2}
  \end{matrix}
  \right).
\end{equation}
  The molecular framework is given by the direction cosine matrix ${\bf D}^\text{m,l}$ (or ${\bf D}^\text{m}$) between molecular framework  and the lab framework
  %How to determine  $({\bf D}^\text{m,l})$?  
%Since we can know the coordinates of $(\hat x^m_3)$ in the lab framework: 
%\begin{equation}
%(\hat x^m_3) = (a_{31}) \hat x^l_1 + (a_{32}) \hat x^l_2 + (a_{31})_{nj} \hat x^l_3,
%\end{equation}
%Therefore, the realation between the unit vecotor  $(\hat x^m_1)$ and $(\hat x^m_2)$ are
%
\begin{subequations}
\begin{align}
  \hat {\bf x}^m ={\bf D}^\text{m} \hat{\bf x}^l.
\end{align}
\end{subequations}
%i.e., we obtain  $({\bf D}^\text{m})$.

% 2a1c.
%\begin{subequations}
%  \begin{align}\left(
%  \begin{matrix}
%  \{\bf D} x^{\epsilon} \\
%  \Delta y^{\epsilon} \\
%  \Delta z^{\epsilon} 
%  \end{matrix}
%    \right)
%    =\Delta r \left(
%  \begin{matrix}
%  \cos\alpha^{\epsilon}\\
%  \cos\beta^{\epsilon}\\
%  \cos\gamma^{\epsilon}
%  \end{matrix}
%    \right)
%\end{align}
%\end{subequations}
%i.e., $\Delta {\bf r}^{\epsilon} =(\Delta x^{\epsilon}, \Delta y^{\epsilon} ,\Delta z^{\epsilon} )^\text{T}$.
%        
%2a2.
%\begin{subequations}
%\begin{align}
%  &   (x'_H)_{\epsilon} -(x_O)_{0} = (x_H)_{\epsilon} -(x_O)_{0} + \Delta  x_{\epsilon} \\
%  &   (y'_H)_{\epsilon} -(y_O)_{0} = (y_H)_{\epsilon} -(y_O)_{0} + \Delta  y_{\epsilon} \\
%  &   (z'_H)_{\epsilon} -(z_O)_{0} = (z_H)_{\epsilon} -(z_O)_{0} + \Delta  z_{\epsilon}
%\end{align}
%\end{subequations}
%
%2a4.We need to calculate the dipole moment of each molecule for $\Delta r \in [-0.05,0.05]$. 
The dipole moment of each OH bond with different length is requird to determine the dipole moment derivative. 
    %(For convionient, we have defined $ \text{inc} =\Delta r/|\Delta r|$.)
    %We elongate (reduce) the bond OH$_{n, j,\epsilon}$, and keep other bonds still, then we obtain a updated coordinate file ${\text{pos}}nj\epsilon\text{inc}.\text{xyz}$, where $(n, j,\epsilon)$ is the index of the H in OH$_{\epsilon}$.
Therefore, we elongate (reduce) one bond ${\epsilon}$ by $\Delta r = 0.05$ \AA ($\Delta r = -0.05 $ \A), and keep other 
bonds in the total system still, then we obtain a updated coordinate.
Then the MLWF centers for the system can be calculated from the updated coordinate, using force and energy calculation at the DFT level.
%which is represented by the coordinate file ${\text{pos}}nj\epsilon\text{inc}.\text{xyz}$.
    %2a5a: before next step, we generated a template input file 1.inp. 
    %2a5b: From the template input file 1.inp, we generate an input file for the disturbed position represented by ${\text{pos}}nj\epsilon\text{inc}.\text{xyz}$ with the elongated bond OH$_{\epsilon}$: $n j \epsilon\text{inc}.\text{inp}$
    %2a5c: Using DFTMD, we use the input file $nj\epsilon\text{inc}.\text{inp}$ to calculate and write out the Wannier centers for the system  ${\text{pos}}n j \epsilon\text{inc}.\text{xyz}$. The output: ${\text{HOMO}}n j \epsilon \text{inc}.\text{xyz}$.
    %2a6: From the Wannier centers of the water molecule (can be found in the output ${\text{HOMO}}nj \epsilon \text{inc}.\text{xyz}$), we can calculate the dipole moment for the elongated (reduced) bond OH$_{\epsilon}$.  The $k$ component of the dipole moment: $(\mu^{b,r+\Delta r}_k)_{\epsilon}$.
    From the Wannier centers of the $j$-th water molecule, we can calculate the dipole moment for the elongated (reduced) 
   % MAYBE I DO NOT NEED TO DO BOTH ELONGATING AND REDUCING THE BOND,IE.,I JUST TAKE $\Delta r=0.05$\AA OR $\Delta r=0.05 $\AA IS ENOUGH!) 
    bond ${\epsilon}$. 
    %Therefore, the $k$ component of the dipole moment $(\mu^{b,r+\Delta r}_k)_{\epsilon}$ can be obtained.
    In the bond frame, $|\mu^b| =|\mu^b_z|$. 
    Therefore, we can calculate the $k=z$ component of the dipole moment for ${\epsilon}$ in water molecule $j$ from the MLWF centers. \cite{Silvestrelli1999} 
      The MLWF centers are computed  and the partial dipole moment for a given molecular species $I$ is defined as \cite{Salanne08}
      %
      \begin{equation}
        \mu^I = \sum_{i\in I}(Z_i {\bf R}_i - 2\sum_{n\in i} {\bf r}^w_n).
      \end{equation}
       %
In particular, here it is expressed as 
%\begin{equation}
%  \mu^{\text{b},r+\Delta r,\epsilon} = \frac{1}{2}Z_O {\bf R}^{0} +Z_H{\bf R}^{\epsilon} +1\times (-2) \times r^{w,\epsilon} +\frac{1}{2}\times 2\times (-2)\times r^{w,0},
%\end{equation}
\begin{equation}
  \mu^{\text{b},r+\Delta r,\epsilon} = \frac{1}{2}Z_{\text{O}} {\bf R}^{0} +Z_{\text{H}}{\bf R}^{\epsilon} -2r^{w,\epsilon} -2r^{w,0},
\end{equation}
where $\epsilon=\pm 1$.
%3c: 
In the $\epsilon$ frame, the $k=z$ component dipole moment derivative with respect to bond length \cite{Wilson1955} for the single OH bond $\epsilon$ in water molecule $j$ is
        %
        \begin{equation}
          \frac{\partial \mu^{\text{b},\epsilon}}{\partial r} = (\mu^{\text{b},r+\Delta r,\epsilon}-\mu^{\text{b},r,\epsilon})/\Delta r.
        \end{equation}
       % for both the case of elongating the OH bond, i.e., $\Delta r > 0$ and reducing the OH bond.
        %3d
       % We represent this vector $\frac{\partial \mu^{\text{b},\epsilon}}{\partial r}$ as $( 0, 0, \frac{\partial \mu^{\text{b},\epsilon}}{\partial r} )^\text{T}$.
        Since the components of the dipole moment in the molecular framework are given by the transformation:
        \begin{subequations}
          \begin{align}
            \left(
            \begin{matrix}
              (\frac{\partial \mu^\text{m}}{\partial r})_1\\
              (\frac{\partial \mu^\text{m}}{\partial r})_2\\
              (\frac{\partial \mu^\text{m}}{\partial r})_3
            \end{matrix}
            \right)
            = {\bf D}^{\text{b},\epsilon}
            \left(
            \begin{matrix}
              0\\
              0\\
              \frac{\partial \mu^{\text{b},\epsilon}}{\partial r}
            \end{matrix}
            \right),
            \end{align}
        \end{subequations}
    % END of r-loop 
    then, to calculate the individual polarizability for a OH bond from Wannier centers, 
    calculations involving finite electric fields (of 0.0001 au intensity) were performed independently 
    along $x$, $y$, and $z$ directions at each sampled time step. \cite{sulpizi2013}
        %4a: 
    For the electric field $\mathscr{E} \in \{\mathscr{E}_x,\mathscr{E}_y, \mathscr{E}_z\}$,
            %similar to 2a5a, but add an electric filed $ \varepsilon$. %The template input file is: 2.inp.
            %4a1: from the template input file, generate a input file $nj\epsilon\varepsilon.\text{inp}$, where if extra electric filed $\varepsilon=\varepsilon_x$, that means we add a electric field along $x$ axis. Similar for $\varepsilon_y$ and $\varepsilon_z$.
            %4a2: similar to 2a5c, for the position file ${\text{pos}}nj\epsilon.\text{xyz}$, the output Wannier centers are in   ${\text{HOMO}}nj\epsilon\varepsilon.\text{xyz}$.
       like in the case of no external electric field, the MLWF centers are calculated. %in ${\text{HOMO}}nj\epsilon\varepsilon.\text{xyz}$.
 For a finite $\Delta r$, the dipole moment is given by 
%\begin{equation}
%  \mu^{b,r+\Delta r,\epsilon,\varepsilon} = Z_H{\bf R}^{\epsilon,\varepsilon} + 
%  \frac{1}{2}Z^0 {\bf R}^{0,\varepsilon} + 
%  1\times (-2) \times {\bf r}^{\text{w},\epsilon,\varepsilon} +\frac{1}{2}\times 2\times (-2)\times ({\bf r}^\text{w})^{0,\varepsilon}
%\end{equation}
\begin{equation}
  \mu^{b,r+\Delta r,\epsilon,\mathscr{E}} = Z_H{\bf R}^{\epsilon,\mathscr{E}} + 
  \frac{1}{2}Z^0 {\bf R}^{0,\mathscr{E}} -2 {\bf r}^{\text{w},\epsilon,\mathscr{E}} -2{\bf r}^{\text{w},0,\mathscr{E}}.
\end{equation}
        %END $ \Delta r$-loop  
%4a4 the polarizability tensors $\alpha^{b,r+|\Delta r|}$ and $\alpha^{b,|r-\Delta r|}$. 
From the relation (obtained from Eq.\space\ref{eq:def_alpha})
      \begin{subequations}
          \begin{align}
            & 0 = \alpha^{b,r+\Delta r}_{11}\mathscr{E}_{1} + \alpha^{b,r+\Delta r}_{12}\mathscr{E}_{2} + \alpha^{b,r+\Delta r}_{13}\mathscr{E}_{3}\\
            & 0 = \alpha^{b,r+\Delta r}_{21}\mathscr{E}_{1} + \alpha^{b,r+\Delta r}_{22}\mathscr{E}_{2} + \alpha^{b,r+\Delta r}_{23}\mathscr{E}_{3}\\
            & \delta \mu^{b,r+\Delta r}_{3} = \alpha^{b,r+\Delta r}_{31}\mathscr{E}_{1} + \alpha^{b,r+\Delta r}_{32}\mathscr{E}_{2} + \alpha^{b,r+\Delta r}_{33}\mathscr{E}_{3}.
          \end{align}
      \end{subequations}
%
where
\begin{equation}
  \delta \mu^{b,r+\Delta r}_{3} = \mu^{b,\epsilon,r+\Delta r,\mathscr{E}}_{3} - \mu^{b,\epsilon,r+\Delta r}_{3}.
\end{equation}
%
and the expressions of the electric filed ${\mathscr{E}^\text{b}}$ in a OH framework for the 3 cases of external eletric field 
which is along $x$, $y$ and $z$ axis in the lab framework, respectively, we obtain 9 equations.
%
For
\begin{equation}
  {\bf\mathscr{E}}^{\text{l}} = (\mathscr{E}_0, 0,0)^T \nonumber
\end{equation}  
where $\mathscr{E}_0 = 0.0001$ au, we can obtain the intensity of the external electric filed in the molecular framework  
\begin{subequations}
  \begin{align}
    &\mathscr{E}^\text{m}_1 = {\bf D}^{\text{m}}_{12} \mathscr{E}^\text{l}_2\\
    &\mathscr{E}^\text{m}_2 = {\bf D}^{\text{m}}_{22} \mathscr{E}^\text{l}_2\\
    &\mathscr{E}^\text{m}_3 = {\bf D}^{\text{m}}_{32} \mathscr{E}^\text{l}_2,
    \end{align}
\end{subequations}
where ${\bf D}^{\text{m}}_{pq}$ is the $pq$-component of ${\bf D}^\text{m}$.
%
In OH bond framework,  
\begin{equation}
  \mathscr{E}^{\text b} = {\bf D}^{\text b} \mathscr{E}^{\text{m}}.
\end{equation}
Similarly, we obtain similar (but different) expansions of the intensity of the electric field for the other two cases: 
when $\mathscr{E}^{\text{l}} = (0,\mathscr{E}_0, 0)^\text{T}$ and when $\mathscr{E}^{\text{l}} = (0,0,\mathscr{E}_0)^\text{T}$, respectively.
Here, $\mathscr{E}_x$ is the electric field along $x$-axis in the lab frame. 
    %4a5: 
    %If we just elongate each single bond, ie., let $\Delta r > 0$,
    Then the dipole polarizability for the bond ${\epsilon}$ is as follows:
\begin{subequations}
  \begin{align}
    &\frac{\partial \alpha^{b,\epsilon}_{31}}{\partial r} = (\alpha^{b,\epsilon,r+\Delta r}_{31} -\alpha^{b,\epsilon,r}_{31})/\Delta r\\
    &\frac{\partial \alpha^{b,\epsilon}_{32}}{\partial r} = (\alpha^{b,\epsilon,r+\Delta r}_{32} -\alpha^{b,\epsilon,r}_{32})/\Delta r\\
    &\frac{\partial \alpha^{b,\epsilon}_{33}}{\partial r} = (\alpha^{b,\epsilon,r+\Delta r}_{33} -\alpha^{b,\epsilon,r}_{33})/\Delta r.
  \end{align}
\end{subequations}
    %If we just reduce each single bond, ie., let $\Delta r < 0$, then the dipole polarizability for the bond ${\epsilon}$ is as follows:
%\begin{subequations}
  %\begin{align}
   % &\frac{\partial \alpha^{b,\epsilon}_{31}}{\partial r} = (\alpha^{b,\epsilon,r}_{31} -\alpha^{b,\epsilon,r+\Delta r}_{31})/|\Delta r|\\
   % &\frac{\partial \alpha^{b,\epsilon}_{32}}{\partial r} = (\alpha^{b,\epsilon,r}_{32} -\alpha^{b,\epsilon,r+\Delta r}_{32})/|\Delta r|\\
   % &\frac{\partial \alpha^{b,\epsilon}_{33}}{\partial r} = (\alpha^{b,\epsilon,r}_{33} -\alpha^{b,\epsilon,r+\Delta r}_{33})/|\Delta r|.
  %\end{align}
%\end{subequations}
          %END 4a
      %END $j$-loop
%END $n$-loop

Therefore, the average for $(\frac{\partial \mu^{\text{b},\epsilon}}{\partial r})_\kappa$ and $(\frac{\partial \alpha^{\text{b},\epsilon}}{\partial r})_{\eta\xi}$ 
(The subscripts $\kappa, \eta, \xi = x^m, y^m, z^m$, or $1, 2, 3$) over all OH bonds gives the molecular dipole and polarizability derivatives. 

