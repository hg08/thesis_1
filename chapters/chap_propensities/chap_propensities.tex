\chapter{Propensities of ions}\label{propensities_of_ions} 
\section{Free energy of the water separated and the contact ion pair}\label{calculate_free_energy} 
From the blue-moon ensemble method\cite{Carter1989,Sprik1998}, we can obtain the constraint force (unit: a.u.force) acting on the atoms. 
If the distance between the ion pair (unit: \A) is chosen as the reaction coordinate, the formula for calculating the free energy (unit: kcal/mol) is given as follows.
%First, note that
%\begin{equation}
%  1\ \text{a.u.force} =8.2387\times 10^{-8}\ \text{N}\nonumber;
%\label{eq:au2n}
%\end{equation}
%\begin{equation}
%  1\ \text{\A} = 10^{-10}\ \text{m} \nonumber;
%\label{eq:aa}
%\end{equation}
%\begin{equation}
%  1\ \text{J} = 1.44\times 10^{20}\ \text{ kcal/mol}.\nonumber
%\label{eq:j2kcpm}
%\end{equation}
The relative free energy is given by
%\begin{equation}
%  F = \sum_{i}^{N}f_i{\Delta{r}} \ \text{ a.u.energy},
%  \label{eq:f-e}
%\end{equation}
\begin{equation}
  F = \sum_{i}^{N}f_i{\Delta{r}},\nonumber
  \label{eq:f-e}
\end{equation}
where $i$ denote a point on the one-dimensional reaction coordinate, 
$N$ is the number of the sampling points of the reaction coordinates,
and $f_i$ denotes the average force on atoms over the trajectory when $i$ is fixed. 
%Furthermore,
%\begin{equation}
%  \begin{split}
%  &F = \sum_{i}^{N}f_i{\Delta{r}}\times 8.2387\times10^{-18}\text{ J}\\
%  &= \sum_{i}^{N}f_i{\Delta{r}}\times 8.2387\times10^{-18}\times1.44\times10^{20}\text{ kcal/mol}.
%  \end{split}
%\label{eq:f-e2}
%\end{equation}
Now we estimate the error of the free energy $\delta{F}$ from the summation approximation. It reads
\begin{equation}
  \delta{F} = \frac{1}{N}\sum_{i}^{N}\delta{f_i}{\Delta{r}}.
  \label{eq:dleta_f}
\end{equation}
Usually, $\delta{f_i}\approx\delta{f}$, thus
\begin{equation}
  \delta{F} = \frac{1}{N}\delta{f}\sum_{i}^{N}{\Delta{r}}.
\label{eq:dleta_f-2}
\end{equation}
Particularly, if $\Delta{r}= 0.2$ \AA, $\delta{f}=0.0075\ \text{a.u.force}$, we get 
%\begin{equation}
%\begin{split}
%  &\delta{F} = \frac{0.0075\times\sum_{i}^{N}237}{N}\ \text{kcal/mol}\\
%  &          \approx 1.78\ \text{kcal/mol}.
%\end{split}
%\label{eq:dleta_f-3}
%\end{equation}
\begin{equation}
\begin{split}
  &\delta{F} \approx 1.78\ \text{kcal/mol}.\nonumber
\end{split}
\label{eq:dleta_f-3}
\end{equation}
%  &= 0.0075\times237\ \text{kcal/mol}\

%\section{Ion--surface distance}
%The ion--surface distances for the aqueous solution/air interfaces are given below. 
%
%Figure \thinspace\ref{fig:dist_K_surf1_I_surf1} shows the time dependence of K$^+$--surface and I$^-$--surface distances. 
%\begin{figure}[H]
%\centering
%\includegraphics [width=0.6 \textwidth] {./diagrams/dist_K_surf1_I_surf1}
%\setlength{\abovecaptionskip}{0pt}
%\caption{\label{fig:dist_K_surf1_I_surf1} Time dependence of K$^+$--surface and I$^-$--surface distances for KI solution.} %, through the trajectory of 20 ps.
%%$15.7787 \times 15.7787 \times 31.5574$ \AA$^3$
%\end{figure}

\section{Surface tension increment}\label{surface_tension_increment}
The surface tension increment $d\gamma/{dm_2}$ is derived as follows.
At constant temperature $T$ and constant pressure $p$, the Gibbs-Duhem equation is\cite{Pegram2006}:
\begin{equation}
Ad\gamma+n_1^{\sigma}d\mu_1 + {n_+^{\sigma}d\mu_+} + {n_-^{\sigma}d\mu_-} =0,
\label{eq:GD}
\end{equation}
where water is component 1, and the chemical potential for the ionic species $i(i=+,-)$ is defined as 
\begin{equation}
\mu_i =\mu_i^0+RT\ln{a}_i+z_iF\phi,
\label{eq:GDb}
\end{equation}
where $F$ is Farady constant, $\phi$ is the electrical potential of this region (surface or bulk), and $z_i$ is the valence of species $i$, with ionic 
activity $a_i=f_im_i$ in that region, where $f_i$ is the activity coefficient, and $m_i$ is the surface or bulk molality of the ion.

Differentiate Eq.\thinspace\ref{eq:GD} with respect to $m_2$, then
the surface tension increment is
\begin{equation}
\frac{d\gamma}{dm_2} = -\frac{1}{m_1^*}\frac{n_1^{\sigma}}{A} [(m_+^{\sigma} -m_+^{b})\frac{d\mu_+}{dm_2} + (m_-^{\sigma} -m_-^b)\frac{d\mu_-}{dm_2}],
\label{eq:h}
\end{equation}
where $m_1^*$ is the solvent molality. 

For species $i$, distribution coefficient $K_{p,i}$, is defined as
\begin{equation}
K_{p,i}= \frac{m_i^{\sigma}}{m_i^b} \approx \frac{m_i^{\sigma}}{\nu_i m_2}.
\label{eq:i}
\end{equation}

\begin{equation}
\frac{d\mu_{\pm}}{dm_2} =\frac{RT}{m_2} (1+\epsilon_{\pm}^b) + z_{\pm}F\frac{d\phi^{b}}{dm_2} \approx \frac{RT}{m_2} (1+\epsilon_{\pm}^b) ,
\label{eq:i1}
\end{equation}
where $\epsilon_{\pm}^b=d($ln$f_{\pm}^b)/d($ln$m_2)$. 

The ion partition coefficients $K_{p,2}$ is defined:
\begin{equation}
\nu K_{p,2}= \nu_+ K_{p,+} + \nu_- K_{p,-},
\label{eq:j}
\end{equation}
where $\nu_i$ is ions' stoichiometry.

%I can obtain eq.(~\ref{eq:p}) 

In general, from Eq.s \thinspace\ref{eq:h}--\thinspace\ref{eq:i1}, for solutions at sufficient low electrolyte concentrations, the surface tension increment is 
\begin{equation}
\frac{d\gamma}{dm_2} = -\frac{1}{m_1^*}\frac{n_1^{\sigma}}{A}\sum_{i=\pm}(m_i^{\sigma} -m_{i}^{b})\frac{d\mu_i}{dm_2} \nonumber
\label{eq:gamma_m2}
\end{equation}\
\begin{equation}
= -\frac{RT\nu}{m_1^*}\frac{n_1^{\sigma}}{A}\sum_{i=\pm} \frac{{\nu_i}(K_{p,i}-1)(1+\epsilon_i^b)}{\nu} \nonumber
\label{eq:l}
\end{equation}\
\begin{equation}
= -\frac{RT\nu}{m_1^*}\frac{n_1^{\sigma}}{A}[(K_{p,2}-1)+\sum_{i=\pm} \frac{(K_{p,i}-1)\nu_i\epsilon_i^b}{\nu}]
\label{eq:n}
\end{equation}\
Under the condition $\epsilon_{\pm}^b << 1$ or $\epsilon_{+}^b \approx \epsilon_{-}^b$
%$1+\epsilon_{\pm}^b$ can be extracted as a common factor,
the surface tension increment is 
\begin{equation}
\frac{d\gamma}{dm_2} 
= -\frac{RT\nu}{m_1^*}\frac{n_1^{\sigma}}{A} (1+\epsilon_i^b)^0 [\frac{\nu_+ K_{p,+} + \nu_- K_{p,-}}{\nu}-1].
\label{eq:p}
\end{equation}
The approximations in Eq.\thinspace\ref{eq:p} regarding the self-interaction {nonideality terms} $\epsilon$ for the salt component, 
for the single ions, and as a geometric mean of the single ion terms, are given in Ref.\cite{Pegram2006}.  
Typically the mean ionic version of $\epsilon$ is of order $\sim$ 0.1, so smaller in magnitude than 1. %(Thomas Record)
Here, \emph{nonideality} means the activity coefficient $f_i$ is not much smaller than 1.
%
%TODO: answer this quesntion 1
%\paragraph{Question 1}
%Is there an approximation required to obtain eq.(~\ref{eq:i1}) from Eq. \ref{eq:GDb}? 
%Since expression $\nu_{\pm}+\epsilon_{\pm}^b$ or $\frac{d (lnm_i^{\sigma})}{d (ln m_2)}+\epsilon_{\pm}^b$ rather than $1+\epsilon_{\pm}^b$ included eq.(~\ref{eq:i1}) is obtained.
%
%Since $\frac{d\mu_i^0}{dm_2} = 0$ and ln$(m_if_i) =$ln$f_i +$ln$m_i$. From eq.(~\ref{eq:GDb}),
%for bulk,  $m_i= \nu_i m_2$, then
%\begin{equation}
%\frac{d\mu_{\pm}}{dm_2} = \frac{RT}{m_2} (\nu_{\pm}+\epsilon_{\pm}^b) + z_{\pm}F\frac{d\phi^{b}}{dm_2} \approx \frac{RT}{m_2} (\nu_{\pm}+\epsilon_{\pm}^b) \nonumber,
%\label{eq:i4}
%\end{equation}
%where $\epsilon_{\pm}^b=: d($ln$f_{\pm}^b)/d($ln$m_2)$. 
%
%For surface, $m_i= m_i^{\sigma}$, then
%\begin{equation}
%\frac{d\mu_{\pm}}{dm_2}=\frac{RT}{m_2}[\frac{d(\ln{m}_i^{\sigma})}{d (\ln{m}_2)}+\epsilon_{\pm}^b] + z_{\pm}F\frac{d\phi^{b}}{dm_2} \approx\frac{RT}{m_2}[\frac{d (\ln{m}_i^{\sigma})}{d (\ln{m}_2)}+\epsilon_{\pm}^b].\nonumber
%\label{eq:i3}
%\end{equation}
%


%\section{Gibbs surface excess}\label{gibbs_surface_excess}
%\paragraph{Gibbs isotherm}
%The Gibbs adsorption isotherm for multicomponent systems is an equation used to relate the changes in concentration 
%of a component in contact with a surface with changes in the surface tension. For a binary system containing two components, 
%the Gibbs adsorption equation (\emph{adsorption isotherm}) in terms of surface excess is:
%\begin{eqnarray}
%\label{1}
%-\mathrm{d}\gamma\ = \Gamma_1\mathrm{d}\mu_1\, + \Gamma_2\mathrm{d}\mu_2\,
%\end{eqnarray}
%where
%$\gamma$ is the surface tension,
%$\Gamma_i$ is the surface excess of component $i$,
%$\mu_i$ is the chemical potential of component $i$.
%%The $T$ is a constant here, this equation has the name \emph{adsorption isotherm}.
%
%\emph{Remark 1}:
%%There is an important metaphor. 
%%Since pressure, temperature, surface tension, chemical potential are all intensive variables, 
%%we can make analogies between $\mathrm{d}\gamma/\mathrm{d}\mu_i$  and $\mathrm{d}P/\mathrm{d}T$. 
%For a general chemical equilibrium
%\begin{eqnarray}
%\label{c1}
%\alpha A +\beta B \cdots \rightleftharpoons \rho R+\sigma S \cdots
%\end{eqnarray}
%the thermodynamic equilibrium constant can be defined such that, at equilibrium,
%\begin{eqnarray}
%\label{c2}
%K^\ominus =\frac{{\{R\}} ^\rho {\{S\}}^\sigma \cdots } {{\{A\}}^\alpha {\{B\}}^\beta \cdots}
%\end{eqnarray}
%where curly brackets denote the thermodynamic activities of the chemical species and 
%${\{R\}}^i= \{R\}\cdot\{R\}\cdot \cdots$. This expression can be derived by considering the Gibbs free energy 
%change for the reaction. If deviations from ideal behavior are neglected, the activities may be replaced by concentrations,
%[A](or $m_\text{A}$), and a concentration quotient
%\begin{eqnarray}
%\label{c3}
%K_\text{c}=\frac{{[R]} ^\rho {[S]}^\sigma \cdots } {{[A]}^\alpha {[B]}^\beta \cdots}.
%\end{eqnarray}
%But $K_\text{c}$ is a little different quantity which describe the \emph{distribution} of products and reactants in a chemical reaction, 
%neglecting distribution of different products or different reactants.
%
%\emph{Remark 2}: The thermodynamic activities of a chemical species is a more general concept than the concept of concentration:
%\begin{eqnarray}
%\label{c4}
%a_i= f_im_i.
%\end{eqnarray}
%
%\emph{Remark 3}: Notice that $f(x)=\ln x$ is a monotonically increasing function, thus $\epsilon_i^b=\mathrm{d}\ln f_i^{b}/\mathrm{d}\ln m_2$ 
%describe the same thing as $\mathrm{d}f_i^{b}/\mathrm{d}m_2$, i.e., the dependence of the activity coefficient on the 
%concentration for the chemical species $i$. However, using $\ln f_i^{b}$ and $\ln m_2$ is more convenient.
%
%Different influences at the interface may cause changes in the composition of the near-surface layer. 
%Substances may either accumulate near the surface or, conversely, move into the bulk. The movement of the molecules 
%characterizes the phenomena of adsorption. Adsorption influences changes in surface tension and colloid stability. 
%Adsorption layers at the surface of a liquid dispersion medium may affect the interactions of the dispersed particles 
%in the media and consequently these layers may play crucial role in colloid stability The adsorption of molecules of 
%liquid phase at an interface occurs when this liquid phase is in contact with other immiscible phases that may be gas, liquid, or solid.
%
%Surface tension describes how difficult it is to extend the area of a surface (by stretching or distorting it).
%If surface tension is high, there is a large free energy required to increase the surface area,
%so the surface will tend to contract and hold together like a rubber sheet.
%There are various factors affecting surface tension, one of which is that the composition of the surface may
%be different from the bulk. For example, if water is mixed with a tiny amount of surfactants (eg., hand soap, or C$_{17}$H$_{35}$COONa),
%the bulk water may be 99\% water molecules and 1\% soap molecules, but the topmost surface of the water may be 50\% water molecules
%and 50\% soap molecules.(C$_{17}$H$_{35}$COONa is floating on the surface.) In this case, the soap has a large and positive {surface excess}.
%In other examples, the surface excess may be negative. For example, if water is mixed with an inorganic salt like sodium chloride, 
%the surface of the water is on average less salty and more pure than the bulk average.
%
%Consider again the example of water with a bit of soap. Since the water surface needs to have higher concentration of soap than the bulk,
%whenever the water's surface area is increased, it is necessary to remove soap molecules from the bulk and add them to the new surface. 
%If the concentration of soap is increased a bit, the soap molecules are more readily available (they have higher chemical potential), 
%so it is easier to pull them from the bulk in order to create the new surface. Since it is easier to create new surface, the surface tension is lowered. 
%The general principle is: When the surface excess of a component is \emph{positive}, \emph{increasing} the chemical potential of that component 
%\emph{reduces} the surface tension:
%\begin{eqnarray}
%\frac{\mathrm{d}\gamma}{\mathrm{d}\mu_i} < 0, (\Gamma_i > 0).
%\label{2}
%\end{eqnarray}
%
%Next consider the example of water with salt. The water surface is less salty than bulk, so whenever the water's surface area is increased, 
%it is necessary to remove ions from the new surface and push them into bulk. If the concentration of salt is increased a bit 
%(raising the salt's chemical potential), it becomes harder to push away the ions. Since it is now harder to create the new surface, 
%the surface tension is higher. The general principle is:
%When the surface excess of a component is negative, increasing the chemical potential of that component increases the surface tension
%\begin{eqnarray}
%\frac{\mathrm{d}\gamma}{\mathrm{d}\mu_i} > 0, (\Gamma_i < 0).
%\label{3}
%\end{eqnarray}
%The Gibbs isotherm equation gives the exact quantitative relationship for these trends.
%
%In the presence of two phases ($\alpha$ and $\beta$, the surface (surface phase) is located in between the phase $\alpha$ and phase $\beta$.
%Experimentally, it is difficult to determine the exact structure of an inhomogeneous surface phase that is in contact with a bulk liquid phase 
%containing more than one solute. Inhomogeneity of the surface phase is a result of variation of mole ratios. A model proposed by Josiah W. Gibbs 
%proposed that the surface phase as an idealized model that had zero thickness. In reality, although the bulk regions of $\alpha$ and $\beta$ phases 
%are constant, the concentrations of components in the interfacial region will gradually vary from the bulk concentration of $\alpha$ to the bulk 
%concentration of $\beta$ over the distance $x$. This is in contrast to the idealized Gibbs model where the distance $x$ takes on the value of zero.
%
%\paragraph{Definition of surface excess}
%In the idealized model, the chemical components of the $\alpha$ and $\beta$ bulk phases remain unchanged except when approaching the dividing surface.
%The total moles of any component (examples include water, ethylene glycol, etc.) remains constant in the bulk phases but varies in the surface phase
%for the real system model.
%
%In the real system, however, the total moles of a component varies depending on the arbitrary placement of the dividing surface. 
%The quantitative measure of adsorption of the $i$-th component is captured by the surface excess quantity. 
%The surface excess represents the difference between the total moles of the $i$-th component in a system and the moles of the $i$-th 
%component in a particular phase (either $\alpha$ and $\beta$ ) and is represented by:
%\begin{eqnarray}
%\Gamma_i = \frac{{n_i}^{\text{TOTAL}} - {n_i}^{\alpha}\, - {n_i}^{\beta}\,}{A},
%\label{4}
%\end{eqnarray}
%where $\Gamma_i$ is the surface excess of the $i$-th component, $n$ are the moles,  $\alpha$ and $\beta$ are the phases, and $A$ is the area of the dividing surface.
%$\Gamma$ represents excess of solute per unit area of the surface over what would be present if the bulk concentration prevailed all the way to the surface, 
%it can be positive, negative or zero. It has units of mol/m$^2$.
%
%%Relation between surface tension and the surface excess concentration
%
%The chemical potential of species i in solution depends on the activity $a$ using the following equation:\cite{Hiemenz1997}
%\begin{eqnarray}
%\mu_i = {\mu_i}^o + RT \ln a_i,
%\label{5}
%\end{eqnarray}
%where $\mu_i$ is the chemical potential of the $i$-th component, $\mu_i^o$ is the chemical potential of the $i$-th component at a reference state, 
%$R= 8.3144$J/mol$\cdot$K is the gas constant, $T$ is the temperature, and $a_i$ is the activity of the $i$-th component.
%Differentiation of the chemical potential equation results in:
%\begin{eqnarray}
%\mathrm{d}\mu_i  = RT \frac{\mathrm{d}a_i}{a_i} = RT \mathrm{d}\ln fm_i\,
%\label{5}
%\end{eqnarray}
%where $f$ is the activity coefficient of component i, and $m_i$ is the concentration of species i in the bulk phase.
%
%If the solutions in the $\alpha$ and $\beta$ phases are dilute (rich in one particular component $i$) then activity coefficient 
%of the component $i$ approaches unity and the Gibbs isotherm becomes:
%\begin{eqnarray}
%\Gamma_i = - \frac{1}{RT} \left( \frac{\partial \gamma}{\partial \ln m_i} \right)_{T,p}.
%\label{6}
%\end{eqnarray}
%In the derivation of the equation it is assumed that the solution is ideal, (so $\mu = \mu^o + RT \ln m$) and surface concentration of the solvent is zero, so it is only valid under these assumptions.
%
%The $a_i$ changes too fast, so we use $\ln a_i$ to describe a solution system. In an ideal solution, we just borrow $\ln m$ to describe the system, since $\mu = \mu^o + RT \ln m$. But in the nonideal case, $\mu = \mu^o + RT \ln f_im_i$, more "symmetry-breaking" exists, thus we have to use a more sophisticated quantity $\ln f_im_i$ to do so.
%
%\paragraph{measurement}
%%How to measure the activity coefficient?
%The most direct way of measuring an activity of a species is {to measure its partial vapor pressure} in equilibrium 
%with a number of solutions of different strength. For some solutes this is not practical, say sucrose or salt (NaCl) 
%do not have a measurable vapor pressure at ordinary temperatures. However, in such cases it is possible to measure 
%the vapor pressure of the solvent instead. Using the Gibbs-Duhem relation it is possible to translate the change in 
%solvent vapor pressures with concentration into activities for the solute.
%
%Another way to determine the activity of a species is through the manipulation of colligative properties, 
%specifically freezing point depression. Using freezing point depression techniques, it is possible to calculate the activity of a weak acid from the relation,
%$m^{\prime} = m(1 + a)$,
%where $m'$ is the total molal equilibrium concentration of solute determined by any colligative property measurement 
%(in this case $\Delta{T_\text{fus}}$, $b$ is the nominal molality obtained from titration and a is the activity of the species.
%
%There are also electrochemical methods that allow the determination of activity and its coefficient.
%The value of the mean ionic activity coefficient of ions in solution can also be estimated with the Debye-Duekel equation, the Davies equation or the Pitzer equations.
