\chapter{Calculation of nonlinear optical susceptibilities}\label{calculation_of_chi}

\paragraph{Definitions and relations}\label{defs_and_relations}
1. Definition of double product of a $m$-order tensor $A$ and a $n$-order tensor $B$ is a tensor with order $m+n-2$.
\begin{equation}
    A:B=A_{ij}B_{lm}\delta_{jl}\delta_{im}.
\label{tensor_double_product}
\end{equation}

2. The components of product of $AB$ is defined by 
\begin{equation}
    (AB)_{ijlm}=A_{ij}B_{lm}.
\label{tensor_product}
\end{equation}

3. For vectors $\bf{a}$, $\bf{b}$, $\bf{c}$ and $\bf{d}$, ${\bf{ab}}:{\bf{cd}}=({\bf{a}}\cdot {\bf{d}})({\bf{b}}\cdot{\bf{c}})$.

Proof:
\begin{align}
    {\bf{ab}}:{\bf{cd}}&=({\bf{ab}})_{ij}({\bf{cd}})_{lm}\delta_{jl}\delta_{im} \nonumber \\
    &=({\bf{ab}})_{ij}({\bf{cd}})_{ji} \nonumber\\
    &=(a_i b_j)(c_j d_i) \nonumber\\
    &=({\bf{b}}\cdot{\bf{c}})({\bf{a}}\cdot{\bf{d}})
\end{align}

\paragraph{Proof of Eq.\thinspace\ref{eq:beta_R_b}} \label{identities}
Since 
\begin{equation}
  \lim_{t\to\infty} |e^{-it(a-ib)}|\le \lim_{t\to\infty} |e^{-bt}| = 0,\nonumber
  \label{integral_identity0}
\end{equation}
we can obtain
\begin{equation}
  \int_0^\infty dt e^{-it(a-ib)}=\frac{1}{b+ia},\nonumber
  \label{integral_identity0}
\end{equation}
i.e.,
\begin{equation}
  i\int_0^\infty dt e^{-it(a-ib)}=\frac{1}{a-ib}.\nonumber
  \label{integral_identity1}
\end{equation}
Set $a = (\omega_{v'} - \omega_{v}) - \omega$, and $b = \gamma_{v'v}$,
then we have
 \begin{align}
 \int_0^\infty dt e^{-it[(\omega_{v'}-\omega_v)-\omega-i\gamma_{v'v}]}=\frac{-i}{(\omega_{v'} -\omega_v)-\omega-i\gamma_{v'v}}.
 \label{ident}
 \end{align}
 Using Eq.\ref{ident}, express Eq.\ref{eq:beta_R} in integral form, we obtain
 \begin{align}
  \beta_{\eta\xi\kappa}^{\text{R}}=i\int_0^\infty dt e^{i\omega t} \langle\alpha_{\eta\xi}(t)\mu_{\eta\xi}(0)\rangle,\nonumber
 \end{align}
which is Eq.\ref{eq:beta_R_b}.

%\paragraph{Derivation of Eq.\thinspace\ref{eq:chi}}
%Now we give the derivation of the expression of $\chi^{\text{(2),R}}$
%\begin{align}
%   \chi^{\text{(2),R}}_{\eta\xi\kappa}&=\frac{-i}{k_{\text{B}}T \omega} \int_0^\infty dt e^{i \omega t}\left\langle \dot{A}_{\eta\xi}(t) \dot{M}_{\kappa}(0)\right\rangle
%\end{align}
%Derivation: 
%The resonant term $\chi^{(2),R}_{\eta\xi\kappa}$ is given by \cite{Morita2008}
%\begin{align}
%  \chi^{\text{(2),R}}_{\eta\xi\kappa}&=\frac{i\omega}{k_{\text{B}}T} \int_0^\infty dt e^{i \omega t}\left\langle {A}_{\eta\xi}(t) {M}_{\kappa}(0)\right\rangle,
%\end{align}

\paragraph{Properties of $\chi^{(2)}$}\label{chi_properties}
%[The below is DETAILS OF SFG. NOT APPROPRIATE FOR THE INTRODUCTION.]
Theoretically, the interaction of electromagnetic radiation with mater in the long-wavelength limit is treated within \emph{electric dipole approximation} (EDA). 
EDA is usually expressed in terms of a scalar potential $-{\bf E}(0,t)\cdot {\bf r}$, where ${\bf E}(0,t)$ is the electric field at the origin, and $\bf r$ is the displacement vector, and a zero vector potential\cite{Kobe1982}. 
%EDA only approximates the effect of the electric field on the medium (atoms) in the long-wavelength limit and the magnetic multipoles are neglected. 
Within EDA, the magnetic multipoles are neglected.
A material system under external electric field $\bf E$ induces dipole moment, or
polarization. The polarization $\bf P$ is defined as the dipole moment per
a unit volume of a bulk material. 
When the system indicates a surface, $\bf P$ is defined as the dipole moment per a unit area. 
The induced polarization $\bf P$ is represented as a power series of the electric field\cite{Morita2018}
\begin{equation}
{P}_{i} =\chi^{(1)}_{ij}E_j + \chi^{(2)}_{ijk}E_{j}E_{k} + \cdots,
\label{eq:polarization_1}
\end{equation}
where the subscripts $i,j,k$ denote the Cartesian components. 
The first term describes the linear response of polarization with respect 
to electric field, where $\chi^{(1)}$ is a second-rank tensor, which is called \emph{linear susceptibility}. 
The second term is responsible to the second-order nonlinear optical processes, and $\chi^{(2)}$ is the second-order nonlinear susceptibility.
Therefore, the generation of a polarization in a nonlinear medium by an optical electric field ${\bf E}$ can be presented by
\begin{equation}
{P}^{(2)}_{i} =\chi^{(2)}_{ijk}E_{j}E_{k}.
\label{eq:polarization_1}
\end{equation}
The second-order nonlinear susceptibility $\chi^{(2)}$ is a $3 \times 3 \times 3$ third-rank tensor, which characterizes the process and whose components $\chi^{(2)}_{ijk}$ are 
restricted by the symmetry of the sample.
The components of nonlinear susceptibility represented in different coordinate frames, $\chi^{(2)}_{\alpha\beta\gamma}$ and $\chi^{(2)}_{ijk}$, satisfy the condition 
\begin{equation}
\chi^{(2)}_{\alpha\beta\gamma} = a_{\alpha i}a_{\beta j}a_{\gamma k}\chi^{(2)}_{ijk},
\label{eq:tensor_chi}
\end{equation}
where $a$ can be written as a $3 \times 3$ matrix representing an arbitrary combination of rotation and inversion.
If $a$ is restricted to be a symmetry transformation $A$, then all the properties of the sample are identically described in both coordinate frames.
Then the elements of $\chi^{(2)}$ are the same in both coordinate frames so that
\begin{equation}
\chi^{(2)}_{\alpha\beta\gamma} = A_{\alpha i}A_{\beta j}A_{\gamma k}\chi^{(2)}_{ijk}.
\label{eq:tensor_chi_2}
\end{equation}
If the sample has inversion symmetry\cite{Franken1963}, i.e., $A_{\alpha i} = -\delta_{\alpha i}$, Eq. \ref{eq:tensor_chi_2} yields
\begin{align}
\chi^{(2)}_{\alpha\beta\gamma} &= (-\delta_{\alpha i}) (-\delta_{\beta j}) (-\delta_{\gamma k})\chi^{(2)}_{ijk} \nonumber\\
    & = -\chi^{(2)}_{\alpha\beta\gamma} \nonumber\\
    & = 0.
\label{eq:tensor_chi_3}
\end{align}
Therefore, for any material exhibiting inversion symmetry, $\chi^{(2)}$ is identically 0, and sum-frequency generation is precluded.
This result implies that, within EDA, the VSFG process is forbidden in any centrosymmetric bulk medium\cite{CheM2012},
such as isotropic liquids and glasses, but it is allowed at interfaces because of the broken inversion symmetry\cite{PF00}.
The advantage is its wide applicability to almost every interface which lack a center of inversion, as long as light can reach them. 

%
The second important property of $\chi^{(2)}$ is that it is a polar tensor, which has vectorial nature\cite{Nihonyanagi2013}. 
$\chi^{(2)}$ can also be represented as
the sum of $\beta$ of all molecules in a probed volume
\begin{equation}
\chi^{(2)} = \sum \beta, \nonumber
\label{eq:tensor_chi_4}
\end{equation}
or more quantitative expression
\begin{equation}
\chi^{(2)} = N_\text{s} \beta_{lmn} R(\langle \cos\theta\rangle, \langle \cos^3\theta\rangle),
\label{eq:tensor_chi_5}
\end{equation}
where $N_\text{s}$ is the number of molecules in a unit area, $R(\langle \cos\theta\rangle, \langle \cos^3\theta\rangle)$ is an orientation function,
$\theta$ is the polar orientation angle of a molecule and $\langle \cdots \rangle$ denotes the ensemble average.
Thus $\chi^{(2)}$ has a sign relating to the molecular orientation when $\chi^{(2)}\neq 0$. 
From Eq.~\ref{eq:tensor_chi_5}, if $\chi^{(2)}$ and $\beta_{lmn}$ is known, the sign of $\langle \cos\theta\rangle$ can be determined. 
Therefore, the up/down orientation of interfacial molecules can be experimentally determined.

\paragraph{Molecular dipole moment and dipole polarizability derivatives} \label{calculate_derivatives} 

The polarizability tensor $\alpha$ is defined by the relation
\begin{equation}
  \delta \boldsymbol{\mu} = {\alpha} \boldsymbol{\mathscr{E}}
  \label{eq:def_alpha}
\end{equation}
where $\delta \boldsymbol{\mu}$ is the electric dipole moment (a vector) induced in the molecule by
the electric field $\boldsymbol{\mathscr{E}}$, with the components $\mathscr{E}_x$,$\mathscr{E}_y$ and $\mathscr{E}_z$.
Now we describe the main algorithm to implement the parametrization of the molecular dipole moment 
derivative $\frac{\partial \mu_k}{\partial r}$ and dipole polarizability derivative $\frac{\partial\alpha_{\eta\xi}}{\partial r}$. 
This result can be used in the VACF-based method for calculating the VSFG spectroscopy intensity.

Given a DFTMD trajectory of total length $\sim 10^2$ ps for bulk water, sampled with a frequency $\sim 1$ ps$^{-1}$.
For the $j$-th water molecule in the $n$-th snapshot of the trajectory, 
we denote the two OH bonds as H$^{n,j,\epsilon=1}$ and H$^{n,j,\epsilon=-1}$. We will calculate statistical average over all time steps and all OH bonds, therefore, 
we just denote the corresponding OH bonds by ${\epsilon=1}$ and ${\epsilon=-1}$, respectively. 
The H atoms in a water molecule are denoted by H$^{\epsilon=1}$ and H$^{\epsilon=-1}$, and the O atom by O$^{0}$.
%For the bond OH$_{\epsilon}$, we calculate vector OH$_{\epsilon}$, and $|\text{OH}_{\epsilon}|$ in the lab framework,
%then the direction cosines ($\cos\alpha_{\epsilon}$, $\cos\beta_{\epsilon}$, $\cos\gamma_{\epsilon}$) of the vector OH$_{\epsilon}$ in the molecular frame MF$_{n,j}$.
%The size of the array to store the direction cosines is $ 40\times 32\times 3 \times 3$.
%Calculate the direction cosine matrix between bond framework and molecular framework. 

We used three different coordinate frames: the lab frame ($x^{\text{l}},y^{\text{l}},z^{\text{l}}$), the molecular frame
($x^{\text{m}},y^{\text{m}},z^{\text{m}}$) and the bond frame ($x^{\text{b}},y^{\text{b}},z^{\text{b}}$) (Fig.\thinspace\ref{fig:frameworks}).
In the lab frame, the $z^{\text{l}}$-axis is perpendicular to the interface.
The molecular frame will be used to decompose the signal into normal modes of water monomers.
For the $j$-th molecule, the $z^{\text m}$ axis is along the bisector of the H-O-H angle, the $x^{\text m}$ axis is in the molecular plane,
and the $y^{\text m}$ axis is out of the molecular plane. 
In the bond frame, $z^{\text{b},\epsilon}$ axis is along the bond $\epsilon$ of a molecule, $z^{\text{b},\epsilon}$
is in the molecular plane and $y^{\text{b},\epsilon}$ is out of the molecular plane.

There are two direction cosine matrices between the bond frames and molecular frame,
we name them as ${D}^{\text{b},\epsilon=-1}$ and  ${D}^{\text{b},\epsilon=1}$\cite{Khatib2017},
or ${D}^{\text{b},-1}$ and  ${D}^{\text{b},1}$ for short. 
Then the direction matrix ${D}^{\text{b},\epsilon} $ can be represented by direction cosines between ${\bf x}^{\text{b},\epsilon}$ and ${\bf x}^{\text m}$, 
where $\epsilon=\pm 1$ and $\theta$ denotes the H-O-H angle in the $j$-th water molecule for the $n$-step:
\begin{subequations}
\begin{align}
  &\hat x^{\text{b},\epsilon}_1 = \epsilon \cos\frac{\theta}{2} \hat x^{\text m}_1 +  \sin\frac{\theta}{2} \hat x^{\text m}_3 \\
  &\hat x^{\text{b},\epsilon}_2 = \epsilon \hat x^m_2 \\
  &\hat x^{\text{b},\epsilon}_3 = -\epsilon \sin\frac{\theta}{2} \hat x^{\text m}_1 + \cos\frac{\theta}{2} \hat x^{\text m}_3
\end{align}
\end{subequations}
i.e., 
\begin{equation}
  {D}^{\text{b},\epsilon} =\left(
  \begin{matrix}
    \epsilon\text{cos}\frac{\theta}{2} &  0  & \text{sin}\frac{\theta}{2}\\
    0 & \epsilon & 0\\
    -\epsilon\text{sin}\frac{\theta}{2} & 0 & \text{cos}\frac{\theta}{2}
  \end{matrix}
  \right).
\end{equation}
  The molecular frame is given by the direction cosine matrix ${D}^\text{m,l}$ (or ${D}^\text{m}$) between molecular frame  and the lab frame
  %How to determine  $({\bf D}^\text{m,l})$?  
%Since we can know the coordinates of $(\hat x^m_3)$ in the lab frame: 
%\begin{equation}
%(\hat x^m_3) = (a_{31}) \hat x^l_1 + (a_{32}) \hat x^l_2 + (a_{31})_{nj} \hat x^l_3,
%\end{equation}
%Therefore, the relation between the unit vector  $(\hat x^m_1)$ and $(\hat x^m_2)$ are
%
\begin{subequations}
\begin{align}
  \hat {\bf x}^{\text{m}} ={D}^\text{m} \hat{\bf x}^{\text{l}}.
\end{align}
\end{subequations}
%i.e., we obtain  $({\bf D}^\text{m})$.

% 2a1c.
%\begin{subequations}
%  \begin{align}\left(
%  \begin{matrix}
%  \{\bf D} x^{\epsilon} \\
%  \Delta y^{\epsilon} \\
%  \Delta z^{\epsilon} 
%  \end{matrix}
%    \right)
%    =\Delta r \left(
%  \begin{matrix}
%  \cos\alpha^{\epsilon}\\
%  \cos\beta^{\epsilon}\\
%  \cos\gamma^{\epsilon}
%  \end{matrix}
%    \right)
%\end{align}
%\end{subequations}
%i.e., $\Delta {\bf r}^{\epsilon} =(\Delta x^{\epsilon}, \Delta y^{\epsilon} ,\Delta z^{\epsilon} )^\text{T}$.
%        
%2a2.
%\begin{subequations}
%\begin{align}
%  &   (x'_H)_{\epsilon} -(x_O)_{0} = (x_H)_{\epsilon} -(x_O)_{0} + \Delta  x_{\epsilon} \\
%  &   (y'_H)_{\epsilon} -(y_O)_{0} = (y_H)_{\epsilon} -(y_O)_{0} + \Delta  y_{\epsilon} \\
%  &   (z'_H)_{\epsilon} -(z_O)_{0} = (z_H)_{\epsilon} -(z_O)_{0} + \Delta  z_{\epsilon}
%\end{align}
%\end{subequations}
%
%2a4.We need to calculate the dipole moment of each molecule for $\Delta r \in [-0.05,0.05]$. 
The dipole moment of each OH bond with different length is required to determine the dipole moment derivative. 
    %(For convenient, we have defined $ \text{inc} =\Delta r/|\Delta r|$.)
    %We elongate (reduce) the bond OH$_{n, j,\epsilon}$, and keep other bonds still, then we obtain a updated coordinate file ${\text{pos}}nj\epsilon\text{inc}.\text{xyz}$, where $(n, j,\epsilon)$ is the index of the H in OH$_{\epsilon}$.
Therefore, we elongate (reduce) one bond ${\epsilon}$ by $\Delta r = 0.05$ \AA ($\Delta r = -0.05 $ \A), and keep other 
bonds in the total system still, then we obtain a updated coordinate.
Then the MLWF centers for the system can be calculated from the updated coordinate, using force and energy calculation at the DFT level.
%which is represented by the coordinate file ${\text{pos}}nj\epsilon\text{inc}.\text{xyz}$.
    %2a5a: before next step, we generated a template input file 1.inp. 
    %2a5b: From the template input file 1.inp, we generate an input file for the disturbed position represented by ${\text{pos}}nj\epsilon\text{inc}.\text{xyz}$ with the elongated bond OH$_{\epsilon}$: $n j \epsilon\text{inc}.\text{inp}$
    %2a5c: Using DFTMD, we use the input file $nj\epsilon\text{inc}.\text{inp}$ to calculate and write out the Wannier centers for the system  ${\text{pos}}n j \epsilon\text{inc}.\text{xyz}$. The output: ${\text{HOMO}}n j \epsilon \text{inc}.\text{xyz}$.
    %2a6: From the Wannier centers of the water molecule (can be found in the output ${\text{HOMO}}nj \epsilon \text{inc}.\text{xyz}$), we can calculate the dipole moment for the elongated (reduced) bond OH$_{\epsilon}$.  The $k$ component of the dipole moment: $(\mu^{b,r+\Delta r}_k)_{\epsilon}$.
    From the Wannier centers of the $j$-th water molecule, we can calculate the dipole moment for the elongated (reduced) 
   % MAYBE I DO NOT NEED TO DO BOTH ELONGATING AND REDUCING THE BOND,IE.,I JUST TAKE $\Delta r=0.05$\AA OR $\Delta r=0.05 $\AA IS ENOUGH!) 
    bond ${\epsilon}$. 
    %Therefore, the $k$ component of the dipole moment $(\mu^{b,r+\Delta r}_k)_{\epsilon}$ can be obtained.
    In the bond frame, $|\mu^b| =|\mu^b_z|$. 
    Therefore, we can calculate the $k=z$ component of the dipole moment for ${\epsilon}$ in water molecule $j$ from the MLWF centers\cite{Silvestrelli1999}. 
      The MLWF centers are computed  and the partial dipole moment for a given molecular species $I$ is defined as\cite{Salanne08}
      %
      \begin{equation}
        \mu^I = \sum_{i\in I}(Z_i {\bf R}_i - 2\sum_{n\in i} {\bf r}^{\text{w}}_n).
      \end{equation}
       %
In particular, here it is expressed as 
%\begin{equation}
%  \mu^{\text{b},r+\Delta r,\epsilon} = \frac{1}{2}Z_O {\bf R}^{0} +Z_H{\bf R}^{\epsilon} +1\times (-2) \times r^{w,\epsilon} +\frac{1}{2}\times 2\times (-2)\times r^{w,0},
%\end{equation}
\begin{equation}
  \mu^{\text{b},r+\Delta r,\epsilon} = \frac{1}{2}Z_{\text{O}} {\bf R}^{0} +Z_{\text{H}}{\bf R}^{\epsilon} -2r^{\text{w},\epsilon} -2r^{\text{w},0},
\end{equation}
where $\epsilon=\pm 1$.
%3c: 
In the $\epsilon$ frame, the $k=z$ component dipole moment derivative with respect to bond length\cite{Wilson1955} for the single OH bond $\epsilon$ in water molecule $j$ is
        %
        \begin{equation}
          \frac{\partial \mu^{\text{b},\epsilon}}{\partial r} = (\mu^{\text{b},r+\Delta r,\epsilon}-\mu^{\text{b},r,\epsilon})/\Delta r.
        \end{equation}
       % for both the case of elongating the OH bond, i.e., $\Delta r > 0$ and reducing the OH bond.
        %3d
       % We represent this vector $\frac{\partial \mu^{\text{b},\epsilon}}{\partial r}$ as $( 0, 0, \frac{\partial \mu^{\text{b},\epsilon}}{\partial r} )^\text{T}$.
        Since the components of the dipole moment in the molecular frame are given by the transformation:
        \begin{subequations}
          \begin{align}
            \left(
            \begin{matrix}
              (\frac{\partial \mu^\text{m}}{\partial r})_1\\
              (\frac{\partial \mu^\text{m}}{\partial r})_2\\
              (\frac{\partial \mu^\text{m}}{\partial r})_3
            \end{matrix}
            \right)
            = {D}^{\text{b},\epsilon}
            \left(
            \begin{matrix}
              0\\
              0\\
              \frac{\partial \mu^{\text{b},\epsilon}}{\partial r}
            \end{matrix}
            \right),
            \end{align}
        \end{subequations}
    % END of r-loop 
    then, to calculate the individual polarizability for a OH bond from Wannier centers, 
    calculations involving finite electric fields (of 0.0001 au intensity) were performed independently 
    along $x$, $y$, and $z$ directions at each sampled time step\cite{Sulpizi2013}.
    %4a: 
    For the electric field $\mathscr{E} \in \{\mathscr{E}_x,\mathscr{E}_y, \mathscr{E}_z\}$,
       like in the case of no external electric field, the MLWF centers are calculated. %in ${\text{HOMO}}nj\epsilon\varepsilon.\text{xyz}$.
 For a finite $\Delta r$, the dipole moment is given by 
\begin{equation}
  \mu^{{\text b},r+\Delta r,\epsilon,\mathscr{E}} = Z_H{\bf R}^{\epsilon,\mathscr{E}} + 
  \frac{1}{2}Z^0 {\bf R}^{0,\mathscr{E}} -2 {\bf r}^{\text{w},\epsilon,\mathscr{E}} -2{\bf r}^{\text{w},0,\mathscr{E}}.
\end{equation}
        %END $ \Delta r$-loop  
%4a4 the polarizability tensors $\alpha^{b,r+|\Delta r|}$ and $\alpha^{b,|r-\Delta r|}$. 
From the relation (obtained from Eq.\thinspace\ref{eq:def_alpha})
      \begin{subequations}
          \begin{align}
            & 0 = \alpha^{{\text{b}},r+\Delta r}_{11}\mathscr{E}_{1} + \alpha^{{\text{b}},r+\Delta r}_{12}\mathscr{E}_{2} + \alpha^{{\text{b}},r+\Delta r}_{13}\mathscr{E}_{3}\\
            & 0 = \alpha^{{\text{b}},r+\Delta r}_{21}\mathscr{E}_{1} + \alpha^{{\text{b}},r+\Delta r}_{22}\mathscr{E}_{2} + \alpha^{{\text{b}},r+\Delta r}_{23}\mathscr{E}_{3}\\
            & \delta \mu^{{\text{b}},r+\Delta r}_{3} = \alpha^{{\text{b}},r+\Delta r}_{31}\mathscr{E}_{1} + \alpha^{{\text{b}},r+\Delta r}_{32}\mathscr{E}_{2} + \alpha^{{\text{b}},r+\Delta r}_{33}\mathscr{E}_{3}.
          \end{align}
      \end{subequations}
%
where
\begin{equation}
  \delta \mu^{{\text{b}},r+\Delta r}_{3} = \mu^{{\text{b}},\epsilon,r+\Delta r,\mathscr{E}}_{3} - \mu^{{\text{b}},\epsilon,r+\Delta r}_{3}.
\end{equation}
%
and the expressions of the electric filed ${\mathscr{E}^\text{b}}$ in a OH frame for the three cases of external electric field 
which is along $x$, $y$ and $z$ axis in the lab frame, respectively, we obtain 9 equations.
%
For
\begin{equation}
  {\bf\mathscr{E}}^{\text{l}} = (\mathscr{E}_0 \  0 \  0){\tran} \nonumber
\end{equation}  
where $\mathscr{E}_0 = 0.0001$ au, we can obtain the intensity of the external electric filed in the molecular frame  
\begin{subequations}
  \begin{align}
    &\mathscr{E}^\text{m}_1 = {D}^{\text{m}}_{12} \mathscr{E}^\text{l}_2\\
    &\mathscr{E}^\text{m}_2 = {D}^{\text{m}}_{22} \mathscr{E}^\text{l}_2\\
    &\mathscr{E}^\text{m}_3 = {D}^{\text{m}}_{32} \mathscr{E}^\text{l}_2,
    \end{align}
\end{subequations}
where ${D}^{\text{m}}_{pq}$ is the $pq$-component of ${D}^\text{m}$.
%
In OH bond frame,  
\begin{equation}
  \mathscr{E}^{\text b} = {D}^{\text b} \mathscr{E}^{\text{m}}.
\end{equation}
Similarly, we obtain expansions of the intensity of the electric field for other two cases: 
when $\mathscr{E}^{\text{l}} = (0,\mathscr{E}_0, 0)^\text{T}$ and when $\mathscr{E}^{\text{l}} = (0,0,\mathscr{E}_0)^\text{T}$, respectively.
Here, $\mathscr{E}_x$ is the electric field along $x$-axis in the lab frame. 
    %4a5: 
    %If we just elongate each single bond, ie., let $\Delta r > 0$,
    Then the dipole polarizability for the bond ${\epsilon}$ is as follows:
\begin{subequations}
  \begin{align}
    &\frac{\partial \alpha^{{\text b},\epsilon}_{31}}{\partial r} = (\alpha^{{\text b},\epsilon,r+\Delta r}_{31} -\alpha^{{\text b},\epsilon,r}_{31})/\Delta r\\
    &\frac{\partial \alpha^{{\text b},\epsilon}_{32}}{\partial r} = (\alpha^{{\text b},\epsilon,r+\Delta r}_{32} -\alpha^{{\text b},\epsilon,r}_{32})/\Delta r\\
    &\frac{\partial \alpha^{{\text b},\epsilon}_{33}}{\partial r} = (\alpha^{{\text b},\epsilon,r+\Delta r}_{33} -\alpha^{{\text b},\epsilon,r}_{33})/\Delta r.
  \end{align}
\end{subequations}
    %If we just reduce each single bond, ie., let $\Delta r < 0$, then the dipole polarizability for the bond ${\epsilon}$ is as follows:
%\begin{subequations}
  %\begin{align}
   % &\frac{\partial \alpha^{b,\epsilon}_{31}}{\partial r} = (\alpha^{b,\epsilon,r}_{31} -\alpha^{b,\epsilon,r+\Delta r}_{31})/|\Delta r|\\
   % &\frac{\partial \alpha^{b,\epsilon}_{32}}{\partial r} = (\alpha^{b,\epsilon,r}_{32} -\alpha^{b,\epsilon,r+\Delta r}_{32})/|\Delta r|\\
   % &\frac{\partial \alpha^{b,\epsilon}_{33}}{\partial r} = (\alpha^{b,\epsilon,r}_{33} -\alpha^{b,\epsilon,r+\Delta r}_{33})/|\Delta r|.
  %\end{align}
%\end{subequations}
          %END 4a
      %END $j$-loop
%END $n$-loop

Therefore, the averages for $(\frac{\partial \mu^{\text{b},\epsilon}}{\partial r})_\kappa$ and $(\frac{\partial \alpha^{\text{b},\epsilon}}{\partial r})_{\eta\xi}$ 
(the subscripts $\kappa, \eta, \xi = x^{\text{m}}, y^{\text{m}}, z^{\text{m}}$, or $1, 2, 3$) over all OH bonds give the molecular dipole and polarizability derivatives, respectively. 

