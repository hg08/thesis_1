\chapter{Introduction}\label{CHAPTER_1}
%Why care interface?
%1
Interfaces of aqueous electrolyte solutions are ubiquitous in biology, atmosphere, chemistry, man-made systems 
and industrial processes\cite{Irwin88,Tobias1999, Benderskii00, 
Asahi2001,Benderskii02,Richmond2002,LiuH2004,
TianCS08,Yamamoto2008, Salmeron2009,ZhangLY2009,
LoNostro2012,Piatkowski2014,Balajka2018}.
The aqueous interface appears in different forms, such as sprays, aerosols, nanoscopic and  microscopic water droplets, water/vapor interfaces.
Many phenomena, such as solvation\cite{Benjamin1996}, adsorption\cite{Chang06}, bubble formation\cite{Craig1993,Craig1993b,Weissenborn1995,Marcelja04,Craig04},
occur at aqueous interfaces\cite{Ball2008,Kuo2004b}. 

%2.
Compared to bulk atoms or molecules, interfacial atomic or molecular layers generally have very different properties. 
At the water/vapor interface, the hydrogen bond (HB) network is sharply terminated, which makes the interface more heterogeneous 
than bulk water\cite{singh2013}. 
The statistical distribution of the orientation of water molecules at the interface is different from that of bulk water,
and thus the local structure, light absorption,  molecular rotations and diffusion are different at the interface than in bulk water\cite{Jedlovszky2004}.
Experiments also demonstrate that there exists on-water catalysis effect, which means that some chemical reactions take place much faster 
on the water surface with respect to those in bulk phase\cite{Rideout1980,Narayan2005,Beattie2010}.
Molecular simulations also show that the effective dielectric constant of interfacial water is significantly lower than its bulk value, 
and it also depends on the curvature of the interface\cite{Dinpajooh2016}. 
%What does above properties imply?
Overall, aqueous/vapor interfaces provide a unique environment for many physical, chemical and biological processes. 

%Then talk about the special form of liquid/vapor interface---- interfaces of aqueous salt solution.
%[FIRST SPECIAL ROLE OF INTERFACES AND THEN MOVE TO THE IONS AT INTERFACES. IN WHICH PROCESSES IONS AT INTERFACES ARE RELEVANT?]
%Introduce the interfaces with ions done
\paragraph{Ions at interfaces}
%[WHY IONS AT WATER/VAPOR INTERFACE IMPORTANT?]
%[Too general]The influence of ions on the liquid interfaces is fundamental in theory and significant in practice.
Due to the presence of ions, many properties of the interface are significantly affected.
%reason 1[EXAMPLE 1]
For example, the stability of cell membranes is affected by the distribution of counter-ions\cite{Veziriglu1990}; 
the free energies of ions across liquid surfaces are essential to solvent extraction processes and phase transfer catalysis\cite{Starks1994};
the uptake of pollutants by water droplets in cloud depends on the ion distribution at the aqueous liquid/vapor interfaces.
% they worth to be studied [后面讲关于离子分布的开放问题]
%
%reason 2[EXAMPLE 2]
The surface tension of the water/vapor interface changes with the addition of electrolytes in water\cite{Pegram2006}.
%ex1无机盐使水界面的表面张力增强 done
%ex2酸会使界面的表面张力减小    done
In most alkali and inorganic salt solutions, the surface tension increases as the concentration of the electrolyte increases; 
however, in most acid solutions, the surface tension decreases as the concentration of the acid increases.
%[后面讲关于界面上MD结果与Gibbs adsorption eq 不符]
%
%reason 3[EXAMPLE 3]
%Furthermore, the solvation and adsorption of ions at the aqueous interfaces is an important process in a large number of chemical and biological systems\cite{Chang06}.
% ADD examples for reason 
It was also observed that there exists specific ion effect. 
The most typical example is that the solvation structure surrounding the more polarizable \I anion at the interface is more anisotropic than the solvation structure around the less polarizable Cl$^-$.
%reason 3a : Hofmeister series [后面讲硝酸根离子和碘离子会趋向于界面,且会使界面水分子的振动频率升高]
%

%reason 4[EXAMPLE 4]
% Ions will affect the dynamics of HBonds.
%[Progress of the study of HB dynamics of the interface (a)]
Ions also modify the HB dynamics at the interface\cite{Raymond2004,McLain2006,Ball2008}.
%[Too general]H-bonding network is important for interfaces of aqueous electrolyte solutions.
The microscopic structure of water is determined by O-H$\cdots$O bonds between the hydroxyl group 
and O atoms of neighboring molecules. 
Because of the H-bonding, electrostatic force and dispersion forces, 
at water/vapor interfaces there exists an interface-specific bonding network\cite{Eisenberg1969,Speedy1976,Poole1994,Soper2008b,Ball2001,Nilsson2011,Pettersson2015}, 
which is different from H-bonding network in corresponding bulk liquid\cite{Allongue96,Velasco-Velez14}.
Experiments have shown that, the presence of ions in aqueous 
electrolyte solutions may significantly change the property of the H-bonding network. 
The formation of H-bonding network indicates a reduction in the orientational degree of freedom, 
an enhancement in the local structure of water around the solute, or change (decrease) of entropy\cite{Frank45a, Frank45b,Frank45c}.
%
%The H-bonding network\cite{Eisenberg1969,Speedy1976,Poole1994,Soper2008b,Ball2001,Nilsson2011,Pettersson2015} as well as electrostatic forces, 
%and van der Waals forces are the main factors that determine the structure of interfaces. 
%Salts change the H-bonding structure of water in the interfacial region\cite{Raymond2004,McLain2006,Ball2008}. 
%The specific distribution of the anions at the interface may have significant influence on the H-bonding network of interfacial water\cite{Morita2008}.

%reason 5 [这一段无非是多一个研究电解质溶液的理由.不要也罢.]
%[REASON 5]
%% ions affect thickness of interface
%%[MS: THIS IS VERY GENERAL...] % 最好不要在具体和一般之间来回跳跃。最好先一般,再具体。
%The presence of ions will change the thickness of the interface.
%Determination of interface layer thickness of a non-ideal two-phase system is an important problem in interface science, molecular biology, 
%and hydromechanics.\cite{LiZhihong2001,Goharzadeh2005,Bano2006} It has direct impact on many technical and natural phenomena, 
%such as the competitive binding of water, ions or denaturants on the macromolecular surface\cite{Arakawa1985,Timasheff2002}, 
%aggregation of proteins\cite{Webb2001} and binding of molecules or drugs on protein surface\cite{Hritz2004}. 

%reason 6[EXAMPLE 6]
A clear microscopic understanding of the interfaces of aqueous salt solutions is also essential to further energy and human development. 
For example, the chemical potential difference of the interface between two seawaters with different salt concentrations can provide new energy for mankind\cite{Pattle1954,Loeb1976}. 
%At the intersection of sea water and river water, brackish water is easily produced
%[There are more than 2 billion kilowatts of usable salinity energy on the earth, and its energy is even greater than the temperature difference energy.]
Many ocean and atmospheric reaction processes, including many key climate properties such as surface activity, 
solubility and interfacial molecular structure, influence climate\cite{Schill2015,Cochran2017}. 

Therefore, understanding the equilibrium properties and dynamics of the electrolyte solution interfaces 
is important in designing and controlling the chemical reactions at liquid interfaces.
%In particular, understanding ion behavior at the air/water interface is crucial in solving environmental problems such as acid rain and water pollution\cite{Chang06}.

\paragraph{Methods to selectively probe the interface}
%[Ions interfaces are so important, then what can we do to expand our understanding to them? What are the benefits? SFG, IHB, IMS,...]
For obtaining the information on the specific mechanism underlying the interfacial phenomena,
gaining molecular-level understanding of the interfacial water organization and ions distribution at interfaces,
we need detection methods with interface selectivity. 

%[THE FIRST POINT TO ADDRESS HERE IS THAT EXPERIMENTS HAVE MADE ENORMOUS PROGRESSES IN THE SELECTIVE CHARACTERIZATION OF LIQUIDS AT INTERFACES. WHY IS SFG SO RELEVANT?]
% 为了研究界面,我们需要具有界面选择性的探测手段。近年来,SFG光谱就是最常用的一种界面探测手段。
% method 1
Experimentally, the Vibrational Sum-Frequency Generation (VSFG) spectroscopy is the most frequently used and powerful interface analytical tool\cite{Shen2016,Morita2018,Shen2020}.
%为什么它可以做到界面选择性呢?
It is based on a simple idea that optical responses of a surface and bulk of a medium follow different selection rules.
The VSFG spectroscopy uses a second-order nonlinear optical process and the resulting signal is very sensitive to surface ions and 
molecules of a sub-monolayer level\cite{Morita2008,WangHongFei2015,WenYuChieh2016,Ishiyama2017,Penalber-Johnstone2018}. 
%
For any material exhibiting inversion symmetry, the experimentally recorded signal, namely nonlinear susceptibility $\chi^{(2)}$ is identically 0, and VSFG is precluded\cite{Franken1963}.
This result implies that, the VSFG process is forbidden in any centrosymmetric bulk medium\cite{CheM2012},
such as isotropic liquids and glasses, but it is allowed at interfaces because of the broken inversion symmetry\cite{PF00}.
The advantage is its wide applicability to almost every interface which lack a center of inversion, as long as light can reach them. 
%
%=========================
%Conventional SFG spectra.
%=========================
Conventional VSFG spectroscopy detect the intensity of VSFG light,
i.e., it gives only the absolute square $|\chi^{(2)}|^2$ of the nonlinear susceptibility\cite{ShenYR1984,Guyot-Sionnest1986,Shen2020}. 
The phase sensitive (PS-)VSFG\cite{JiN2008} and heterodyne detected (HD-)VSFG technology permit to overcome this shortcoming. 
The most important feature is that they can measure complex $\chi^{(2)}$ spectra. 
In other words, they can provide $\Re \chi^{(2)}$ and $\Im \chi^{(2)}$, or the amplitude and phase of the components of $\chi^{(2)}$. 
Since $\Im \chi^{(2)}$ spectra shows an absorptive line shape that directly represents a vibrational resonance, 
it can be interpreted more straightforwardly\cite{Nihonyanagi2013}.
The sign of $\Im \chi^{(2)}$ is also closely related to the orientation of molecules at the interface\cite{TianCS2008,TianCS2009,TianCS2011}.
Recently, broadband HD-VSFG spectroscopy that has a high phase stability has been developed and has been used to study the polar orientation and HB structure of interfacial
water\cite{Nihonyanagi2009,Shen2013}. 

Therefore, the VSFG spectroscopy can be used to probe many types of interfaces, namely, liquid-liquid and 
solid-liquid interfaces\cite{Guyot-Sionnest1987,RS91,DuQ1993,DuQ1994,Richmond2002,Gopalakrishnan2006,ShenYR2006,Morita2008}, metal and semiconductor surfaces\cite{Harris87,Superfine88},
and to determine the molecular orientation at the aqueous solution/vapor interfaces.
It allows to detect intramolecular vibrational modes, and molecular orientation by detecting polarization dependence of the VSFG signals\cite{Vidal05}.  
The VSFG spectra suggest that the interfacial hydrogen (H-) bonding between water molecules is changed by the presence of salt, 
especially the anions\cite{Raymond2004}.
%[The progress of theoretical support for the VSFG spectra?]
Molecular-level properties of interfacial materials arising from interactions between water and minerals, 
such as swelling, wetting, hydrodynamics can also be studied by the VSFG spectroscopy\cite{Rotenberg14}.

%[Method 2 for surface selectivity; NO format problem]
%[MD --> AIMD]
%MD 用在界面之模拟上有何优势呢?
On the comuputational side, Molecular Dynamics (MD) simulations are also powerful tools in the microscopic study of interfaces,
allowing a straightforward investigation of structure and dynamics of molecules at interfaces\cite{Morita2008}.
%既然有MD,为何还需要AIMD呢?
They gives insights into direct interactions between ions and water, and have been used for calculating properties, 
such as the depth profile of ion concentrations of interfaces\cite{Jungwirth2001,Jungwirth2002}, and the VSFG spectra 
of electrolyte solution surfaces\cite{Gopalakrishnan2006,Johnson2014,Ishiyama2014,Ishiyama2017}.
Polarizable force field have been widely used to calculate VSFG spectra\cite{LXD03,Petersen2004,TI07,MM05}, however, they lack
of transferability when complex interfaces are considered.
To overcome this problem, \abinitio MD (AIMD) simulations\cite{CP1985,Pastore1991,Hutter2012} can be used.
In particular density functional theory-based molecular dynamics (DFTMD) simulations\cite{CP1985,Kuehne2020} 
can provide the dynamic trajectory of 
the liquid interface, from which macroscopic properties, such as distributions and
orientation of ions, as well as the VSFG spectra can be calculated.  
%[AIM: compute the interfacial VSFG spectra of electrolyte solutions and to provide their molecular interpretation.]
The advantage of AIMD is that it does not require a priori parameterization and it is capable to include polarization effects\cite{Ufimtsev2011},
also including electronic polarization. \emph{Ab initio} MD at the gradient corrected level, and also including dispersion corrections\cite{Grimme2004,Grimme2006,Grimme2007,Grimme2010,Baer2011}
has been shown to provide an accurate description of the vibrational properties at interface
\cite{LeeCM2013,LeeCM2015,Sulpizi2013,Ohto2015,TO15,Fornaro2015, Khatib2016, Nagata2016, Ohto2016, Nagata2016b, Hosseinpour2017,Santos2018,Ohto2018}
%[ALL SFG Calculation REF.s].
Recently, progresses in the calculations of the SFG spectra have permitted to reduce their computational cost and to investigate more complex systems\cite{Khatib2016b,Khatib2017}.

%[HERE THERE IS QUITE AN ABRUPT JUMP FROM SFG TECHNIQUE TO HOFFMEISTER SERIES. MAYBE THE NEXT TWO PARAGRAPHS SHOULD GO BEFORE INTRODUCING THE SFG.SULPIZI]
%[OPEN QESTION 1]
%In recent years, more accurate and consistent results about the interface have been reported experimentally\cite{TianCS2009,Shen2013}. 
%However, the quantitative interpretation of the VSFG spectra is not straightforward, because the VSFG intensity is influenced by several factors, including ions' concentration, 
%molecular orientation and distribution and local field correction\cite{Morita2008}.

%Report the available experimental data for the VSFG spectra of salty interfaces.
%[Q1: SOME EXPANSION ;
%Q2: ESTABLISH A CONNECTION BETWEEN THESE EXP.S AND YOUR THESIS. IN PARTICULAR, WHICH OPEN QUESTIONS RAISE FROM THE EXPERIMENTS? WHICH ONE WE WANT TO ADDRESS WITH OUR SIMULATIONS]
\paragraph{Selected experimental data and simulation works}

% WE HAD KNOW THE PROBLEM, THEN HOW TO SET UP OUR MODELS and START TO TRYING TO SOLVE THEM
The ions propensity for the liquid/vapor interface, as well as their influence on  
the water's HB network are of special interest to the atmospheric chemistry community.%\cite{Pitts2000,Pitts2003}. 
Various ions play critical roles in the kinetics and mechanisms of heterogeneous chemical reactions 
at the water/vapor interface of atmospheric aerosols. 
A first question which has been subject of intensive investigation regards the ions distribution in the proximity of an interface for the different ionic species.

%%Although we have SFG and powerful computer simulation technology, some problems, such as ions' propensities and surface thickness, are still unresolved regarding the solution interface.
%%\paragraph{Ions' adsorption at solution/vapor interface}
%For the interfaces of electrolyte solutions, theoretical methods, experiments and computer simulations have given some conclusions that seem incompatible with traditional thermodynamic theories, which result in the conclusion that inorganic salts are usually negatively adsorbed and the tension may
%be raised in strong solution\cite{Gibbs1928, Adam1941}.
%%The following results seem to be inconsistent with the theoretical results given by the Gibbs adsorption equation. 
%%clearly state
As early as the beginning of the last century, Heydweiller discovered that anions affected the surface tension significantly
and the magnitude of the variation of the surface tension follows the same sequence discovered by Hofmeister earlier\cite{dosSantos10},
i.e., namely 
CO$_3^{2-}$ $>$  SO$_4^{2-}$ $>$ F$^-$ $>$ Cl$^-$ $>$ Br$^-$ $>$ NO$_3^-$ $>$ I$^-$ $>$ ClO$_4^-$ $>$ SCN$^-$\cite{Jungwirth2006,Pegram2007,ZhangJY2010,Tobias2008,Parsons2011,HuaWei2013}. 
%[DONE for water/vapor interface--Simulations] 
%Progress of the study of interfacial structure (b)
Langmuir\cite{Langmuir1917} was the first to attempt a theoretical explanation of the physical mechanism for the increase of the surface tension by added electrolytes.
The adsorption, or surface excess per unit area, of electrolytes can be described by the well-known Gibbs adsorption equation, which relate 
surface tension, surface excesses and chemical potentials for a system of any number of components.
%[THE INTRO SHOULD BE MORE DISCORSIVE AND AVOID FORMULA.]
%The general form of Gibbs's relation between surface tension $\gamma$, surface excesses $\Gamma_i$, and chemical potentials $\mu_i$ for a system of any number of components is
%\begin{equation}
%d\gamma = -\sum_i \Gamma_i d\mu_i,
%\label{eq:gibbs_relation}
%\end{equation}
%for a system with constant temperature $T$.
Using the Gibbs adsorption equation, Langmuir concluded that this phenomenon was a consequence of ion depletion 
near the water/vapor interface, i.e., the \emph{increase} in surface tension 
implies a \emph{deficiency} of solute in the surface layer\cite{Jarvis1968},
and estimated the thickness of 
the depleted layer in the range 3.3 to 4.2 \A.
Later, MD simulations have shown that more polarizable anions (e.g., larger halide anions) 
can be present in the surface region\cite{Jungwirth2001,Jungwirth2002}. 
Tian and coworkers\cite{TianCS2011} have also predicted that some ions, such as \I and Br$^{-}$, could accumulate at the interface.
These seemingly contradictory conclusions with the Gibbs adsorption equation mean that our understanding of the interface is still incomplete. 
%Specifically, the kinetics of halogen uptake of solutions\cite{HuJH95}, the photoelectron emission experiments\cite{Markovich1991,Ghosal05,Garrett04} and the polarizable 
%force-field simulations\cite{Perera1991,Dang1993,Knipping00,Jungwirth2001,Jungwirth2002,PJ06,Horinek07,Brown08,CST11} showed that 
%some heavier halides anions are able to \emph{approach} the interface closer than the cations, 
%while the surface tension of these solutions is also \emph{increased} compared to pure water. 
%%BUT this result is diff from Gibbs'
%The Poisson-Boltzmann approach also predicts, for a charged interface, an increase in the total number of ions in the vicinity of the interface as compared to the bulk\cite{Manciu2003}.
%These examples imply that our definition of the solution/vapor interface might be not clear enough, 
%motivating us to try to find an approach to better define the thickness of the aqueous solution/vapor interface.
%Another question of widespread concern is the thickness of the interface.

%
%[Known]
%%Progress of the study of interfacial structure (a)
%It was known that a variety of processes, such as propensity to salt-out proteins, enzymatic activity, and polymer and protein folding display a Hofmeister effect.
%Gurau \etal\cite{Gurau2004} demonstrated a novel Hofmeister effect in an octadecylamine monolayer spread on salt solutions by using VSFG spectroscopy.
%Although there is some general consensus on the fact that anions propensity for the interface inversely correlates with
%the order of the Hofmeister series, namely 
%CO$_3^{2-}$ $>$  SO$_4^{2-}$ $>$ F$^-$ $>$ Cl$^-$ $>$ Br$^-$ $>$ NO$_3^-$ $>$ I$^-$ $>$ ClO$_4^-$ $>$ SCN$^-$,
%the driving force and the microscopic details of the solvation structure are still debated. 

The use of surface specific vibrational spectroscopy techniques has 
permitted to elucidate some aspects of surface HB structure for water in 
the presence of ions\cite{Jubb2012,AGL05}. 
% Other result exp. on ions water interfaces>>
Raymond and Richmond\cite{Raymond2004} have shown that there exists anions in the surface region of alkali halide salt solutions, 
comparing the VSFG signal from four alkali halide salt solutions---NaF, NaCl, NaBr 
and NaI. Their experiments showed that the changes of the interfacial H-bonding depend on the anion's species, 
indicating the presence of anions in the interfacial region. The frequency shifts in the peaks of the VSFG spectra display 
the structure-breaking characteristics of larger halogen ions.

The influence of molecular ions such as nitrate (\nitrate), sulfate­ and 
carbonate ions­ has also been analyzed, but proved more elusive than that for halide solution\cite{SG05,Salvador2003}.
Recently considerable attention has been given to the nitrate ions in aqueous phase 
for their ubiquitous and diverse role in atmospheric aerosols, polluted water, 
and the remote troposphere\cite{Pitts2000,XuM2009,Jubb2012,Banerjee2016,Yadav2017b,Cochran2017,Yadav2017,Robinson2020}.
Hua \etal\cite{HuaWei2014} have recently measured the VSFG spectra of water/vapor interface of \LiN salt solutions in the OH stretching region
using HD-VSFG spectroscopy\cite{HuaWei2011,HuaWei2011b,ChenXiangKe2010}. 
At a difference with the spectra for the water/vapor interface, changes in the spectra of 
\LiN solutions are observed and interpreted as caused by the presence of ions at the interface.  
%[a depletion of the 3200 \cm peak is observed, with an 
%enhancement of the 3400 \cm peak.]
A similar behaviour had been observed for the interface of NaNO$_3$ and 
Mg(NO$_3$)$_2$ solutions\cite{Jubb2012,HuaWei2014}. 
It has been suggested that the depletion, and in some cases 
the enhancement of peaks in the spectra, are indications that nitrate 
ions reside at the interface. On the other hand the small 
cations should have little surface propensity. 
It has also been argued that the positive electric field found at the interface of NaCl, NaI and 
NaNO$_3$ salt solutions is due to the formation of an ionic double layer 
between anions located near the surface and their counter-cations (e.g.
Na$^+$) located further below. In PS-VSFG experiments the 
magnitude of the induced change in the $\Im\chi^{(2)}$ spectra comparatively
to that of the neat water suggested that \nitrate has a surface propensity 
just in between I$^-$ and Cl$^-$\cite{Verreault2013,Verreault2009}. 

%From the experimental data of surface tension dependence on solute concentration $\text{d}\gamma/\text{d}m_2$ 
%at low electrolyte concentrations ($\leq$1.5 M )\cite{Weissenborn1995,Hey1981,Jarvis1968,Jarvis1972}, 
%the relation of the surface/bulk molar concentration ratio $K_{\text{p}}$\cite{Pegram2006} among \li, \Na and \K is: 
%\begin{equation}
%0=K_{\text{p,Na}^+}< K_{\text{p,K}^+}< K_{\text{p,Li}^+}.\label{eq:bscr}
%\end{equation}
%i.e., \Na is the most surface-excluded in the alkali nitrate solution RNO$_3$, \K is less excluded, 
%and \Li is the least excluded cation (see Appendix \ref{surface_tension_increment} for details).
%In modeling the interfaces of aqueous solutions of alkali nitrates, we started with LiNO$_3$, 
%because the \Li ion is the least excluded of the solution/vapor interface among the alkali metal ions (see Appendix \ref{surface_tension_increment} for details) 

%%How to calculate VSFG spectra?
%Unlike an absorption spectrum, the vibrational sum-frequency generation (VSFG) signal can be considered as a sum of signed contributions 
%from different hydrogen-bonded species in the sample.\cite{Pieniazek11}
%%[Done]O: 
%Pieniazek and coworkers\cite{Pieniazek11} had shown that the observed positive feature at low frequency, in the imaginary part of
%the VSFG signal, is a result of cancellation between the positive contributions from four-hydrogen-bonded
%molecules and negative contributions from those molecules with one or two broken H-bonds.

%Via the Gibbs adsorption equation, it has been known that, in 
%aqueous solutions of simple inorganic salts, such as the
%alkali halides, the surface tension increases with solute concentration.\cite{PJ01}

%
%%DONE for HBD
%The femtosecond InfraRed Spectroscopy (fsIRS) has been used as a new experimental tool to see the HB dynamics in ionic hydration shells\cite{Laage2007}. 
%Tominaga and coworkers studied the dynamical structure of water in the presence of alkali-metal and halide ions as functions of temperature and concentration, 
%by using the low frequency Raman spectroscopy. 
%They found that the water-water intermolecular stretching frequency decrease with increasing ion concentration\cite{KM98,Amo00}.
%H-Bonds act as bridges between protein binding sites and their substrates\cite{Ball05}.
%The structure and dynamics of H-bonds play an important role in determining the thermodynamic properties of biomolecules in aqueous solutions\cite{HX01}. 
%Water’s HB dynamics is also intimately connected to its ultrafast vibrational dynamics\cite{Nagata15}. 
%The dynamic process of rupturing and reforming of H-bonds is water solution can be indirectly probed by a number of experimental methods\cite{OC84,JT85}.
%The dynamical response of water is intimately related to the lifetime of H-bonds\cite{SP05}. 
%Although it can not be yield by experimental methods, it can be studied by computer simulations\cite{Rapaport1983,Voloshin2009}.
%Computer simulations are tools for the study of HB dynamics near the solvated ions and biomolecules\cite{PJR79,YKC98}.

%[WHY THIS THESIS?]\
\paragraph{This thesis}
%[anisotropy dynamics is important] 
%In bulk water, measuring the anisotropy dynamics of O-H stretch vibrations\cite{Woutersen99} demonstrated the rapid F{\"o}rster resonant energy transfer between O-H vibrations of different water molecules.
%
Density functional theory-based MD simulations 
can provide detailed information on the structure and dynamics  
of water in these simple but non-the-less still puzzling systems: alkali metal nitrate and alkali metal halide solutions\cite{Mizoguchi1998}.

% reason 1: SFG calculation 
As first we provide a molecular interpretation of the main characteristics of the
VSFG spectra of the electrolytic solution/vapor interfaces.
We focus on nitrate solutions and alkali halide solutions because these anions have a significant impact on the structure and dynamics of the interface.
We have used AIMD to provide a realistic description of the electrolytic solution/vapor interface and calculated the VSFG spectra.
The nonlinear susceptibility of solution/vapor interfaces is obtained by the Fourier transform (FT) of 
the time correlation function of the dipole moment and the polarization tensor. This requires, for each step of DFTMD, 
the calculation of the dipole moment and polarization tensor of molecules in the interface. 
To simplify the calculation, we express the correlation function 
as the autocorrelation function of the velocity of the atoms. 
Accordingly, we calculated the VSFG spectrum of the interface for \LiN solution,
which is consistent with the experiment. 
In addition to the spectra interpretation, we also provide the distribution of anions 
and cations relative to the interface: nitrate ions are located in the uppermost layer of the interface, 
and alkali metal ions are located in the layer below and form with the anion a water separated ion pair. 
Using free energy calculations, we also proved that this water-separated ion configuration is more stable than the contact ion pair.

The adsorption of halide anions, in both surface tension experiments and MD simulations, 
follows the Hofmeister series.
These results suggest that heavier halide anions have a similar effect to nitrate ions on the structure and dynamics of the water/vapor interface. 
To verify the experimental suggestion, we also simulated the interface of lithium iodide, sodium iodide and potassium iodide  using DFTMD simulation,
and we calculated the corresponding VSFG spectra.
The results show that they have some common characteristics: in all the cases, 
the H-bonded OH-stretching peak of $|\chi^{(2),\text{R}}|^2$ is blue-shifted, 
indicating that the water molecules at the interfaces are forming weak H-bonding.
% reason 1a:
In addition, we also confirmed that the vibrational properties of interfaces of water and aqueous electrolyte solutions cannot be obtained from small water cluster containing the same ion pair.

%Open question 2
%[AIMD Dynamics]
%The second problem we want to deal with is the dynamics of the hydrogen bond of the solution/vapor interface.
%已知,界面会加快氢键动力学,加快化学反应等\cite{}.
A second aspect which we have investigated in the thesis is the HB dynamics. 
Hydrogen bond dynamics of aqueous solution/vapor interface is different from bulk as suggested by 
Vibrational two-dimensional SFG (2DSFG) spectroscopy\cite{Asbury2004,Eaves2005,Loparo2006b,Auer2007,Nicodemus2010,Ramasesha2011,NiYicun2012} 
and force-field MD simulations\cite{Luzar1996,AL00,LiuPu2005}. 
Using the polarizable water models, Liu \etal\cite{LiuPu2005} showed that
the dynamics of breaking and forming H-bonds at the water/vapor interface is faster than that in bulk water. 
Benjamin\cite{Benjamin2005} showed that, for some water/organic liquid interfaces, the dynamics are slower at the interface 
than in the bulk and are sensitive to the location of the water molecules along the interface normal.
%[what we learn form this?], but depends on force-field.  
%AIMD 不依赖于具体的任何参数,直接从第一性原理出发给出了界面上的微观动力学。它能更真实地描述界面上的动力学性质(how atoms move as a functon of time)。
However, all these simulation results depend on the force-field. 
Here we have investigated 
the HB dynamics of the interface using DFTMD simulations, which do not require an apriori parameterization.
For example, Usui and coworkers\cite{Usui2015} had shown 
that the long-lived O−D$\cdots$O$_{\text{TMAO}}$ H-bonds between Trimethylamine N-oxide (TMAO) and water (D$_2$O) in aqueous TMAO solutions 
can be only reproduced with AIMD instead of force field-based MD simulations. Recent works based on novel \abinitio 
computational methods\cite{Kuehne2007,Mallik2008,Berkelbach2009,Khaliullin2013}
also show that AIMD can provide unprecedented insights into the nature of H-bonding between water molecules.  
%we adopted the DFTMD simulation.
% what does the ions affect HB dynamics? relaxation? hydrogen bond lifetime? free OH bonds' number?  
%
%reason 2: instantaneous interface & IHB.
% 在DFTMD模拟时,我们也实现了气液溶液的界面选择性。与一般的分子动力学里所采用的平面界面层不同的时,我们将气液界面在空间上的涨落考虑进界面层的定义之中。换句话说,我们选择出的是以实时Willard-Chandler表面作为边界,且厚度作为参量的界面层。以此界面层为基础,我们定义出了这个界面层中的水分子之间的氢键动力学(IHB),OH取向关联函数。由它们出发,我们可以求出这种界面层内的氢键断开和生成的特征时间,以及水分子取向驰豫的特征时间等物理量。我们还研究了这两个性质以及水分子平均拥有的氢键数目与界面厚度的依赖关系。更进一步,这里以Willard-Chandler表面作为边界的界面层,也被推广到了离子外一定半径的球体,例如离子第一溶解球壳内的空间。进而,这个球体内的氢键动力学(SHB)等性质也可以求出。
%[DOUBLE CHECK PREVIOUS KEY RESULTS FROM AB INITIO MD, KOTA USUI, TMAO, Other DFTMD on HBD? Thomas obtaiined ANY NEW RESULTS FROM DFTMD?]
In the analysis of the DFTMD simulations, to selectively address the interface, we 
have used the concept of instantaneous interface. 
%当我们计算气液界面系统的宏观性质时,如果对所有分子做平均,那么所得到的界面性质也会被某些类似体相分子的性质所掩盖。因此,我们有必要考虑那些只处于界面上的分子,即,定义一个界面层,或者说,对具有特定厚度的界面层的水分子采样。传统的方法是将界面层定义为一个长方体薄层,且视其边界为一个平面。
%When we calculate the macroscopic properties of the simulated liquid/vapor interface system, 
%if all molecules are averaged, the interface properties obtained will also be masked by the properties of some bulk molecules. 
%Therefore, it is necessary to consider those molecules that are only at the interface, 
%that is, to define an interface layer, or to sample molecules in an interface layer with a specific thickness. 
The traditional method to describe the interface layer makes use of the Gibbs dividing surface, and therefore regard its boundary as a plane.
We consider here the spatial fluctuations of the aqueous interface into the definition of the interface layer. 
Specifically, we choose the interface layer using the instantaneous Willard-Chandler surface. 
Within the instantaneous interfacial layer, the characteristic time of HB breaking and forming, 
and the characteristic time of the orientation relaxation of water molecules have been calculated. 
The dependence of the average number of H-bonds possessed by water molecules 
on the thickness of the interface has also been studied. 

% Open question 3: thickness of w/v interface
% possible methods?
% significance of these problems?
%We are trying to answer a question: How thick is the interface of the solution? 
As the interface thickness cannot be directly inferred from the VSFG experiments,
MD simulations are a precious tool to understand how far is the interface extending.
%从实时界面层内的氢键动力学这个角度,结合界面分子采样的方法我们给出了估算界面厚度的一种方法。这里的“界面分子采样”是与基于Willard-Chandler表面的界面层目的一样,都是选出位于界面的水分子,以便于我们计算它们的氢键动力学性质。巧妙的是,可以证明,这两种界面选择方式给出的氢键动力学都不是真实的,真实的界面氢键动力学位于二者之间。利用这一特点,只要我们逐渐地增大界面的厚度,就可以发现两种情况下的氢键动力学将收敛到一样。我们可以将这两种氢键动力学收敛到一致时的界面层厚度解释为气液界面的厚度。
%[这半段需要查重]
Due to molecular motions, the identity of molecules at the interface also changes over time\cite{Willard2010}, 
and the fixed interface will no longer apply. 
Therefore, the instantaneous interfaces are calculated for the aqueous electrolyte solutions, and are used in the analysis of interfacial HB dynamics
and reorientation relaxation rate of water molecules.
Combining two different sampling methods, the analysis of interfacial HB dynamics allows us to calculate the interface thickness.
%From the perspective of instantaneous HB dynamics in the interface layer, combined with the method of interface molecular sampling (IMS) from AIMD trajectories,
%we give a method to estimate the interface thickness. The "interface molecule sampling" here is the same purpose as the interface layer 
%based on the instantaneous Willard-Chandler surface, which is to select the water molecules located at the interface so that we can calculate their HB dynamics. 
%It can be proved that the HB dynamics given by these two interface selection methods are not real, and the real interface HB dynamics lies between the two. 
%Using this feature, as long as we gradually increase the thickness of the interface, we can find that the HB dynamics in the two cases will converge to the same. 
%We can interpret the thickness of the interface layer when the two HB dynamics converge to the same as the thickness of the aqueous interface.

Finally, the concept of instantaneous interface 
has been extended to analyze the first solvation shell around the ions.
Furthermore, properties such as solvation shell HB (SHB) dynamics have also been obtained.
%%这后面的,能删的尽量删吧!

%This thesis will study the solution(water)/vapor interfaces from the calculation results of the nitrate and iodide solution's VSFG spectrum,
%the calculation of Willard-Chandler instantaneous interface, 
%the interfacial HB dynamics, ion density distribution among instantaneous interface layers, the orientation relaxation of the OH bond at the interface, etc., 
%in an attempt to fully understand the structure and dynamics of interfaces from different perspectives. 

%Structure of the thesis
The thesis is organized as follows. 
In Chapter \ref{CHAPTER_Methods}, we present the fundations of AIMD 
and the method to calculate the VSFG spectra.
In Chapter \ref{CHAPTER_Clusters}, we present the results of the vibrational properties of water \emph{clusters} including alkaline nitrates, 
which are calculated for interpreting the vibrational characteristics of water molecules in water clusters.
The theoretical results of VSFG spectra of solution/vapor interfaces of alkali metal nitrate solutions are included in Chapter \ref{CHAPTER_SFG}. 
Chapter \ref{CHAPTER_HBD} focuses on the HB dynamics of the water/vapor interface, 
and introduce the method for studying the HB dynamics of the instantaneous interface layer of the water/vapor interface.
% (based on a specific geometric definition of H-bond,)
%最后,我们研究了溶液及其中的氢键动力学,水分子的重定向动力学,以及离子附近的水分子构型。除了硝酸根溶液以外,
%我们也对与之在诸多方面有相似性质的碘离子溶液及其界面做了模拟和相应的分析。这部分内容主要包含在第七章.
In addition to the alkaline nitrate solutions, in Chapter \ref{CHAPTER_HBD_Solutions}, we simulated and analyzed the alkaline iodide solutions 
and their interfaces, which have similar properties in terms of ions' surface propensities and HB dynamics to those of the alkali nitrate solutions. 
The conclusions are summarized in Chapter \ref{CHAPTER_Summary}.
