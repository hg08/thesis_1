\chapter{Introduction}\label{CHAPTER_1}

%Why care interface?

%1
%Why ions at water-vpaor important?
Interfaces of aqueous electrolyte solutions are ubiquitous in the biology, atmosphere, chemistry, man-made systems 
and industrial processes\cite{Irwin88,Tobias99, Benderskii00, 
Asahi01,Benderskii02,Richmond02,LiuH04,
TianCS08,Yamamoto2008, Salmeron2009,ZhangLY09,
LoNostro2012,Piatkowski2014,Balajka2018}.
The aqueous interface will appear in different forms, such as sprays, aerosols, nanoscopic ad  microscopic water droplets, water/vapor interfaces, and so on.
Many phenomena, such as solvation\cite{Benjamin1996}, adsorption\cite{Chang06}, bubble formation\cite{Craig1993,Craig1993b,Weissenborn1995,Marcelja04,Craig04},
occur at aqueous interfaces\cite{Ball2008,Kuo2004b}. 

%2.
Compared to bulk atoms or molecules, interfacial atomic or molecular layers generally have very different properties. 
For example, at the water/vapor interface, the hydrogen bond network is sharply terminated, which makes the interface more hetergeneous 
than bulk water\cite{singh2013}. 
The statistical distribution of the orientation of water molecules at the interface is different from that of bulk water.
Therefore, the local structure, light absorption,  molecular rotations and diffusion are different at the interface than in bulk water\cite{Jedlovszky2004}.
Experiments also demonstrate there exists "on-water" catalysis effect, which means that some chemical reactions take place much faster at the water/vapor 
interface than in bulk phase\cite{Rideout1980,Narayan2005,Beattie2010}.
Molecular simulations also show that the effective dielectric constant of interfacial water is significantly lower than its bulk value, 
and it is also depends on the curvature of the interface\cite{Dinpajooh2016}. 
%What does above properties imply?
Thereforee, it can be seen that aqueous/vapor interfaces provide a unique environment for many physical, chemical and biological processes. 

[EXPLAIN WHICH PROPERTIES ARE DIFFERENT WITH. EXAMPLES AND PROPER SITATIONS. HERE CONCENTRATE
HOW WATER PROPERTIES ARE DIFFERENT.
EX1: HBOND NETWORK IS INTERRUPTED;
WATER ORIENTATION;
THE SPECIFIC ORIENTATION DETERMINE AN INTERFACIAL FIELD;
DIELECTRIC CONSTANT AT INTERFACE IS DIFFERNT FROM BULK;
REACTIVITY IS DIFF FROM BULK;
HIGHER REACTIVITY AT THE WATER SURFACE.]

%Then talk about the special form of liquid/vapor interface---- interfaces of aqueous salt solution.
[FIRST SPECIAL ROLE OF INTERFACES AND THEN MOVE TO THE IONS AT INTERFACES. IN WHICH PROCESSES IONS AT INTERFACES ARE RELEVANT?]
%有这样一类界面--含离子的界面。

% 它们特别值得关注和研究

% 为什么呢?
%原因1

%原因2
In recent years, both experiments\cite{HuJH95} and computer simulations\cite{Knipping00,PJ01,PJ02} have given some conclusions that seem incompatible with traditional thermodynamic theories regarding the properties of the interfaces of solutions containing ions.
%具体地说一下这种不相容的表现

%原因3及举例

%原因4及举例
Understanding ion behavior at the air/water interface is crucial in solving environmental problems such as acid
rain and water pollution\cite{Chang06}.
In atmospheric chemistry, the uptake of pollutants by water clouds depends on the ion
distribution at the aqueous solution/vapor interface. 

%原因5及举例
Besides, our understanding of the interfaces of aqueous salt solutions is also closely related to energy and human development. 
For example, the chemical potential difference of the interface between two seawaters with different salt concentrations can provide new energy for mankind. 
At the intersection of sea water and river water, brackish water is easily produced\cite{Pattle1954,Loeb1976}. 
There are more than 2 billion kilowatts of usable salinity energy on the earth, and its energy is even greater than the temperature difference energy.


Ions at the solution/vapor interface can undergo heterogeneous or interfacial reactions\cite{HuJH95,LiuDF04,Clifford07,Manna13,Pillar2014}.
[DELETE: For example, water accelerate organic reactions under heterogeneous condition\cite{Manna13}. 
Heterogeneous reactions of ozone with bromide at the water/vapor interface of NaBr aerosol were observed, 
with measuring pH changes associated with the interfacial reaction of ozone and bromide\cite{Clifford07}.
Pillar and coworkers'\cite{Pillar2014} work shows a scheme that catechol, a molecular probe of the oxygenated aromatic hydrocarbons, 
present in secondary organic aerosols, contribute interfacial reactive species, which enhance the production 
of humic-like substances under atmospheric conditions. It also implys that catechol undergoes fast oxidation 
at the water/vapor interface by some competing pathways.
Reactions between gases and halide anions are enhanced at the interfaces.\cite{HuJH95,LiuDF04}]

%3.
The influence of ions on the liquid interfaces is of fundamental interest and of practical significance.[TOO GENERIC!] 
Understanding the equilibrium properties and dynamics of ions at
aqueous interfaces is essential in controlling these chemical reactivities at interfaces.


%含溶液的界面如此重要,那么我们能做些什么以便能增进人们对它的认识呢?做了这些又有哪些好处呢?
For obtaining the informations on the specific mechanism underlying the interfacial phenomena, for example,
gaining molecular-level understanding of the interfacial water organization and obtaining the ion distributions at interfaces,
studying the structure and dynamics of aqueous interfaces (solution/vapor interfaces) is essential.
%我们可以做什么呢?
% plan 1 
% 目标是什么?
% 工具是什么?
% 结论如何?

% plan 2
% 目标是什么?
% 工具是什么?
% 结论如何?

% plan 3
% 目标是什么?
% 工具是什么?
% 结论如何?

% plan 4
% 目标是什么?
% 工具是什么?
% 结论如何?


%还有什么重要问题没有解决呢?
% 有什么可能的办法可以使用?
% 这些问题的意义何在?
%潜在问题1
%可能的办法
%意义

%潜在问题2
%可能的办法
%意义

%潜在问题3
%可能的办法,
%意义

This thesis will study the structure and dynamics of the liquid/vapor interfaces from two perspectives: 
vibritional sum-frequency generation (VSFG) spectra and interfacial hydrogen bond (HB) dyanamics.
 
%Fact: 
%\paragraph{Sum-frequency generation}
[THE FIRST POINT TO ADDRESS HERE IS THAT EXPERIMENTS HAVE MADE ENORMOUS PROGRESSES IN THE SELECTIVE CHARACTERIZATION OF LIQUIDS AT INTERFACES. WHY IS SFG SO RELEVANT?]
Experimentally, sum-frequency generation (SFG) spectroscopy has become a powerful surface analytical tool 
for surface studies in many disciplines\cite{Shen2016,Morita2018,Shen2020}. 
It is based on a simple idea that optical responses of a surface and bulk of a medium follow different selection rules.
To determine the structure of an interface, one can probe the \emph{second-order nonlinear susceptibility} $\chi^{(2)}$ of 
the interface\cite{Shen84,Guyot-Sionnest1986,Shen2020}.
For this purpose, the vibrational SFG (VSFG) spectroscopy technology has been frequently used.
It utilizes a second-order nonlinear optical process and the resulting signal is very sensitive to surface ions and 
molecules of a sub-monolayer level\cite{Morita2008,WangHongFei2015,WenYuChieh2016,Ishiyama2017,Penalber-Johnstone2018}. 
This technique allows for detecting intramolecular vibrational modes, and molecular orientation by detecting polarization dependence of the VSFG signals\cite{Vidal05}.  
Furthermore, the VSFG spectroscopy does not require ultrahigh-vacuum environment,
because the interface selectivity is attributed to the symmetry reasons\cite{WeiX02,Morita2018}.
This important property can be simply explained as following.

[DETAILS OF SFG. NOT APPROPRIATE FOR THE INTERODUCTION.]
Theoretically, the interactertion of electromagnetic radiation with mater in the long-wavelenght limit is treated within the \emph{electric dipole approximation} (EDA). 
The EDA is usually expressed in terms of a scalar potential $-{\bf E}(0,t)\cdot r$, where ${\bf E}(0,t)$ is the electric field at the origin, and $\bf r$ is the displacement vector, and a zero vector potential\cite{Kobe1982}. 
%The EDA only approximates the effect of the electric field on the medium (atoms) in the long-wavelength limit and the magnetic multipoles are neglected. 
Within the EDA, the magnetic multipoles are neglected.
A material system under external electric field $\bf E$ induces dipole moment, or
polarization. The polarization $\bf P$ is defined as the dipole moment per
a unit volume of a bulk material. 
When the system indicates a surface, $\bf P$ is defined as the dipole moment per a unit area. 
The induced polarization $\bf P$ is represented as a power series of the electric field\cite{Morita2018}
\begin{equation}
{P}_{i} =\chi^{(1)}_{ij}E_j + \chi^{(2)}_{ijk}E_{j}E_{k} + \cdots,
\label{eq:polarization_1}
\end{equation}
where the subscripts $i,j,k$ denote the Cartesian components. 
The first term describes the linear response of polarization with respect 
to electric field, where $\chi^{(1)}$ is a second-rank tensor, which is called \emph{linear susceptibility}. 
The second term is responsible to the second-order nonlinear optical processes, and $\chi^{(2)}$ is the second-order nonlinear susceptibility.
Therefore, the generation of a polarization in a nonlinear medium by an optical electric field ${\bf E}$ can be presented by
\begin{equation}
{P}^{(2)}_{i} =\chi^{(2)}_{ijk}E_{j}E_{k}.
\label{eq:polarization_1}
\end{equation}
The second-order nonlinear susceptibility $\chi^{(2)}$ is a $3 \times 3 \times 3$ third-rank tensor, which characterizes the process and whose components $\chi^{(2)}_{ijk}$ are 
restricted by the symmetry of the sample.
The components of nonlinear susceptibility represented in different coordinate frames, $\chi^{(2)}_{\alpha\beta\gamma}$ and $\chi^{(2)}_{ijk}$, satisfy the condition 
\begin{equation}
\chi^{(2)}_{\alpha\beta\gamma} = a_{\alpha i}a_{\beta j}a_{\gamma k}\chi^{(2)}_{ijk},
\label{eq:tensor_chi}
\end{equation}
where $a$ can be written as a $3 \times 3$ matrix representing an arbitrary combination of rotation and inversion.
If $a$ is restricted to be a symmetry transformation $A$, then all the properties of the sample are identically described in both coordinate frames.
Then the elements of $\chi^{(2)}$ are the same in both coordinate frames so that
\begin{equation}
\chi^{(2)}_{\alpha\beta\gamma} = A_{\alpha i}A_{\beta j}A_{\gamma k}\chi^{(2)}_{ijk}.
\label{eq:tensor_chi_2}
\end{equation}
If the sample has inversion symmetry\cite{Franken1963}, i.e., $A_{\alpha i} = -\delta_{\alpha i}$, Eq.\thinspace\ref{eq:tensor_chi_2} yields
\begin{align}
\chi^{(2)}_{\alpha\beta\gamma} &= (-\delta_{\alpha i}) (-\delta_{\beta j}) (-\delta_{\gamma k})\chi^{(2)}_{ijk} \nonumber\\
    & = -\chi^{(2)}_{\alpha\beta\gamma} \nonumber\\
    & = 0.
\label{eq:tensor_chi_3}
\end{align}
Therefore, for any material exihibiting inversion symmetry, $\chi^{(2)}$ is identically 0, and SFG is precluded.
This result implies that, within the EDA, the VSFG process is forbidden in any centrosymmetric bulk medium\cite{Che2012},
such as isotropic liquids and glasses, but it is allowed at interfaces because of the broken inversion symmetry\cite{PF00}.
The advantage is its wide applicability to almost every interface which lack a center of inversion, as long as light can reach them. 

%
The second important property of $\chi^{(2)}$ is that it is a polar tensor, which has vectorial nature.\cite{Nihonyanagi2013} 
$\chi^{(2)}$ can also be represented as
$\chi^{(2)}$ can be expressed as the sum of $\beta$ of all molecules in a probed volume
\begin{equation}
\chi^{(2)} = \sum \beta, \nonumber
\label{eq:tensor_chi_4}
\end{equation}
or more quantitative expression
\begin{equation}
\chi^{(2)} = N_\text{s} \beta_{lmn} R(\langle \cos\theta\rangle, \langle \cos^3\theta\rangle),
\label{eq:tensor_chi_5}
\end{equation}
where $N_\text{s}$ is the number of molecules in a unit area, $R(\langle \cos\theta\rangle, \langle \cos^3\theta\rangle)$ is an orientation function,
$\theta$ is the polar orientation angle of a molecule and $\langle \cdots \rangle$ denotes the ensemble average.
Thus $\chi^{(2)}$ has a sign relating to the molecular orientation when $\chi^{(2)}\neq 0$. 
From Eq.~\ref{eq:tensor_chi_5}, if $\chi^{(2)}$ and $\beta_{lmn}$ is known, the sign of $\langle \cos\theta\rangle$ can be determined. 
Therefore, the up/down orientation of interfacial molecules can be experimentally determined.


Therefore, the VSFG spectroscopy can be used to probe many types of interfaces, namely, liquid-liquid and 
solid-liquid interfaces\cite{Guyot-Sionnest1987,RS91,Du93,QD94,Richmond02,Gopalakrishnan2006,ShenYR2006,Morita2008}, metal and semiconductor surfaces\cite{Harris87,Superfine88},
and determine the molecular orientation of molecules at the aqueous solution/vapor interfaces.
The VSFG spectra suggest that the interfacial hydrogen (H-) bonding between water molecules is changed by the presence of salt, 
especially the anions\cite{EAR04}.
%The progress of theoretical support for the VSFG spectra?
Molecular-level properties of interfacial materials arising from interactions between water and minerals, 
such as swelling, wetting, hydrodynamics can also be studied by the VSFG spectroscopy\cite{Rotenberg14}.

%=========================
%Conventional SFG spectra.
%=========================
Conventional SFG spectroscopy detect the intensity of SFG light
\begin{equation}
I_{\text{SFG}}\propto |E_{\text{SFG}}|^2 \propto |\chi^{(2)}|,
\label{eq:sfg_intensity}
\end{equation}
i.e., it gives only the absolute square of the nonlinear susceptibility. Therefore, conventional SFG spectroscopy has the following drawbacks\cite{Nihonyanagi2013}:
(1) the sign of $\chi^{(2)}$ is lost; 
(2) in $|\chi^{(2)}|^2$ SFG spectra, a weak signal $\chi^{(2)}$  will becomes even weaker;
(3) interference among multiple signal components distort SFG spectra, which makes interpretation very difficult.
The distortion of the $\chi^{(2)}$ spectra means: unlike the $\Im \chi^{(2)}$ spectra, which has a Lorenzian band, the peak frequency of the $\chi^{(2)}$ is shifted from the resonant frequency and the band shape become asymmetric. When multiple resonances exist in the same frequency region, unfortunately, the distortion will becomes more complicated. 
Therefore, to interpret the conventional SFG spectrum, one need a model fitting analysis to obtan the parameters of the vibrational resonance of the sample\cite{Nihonyanagi2013}.
Because of the lost of phase information, sometimes this analysis leads to an incorrect conclusion. 
A new technology, phase sensitive (PS-)VSFG\cite{Ji2008} or heterodyne detected (HD-)VSFG have been invented to solve the above 
issues. The most important character is that it can measure complex $\chi^{(2)}$ spectra. In other words, it can provide $\Re \chi^{(2)}$ and $\Im \chi^{(2)}$,
or the applitude and phase of the components of $\chi^{(2)}$. Since $\Im \chi^{(2)}$ spectra shows an absorptive lineshape that directly represents a vibrational resonance, 
it can be interpreted more straightforwardly\cite{Nihonyanagi2013}.
Recently, broadband HD-VSFG spectroscopy that has a high phase stability is developed and is used in studying the polar orientation and HB structure of interfacial
water\cite{Nihonyanagi2009,Shen2013}. 

%There are still two problems in VSFG spectra.
[THIS IS VAGUE. YOU SHUOLD EMPHASIZE THAT WITH THE ps-SFG, YOU CAN GET THE MOLECULAR ORIENTATION.]
With the advent of PS-VSFG technology, more accurate and consistent results about the interface have been reported\cite{TianCS2009,Shen2013}. However, the quantitative interpretation of the VSFG spectra is not straightforward, because the VSFG intensity is influenced by several factors, including ions' concentration, 
molecular orientation and distribution and local field correction\cite{Morita2008}.
%[Known]
%Progress of the study of interfacial structure (a)
Although there is some general consensus on the fact that anions propensity for the interface inversely correlates with
the order of the Hofmeister series, namely 
CO$_3^{2-}$ $>$  SO$_4^{2-}$ $>$ F$^-$ $>$ Cl$^-$ $>$ Br$^-$ $>$ NO$_3^-$ $>$ I$^-$ $>$ ClO$_4^-$ $>$ SCN$^-$\cite{PJ06,ZYJ10,DT08,Parsons2011,HuaWei2013}, the driving force and the microscopic details of the solvation structure are still debated. 
% Other result exp. on ions water interfaces>>

%\paragraph{Ions' adsorption at solution/vapor interface}
As early as the beginning of the last century, Heydweiller discovered that anions affected the surface tension significantly
and the the magnitude of the variation of the surface tension follows the same sequence discovered by Hofmeister earlier\cite{dosSantos10}.
%[DONE for water/vapor interface--Simulations] 
%Progress of the study of interfacial structure (b)
Langmuir\cite{Langmuir1917} was the first to attempt a theoretical explanation of the physical mechanism for the increase of the surface tension by added electrolytes.
The adsorption, or surface excess per unit area, of electrolytes can be described by the well-known Gibbs adsorption equation\cite{Gibbs1928, Adam1941}.
[THE INTRO SHOULD BE MORE DISCORSIVE AND AVOID FORMULA.]
The general form of Gibbs's relation between surface tension $\gamma$, surface excesses $\Gamma_i$, and chemical potentials $\mu_i$ for a system of any number of components is
\begin{equation}
d\gamma = -\sum_i \Gamma_i d\mu_i,
\label{eq:gibbs_relation}
\end{equation}
for a system with constant temperature $T$.
Using the Gibbs adsorption equation, Langmuir concluded that this phenomenon was a consequence of ion depletion 
near the water/vapor interface, i.e., the \emph{increase} in surface tension 
implies a \emph{deficiency} of solute in the surface layer\cite{Jarvis1968}, and
calculated the 'thickness' of the adsorbed layer of pure water, finding 
the depleted layer from 3.3 to 4.2 \A. 
%
Some traditional theories, such as Wagner's theory\cite{Wagner1924,dosSantos10}, 
Onsager and Samaras's theory\cite{Onsager1934}, 
also support the conclusion of ionic depletion near the water/vapor interface.
However, the photoelectron emission experiments\cite{Markovich1991,Ghosal05,Garrett04} and the polarizable 
force-field simulations\cite{Perera1991,Dang1993,Jungwirth2001,Jungwirth2002,PJ06,Horinek07,Brown08,CST11} showed that 
some heavier halies anions are able to \emph{approach} the interface closer than the cations, 
while the surface tension of these solutions is also \emph{increased} compared to pure water. 

%
In particular, molecular dynamics (MD) simulations have shown that more polarizable anions (e.g., larger halide anions) 
are present in the surface region\cite{Jungwirth2001,Jungwirth2002}. 
Tian and coworkers\cite{CST11} have also predicted that some ions, such as \I and Br$^{-}$, could accumulate at the interface.
These seemingly contradictory conclusions with the Gibbs adsorption equation mean that our understanding of the interface is still incomplete. 
It implies that our definition of the solution/vapor interface is not clear enough. 

This thesis will study the solution(water)/vapor interfaces from the calculation results of the nitrate and iodide solution's VSFG spectrum,
the calculation of Willard-Chandler instantaneous interface, 
the interfacial HB dynamics, ion density distribution among instantaneous interface layers, the orientation relaxation of the OH bond at the interface, etc., 
in an attempt to fully understand the structure and dynamics of interfaces from different perspectives. 

Moreover, determination of interface layer thickness of a non-ideal two-phase system is an important problem in interface science, molecular biology, 
and hydromechanics.\cite{LiZhihong2001,Goharzadeh2005,Bano2006} It has direct impact on many technical and natural phenomena, 
sush as the competitive binding of water, ions or denaturants on the macromolecular surface\cite{Arakawa1985,Timasheff2002}, 
aggregation of proteins\cite{Webb2001}, binding of molecules or drugs on protein surface\cite{Hritz2004}, etc. 
In particular, we are trying to answer a question: How thick is the interface of the solution? 
The conclusion will be given from the three perspectives: VSFG spectra, interfacial HB dynamics 
and the orientation relaxation of water molecules at aqueous/vapor interfaces.

%Report the available experimental data for the VSFG spectra of salty interfaces.
%{Selected experimental data on electrolyte interfaces}\label{section_SFG_Exp}
%[Q1: SOME EXPANSION ;
%Q2: ESTABLISH A CONNECTION BETWEEN THESE EXP.S AND YOUR THESIS. IN PARTICULAR, WHICH OPEN QUESTIONS RAISE FROM THE EXPERIMENTS? WHICH ONE WE WANT TO ADDRESS WITH OUR SIMULATIONS]

%\paragraph{Selected experimental data}
Here, we report the experimental results available on salty solutions containing alkali cations and nitrate (iodide) anions\cite{PS03,AJ12,HuaWei2014}. 
From the experimental data of surface tension dependence on solute concentration $\text{d}\gamma/\text{d}m_2$ 
at low electrolyte concentrations ($\leq$1.5 M )\cite{Weissenborn95,Hey81,Jarvis68,Jarvis72}, 
the relation of the surface/bulk molar concentration ratio $K_{\text{p}}$\cite{Pegram2006} among \li, \Na and \K is: 
\begin{equation}
0=K_{\text{p,Na}^+}< K_{\text{p,K}^+}< K_{\text{p,Li}^+}.
\label{eq:bscr}
\end{equation}
i.e., \Na is the most surface-excluded in the alkali nitrate solution RNO$_3$, \K is less excluded, 
and \Li is the least excluded cation (see Appendix \ref{surface_tension_increment} for details).
In modeling the interfaces of aqueous solutions of alkali nitrates, we started with LiNO$_3$, 
because the \Li ion is the least excluded of the solution/vapor interface among the alkali metal ions. 

% WE HAD KNOW THE PROBLEM, THEN HOW TO SET UP OUR MODELS and START TO TRYING TO SOLVE THEM
The structural and dynamical properties of aqueous solutions have been long studied, but the basic physical
property has not been fully clarified.
Complex ions, such as nitrate (\nitrate) and ammonium (NH$_4^+$) ions,
are abundant in environmental and  atmospheric chemistry.\cite{SG05,Yadav2017} 
Nitrate ions containing in water affect the properties of water, and therefore human beings as well.\cite{Comly45,Knobeloch00} 
Raymond and Richamond have shown that there exists anions in the surface region of aqueous solutions of alkali halide salts, 
by using VSFG spectroscopy and comparing the VSFG signal from four alkali halide salt solutions---NaF, NaCl, NaBr 
and NaI. 

% exp. results.
\begin{figure}[H] %[htbp]
\centering
  \includegraphics [width=0.6 \textwidth] {./diagrams/vsfg_alkali_nitrate}
\setlength{\abovecaptionskip}{0pt}
  \caption{\label{fig:Allen12}Experimental VSFG intensity of \LiN solutions, compared with that of neat water\cite{HuaWei2014}.}
\end{figure}
%
Hua \etal\cite{HuaWei2014} have recently measured the VSFG spectra of water/vapor interface of \LiN salt solutions in the OH stretching region
(3000--3800 \centimeter) using HD-VSFG spectroscopy\cite{HuaWei2011,HuaWei2011b,ChenXiangKe2010}. 
The experimental result of the VSFG intensity of the alkali nitrate interfaces is given in Fig.\thinspace\ref{fig:Allen12}. 
At a difference with the spectra for the water/vapor interface, in the spectra of 
\LiN solutions, a depletion of the 3200 \cm peak is observed, with an 
enhancement of the 3400 \cm peak.
A similar behaviour had been observed for the interface of NaNO$_3$ and 
Mg(NO$_3$)$_2$ solutions\cite{AJ12,HuaWei2014}. It has been 
suggested that this depletion of the 3200 \cm peak, and in some cases 
the enhancement of the 3400 \cm peak, is an indication that nitrate 
ions reside at the interface. On the other hand the small 
cations should have little surface propensity. 
It has also been argued that the positive electric field found at the interface of NaCl, NaI and 
NaNO$_3$ salt solutions is due to the formation of an ionic double layer 
between anions located near the surface and their counter-cations (e.g.
Na$^+$) located further below. In Phase-Sensitive (PS) VSFG experiments the 
magnitude of the induced change in the $\Im\chi^{(2)}$ spectra comparatively
to that of the neat water suggested that \nitrate has a surface propensity 
just in between I$^-$ and Cl$^-$\cite{Verreault2013,Verreault2009}. 

Auer and Skinner\cite{Auer08} found that water molecules, at the water/vapor interface, without an HB to the H atom are responsible for the distinct shoulder on the blue side of the Raman spectra, by decomposing the transition frequency distribution into subdistributions for water molecules in different H-bonding environments.
%Caleman and coworkers tried to interpret halide ions' surface preference by molecular simulations of alkali and halide ions in water droplets, 
%by using physical properties of ions, such as water-water interaction, ion-water energy, entropy and polarizability, without using chemical properties.\cite{Caleman11}
%Done for infering properties of water/vapor interface
%The surface tension has been used to infer the composition of the water/vapor interface, 
%since the surface tension of the interface is generally altered by dissolved substance\cite{PJ02}.

The ions propensity for the water/vapor interface, as well as their influence on  
the water's HB network are of special interest to the atmospheric chemistry 
community.\cite{FPBJ,BJ} Various ions play critical roles in the kinetics 
and mechanisms of heterogeneous chemical reactions at the water/vapor interface of atmospheric aerosols. 
The use of surface specific vibrational spectroscopy techniques has 
permitted to elucidate some aspects of surface HB structure for water in 
the presence of ions.\cite{AJ12,AGL05} The influence of molecular ions such as nitrate­, sulfate­ and 
carbonate ions­ has also been analyzed, but proved more elusive than that for halide solution. \cite{SG05,PS03}
Recently lots of attention has been driven by the nitrate ions in aqueous phase for their 
ubiquitous role in atmospheric aerosols from polluted water to the remote troposphere.
\cite{BJ}

%%How to calculate VSFG spectra?
%Unlike an absorption spectrum, the vibrational sum-frequency generation (VSFG) signal can be considered as a sum of signed contributions 
%from different hydrogen-bonded species in the sample.\cite{Pieniazek11}
%%[Done]O: 
%Pieniazek and coworkers\cite{Pieniazek11} had shown that the observed positive feature at low frequency, in the imaginary part of
%the VSFG signal, is a result of cancellation between the positive contributions from four-hydrogen-bonded
%molecules and negative contributions from those molecules with one or two broken H-bonds.

%Via the Gibbs adsorption equation, it has been known that, in 
%aqueous solutions of simple inorganic salts, such as the
%alkali halides, the surface tension increases with solute concentration.\cite{PJ01}

% 5th
%[Progress of the study of HB dynamics of the interface (a)]
%\paragraph{Hydrogen bonding networks}
H-bonding network is important for interfaces of aqueous electrolyte solutions.
The microscopic structure of water is determined by O-H$\cdots$O bonds between the hydroxy group 
and O atoms of neighboring molecules. Therefore, the H-bonded network of water molecules determines complex structural changes 
and properties on ultrafast time scales.\cite{Stenger01, Jimenez1994,Chowdhuri2002}
Processes such as aqueous solvation and the transport of protons is governed by liquid water's properties, 
and these properties arise from the motions of water molecules within a constantly changing H-bonding network.\cite{CJF03}
Because of the H-bonding, electrostatic force and dispersion forces, 
at water/vapor interfaces there exists an interface-specific bonding network, 
which is different from H-bonding network in corresponding bulk liquid.\cite{Allongue96,Velasco-Velez14}
The structural relaxation of the protein also arise from the relaxation of the H-bonding network 
via solvent translational displacement.\cite{Tarek02} In experimental situations, the presence of ions in aqueous 
electrolyte solutions may significantly change the property of the H-bonding network. 
The formation of H-bonding network indicates a reduction in the orientational degree of freedom, 
an enhancement in the local structure of water around the solute, or change (decrese) of entropy.\cite{Frank45a, Frank45b,Frank45c}
The iceberg model was proposed to consider the anomalously large decrease in entropy during hydration.\cite{Frank45c}

%
The H-bonding network \cite{Eisenberg1969,Speedy1976,Poole1994,Soper2008b,Nilsson2011,Ball2001,Pettersson2015} as well as electrostatic forces, 
and van der Waals forces are the main factors that determine the structure of interfaces. 
Salts change the H-bonding structure of water in the interfacial region\cite{EAR04,McLain2006,Ball2008}. 
The specific distribution of the anions at the interface may have significant influence on the H-bonding network of interfacial water\cite{Morita2008}.
HB dynamics is studied to obtain the structural characteristics of the solution/vapor interfaces.
Due to molecular motions, the identity of molecules that on the interface also changes over time\cite{Willard2010}, and the fixed interface will no longer apply. 
To represent the interface at molecular level, the technique for calculating instantaneous liquid interface are used to define a liquid interface from atomic coordinates.
The instantaneous interfaces are calculated for the aqueous electrolyte solutions, and are used in the analyse of interfacial HB dynamics, 
reorientation relaxation rate of water molecules, etc.

%DONE for HBD
The femtosecond InfraRed Spectroscopy (fsIRS) has been used as a new experimental tool to see the HB dynamics in ionic hydration shells\cite{Laage2007}. 
Tominaga and coworkers studied the dynamical structure of water in the presence of alkali-metal and halide ions as functions of temperature and concentration, 
by using the low frequency Raman spectroscopy. 
They found that the water-water intermolecular stretching frequency decrease with increasing ion concentration\cite{KM98,Amo00}.
H-Bonds act as bridges between protein binding sites and their substrates\cite{Ball05}.
The structure and dynamics of H-bonds play an important role in determining the thermodynamic properties of biomolecules in aqueous solutions\cite{HX01}. 
Water’s HB dynamics is also intimately connected to its ultrafast vibrational dynamics\cite{Nagata15}. 
The dynamic process of rupturing and reforming of H-bonds is water solution can be indirectly probed by a number of experimental methods\cite{OC84,JT85}.
The dynamical response of water is intimately related to the lifetime of H-bonds\cite{SP05}. 
Although it can not be yield by experimental methods, it can be studied by computer simulations\cite{Rapaport1983,Voloshin2009}.
Computer simulations are tools for the study of HB dynamics near the solvated ions and biomolecules\cite{PJR79,YKC98}.
%Reactive halogen atoms involved in catalytic reactions are main resource to the Arctic tropospheric ozone depletion. \cite{Foster97,Knipping00,Oum98} 


%DONE by others
In recent years, MD simulations have been used for calculating properties, 
such as the depth profile of ion concentrations of interfaces\cite{Jungwirth2001,Jungwirth2002}, and the VSFG spectra 
of electrolyte solution surfaces\cite{Gopalakrishnan2006,Johnson2014,Ishiyama2014,Ishiyama2017},
but the results depend heavily on the molecular model and interaction potentials used\cite{LXD03,MKP04,TI07,MM05}.
In this thesis, we use density functional theory-based molecular dynamics (DFTMD) simulations to generate the dynamic trajectory of 
the liquid interface, and use them to calculate the various properties mentioned above, including the VSFG spectrum.  
%AIM: compute the interfacial VSFG spectra of electrolyte solutions and to provide their molecular interpretation.
The advantage of DFTMD is that it does not require a priori parameterization and it is capable to include polarization effects\cite{Ufimtsev2011},
also including electronic polarization. DFTMD at the gradient corrected level, and also including dispersion corrections\cite{Grimme04,Grimme06,Grimme07,Grimme10,Baer2011}
has been shown to provide an accurate description of the vibrational properties at interface\cite{Fornaro2015}.
%[anisotropy dynamics is important] 
%In bulk water, measuring the anisotropy dynamics of O-H stretch vibrations\cite{Woutersen99} demonstrated the rapid F{\"o}rster resonant energy transfer between O-H vibrations of different water mmolecules.

[WHY THIS THESIS?]
Recently considerable attention has been given to the nitrate ions in aqueous phase 
for their ubiquitous and diverse role in atmospheric aerosols, polluted water, 
and the remote troposphere\cite{XuM2009,Jubb2012}.
In view of the uniqueness of the adsorption of heavier halide anions in surface tension experiments and molecular dynamics simulations, 
we also chose the interface of lithium iodide, sodium iodide, potassium iodide and other solutions for DFTMD simulation.
In order to clarify the variations of the structural and dynamical properties 
of water containing ions with high surface propensity, MD simulations are a valuable tool, 
which can provide detailed information on the structure and dynamics  
of water in these simple and well-known systems: alkali metal nitrate and alkali metal halide solutions\cite{KM98}.


%Useful models for investigating HB dynamics includes the jump model, extended jump model\cite{Laage2007};
%The analytic kinetic model connected to $C(t)$\cite{Laage2007};
%monoexponential decay, characterized by the reorientation time $\tau_2$, of the orientational time correltion function:
%$C(t)=\langle P_2[{\bf u}(0)\cdot{\bf u}(t)]$, where $P_2$ is the second-rank Legendre polynomial and ${\bf u}$ is the OH-bond direction vector.
%Chloride diffusion constant in water, is calculated from simulations through the ion mean-square displacement 
%$D=\lim\limits_{t\to\infty} 1/6t \langle |r(t)-r(0)|^2\rangle$\cite{Laage2007}.

%[implies]
Based on the instantaneous interface layer, we obtained a series of properties for the aqueous interfaces that are consistent with the experimental results, 
such as the thickness of the water/vapor interface, the relaxation time of the H-bonds at the interface, 
and the distribution of free OH bonds at the interface that have an important contribution to the VSFG spectrum.
These results suggest that considering instantaneous interfaces is essential for understanding the experimental results about aqueous interfaces.

%本论文的结构
The thesis is organized as follows. 
In Chapter \ref{CHAPTER_Methods}, we present the methods of \abinitio molecular dynamics
and the method to calculate the VSFG spectra.
In Chapter \ref{CHAPTER_Clusters}, we present the results of the vibrational properties of water \emph{clusters} including alkaline nitrates, 
which are calculated for interpreting the vibrational characteristics of water molecules in a special water/vapor interface---the water clusters.
The theoretical results of VSFG spectra of solution/vapor interfaces of alkali metal nitrate solutions are included in Chapter \ref{CHAPTER_SFG}. 
Chapter \ref{CHAPTER_HBD} focuses on the HB dynamics of the water/vapor interface, 
and introduce the method for studying the HB dynamics of the instantaneous interface layer of the water/vapor interface.
% (based on a specific geometric definition of H-bond,)
%最后,我们研究了溶液及其中的氢键动力学,水分子的重定向动力学,以及离子附近的水分子构型。除了硝酸根溶液以外,
%我们也对与之在诸多方面有相似性质的碘离子溶液及其界面做了模拟和相应的分析。这部分内容主要包含在第七章.
In addition to the alkaline nitrate solutions, in Chapter \ref{CHAPTER_HBD_Solutions}, we also simulated and analyzed the alkaline iodide solutions and its interface, which have similar properties in ions' surface propensities and HB dynamics. 
The conclusions are summarized in Chapter \ref{CHAPTER_Summary}.
