\chapter{Introduction}\label{CHAPETR_1}
%Why ions at water-vpaor important?
%Water and aqueous solution are indispensable to life.
Interfaces of water and electrolyte solutions exist in many biological and industrial chemical systems, and
are essential for all kinds of physical \cite{Yamamoto2008, Salmeron2009,Balajka2018} and chemical processes. \cite{Tobias99,Benderskii00,Benderskii02}
Many phenomena in biology and chemistry, such as adsorption,
bubble formation, occur at aqueous interfaces. \cite{Ball2008}
Particularly, water/vapor interfaces are the most common liquid interface. \cite{Kuo2004b} 
Water/vapor interfaces of aqueous solutions play a important role in environmental chemistry, biological systems, \cite{ZhangLY09,LoNostro2012} 
man-made systems \cite{Richmond02,LiuH04,Asahi01} and atmospheric science. \cite{TianCS08,Irwin88} 
%2.
Ions at the water/vapor interface can undergo heterogeneous or interfacial reactions. \cite{HuJH95,LiuDF04,Clifford07,Manna13,Pillar2014}
%for example, water accelerate organic reactions under heterogeneous condition.\cite{Manna13} 
%Heterogeneous reactions of ozone with bromide at the water/vapor interface of NaBr aerosol were observed, 
%by Clifford and Donaldson, with measuring pH changes associated with the interfacial reaction of ozone and bromide\cite{Clifford07}.
%In 2014, Pillar et.al.'s work shows a scheme that catechol, a molecular probe of the oxygenated aromatic hydrocarbons, 
%present in secondary organic aerosols, contribute interfacial reactive species, which enhance the production 
%of humic-like substances under atmospheric conditions.\cite{Pillar2014} It also implys that catechol undergoes fast oxidation 
%at the water/vapor interface by some competing pathways.
%Reactions between gases and halide anions are enhanced at the interfaces.\cite{HuJH95,LiuDF04}
%3.
Therefore, the distribution of ions at water/vapor interfaces is essential for understanding the structure and dynamics of interfaces. 
The hydrogen (H-) bonding network \cite{Eisenberg1969,Speedy1976,Poole1994,Soper2008b,Nilsson2011,Ball2001,Pettersson2015} as well as electrostatic force, and van der Waals force are the main factors that determine the structure of interfaces. 
Salts change the H-bonding structure of water in the interfacial region. \cite{EAR04,McLain2006,Ball2008} 
The difference between anions' distribution may have significant influence on the H-bonding network of interfacial water. \cite{Morita2008}

%Fact: 
Compared to bulk atoms or molecules,
interfacial atomic or molecular layers generally have very different optical properties. 
Experimentally, in order to determine the structure of an interface, one can use Vibrational Sum-Frequency Generation (VSFG) spectroscopy.
%What VSFG can provide for water/vapor interface?
The VSFG spectroscopy utilizes a second-order nonlinear optical process and the resulting signal is very sensitive to surface ions and 
molecules of a submonolayer level. \cite{Morita2008,WangHongFei2015,WenYuChieh2016,Ishiyama2017,Penalber-Johnstone2018} 
This technique allows for detecting intramolecular vibrational modes, and molecular orientation by detecting polarization dependence of the VSFG signals. \cite{Vidal05}  
Furthermore, the VSFG spectroscopy does not require ultrahigh-vacuum environment. \cite{WeiX02}
The advantage is its wide applicability to almost every interface as long as light can reach them. 
Therefore, the VSFG spectroscopy can be used to probe many types of interfaces, namely, liquid-liquid and 
solid-liquid interfaces. \cite{Guyot-Sionnest1987,RS91,Du93,QD94,Richmond02,Gopalakrishnan2006,ShenYR2006,Morita2008}
With the electric dipole approximation, the VSFG process is forbidden in any centrosymmetric bulk medium, \cite{Che2012}
such as isotropic liquids and glasses,  but it is allowed at interfaces because of the broken inversion symmetry at the interfaces. \cite{PF00}
%DONE from vsfg spectra?
The VSFG spectroscopy can also be applied to metal and semiconductor surfaces. \cite{Harris87,Superfine88}
The VSFG spectra suggest that the interfacial H-bonding between water molecules is changed by the presence of salt, 
especially the anions. \cite{EAR04}
%The progress of theoretical support for the VSFG spectra?
Molecular-level properties of interfacial materials arising from interactions between water and minerals, 
such as swelling, wetting, hydrodynamics can also be studied by the VSFG spectroscopy. \cite{Rotenberg14}

%There are still two problems in VSFG spectra.
However, the quantitative interpretation of the VSFG spectra is not straightforward,
because the VSFG intensity is influenced by several factors, including ions' concentration, 
molecular orientation and distribution and local field correction. \cite{Morita2008}

%[Known]
%Progress of the study of interfacial structure (a)
Although there is some general consensus on the fact that anions propensity inversely correlates with
the order of the Hofmeister series, namely 
CO$_3^{2-}$ $>$  SO$_4^{2-}$ $>$ F$^-$ $>$ Cl$^-$ $>$ Br$^-$ $>$ NO$_3^-$ $>$ I$^-$ $>$ ClO$_4^-$ $>$ SCN$^-$, \cite{PJ06,ZYJ10,DT08,Parsons2011}
the driving force and the microscopic details of the solvation structure are still debated. 
%[DONE for water/vapor interface--Simulations] 
%Progress of the study of interfacial structure (b)
In terms of the Gibbs adsorption equation, the traditional theory predicts that 
no atomic ions will exist at interfacial regions of solutions. However, 
Molecular Dynamics (MD) simulations have shown that more polarizable anions (e.g., larger halide anions) are present in the surface region. \cite{Jungwirth2001,Jungwirth2002} 
Tian and coworkers \cite{CST11} also have predicted that some ions, such as \I and Br$^{-}$, could accumulate at the interface.
%Auer and Skinner\cite{Auer08} found that water molecules, at the water/vapor interface, without an H-bond to the H atom are responsible for the distinct shoulder on the blue side of the Raman spectra, by decomposing the transition frequency distribution into subdistributions for water molecules in different H-bonding environments.
%Caleman and coworkers tried to interpret halide ions' surface preference by molecular simulations of alkali and halide ions in water droplets, 
%by using physical properties of ions, such as water-water interaction, ion-water energy, entropy and polarizability, without using chemical properties.\cite{Caleman11}
%Done for infering properties of water/vapor interface
The surface tension has been used to infer the composition of water/vapor interface, 
since the surface tension of the water/vapor interface is generally altered by dissolved substance. \cite{PJ02}

%[The ions propensity for the water/vapor interface, as well as their influence on  
%the water's hydrogen bond (HB) network are of special interest to the atmospheric chemistry 
%community.\cite{FPBJ,BJ} Various ions play critical roles in the kinetics 
%and mechanisms of heterogeneous chemical reactions at the water/vapor interface of atmospheric aerosols. 
%The use of surface specific vibrational spectroscopy techniques has 
%permitted to elucidate some aspects of surface HB structure for water in 
%the presence of ions.\cite{AJ12,AGL05} The influence of molecular ions such as nitrate­, sulfate­ and 
%carbonate ions­ has also been analyzed, but proved more elusive than that for halide solution. \cite{SG05,PS03}
%Recently lots of attention has been driven by the nitrate ions in aqueous phase for their 
%ubiquitous role in atmospheric aerosols from polluted water to the remote troposphere.
%\cite{BJ}

%How to calculate VSFG spectra?
%Unlike an absorption spectrum, the Vibrational Sum-Frequency Generation (VSFG) signal can be considered as a sum of signed contributions 
%from different hydrogen-bonded species in the sample.\cite{Pieniazek11}
%[Done]O: 
%Pieniazek and coworkers\cite{Pieniazek11} had shown that the observed positive feature at low frequency, in the imaginary part of
%the VSFG signal, is a result of cancellation between the positive contributions from four-hydrogen-bonded
%molecules and negative contributions from those molecules with one or two broken H-bonds.

%Via the Gibbs adsorption equation, it has been known that, in 
%aqueous solutions of simple inorganic salts, such as the
%alkali halides, the surface tension increases with solute concentration.\cite{PJ01}
%Traditional thermodynamic model suggests ions in water should be
%repelled from the water/vapor interface, but recent 
%MD simulations have predicted that some ions, such as
%\I and Br$^{-}$, could accumulate at the interface.\cite{CST11,PJ01}

% 5th
%[REMOVE] 
%[Progress of the study of HB dynamics of the interface (a)]
% Second, H-bonding network is important for interfaces of aqueous electrolyte solutions.
%The microscopic structure of water is determined by O-H$\cdots$O H-bonds between the hydroxy group 
%and O atoms of neighboring molecules. Therefore, the H-bonded network of water molecules determines complex structural changes 
%and properties on ultrafast time scales.\cite{Stenger01, Jimenez94,Chowdhuri2002}
%Processes such as aqueous solvation and the transport of protons is governed by liquid water's properties, 
%and these properties arise from the motions of water molecules within a constantly changing H-bonding network.\cite{CJF03}
%Because of the H-bonding, electrostatic force and dispersion forces, 
%at water/vapor interfaces there exists an interface-specific bonding network, 
%which is different from H-bonding network in corresponding bulk liquid.\cite{Allongue96,Velasco-Velez14}
%The structural relaxation of the protein also arise from the relaxation of the H-bonding network 
%via solvent translational displacement.\cite{Tarek02} In experimental situations, the presence of ions in aqueous 
%electrolyte solutions may significantly change the property of the H-bonding network. 
%The formation of H-bonding network indicates a reduction in the orientational degree of freedom, 
%an enhancement in the local structure of water around the solute, or change (decrese) of entropy.\cite{Frank45a, Frank45b,Frank45c}
%The iceberg model was proposed to consider the anomalously large decrease in entropy during hydration.\cite{Frank45c}

%DONE for HBD
%The femtosecond InfraRed Spectroscopy (fsIRS) has been used as a new experimental tool to see the HB dynamics in ionic hydration shells.\cite{Laage07} 
%Tominaga and coworkers studied the dynamical structure of water in the presence of alkali-metal and halide ions as functions of temperature and concentration, 
%by using the low frequency Raman spectroscopy. 
%They found that the water-water intermolecular stretching frequency decrease with increasing ion concentration.\cite{KM98,Amo00}

%H-Bonds act as bridges between protein binding sites and their substrates. \cite{Ball05}
%The structure and dynamics of H bonds play an important role in determining the thermodynamic properties of biomolecules in aqueous solutions.\cite{HX01}  Water’s HB dynamics is also intimately connected to its ultrafast vibrational dynamics.\cite{Nagata15} The dynamic process of rupturing and reforming of H-bonds is water solution can be indirectly probed by a number of experimental methods.\cite{OC84,JT85}  The dynamical response of water is intimately related to the lifetime of H-bonds.\cite{SP05}) Although it can not be yield by experimental methods, it can be studied by computer simulations.\cite{DCR83,VPV09}
%Computer simulations are tools for the study of  hydrogen-bond kinetics near the solvated ions and biomolecules. \cite{PJR79,YKC98}
%Reactive halogen atoms involved in catalytic reactions are main resource to the Arctic tropospheric ozone depletion. \cite{Foster97,Knipping00,Oum98} 

% WE HAD KNOW THE PROBLEM, THEN HOW TO SET UP OUR MODELS and START TO TRYING TO SOLVE THEM
%The structural and dynamical properties of aqueous solutions have been long studied, but the basic physical
%property has not been fully clarified.
%Complex ions, such as nitrate (\nitrate) and ammonium (NH$_4^+$) ions,
%are abundant in environmental and  atmospheric chemistry.\cite{SG05,Yadav2017} 
%Nitrate ions containing in water affect the properties of water, and therefore human beings as well.\cite{Comly45,Knobeloch00} 
%Raymond and Richamond have shown that there exists anions in the surface region of aqueous solutions of alkali halide salts, 
%by using VSFG spectroscopy and comparing the VSFG signal from four alkali halide salt solutions---NaF, NaCl, NaBr 
%and NaI. 
Recently considerable attention has been given to the nitrate ions in aqueous phase 
for their ubiquitous and diverse role in atmospheric aerosols, polluted water, 
and the remote troposphere. \cite{XuM2009,Jubb2012}
In order to clarify the variations of the structural and dynamical properties 
of water containing ions with high surface propensity, MD simulations are a valuable tool, 
which can provide detailed information on the structure and dynamics  
of water in a simple and well-known system like alkali metal nitrate and alkali metal halide solutions. \cite{KM98}
%Lots of researchers have studied the dynamical structure of aqueous solutions by using low-frequency Raman spectroscopy (LF-RS).\cite{Ujike99} 
%A femtosecond laser technique enables one to investigate the dynamical property of water by Raman induced Kerr effect
%spectroscopy (RIKES).\cite{Mizoguchi92,Righini93}
%
%Useful models for investigating HB dynamics includes the Jump model, extended jump model\cite{Laage07};
%The analytic kinetic model connected to $C(t)$;\cite{Laage07}
%monoexponential decay, characterized by the reorientation time $\tau_2$, of the orientational time correltion function:
%$C(t)=\langle P_2[{\bf u}(0)\cdot{\bf u}(t)]$, where $P_2$ is the second-rank Legendre polynomial and ${\bf u}$ is the OH-bond direction vector.
%Chloride diffusion constant in water, is calculated from simulations through the ion mean-square displacement 
%$D=\lim\limits_{t\to\infty} 1/6t \langle |r(t)-r(0)|^2\rangle$\cite{Laage07}.
%
%DONE by others
In recent years, MD simulations have been used for calculating properties, 
such as the depth profile of ion concentrations of interfaces, \cite{Jungwirth2001,Jungwirth2002}
and the VSFG spectra of electrolyte solution surfaces, \cite{Gopalakrishnan2006,Johnson2014,Ishiyama2014,Ishiyama2017}
but the results depend heavily on the molecular model and interaction potentials used. \cite{LXD03,MKP04,TI07,MM05}
In this thesis, we use Density Functional Theory-based Molecular Dynamics (DFTMD) simulations to compute 
the interfacial VSFG spectra of eletrolyte solutions and to provide their molecular interpretation.
The advantage of DFTMD is that it does not require a priori parameterization and it is capable to include polarization effects, \cite{Ufimtsev2011}
also including electronic polarization. DFTMD at the gradient corrected level, and also including dispersion corrections \cite{Grimme04,Grimme06,Grimme07,Grimme10,Baer2011}
has been shown to provide an accurate description of the vibrational properties at interface. \cite{Fornaro2015}
%[anisotropy dynamics is important] 
%In bulk water, measuring the anisotropy dynamics of O-H stretch vibrations\cite{Woutersen99} demonstrated the rapid F{\"o}rster resonant energy transfer between O-H vibrations of different water mmolecules.

%This thesis gives the microscopic structural characters of water/vapor interfaces of alkali-nitrate and alkali-halide solutions,
%by calculating the sum-frequency vibrational spectroscopy (SFVS), HB dynamics and rotational anisotropy decay of water molecules on these water interfaces. 
The thesis is organized as follows. 
In chapter \ref{CHAPTER_Methods}, we present the methods of \abinitio Molecular Dynamics and one of its implementations--- 
the DFT, the calculation of the Vibrational Density of States (VDOS), and the method to calculate the VSFG spectra.
In chapter \ref{CHAPTER_SFG_Exp}, we showed the experimental VSFG spectra of salty interfaces.
In chapter \ref{CHAPTER_results_clusters}, we present the results of VDOS for water clusters including alkali metal nitrates, 
 which are calculated for interpreting the vibrational characteristics of water molecules in a special water/vapor interface---the water clusters.
The theoretical results of VSFG spectra of water/vapor interfaces of alkali metal nitrate solutions are included in chapter \ref{CHAPTER_SFG_Calculation}. 
Chapter \ref{CHAPTER_HB} focuses on the Hydrogen Bond (HB) dynamics of water/vapor interfaces, 
% (based on a specific geometric definition of H-bond,)
and the rotational anisotropy decay of water at interfaces of alkaline iodine solutions. 
%最后,我们研究了溶液及其中的氢键动力学,水分子的重定向动力学,以及离子附近的水分子构型。除了硝酸根溶液以外,
%我们也对与之在诸多方面有相似性质的碘离子溶液及其界面做了模拟和相应的分析。这部分内容主要包含在第七章.
Finally, we studied the solution and its HB dynamics, the reorientation dynamics of water molecules, and the configuration 
of water molecules near ions.
In addition to the alkali nitrate solutions, we also simulated and analyzed the alkali iodide solutions and its interface, which have similar properties in many aspects. 
This part of the content is mainly contained in Chapter \ref{CHAPTER_HB_SOLUTIONS}.
The conclusions are summarized in chapter \ref{CHAPTER_Summary}.
