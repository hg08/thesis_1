\chapter{Introduction}\label{CHAPETR_1}

%Why care interface?
The studies of interfacial systems are extremely important in physics, chemistry, biology, materials science, environmental and atmospheric sciences.
%1
%Why ions at water-vpaor important?
Because of extremely high polarity, water is a unique solvent for salts.
Interfaces of aqueous electrolyte solutions are ubiquitous in the biology, atmosphere, chemistry, man-made systems 
and industrial processes\cite{Irwin88,Tobias99, Benderskii00, 
Asahi01,Benderskii02,Richmond02,LiuH04,
TianCS08,Yamamoto2008, Salmeron2009,ZhangLY09,
LoNostro2012,Piatkowski2014,Balajka2018}.
Many phenomena, such as solvation, adsorption\cite{Chang06}, bubble formation\cite{Craig1993,Craig1993b,Weissenborn1995,Marcelja04,Craig04},
occur at aqueous interfaces\cite{Ball2008,Kuo2004b}. 

%2.
Compared to bulk atoms or molecules, interfacial atomic or molecular layers generally have very different properties. 
Ions at the solution/vapor interface can undergo heterogeneous or interfacial reactions\cite{HuJH95,LiuDF04,Clifford07,Manna13,Pillar2014}.
For example, water accelerate organic reactions under heterogeneous condition\cite{Manna13}. 
Heterogeneous reactions of ozone with bromide at the water/vapor interface of NaBr aerosol were observed, 
by Clifford and Donaldson, with measuring pH changes associated with the interfacial reaction of ozone and bromide\cite{Clifford07}.
In 2014, Pillar et.al.'s work shows a scheme that catechol, a molecular probe of the oxygenated aromatic hydrocarbons, 
present in secondary organic aerosols, contribute interfacial reactive species, which enhance the production 
of humic-like substances under atmospheric conditions\cite{Pillar2014}. It also implys that catechol undergoes fast oxidation 
at the water/vapor interface by some competing pathways.
%Reactions between gases and halide anions are enhanced at the interfaces.\cite{HuJH95,LiuDF04}

%3.
The influence of ions on the liquid interfaces is of fundamental interest
and of practical significance. 
Understanding ion behavior at the air/water interface is crucial in solving environmental problems such as acid
rain and water pollution.\cite{Chang06} In atmospheric chemistry,
the uptake of pollutants by water clouds depends on the ion
distribution at the aqueous solution/vapor interface. Understanding the equilibrium properties and dynamics of ions at
aqueous interfaces is essential in controlling the chemical reactivity at interfaces.

%4.
Besides, our understanding of the liquid interface is also closely related to energy and human development. 
For example, the chemical potential difference of the interface between two seawaters with different salt concentrations. 
At the intersection of sea water and river water, brackish water is easily produced. 
There are more than 2 billion kilowatts of usable salinity energy on the earth, and its energy is even greater than the temperature difference energy.

%5.
Therefore, studying the structure and dynamics of aqueous interfaces (solution/vapor interfaces) is essential for 
obtaining the informations on the specific mechanism underlying the interfacial phenomena.
 
%Fact: 
Experimentally, to determine the structure of an interface, one can probe the second-order nonlinear susceptibility $\chi^{(2)}$ of 
the interface\cite{Shen84,Guyot-Sionnest1986,Shen2020}.
For this purpose, the Vibrational Sum-Frequency Generation (VSFG) spectroscopy technology has been frequently used in recent years.
It utilizes a second-order nonlinear optical process and the resulting signal is very sensitive to surface ions and 
molecules of a sub-monolayer level\cite{Morita2008,WangHongFei2015,WenYuChieh2016,Ishiyama2017,Penalber-Johnstone2018}. 
This technique allows for detecting intramolecular vibrational modes, and molecular orientation by detecting polarization dependence of the VSFG signals\cite{Vidal05}.  
Furthermore, the VSFG spectroscopy does not require ultrahigh-vacuum environment,
because the interface selectivity is attributed to the symmetry reasons\cite{WeiX02,Morita2018}.
This important property can be explained as following.


In general, the components of nonlinear susceptibility represented in different coordinate frames, $\chi^{(2)}_{\alpha\beta\gamma}$ and $\chi^{(2)}_{ijk}$, satisfy the condition 
\begin{equation}
\chi^{(2)}_{\alpha\beta\gamma} = a_{\alpha i}a_{\beta j}a_{\gamma k}\chi^{(2)}_{ijk},
\label{eq:tensor_chi}
\end{equation}
where $a$ can written as a $3 \times 3$ matrix representing an arbitrary combination of rotation and inversion.
If $a$ is restricted to be a symmetry transformation $A$, then all the properties of the sample are identically described in both coordinate frames.
Then the elements of $\chi^{(2)}$ are the same in both coordinate frames so that
\begin{equation}
\chi^{(2)}_{\alpha\beta\gamma} = A_{\alpha i}A_{\beta j}A_{\gamma k}\chi^{(2)}_{ijk}.
\label{eq:tensor_chi_2}
\end{equation}
If the sample has inversion symmetry\cite{Franken1963}, i.e., $A_{\alpha i} = -\delta_{\alpha i}$, Eq.\thinspace\ref{eq:tensor_chi_2} yields
\begin{align}
\chi^{(2)}_{\alpha\beta\gamma} &= (-\delta_{\alpha i}) (-\delta_{\beta j}) (-\delta_{\gamma k})\chi^{(2)}_{ijk} \nonumber\\
    & = -\chi^{(2)}_{\alpha\beta\gamma} \nonumber\\
    & = 0.
\label{eq:tensor_chi_3}
\end{align}
Therefore, for any material exihibiting inversion symmetry, $\chi^{(2)}$ is identically 0, and sum-frequency generation is precluded.
This result implies that, with the electric dipole approximation, the VSFG process is forbidden in any centrosymmetric bulk medium\cite{Che2012},
such as isotropic liquids and glasses, but it is allowed at interfaces because of the broken inversion symmetry\cite{PF00}.
The advantage is its wide applicability to almost every interface which lack a center of inversion, as long as light can reach them. 
Therefore, the VSFG spectroscopy can be used to probe many types of interfaces, namely, liquid-liquid and 
solid-liquid interfaces\cite{Guyot-Sionnest1987,RS91,Du93,QD94,Richmond02,Gopalakrishnan2006,ShenYR2006,Morita2008}.

%DONE from vsfg spectra?
The VSFG spectroscopy can also be applied to metal and semiconductor surfaces\cite{Harris87,Superfine88}.
The VSFG spectra suggest that the interfacial H-bonding between water molecules is changed by the presence of salt, 
especially the anions\cite{EAR04}.
%The progress of theoretical support for the VSFG spectra?
Molecular-level properties of interfacial materials arising from interactions between water and minerals, 
such as swelling, wetting, hydrodynamics can also be studied by the VSFG spectroscopy\cite{Rotenberg14}.

%There are still two problems in VSFG spectra.
However, the quantitative interpretation of the VSFG spectra is not straightforward,
because the VSFG intensity is influenced by several factors, including ions' concentration, 
molecular orientation and distribution and local field correction\cite{Morita2008}.
%[Known]
%Progress of the study of interfacial structure (a)
There is some general consensus on the fact that anions propensity for the interface inversely correlates with
the order of the Hofmeister series, namely 
CO$_3^{2-}$ $>$  SO$_4^{2-}$ $>$ F$^-$ $>$ Cl$^-$ $>$ Br$^-$ $>$ NO$_3^-$ $>$ I$^-$ $>$ ClO$_4^-$ $>$ SCN$^-$\cite{PJ06,ZYJ10,DT08,Parsons2011}. 
Heydweiller discovered that anions affected the surface tension significantly and the the magnitude of the variation of the surface tension 
follows the same sequence discovered by Hofmeister earlier\cite{dosSantos10}.
%However, the driving force and the microscopic details of the solvation structure are still debated. 
%[DONE for water/vapor interface--Simulations] 
%Progress of the study of interfacial structure (b)
Langmuir\cite{Langmuir1917} was the first to attempt a theoretical explanation of the physical mechanism for the increase of the surface tension by added electrolytes.
The adsorption, or surface excess per unit area, of electrolytes can be described by the well-known Gibbs adsorption equation\cite{Gibbs1928, Adam1941}.
Using the Gibbs adsorption equation, Langmuir concluded that this phenomenon was a consequence of ion depletion 
near the water/vapor interface, i.e., the increase in surface tension 
implies a deficiency of solute in the surface layer\cite{Jarvis1968}, and
calculated the 'thickness' of the adsorbed layer of pure water, finding 
the depleted layer from 3.3 to 4.2 \A. 

%
Some traditional theories, such as Wagner's theory\cite{Wagner1924,dosSantos10}, 
Onsager and Samaras's theory\cite{Onsager1934}, 
also support the conclusion of ionic depletion near the water/vapor interface.
However, the photoelectron emission experiments\cite{Markovich1991,Ghosal05,Garrett04} and the polarizable 
force-field simulations\cite{Perera1991,Dang1993,Jungwirth2001,Jungwirth2002,PJ06,Horinek07,Brown08,CST11} showed that 
some heavier halies anions are able to approach the interface closer than the cations, 
while the surface tension of these solutions is also increased compared to pure water. 

%
In particular, Molecular Dynamics (MD) simulations have shown that more polarizable anions (e.g., larger halide anions) 
are present in the surface region\cite{Jungwirth2001,Jungwirth2002}. 
Tian and coworkers\cite{CST11} have also predicted that some ions, such as \I and Br$^{-}$, could accumulate at the interface.
These seemingly contradictory conclusions with the Gibbs adsorption equation may mean that our understanding of the interface is still incomplete. 
One possible guess is that our definition of the solution/vapor interface is not clear enough. 
This thesis will study the solution(water)/vapor interfaces from the calculation results of the nitrate and iodide solution's SFG spectrum, 
the interfacial hydrogen bond dynamics, ion density distribution, the orientation relaxation of the OH bond at the interface, etc., 
in an attempt to fully understand the interface from different perspectives. 
Especially, we are trying to answer a question: How thick is the interface of the solution?

%Report the available experimental data for the VSFG spectra of salty interfaces.
%{Selected experimental data on electrolyte interfaces}\label{section_SFG_Exp}
%[Q1: SOME EXPANSION ;
%Q2: ESTABLISH A CONNECTION BETWEEN THESE EXP.S AND YOUR THESIS. IN PARTICULAR, WHICH OPEN QUESTIONS RAISE FROM THE EXPERIMENTS? WHICH ONE WE WANT TO ADDRESS WITH OUR SIMULATIONS]

Here, we report the experimental results available on salty solutions containing alkali cations and nitrate (iodide) anions\cite{PS03,AJ12,HuaWei2014}. 
From the experimental data of surface tension dependence on solute concentration $\text{d}\gamma/\text{d}m_2$ 
at low electrolyte concentrations ($\leq$1.5 M )\cite{Weissenborn95,Hey81,Jarvis68,Jarvis72}, 
the relation of the surface/bulk molar concentration ratio $K_{\text{p}}$\cite{Pegram2006} among \li, \Na and \K is: 
\begin{equation}
0=K_{\text{p,Na}^+}< K_{\text{p,K}^+}< K_{\text{p,Li}^+}.
\label{eq:bscr}
\end{equation}
i.e., \Na is the most surface-excluded in the alkali nitrate solution RNO$_3$, \K is less excluded, 
and \Li is the least excluded cation (see Appendix \ref{surface_tension_increment} for details).
In modeling the interfaces of aqueous solutions of alkali nitrates, we started with LiNO$_3$, 
because the \Li ion is the least excluded of the solution/vapor interface among the alkali metal ions. 

Hua \etal\cite{HuaWei2014} have recently measured the VSFG spectra of water/vapor interface of \LiN salt solutions in the OH stretching region
(3000--3800 \centimeter) using heterodyne detected VSFG spectroscopy\cite{HuaWei2011,HuaWei2011b,ChenXiangKe2010}. 
The experimental result of the VSFG intensity of the alkali nitrate interfaces is given by in Fig.\thinspace\ref{fig:Allen12}. 
At a difference with the spectra for the water/vapor interface, in the spectra of 
\LiN solutions, a depletion of the 3200 \cm peak is observed, with an 
enhancement of the 3400 \cm peak.
A similar behaviour had been observed for the interface of NaNO$_3$ and 
Mg(NO$_3$)$_2$ solutions\cite{AJ12,HuaWei2014}. It has been 
suggested that this depletion of the 3200 \cm peak, and in some cases 
the enhancement of the 3400 \cm peak, is an indication that nitrate 
ions reside at the interface. On the other hand the small 
cations should have little surface propensity. 
It has also been argued that the positive electric field found at the interface of NaCl, NaI and 
NaNO$_3$ salt solutions is due to the formation of an ionic double layer 
between anions located near the surface and their counter-cations (e.g.
Na$^+$) located further below. In Phase-Sensitive (PS) VSFG experiments the 
magnitude of the induced change in the $\Im\chi^{(2)}$ spectra comparatively
to that of the neat water suggested that \nitrate has a surface propensity 
just in between I$^-$ and Cl$^-$\cite{Verreault2013,Verreault2009}. 
% exp. results.
\begin{figure}[H] %[htbp]
\centering
  \includegraphics [width=0.6 \textwidth] {./diagrams/vsfg_alkali_nitrate}
\setlength{\abovecaptionskip}{0pt}
  \caption{\label{fig:Allen12}Experimental VSFG intensity of \LiN solutions, compared with that of neat water\cite{HuaWei2014}.}
\end{figure}


%Auer and Skinner\cite{Auer08} found that water molecules, at the water/vapor interface, without an H-bond to the H atom are responsible for the distinct shoulder on the blue side of the Raman spectra, by decomposing the transition frequency distribution into subdistributions for water molecules in different H-bonding environments.
%Caleman and coworkers tried to interpret halide ions' surface preference by molecular simulations of alkali and halide ions in water droplets, 
%by using physical properties of ions, such as water-water interaction, ion-water energy, entropy and polarizability, without using chemical properties.\cite{Caleman11}
%Done for infering properties of water/vapor interface
%The surface tension has been used to infer the composition of the water/vapor interface, 
%since the surface tension of the interface is generally altered by dissolved substance\cite{PJ02}.

%[The ions propensity for the water/vapor interface, as well as their influence on  
%the water's hydrogen bond (HB) network are of special interest to the atmospheric chemistry 
%community.\cite{FPBJ,BJ} Various ions play critical roles in the kinetics 
%and mechanisms of heterogeneous chemical reactions at the water/vapor interface of atmospheric aerosols. 
%The use of surface specific vibrational spectroscopy techniques has 
%permitted to elucidate some aspects of surface HB structure for water in 
%the presence of ions.\cite{AJ12,AGL05} The influence of molecular ions such as nitrate­, sulfate­ and 
%carbonate ions­ has also been analyzed, but proved more elusive than that for halide solution. \cite{SG05,PS03}
%Recently lots of attention has been driven by the nitrate ions in aqueous phase for their 
%ubiquitous role in atmospheric aerosols from polluted water to the remote troposphere.
%\cite{BJ}

%How to calculate VSFG spectra?
%Unlike an absorption spectrum, the Vibrational Sum-Frequency Generation (VSFG) signal can be considered as a sum of signed contributions 
%from different hydrogen-bonded species in the sample.\cite{Pieniazek11}
%[Done]O: 
%Pieniazek and coworkers\cite{Pieniazek11} had shown that the observed positive feature at low frequency, in the imaginary part of
%the VSFG signal, is a result of cancellation between the positive contributions from four-hydrogen-bonded
%molecules and negative contributions from those molecules with one or two broken H-bonds.

%Via the Gibbs adsorption equation, it has been known that, in 
%aqueous solutions of simple inorganic salts, such as the
%alkali halides, the surface tension increases with solute concentration.\cite{PJ01}

% 5th
%[REMOVE] 
%[Progress of the study of HB dynamics of the interface (a)]
% Second, H-bonding network is important for interfaces of aqueous electrolyte solutions.
%The microscopic structure of water is determined by O-H$\cdots$O H-bonds between the hydroxy group 
%and O atoms of neighboring molecules. Therefore, the H-bonded network of water molecules determines complex structural changes 
%and properties on ultrafast time scales.\cite{Stenger01, Jimenez94,Chowdhuri2002}
%Processes such as aqueous solvation and the transport of protons is governed by liquid water's properties, 
%and these properties arise from the motions of water molecules within a constantly changing H-bonding network.\cite{CJF03}
%Because of the H-bonding, electrostatic force and dispersion forces, 
%at water/vapor interfaces there exists an interface-specific bonding network, 
%which is different from H-bonding network in corresponding bulk liquid.\cite{Allongue96,Velasco-Velez14}
%The structural relaxation of the protein also arise from the relaxation of the H-bonding network 
%via solvent translational displacement.\cite{Tarek02} In experimental situations, the presence of ions in aqueous 
%electrolyte solutions may significantly change the property of the H-bonding network. 
%The formation of H-bonding network indicates a reduction in the orientational degree of freedom, 
%an enhancement in the local structure of water around the solute, or change (decrese) of entropy.\cite{Frank45a, Frank45b,Frank45c}
%The iceberg model was proposed to consider the anomalously large decrease in entropy during hydration.\cite{Frank45c}



% WE HAD KNOW THE PROBLEM, THEN HOW TO SET UP OUR MODELS and START TO TRYING TO SOLVE THEM
%The structural and dynamical properties of aqueous solutions have been long studied, but the basic physical
%property has not been fully clarified.
%Complex ions, such as nitrate (\nitrate) and ammonium (NH$_4^+$) ions,
%are abundant in environmental and  atmospheric chemistry.\cite{SG05,Yadav2017} 
%Nitrate ions containing in water affect the properties of water, and therefore human beings as well.\cite{Comly45,Knobeloch00} 
%Raymond and Richamond have shown that there exists anions in the surface region of aqueous solutions of alkali halide salts, 
%by using VSFG spectroscopy and comparing the VSFG signal from four alkali halide salt solutions---NaF, NaCl, NaBr 
%and NaI. 


%DONE by others
In recent years, MD simulations have been used for calculating properties, 
such as the depth profile of ion concentrations of interfaces\cite{Jungwirth2001,Jungwirth2002}, and the VSFG spectra 
of electrolyte solution surfaces\cite{Gopalakrishnan2006,Johnson2014,Ishiyama2014,Ishiyama2017},
but the results depend heavily on the molecular model and interaction potentials used\cite{LXD03,MKP04,TI07,MM05}.
In this thesis, we use Density Functional Theory-based Molecular Dynamics (DFTMD) simulations to generate the dynamic trajectory of 
the liquid interface, and use them to calculate the various properties mentioned above, including the VSFG spectrum.  
%AIM: compute the interfacial VSFG spectra of electrolyte solutions and to provide their molecular interpretation.
The advantage of DFTMD is that it does not require a priori parameterization and it is capable to include polarization effects\cite{Ufimtsev2011},
also including electronic polarization. DFTMD at the gradient corrected level, and also including dispersion corrections\cite{Grimme04,Grimme06,Grimme07,Grimme10,Baer2011}
has been shown to provide an accurate description of the vibrational properties at interface\cite{Fornaro2015}.
%[anisotropy dynamics is important] 
%In bulk water, measuring the anisotropy dynamics of O-H stretch vibrations\cite{Woutersen99} demonstrated the rapid F{\"o}rster resonant energy transfer between O-H vibrations of different water mmolecules.

Recently considerable attention has been given to the nitrate ions in aqueous phase 
for their ubiquitous and diverse role in atmospheric aerosols, polluted water, 
and the remote troposphere\cite{XuM2009,Jubb2012}.
In view of the uniqueness of the adsorption of heavier halide anions in surface tension experiments and molecular dynamics simulations, 
we also chose the interface of lithium iodide, sodium iodide, potassium iodide and other solutions for DFTMD simulation.
In order to clarify the variations of the structural and dynamical properties 
of water containing ions with high surface propensity, MD simulations are a valuable tool, 
which can provide detailed information on the structure and dynamics  
of water in these simple and well-known systems: alkali metal nitrate and alkali metal halide solutions\cite{KM98}.

%
The hydrogen (H-) bonding network \cite{Eisenberg1969,Speedy1976,Poole1994,Soper2008b,Nilsson2011,Ball2001,Pettersson2015} as well as electrostatic forces, 
and van der Waals forces are the main factors that determine the structure of interfaces. 
Salts change the H-bonding structure of water in the interfacial region\cite{EAR04,McLain2006,Ball2008}. 
The specific distribution of the anions at the interface may have significant influence on the H-bonding network of interfacial water\cite{Morita2008}.
Hydrogen bond (HB) dynamics is studied to obtain the structural characteristics of the solution/vapor interfaces.
Due to molecular motions, the identity of molecules that on the interface also changes over time\cite{Willard2010}, and the fixed interface will no longer apply. 
To represent the interface at molecular level, the technique for calculating instantaneous liquid interface are used to define a liquid interface from atomic coordinates.
The instantaneous interfaces are calculated for the aqueous electrolyte solutions, and are used in the analyse of interfacial HB dynamics, 
reorientation relaxation rate of water molecules, etc.

%DONE for HBD
The femtosecond InfraRed Spectroscopy (fsIRS) has been used as a new experimental tool to see the HB dynamics in ionic hydration shells\cite{Laage2007}. 
Tominaga and coworkers studied the dynamical structure of water in the presence of alkali-metal and halide ions as functions of temperature and concentration, 
by using the low frequency Raman spectroscopy. 
They found that the water-water intermolecular stretching frequency decrease with increasing ion concentration\cite{KM98,Amo00}.
H-Bonds act as bridges between protein binding sites and their substrates\cite{Ball05}.
The structure and dynamics of H-bonds play an important role in determining the thermodynamic properties of biomolecules in aqueous solutions\cite{HX01}. 
Water’s HB dynamics is also intimately connected to its ultrafast vibrational dynamics\cite{Nagata15}. 
The dynamic process of rupturing and reforming of H-bonds is water solution can be indirectly probed by a number of experimental methods\cite{OC84,JT85}.
The dynamical response of water is intimately related to the lifetime of H-bonds\cite{SP05}. 
Although it can not be yield by experimental methods, it can be studied by computer simulations\cite{Rapaport1983,Voloshin2009}.
Computer simulations are tools for the study of hydrogen-bond kinetics near the solvated ions and biomolecules\cite{PJR79,YKC98}.
%Reactive halogen atoms involved in catalytic reactions are main resource to the Arctic tropospheric ozone depletion. \cite{Foster97,Knipping00,Oum98} 

%Useful models for investigating HB dynamics includes the jump model, extended jump model\cite{Laage2007};
%The analytic kinetic model connected to $C(t)$\cite{Laage2007};
%monoexponential decay, characterized by the reorientation time $\tau_2$, of the orientational time correltion function:
%$C(t)=\langle P_2[{\bf u}(0)\cdot{\bf u}(t)]$, where $P_2$ is the second-rank Legendre polynomial and ${\bf u}$ is the OH-bond direction vector.
%Chloride diffusion constant in water, is calculated from simulations through the ion mean-square displacement 
%$D=\lim\limits_{t\to\infty} 1/6t \langle |r(t)-r(0)|^2\rangle$\cite{Laage2007}.

%[implies]
Based on the instantaneous interface layer, we obtained a series of properties for the aqueous interfaces that are consistent with the experimental results, 
such as the thickness of the water/vapor interface, the relaxation time of the H-bonds at the interface, 
and the distribution of free OH bonds at the interface that have an important contribution to the VSFG spectrum.
These results suggest that considering instantaneous interfaces is essential for understanding the experimental results about aqueous interfaces.
%

The thesis is organized as follows. 
In Chapter \ref{CHAPTER_Methods}, we present the methods of \abinitio Molecular Dynamics
and the method to calculate the VSFG spectra,
In Chapter \ref{CHAPTER_Clusters}, we present the results of the vibrational properties of water clusters including alkaline nitrates, 
which are calculated for interpreting the vibrational characteristics of water molecules in a special water/vapor interface---the water clusters.
The theoretical results of VSFG spectra of solution/vapor interfaces of alkali metal nitrate solutions are included in Chapter \ref{CHAPTER_SFG}. 
Chapter \ref{CHAPTER_HBD} focuses on the HB dynamics of the water/vapor interface, 
and introduce the method for studying the HB dynamics of the instantaneous interface layer of the water/vapor interface.
% (based on a specific geometric definition of H-bond,)
%最后,我们研究了溶液及其中的氢键动力学,水分子的重定向动力学,以及离子附近的水分子构型。除了硝酸根溶液以外,
%我们也对与之在诸多方面有相似性质的碘离子溶液及其界面做了模拟和相应的分析。这部分内容主要包含在第七章.
In addition to the alkaline nitrate solutions, in Chapter \ref{CHAPTER_HBD_Solutions}, we also simulated and analyzed the alkaline iodide solutions and its interface, which have similar properties in ions' surface propensities and HB dynamics. 
The conclusions are summarized in Chapter \ref{CHAPTER_Summary}.
