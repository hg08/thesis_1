\chapter{HB Lifetime Distributions and Instantaneous Interfaces}
\section{Relations between HB Lifetime Distributions}\label{relation_hbd}
\paragraph{Different HB lifetime Distributions}\label{diff_distr}
From the probability $P_{tc}(t)$ of the total HB lifetime in a configuration, and the probability $P_{a}$ of 
the first HB breaking in time $t$ after it have been detected at the moment $t$, one can introduce the average time 
$\langle \tau_{tc}\rangle$ and $\langle \tau_{a}\rangle$ :
\begin{equation}
\langle \tau_{tc}\rangle = \int_0^\infty t P_{tc}(t) dt,
\label{eq:tau_tc}
\end{equation}
\begin{equation}
\langle \tau_{a}\rangle = \int_0^\infty t P_a(t) dt. 
\label{eq:tau_a}
\end{equation}
Since $P_{\mathrm{a}}(t)=\int_{t}^{\infty} P_{\mathrm{tc}}(\tau) \frac{d \tau}{\tau}$, i.e., 
\begin{equation}
P_{tc}(t) = -t\frac{dP_a(t)}{dt}, \nonumber
\label{eq:relation_Ptc--Pa}
\end{equation}
integrating by parts, we obtain
\begin{eqnarray}
&&\langle \tau_{tc}\rangle = -\int_0^\infty t^2 \frac{dP_a(t)}{dt}dt \nonumber \\
&&= -\int_0^\infty t^2 dP_a(t) = -[-2\langle \tau_{a}\rangle + 0] \nonumber\\
&&= 2\langle \tau_{a}\rangle,\nonumber
\end{eqnarray}
in which we used $\int_0^\infty d(t^2 P_a)=0$.
Therefore, there is a relationship between $\langle \tau_{tc} \rangle$ and $\langle \tau_a \rangle$:
\begin{eqnarray}
\langle \tau_{tc}\rangle = 2\langle \tau_{a}\rangle.
\label{eq:relation_tau_tc--t_a}
\end{eqnarray}

We denote the probability of the total HB lifetime along a trajectory as $P_{\mathrm{tt}}(t)$,
then the average HB lifetime over the trajectory is
\begin{eqnarray}
\left\langle\tau_{\mathrm{tt}}\right\rangle=\int_{0}^{\infty} t P_{\mathrm{tt}}(t) d t.
\label{eq:relation_tau_tt}
\end{eqnarray}
Because $\int_{0}^{\infty} P_{\mathrm{tc}}(t) d t=\frac{1}{\langle \tau_{tt}\rangle} \int_{0}^{\infty} t P_{\mathrm{tt}}(t) d t = 1$, 
we get 
\begin{eqnarray}
P_{\mathrm{tt}}(t)=\left\langle \tau_{\mathrm{tt}}\right\rangle P_{\mathrm{tc}}(t) / t.
\label{eq:relation_P_tt--P_tc}
\end{eqnarray}
The difference between the two distribution functions, $P_{\mathrm{tt}}(t)$ and $P_{\mathrm{tc}}(t)$, can be described as follows.
The $P_{\mathrm{tt}}(t)$ represents the percentage of pairs of molecules that had a continuous H-bonds during time $t$, 
while the $P_{\mathrm{tc}}(t)$ the percentage of the number of H-bonds with a given lifetime $t$ to the number of all H-bonds in any configuration. 
\cite{Voloshin2009}

Since $P_{\mathrm{a}}(t)=\int_{t}^{\infty} P_{\mathrm{tc}}(\tau) \frac{d \tau}{\tau}$,
we can obtain
\begin{eqnarray}
&& P_{\mathrm{a}}(t)=\int_t^\infty \frac{P_{tt}}{\langle\tau_{tt}\rangle} \frac{\tau}{\tau} d\tau \nonumber \\
&& =  \int_t^\infty \frac{P_{tt}}{\langle\tau_{tt}\rangle}d\tau. \nonumber
\label{eq:P_a}
\end{eqnarray}
Let $t=0$, we obtain
\begin{eqnarray}
P_{\mathrm{a}}(0)=1 /\left\langle t_{\mathrm{tt}}\right\rangle = 1 /\left\langle t_{\mathrm{HB}}\right\rangle.
\label{eq:P_a0}
\end{eqnarray}

From Eq. \ref{eq:tau_tc} and the relation between \SHB and $P_{\mathrm{a}}(t)$
\begin{eqnarray}
S_{\mathrm{HB}}(t)=\int_{t}^{\infty} P_{\mathrm{a}}(\tau) d \tau,
\label{eq:P_a}
\end{eqnarray}
we can obtain
%
\begin{eqnarray}
&&\int_{0}^{\infty} \int_{t}^{\infty} P_{a}(t) d \tau d t = \int_{0}^{\infty} \int_{0}^{\tau} P_{a}(\tau) d t d \tau \nonumber \\
&& = \int_{0}^{\infty} \tau P_{a}(\tau) d \tau \nonumber \\
&& = \langle \tau_{\mathrm{a}} \rangle, \nonumber
\end{eqnarray}
i.e., 
\begin{eqnarray}
\int_{0}^{\infty}  S_{\mathrm{HB}}(t) d t = \langle \tau_{\mathrm{a}} \rangle.
\label{eq:int_Ca}
\end{eqnarray}
\paragraph{Calculation of HB lifetime distributions}
In this paragraph, we describe the method to calculate the above lifetime distributions $P_{tc}(\tau)$, $P_{a}(\tau)$, and $P_{tt}(\tau)$.

First, we describe the method of calculating $P_{tc}(\tau)$.
Theoretically speaking, in order to calculate $P_{tc}(\tau)$, our detection time $t$ must meet the following conditions: 
$t-t_0 > \tau_{hb}^{\max}$, where $t_0$ is the initial time and $\tau_{hb}^{\max}$ is the maximum lifetime value of the H-bonds in the system. 
However, the value of $\tau_{hb}$ cannot be known in advance. 
In order to reduce the error, the method we can adopt is to set an empirical value as large as possible for
 $\tau_{hb}^{\max}$ if conditions permit. Since the value of $\tau_{hb}^{\max}$ is limited, in principle the lifetime value of the HB 
can always be greater than $\tau_{hb}^{\max}$. 
Therefore, the average value of the HB lifetime distribution 
calculated in this approximate way will move to a shorter lifetime than the average value of the true HB lifetime distribution:
\begin{eqnarray}
\int_{0}^{\infty} \tau P_{\mathrm{tc}}^{\mathrm{approx}}(\tau) d \tau < \left\langle\tau_{\mathrm{tc}}\right\rangle.
\label{eq:upper_bound_1_for_tau_tc}
\end{eqnarray} 

Among the hydrogen bonds detected at time $t$, if there are some H-bonds that have existed at the beginning $t_0$ and 
remain in existence until time $t$, then we can approximately express the lifetime of these H-bonds as $\delta t^{(j)}=t^{(j)} -t_0$, 
where $j=1,\cdots, m$ is the labels of the $m$ H-bonds and $t^{(j)}$ is the moment when the $j$-th HB is broken. 
If we use $\tau^{(j)},j=1,\cdots,m$ to represent the true lifetimes of these $m$ H-bonds,
then we can find that $\tau^{(j)}-\delta t^{(j)} >0 $.
Since we cannot judge the true lifetime of these $m$ hydrogen bonds, we can use $\delta t^{(j)}$ to approximate 
the lifetime of these $m$ H-bonds, that is
\begin{eqnarray}
\tau^{(j)} = \delta t^{(j)}.
\end{eqnarray}
For those H-bonds that did not exist at the beginning, the method of calculating their lifetime is very straightforward, 
the lifetime $\tau^{(j)}$ is equal to the time $t^{(j)}$ when the HB is broken minus the moment $t^{{(j)}}_0$ of its formation:
\begin{eqnarray}
\tau^{(j)} = t^{(j)} - t^{(j)}_0,
\end{eqnarray} 
where the superscript $j = 1, \cdots, m'$, identifies $m'$ H-bonds 
detected at time $t$, and
formed after $t_0$ and broken at $t^{(j)}$.

Specifically, for the AIMD simulation results we obtained, we also approximate $P_{tc}$ as follows: 
We select evenly distributed $n$ time points $t=t_1,...,t_n$, from the trajectory obtained by the simulation, 
and count the HB lifetimes at each time point $t_i$.
The distribution function $P_{tc}(\tau)$ 
can be obtained by the average of the lifetime distribution detected at a certain time $t_i$, $i=1,\cdots, n$, 
where $t_i-t_{i-1} = \tau_{hb}^{\max}$. 

If only from the perspective of simulation data, we have another way to obtain $P_{tc}(\tau)$: 
Count the lifetimes of H-bonds that are formed after the initial time $t_0$ and are broken before the end time $t_f$.
Although the distribution obtained in this method cannot be verified experimentally, 
it is the true distribution of HB bonds in simulated systems. 

%TODO
%For $P_a(\tau)$
%For $P_{tt}(\tau)$

\paragraph{Calculation of the Reactive Flux}\label{calc_rf}
For dynamical variables $x_i(t)$ and $x_k(t)$, their correlation functions have the following relationship:\cite{Landau1980}
\begin{equation}
\langle x_i(t') x_k(t)\rangle = -\langle x_i(t) x_k(t')\rangle,
\label{eq:correlation_relation}
\end{equation}
or 
\begin{equation}
\phi_{ik}(t) = -\phi_{ki}(t). \nonumber
\label{eq:correlation_relation2}
\end{equation}
Let $x_i = h$, $x_k = \dot h$,
then, we obtain
\begin{equation}
\langle h(0) \dot{h}(t)\rangle=-\langle\dot{h}(0) h(t)\rangle. 
\label{eq:h_correl_relation}
\end{equation}
From the definition of the reactive flux $k(t) = -dc/dt$, we obtain 
\begin{equation}
k(t)=-\langle h(0) \dot{h}(t)\rangle /\langle h\rangle. 
\label{eq:rf1}
\end{equation}
Then from Eq. \ref{eq:h_correl_relation},
we get 
\begin{equation}
k(t) =  \langle \dot{h}(0)h(t)\rangle /\langle h\rangle. \nonumber
\label{eq:rf2}
\end{equation}
Since $\langle\dot{h}(0)\rangle=0$, $k(t)$ can be calculated by
\begin{equation}
k(t) = - \langle \dot{h}(0)[1-h(t)]\rangle /\langle h\rangle.
\label{eq:rf3}
\end{equation}

%\paragraph{Fitting $C_2(t)$ with a Single Exponential Function}\label{single_exp}
%We assume that the anisotropy decay $C_2(t)$ is a single exponential given by 
%\begin{equation}
%C_2(t) = A e^{-\kappa t},
%\label{eq:tcf2}
%\end{equation}
%where $\kappa$ is a relaxation rate constant of the anisotropy decay. For each value of $\kappa$, we denote the relaxation period as $1/\kappa$.
%The $C_2(t)$ for water molecules in different environments in LiI solution at 330 K is shown in
%Fig.\space\ref{fig:2LiI-124w_c2_fit_5_single-exp}.
%%
%\begin{figure} [htbp]
%\centering
%	\includegraphics [width=0.60\textwidth] {./diagrams/2LiI-124w_c2_fit_5_single-exp}
%\setlength{\abovecaptionskip}{0pt}
%	\caption{\label{fig:2LiI-124w_c2_fit_5_single-exp} The anisotropy decay of OH bonds in water molecules at the water/vapor interface of LiI solutions (5 ps).}
%\end{figure} 
%It is evident, from the figure, that $C_2(t)$ for water molecules neither in the bulk or at interface of the LiI solution can \emph{not} be described as a single exponential dynamics.  Besides, we also have fitted the $C_2(t)$ for water molecules in NaI solution and the result is the same as above. The fitting parameters for LiI and NaI solution are presented in table \ref{tab:c2_single-exp-fitting_LiI} and \ref{tab:c2_single-exp-fitting_NaI}, respectively.
%%-- 
%\begin{table}
%\centering
%\caption{\label{tab:c2_single-exp-fitting_LiI}%
%	The single-exponentially fitted parameters---the amplitude $A$, the decay rate $\kappa$, the relaxation period $1/\kappa$, of anisotropy decay for water molecules 
%  at the water/vapor interface of 0.9 M LiI solutions, at 330 K.} 
%\begin{tabular}{lccc}
%	water molecules &  $A$ & $\kappa$ (THz) & $1/\kappa$ (ps)  \\
%\hline
%	I$^-$-shell  & 0.83  & 0.26  & 3.85  \\
%	Li$^+$-shell & 0.89  & 0.07  & 14.29 \\
%	bulk & 0.86 & 0.12 & 8.33 \\
%	surface & 0.75 & 0.29 & 3.45 \\
%\end{tabular}
%\end{table}
%%-- 
%\begin{table}[H]
%\centering
%\caption{\label{tab:c2_single-exp-fitting_NaI}%
%	The single-exponentially fitted parameters---the amplitude $A$, the decay rate $\kappa$, the relaxation period $1/\kappa$, of anisotropy decay for water molecules 
%	at the water/vapor interface of 0.9 M NaI solutions, at 330 K.} 
%\begin{tabular}{lccc}
%	water molecules &  $A$ & $\kappa$ (THz) & $1/\kappa$ (ps)  \\
%\hline
%	I$^-$-shell  & 0.86  & 0.14  & 7.14 \\
%	Na$^+$-shell & 0.79 & 0.07  & 14.29 \\
%	bulk & 0.83 & 0.06  & 16.67 \\
%	surface & 0.78 & 0.12 & 8.34 \\
%\end{tabular}
%\end{table}

\paragraph{Relaxation Time of Hydrogen Bonds in Bulk Water}\label{rate_const_and_tau_R_128w}
% 
In addition to the simulation in the main text of this thesis, we also use different methods for the bulk water system with the same temperature,
volume and number of molecules. The number of water molecules in the system is 128, the temperature is still $T=300$ K, and the box is a cube with a side length of 15.6404 \A. 
In this simulation, we relax the value of the target accuracy for the SCF convergence to 10$^{-6}$. 
Other settings, such as exchange correlation functional, correction of dispersion force, etc. are the same as the text. 
For the dynamic trajectory of such a system, we also calculated the self-correlation function $c(t)$ of the HB population operator, 
and the functions $k(t)$ and $n(t)$ derived from it. 
Table \ref{tab:k_k_prime_128w_1} and \ref{tab:k_k_prime_128w_2} shows the rate constant $k$, $k'$ and 
the relaxation time $\tau_{HB}$ obtained by the least squares fit method. 
It can be seen from the tables that the accuracy of the calculation of $k$, $k'$ are accurate to at least two decimal places.

%
\begin{table}[htbp]
\centering
\caption{\label{tab:k_k_prime_128w_1} 
    The $k$ and $k'$ for the bulk water. We carried on the short time region 0.2 ps $< t <$ 2 ps. 
The unit for $k$ ($k'$) is ps$^{-1}$, and that for $\tau_{\text{HB}}$ ($=1/k$) is ps. The $h(t)$ is bond-based.} 
\begin{tabular}{cccc}
 Criterion & $k$  (bulk) & $k'$ (bulk) & $\tau_{\text{HB}}$ (bulk) \\
\hline
  ADH & 0.3346 & 0.9374 & 2.9884  \\
  ADH(from $k_{in}$) & 0.2988  & 1.0290 & 3.3467   \\
  AHD & 0.3224 & 1.0059 & 3.1014 \\ 
  AHD(from $k_{in}$) & 0.2884 & 1.1210 & 3.4679 \\ 
\end{tabular}
\end{table}
%
\begin{table}[htbp]
\centering
\caption{\label{tab:k_k_prime_128w_2} 
    The $k$ and $k'$ for the bulk water. We carried on the long time region 2 ps $< t <$ 12 ps. 
The unit for $k$ ($k'$) is ps$^{-1}$, and that for $\tau_{\text{HB}}$ ($=1/k$) is ps. The $h(t)$ is bond-based.} 
\begin{tabular}{cccc}
 Criterion & $k$  (bulk) & $k'$ (bulk) & $\tau_{\text{HB}}$ (bulk) \\
\hline
  ADH & 0.1032 & 0.0237 & 9.6901 \\
  ADH(from $k_{in}$) & 0.1028  & 0.0275 & 9.7283 \\
  AHD & 0.1036 & 0.0339 & 9.6558  \\
  AHD(from $k_{in}$) & 0.1031  & 0.0401 & 9.7016  \\
\end{tabular}
\end{table}

\paragraph{Different Definitions of the $h(t)$}\label{DEF_POPULATION_OPERATOR}
The correlation functions $C_{HB}(t)$ of water molecules in bulk water is shown in Fig. \ref{fig:bk_water_c_two_population_operators_with_ADH}.
It shows that there are big differences in the correlation function $C_{HB}(t)$ of the two definitions of $h(t)$,
because hydrogen exchange is considered in the O--H pair-based definition of $h(t)$, but not in the water--water pair-based definition.
% bk_water_c_two_population_operators_with_ADH.eps
% /home/gang/Github/water_pair_HB_dynamics/plot/plot_c/bk_water_c_two_population_operators_with_ADH.eps
% old version: 128w_bk--water-pair-based_and_bond-based_c.eps
\begin{figure} [htbp]
\centering
	\includegraphics [width=0.36\textwidth] {./diagrams/bk_water_c_two_population_operators_with_ADH}
\setlength{\abovecaptionskip}{0pt}
	\caption{\label{fig:bk_water_c_two_population_operators_with_ADH} The correlation functions $C_{HB}(t)$ of water molecules in bulk water (ADH criterion). 
        Here $h(t)$ is based on two different definitions, one is based on a pair of water molecules (solid line),\cite{Khaliullin2013} 
and the other is based on O--H pairs between water molecules (dashed line).}
\end{figure} 

\section{Dynamics of Interfacial Hydrogen Bonds} \label{ihb_and_selection}
\paragraph{Instantaneous Surfaces}
Let us consider the interfacial system of pure water.  At a certain time $t$, the instantaneous surface ${\mathbf s}^0(x(t),y(t))$ can be determined by calculating 
the coarse-graining density field: we can specify that the coarse-grained density is the reference density $\rho_\text{ref} = 0.016 $ \A$^{-3}$,
and those grid points constitute the surface ${\mathbf s}^0(x,y)$ of pure water. In our simulated pure water system, there are two such surfaces.
The code for calculating the instantaneous surfaces of interfacial systems can be found on \url{https://github.com/hg08/interface}. 
%===================
\paragraph{Griding and Layering}
Wee assume that the normal is along the $z$-axis direction. 
First, we need to discretize the coordinates of the $xy$ plane. 
We divide the edges along the $x$-axis direction and the edges along the $y$-axis direction of the simulated box into $N$ parts uniformly, 
and the $xy$ plane can be approximated by $N\times N$ discrete points. 
Then $N\times N$ ordinates of the surface ${\mathbf s}^0(x(t_0),y(t_0))$ at the initial time $t_0$ can be represented as components of an $N\times N$-vector ${\mathbf z}^0(t_0$). 
The surface ${\mathbf s}^0(x(t_0),y(t_0))$ is also the upper boundary of the interface.
Secondly, we define a layering strategy, or define the thickness $d$ of the interface layer. 
In this way, we then determine the lower boundary ${\mathbf s}^1(x(t_0),y(t_0))$ of the interface. 
It can be expressed as a $N\times N$-vector ${\mathbf z}^1(t_0)={\mathbf z}^0(t_0)-{\mathbf d}$, and ${\mathbf d}$ is a $N\times N$-dimensional constant 
vector in which all the entries are $d$. The superscripts 0 and 1 respectively identify the upper and lower boundaries of the interface. 
Similarly, the other interface can also be determined.

So far, our interfaces are still static, because we only considered the initial moment. 
But the interfaces defined here can be defined for different moments naturally. 
Therefore, according to the same method, for the molecular motion trajectory of an interface system, 
we can obtain the dynamically changing interface layer, i.e., ${\mathbf z}^1(t)={\mathbf z}^0(t)-{\mathbf d}$. 
When the interfaces are determined, we can answer the question: For an atom with coordinates $(x, y, z)$ at any time $t$, whether it is in the interface?
To answer this question, we need to map the atom's coordinates $(x, y)$ to an ordered integer pair $(i, j)$, $i,j \in  \{0,\cdots,N\}$, 
where $i = \text{int}(x /\Delta x), j = \text{int}(y/\Delta y)$, $\Delta x = a/N$, and $\Delta y = b/N$.
 
At a certain time $t$, the ordinate $z$ of an atom can be approximately represented as a function defined on point $(i, j)$, $z=z(i,j)$. 
Then we compare $z(i,j)$ and $z^l_{i,j}$, $l=0,1$ to determine if the atom is in the interface. 
That is, if $z^0_{i,j}(t)<z(i,j)<z^1_{ij}(t)$, then we know that at time $t$, the atom at $(x,y,z)$ is located in the interface.

\paragraph{Selcetion of Water Molecules from Instantaneous Interface}
The first method to extract the interfacial molecules is as follows. 
1) Determine the instananeous surface of a system;
2) Define a interface system with a fixed thickness below the instantaneous surface; 
3) Select the atoms which are near the instananeous surface;
4) Calculate the HB dynamics for molecules in interface at different time $t$.
Given the thickness of layer $d$, at any time $t_0$, $t_1$, $t_2$, $\cdots$, $t_n$, the set of molecules in the interface can be determined. 
Since the molecules are always in motion, generally, the set $S^0(t_i)$ of the molecules in the interface at time $t_i$ 
and the set $S^0(t_j)$ of the molecules in the interface at a different time $t_j$ is different. 
Therefore, to calculate the HB dynamics of molecules in the interface, we need to calculate the correlation functions of 
the HB population $h(t)$ of the molecules in the interface at different times, and average the correlation functions obtained at all times.
 
\paragraph{Interfacial Hydrogen Bond Population Operator $h^s(t)$}
The second method to extract the interfacial molecules is as follows. 
1) Determine the instananeous surface of a system;
2) Define a interface system with a fixed thickness below the instantaneous surface; 
3) Define the interfacial hydrogen bond population operator $h^s(t)$.
4) Calculate the autocorrelation function of $h^s(t)$ and then the related observations. 

If we want to calculate only the dynamic characteristics of molecules located in the interface, 
the above statistical method will have a error in select correct molecules in interface. 
In chapter \ref{CHAPTER_HB}, we had combined the recognition technology of the instantaneous liquid interfaces \cite{Willard2010} 
with the definition of H-bonds \cite{AL96b,Luzar1996} to define H-bonds that depend on the environment. 
With this method, we can have more understanding of the differences and commonalities in the kinetics of the breaking and reconstruction of the hydrogen bonds 
of the water molecules in the interface system. At the same time, we can compare the HB dynamcis in the interface of different thicknesses 
with the HB dynamics in bulk water to obtain the thickness of the interface from the perspective of HB dynamics. 
In addition, for the interface system of different solutions, we can also use the layering technique to study the HB dynamics,
so that we can understand the particularity of the interface of various solutions relative to pure water. 
%These results may help us have a better understanding of the physical and chemical processes related to interfaces.

In fact, none of the above-mentioned two methods for obtaining the HB dynamics of the interface water molecules 
can give the true interface HB dynamics completely and accurately. But they respectively give an extreme case of interface HB dynamics. 
In the first method, the formation and breaking of intermolecular H-bonds can be truly described, 
but the selection of interface water molecules is not accurate enough. Since the configuration of the molecule will change over time, 
the contribution of the H-bonds in the bulk phase will be included. 
In the second method, we are very accurate in choosing the interface water molecules and the H-bonds in the interface, 
but we may artificially destroy some H-bonds in the interface that were not broken. 
To a certain extent, the HB dynamics obtained by the second method is \emph{accelerated}. 
Therefore, the comparison of the results obtained by these two methods may give a true picture of the interface HB dynamics.

The code for calculating the HB dynamics for instantaneous interfaces can be found on \url{https://github.com/hg08/hb_in_interface}. 
%===================

