\chapter{Experimental VSFG spectra of solution/vapor interfaces}\label{CHAPTER_SFG_Exp}
In this chapter, we give the experimental results obtained on salty solutions containing alkali cations and nitrate (iodide) anions. \cite{PS03,AJ12,HuaWei2014} 

From the experimental data of surface tension dependence on solute concentration $\text{d}\gamma/\text{d}m_2$ 
at low electrolyte concentrations ($\leq$1.5 M ), \cite{Weissenborn95,Hey81,Jarvis68,Jarvis72} 
the relation of the surface/bulk molar concentration ratio $K_{\text{p}}$ \cite{Pegram2006} among \li, \Na and \K is: 
\begin{equation}
0=K_{\text{p,Na}^+}< K_{\text{p,K}^+}< K_{\text{p,Li}^+}.
\label{eq:bscr}
\end{equation}
i.e., \Na is the most surface-excluded in the water solution RNO$_3$, \K is less excluded, 
and \Li is the least excluded cation. (See Appendix \ref{surface_tension_increment} for details.)
In modeling the interfaces of aqueous solutions of alkali metal nitrates, we decided to start with LiNO$_3$, because the \Li ion is the least excluded of the vapor-liquid interface 
among the alkali metal ions. 

Hua \etal \cite{HuaWei2014} have recently measured the VSFG spectra of water/vapor interface of \LiN salt solutions in the OH stretching region
(3000--3800 \centimeter) using Heterodyne Detected VSFG spectroscopy. \cite{HuaWei2011,HuaWei2011b,ChenXiangKe2010} 
The experimental result of the VSFG intensity of the alkali nitrate interfaces is given by in Fig.\space\ref{fig:Allen12}. 
At a difference with the spectra for the water interface, in the spectra of 
\LiN solutions, a depletion of the 3200 \cm peak is observed, with an 
enhancement of the 3400 \cm peak.
A similar behaviour had been observed for the interface of NaNO$_3$ and 
Mg(NO$_3$)$_2$ solutions. \cite{AJ12,HuaWei2014} It has been 
suggested that this depletion of the 3200 \cm peak, and in some cases 
the enhancement of the 3400 \cm peak, is an indication that nitrate 
ions reside at the interface. On the other hand the small 
cations should have little surface propensity. 
It has also been argued that the positive electric field found at the interface of NaCl, NaI and 
NaNO$_3$ salt solutions is due to the formation of an ionic double layer 
between anions located near the surface and their counter-cations (e.g.
Na$^+$) located further below. In Phase-Sensitive (PS) VSFG experiments the 
magnitude of the induced change in the Im$\chi^{(2)}$ spectra comparatively
to that of the neat water suggested that \nitrate has a surface propensity 
just in between I$^-$ and Cl$^-$. \cite{Verreault2013,Verreault2009} 
% exp. results.
\begin{figure}[H] %[htbp]
\centering
  \includegraphics [width=0.6 \textwidth] {./diagrams/vsfg_alkali_nitrate}
\setlength{\abovecaptionskip}{0pt}
  \caption{\label{fig:Allen12}Experimental VSFG intensity of \LiN solutions, compared with that of neat water. \cite{HuaWei2014}}
\end{figure}
