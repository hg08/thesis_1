  \chapter{Alkali nitrate clusters}\label{CHAPTER_Clusters}
There are two ways of obtaining highly specific information on solvation structures: studying gas-phase clusters consisting of ions surrounded by a few water molecules\cite{Weber2000,Kropman2001},
and exciting and detecting a dissolved prob molecule\cite{Jimenez1994}.
Due to the experimental difficulties, most information on the dynamics of aqueous solvation shells was obtained from MD simulations\cite{Smith1994,Chandra2000}.
  In this chapter, DFTMD simulations of gas phase clusters including alkali cations, nitrate ions and a few water molecules have been used to obtain these specific information and to understand the
  effects of alkali cations and nitrate anions on H-bonding\cite{jiangling2010,heine2015}. 
  The VDOS is used to extract the vibrational signatures for the water molecules in these systems.
  In paragraph \ref{paragraph_3w_nitrate} and \ref{paragraph_clusters_alkali_water}, 
  the effect of the anion and the cation are separately investigated. 
  The two clusters, [NO$_3\cdot$(H$_2$O)$_3$]$^-$ and [Li$\cdot$(H$_2$O)$_4$]$^+$, are used 
  to study the structural and dynamical properties of water clusters with nitrate ions and with alkali cations. 
  In paragraph \ref{paragraph_clusters_alkali_nitrate_and_water_molecules}, the effects of alkali metal cations 
  and nitrate anion are discussed within clusters containing both cations and anions and an increasing number of water molecules.
  %
  \section{Cluster of nitrate and water molecules}\label{paragraph_3w_nitrate}
  %EXAMPLE of FORMULA
  %$$X_n=X_k \qquad\hbox{if and only if}\qquad Y_n=Y_k \quad\hbox{and}\quad Z_n=Z_k.$$
  
  %\begin{wrapfigure}{l}[0.05cm]{8.0cm}
  %\centering
  %\includegraphics[width=0.3\textwidth]{./diagrams/3_NO3_small}
  %\setlength{\abovecaptionskip}{10pt}
  %\caption{The geometry optimized structure of the cluster NO$_3^-$(H$_2$O)$_3$. The red dotted lines denote the H-bonds.}\label{fig:3_NO3_small}
  %\end{wrapfigure}
  \begin{figure}[H]
  \centering
  \includegraphics[width=0.5\textwidth]{./diagrams/clusters_4}
  \setlength{\abovecaptionskip}{0pt}
    \caption{\label{fig:clusters_4}Geometry optimized structure of the clusters: 
    (a) [NO$_3\cdot$(H$_2$O)$_3$]$^-$; 
    (b) RNO$_3$(H$_2$O)$_3$; 
    (c) RNO$_3$(H$_2$O)$_4$; 
    (d) RNO$_3$(H$_2$O)$_5$ (R=Li, Na, K).
    More structural properties are given in Appendix \ref{structure_of_clusters}.}
  \end{figure}
  First, we consider the nitrate--water cluster, [NO$_3\cdot$(H$_2$O)$_3$]$^-$. The symmetric isomer of the cluster, 
  as shown in Fig.\thinspace\ref{fig:clusters_4} a, is obtained by geometry optimization at the BLYP/TZV2P level of theory. 
  According to the definition of a HB\cite{JT90,SB02}, there are three H-bonds in it,
  i.e., only one of the two OH bonds is H-bonded to \nitrate in each water molecule. 
  Therefore, the two OH bonds in each water molecule exhibit different vibrational features. 
  Fig.\thinspace\ref{fig:vdos_NO3-3w_2_H6H7} shows the VDOS for the OH bonds in the cluster.
  For each water molecule, one OH bond is vibrating in the frequency range 3680--3700 cm$^{-1}$, 
  while the other in the range 3380--3440 cm$^{-1}$. 
  The difference of frequencies between the vibrational modes is about $\Delta\nu=$ 180 \centimeter.
  %
  \begin{figure}[H] %[htbp]
  \centering
  \includegraphics [width=0.36\textwidth] {./diagrams/vdos_NO3-3w_2_H6H7_simple}%
  \setlength{\abovecaptionskip}{0pt}
    \caption{\label{fig:vdos_NO3-3w_2_H6H7}The VDOS for the two OH bonds in w1 (Fig.\thinspace\ref{fig:clusters_4} a) of [NO$_3\cdot$(H$_2$O)$_3$]$^-$.} 
    %The length of the DFTMD trajectory is 5 ps.
  \end{figure}  %(Calculated from the function vdos3.f and ft$\_$5s.sh)
  %(MS:repetition)A quasi-HB is formed if the O--H distance $r_\text{OH}$ satisfies the condition $r_\text{OH}<3.5$ \AA, but not that the O--H$\cdots$O angle is less than $\frac{\pi}{6}$. \cite{JT90} 
  Additionally, we label the three water molecules as w1, w2, and w3, respectively (Fig.\thinspace\ref{fig:clusters_4} a). 
  For the three water molecules, we find some differences in the structural parameters.
  Table\thinspace\ref{tab:3_nitrate_bond} gives the calculated lengths of H-bonds in the cluster [NO$_3\cdot$(H$_2$O)$_3$]$^-$. 
  The average differences $\Delta{d}$ between the H-bonds and other three bonds are 0.69 \AA (Table\thinspace\ref{tab:3w_nitrate}). 
  %============
  %section 4_Li
  %============
  \section{Cluster of alkali metal cation and water molecules}\label{paragraph_clusters_alkali_water}
  \begin{wrapfigure}{l}[0.05cm]{6.5cm}
  \centering
  \includegraphics[width=0.25\textwidth]{./diagrams/4_Li}
  \setlength{\abovecaptionskip}{0pt}
  \caption{\label{fig:4_Li}The cluster [Li$\cdot$(H$_2$O)$_4$]$^+$.}
  \end{wrapfigure}
  %====
  To find the effects of alkali cations on the structural properties of water, we investigated the cluster \LiFourW 
  (Fig.\thinspace\ref{fig:4_Li}). We concentrate on two aspects: the radial distribution function (RDF), 
  and the VDOS for water molecules of this cluster.

  The sharp peaks in the RDF given in Fig.\thinspace\ref{gdr_4_Li} show that the solvation shell of \Li is bound to all the four water molecules.
  The peak for $g_{\text{LiO}}$ is at 2.02 \AA, and for $g_{\text{LiH}}$ is at 2.69 \AA. 
  \begin{figure}[b!]
  \centering
  \includegraphics[width=0.36\textwidth] {./diagrams/gdr_4_Li}
  \setlength{\abovecaptionskip}{0pt}
    \caption{\label{gdr_4_Li}RDFs $g_{\text{Li-O}}$ and $g_{\text{Li-H}}$ for the cluster \LiFourW.} 
  \end{figure}
  %

  The VDOS for water molecules in the cluster [Li$\cdot$(H$_2$O)$_4$]$^+$ is calculated from a 20-ps trajectory,
  during which one water molecule escaped from the bonding of the Li and then formed a new HB to 
  another water molecule. Figure \thinspace\ref{fig:vdos_4_Li} (A) 
  shows that, in this case, there are two types of OH stretching modes in the cluster \LiFourW:
  free OH stretch which peaks at 3705 \cm and bonded OH stretch at 3625 \centimeter. 
  However, the water molecules just bound to Li has two degenerate free O-H stretching modes. 
  %Second, Fig.\thinspace\ref{vdos_4_Li}  shows that if a H-bonded water molecule is in the solvation shell of \Li, 
  %the H-bonded OH streching frequency is 75 \cm higher than that of the H-bonded OH strech outside the solvation shell of \Li. 
\begin{figure}%
    \centering
    \subfloat[]{{\includegraphics[width=5.3cm]{./diagrams/vdos_4_Li} }}
    \qquad
    \subfloat[]{{\includegraphics[width=5.0cm]{./diagrams/vdos_4_Li-wat_w1_5ps} }}
    \caption{
(A) 
The VDOS for the four water molecules (all water molecules) in the cluster \LiFourW.
(B)
The VDOS for the water molecules (three water molecules) bound to Li in the cluster \LiFourW.
}%
    \label{fig:vdos_4_Li}%
\end{figure}
  %
  The VDOS for water molecules only bound to Li (Fig.\thinspace\ref{fig:vdos_4_Li} (B)) shows that these water molecules only have free OH stretch, 
  since there is only a broad stretching mode at 3705 cm$^{-1}$.

  \section{Clusters of alkali nitrate and water molecules}\label{paragraph_clusters_alkali_nitrate_and_water_molecules}
  %\begin{wrapfigure}{r}[0.05cm]{7.5cm}
  %\centering
  %\includegraphics[width=0.3\textwidth]{./diagrams/3_RNO3}
  %\setlength{\abovecaptionskip}{10pt}
  %\caption{\label{fig:3_RNO3}The clusters RNO$_3$(H$_2$O)$_3$ (R=Li, Na, K).}
  %\end{wrapfigure}
  %===================
  %MS:INSERT AN INTRO:
  %===================
  As a first minimal model system for the interfaces of alkali nitrate solution, we consider alkali nitrate water clusters including 3 to 5 waters. 
  The idea is to investigate the effect of the alkali nitrate on the vibrational properties of those water molecules which are directly 
  H-bonded to the ions.
  In our simulations, the clusters are geometry optimized and the most stable configurations are determined (Fig.\thinspace\ref{fig:clusters_4} b--d).
  The first interesting result is that for all the clusters containing 3 to 5 water molecules, a contact ion pair is maintained during the 
  simulation trajectories where a \emph{direct} interaction involves the cation and one of the nitrate oxygen's. 
  %The most stable isomer of the RNO$_3$(NO$_3$)$_3$ complex (R=Li,Na,K) is shown in Fig.~\ref{fig:clusters_4}(b) and compared to the symmetric slovation of the simple NO$_3$(H$_2$O)$_3$. 
  %Independently of the type of alkali cations (Li,Na,K), the most stable structure is the same.

In the LiNO$_3$(H$_2$O)$_3$ cluster, there are three H-bonds and three Li-O bonds. 
The average lengths of them are given in Table\thinspace\ref{tab:table_lino3}. 
We use HB1, HB2 and HB3 to denote the HB between w1 and w2, w2 and \nitrate, and w3 and \nitrate, 
respectively (Fig.\thinspace\ref{fig:clusters_4} b). Both the average lengths of HB1 and HB3 are very close 
to each other and both of them are smaller than that of HB2. 
Since both w1 and w3 are bound to \li, we calculate an average value $\bar{d}_{\text{HB}}=1.81$ \AA of the lengths of HB1 and HB3.
The difference between length of HB2 and $\bar{d}_{\text{HB}}$ is $\delta d_{\text{HB}}=0.19$ \AA.
By testing the difference of environment of each H-bonds,  we obtain that $\delta d_{\text{HB}}$ comes from the 
difference between Li-O bonds and H-bonds.
\begin{table}[htbp]
\centering
\caption{\label{tab:table_lino3}%
  The average length $r_a$ of H-bonds (Li-O bonds) in the cluster LiNO$_3$(H$_2$O)$_3$.}
\begin{tabular}{ccc}
Bonds& $r_a$( \AA) \\ 
\hline
HB1 &1.83 $\pm$ 0.14\\
HB2 &2.00 $\pm$ 0.25 \\
HB3 &1.79 $\pm$ 0.16 \\
O(w1)--Li &1.95 $\pm$ 0.09 \\
O(w3)--Li &1.92 $\pm$ 0.07 \\
nitrate O--Li &1.91 $\pm$ 0.08
\end{tabular}
\end{table}

%\subsection{Structural Characterization}
 Besides, the RDF between the alkali (\li, \na\space or \pot) and the water O (panel a) and the nitrate O -- water H (panel b) are reported in Fig.\thinspace\ref{fig:gdr_3_RNO3}. 
The sharp and left-shifed peaks in the RDF (Fig.\thinspace\ref{fig:gdr_3_RNO3} b) 
show that the nitrate is solvated and in particular a stronger HB is formed in the presence of the cation in all three clusters. 
%=============
%MS: added XXX
% To express number ranges like 2--4, use en-dash, which can be implemented by typing two hyphens (--).
%=============

 The vibrational features associated to the small clusters are calculated from the VDOS and reported in
Fig.\thinspace\ref{fig:vdos_Li_Na_K-NO3-3w}.
In the frequency range 3000--3800 \centimeter, each water molecules has two vibrational bands. In addition
to the free OH peak at 3700 \centimeter, we see that the HB band is characterized by quite strong \emph{red-shifted} 
peaks around 3200 \centimeter. These red-shifted peaks are associated to water molecules which are bound either to 
the cation or to both cation and anion and are different with respect to the peaks associated to the water molecules
which only bound to the nitrate in the simple cluster [NO$_3\cdot$(H$_2$O)$_3$]$^-$ (3430 \centimeter, see Fig.\thinspace\ref{fig:vdos_NO3-3w_2_H6H7}). 
%
%\egin{figure}[h!]
\begin{figure}[H]
\centering
  \includegraphics [width=0.72\textwidth] {./diagrams/gdr_3_RNO3}
\setlength{\abovecaptionskip}{0pt}
\caption{\label{fig:gdr_3_RNO3} (a) RDF $g_{\text{R-O}}$ for clusters RNO$_3$(H$_2$O)$_3$ (R=Li, Na, K);
  (b)RDF $g_{\text{O-H}}$ for clusters RNO$_3$(H$_2$O)$_3$ and [NO$_3\cdot$(H$_2$O)$_3$]$^-$ (no alkali metal cation, denoted as "R=-").}
\end{figure}
%
%\begin{figure}[htbp]
\begin{figure}[H]
 \centering
 \includegraphics [width=0.90\textwidth, center] {./diagrams/vdos_Li_Na_K-NO3-3w}
 \setlength{\abovecaptionskip}{0pt}
  \caption{\label{fig:vdos_Li_Na_K-NO3-3w}The VDOS for H$_2$O in clusters: (a) LiNO$_3$(H$_2$O)$_3$, (b) NaNO$_3$(H$_2$O)$_3$ and (c) KNO$_3$(H$_2$O)$_3$.  
 w1: \water bound to R and \water;
 w2: H$_2$O bound to nitrate and \water;
 w3: \water bound to R and nitrate.
 } 
\end{figure}
%
\begin{figure}[H]
\centering
\includegraphics [width=0.90\textwidth, center] {./diagrams/vdos_LiNO3-3-5w}
\setlength{\abovecaptionskip}{0pt}
\caption{\label{fig:vdos_LiNO3-3-5w}The VDOS for H$_2$O in clusters LiNO$_3$(H$_2$O)$_n$: 
  (a) $n=3$;  (b) $n=4$; (c) $n=5$.
  w1: H$_2$O bound to Li and \water;
  w2: H$_2$O bound to nitrate and \water;
  w3: H$_2$O bound to Li and nitrate;
  w4: H$_2$O bound to \water;
  w5: H$_2$O only bound to Li.}
%w1 : the water molecule bonded to \Li and \water,
%w2 the water molecule bound to \nitrate and \wat, 
%and w3 the water molecule bound to \Li and \nit.
\end{figure}
%

To explore the effect of adding some additional water molecules to
the cluster, we considered the clusters RNO$_3$(H$_2$O)$_n$ ($n$=4, 5; R=Li, Na, K).
The most stable configurations are shown in Fig.\thinspace\ref{fig:clusters_4} c and d,
and the corresponding VDOS for water molecules are shown in
Fig.\thinspace\ref{fig:vdos_LiNO3-3-5w} b and c for the clusters LiNO$_3$(H$_2$O)$_n$
($n$=4 and 5). We find that the OH stretching peaks in the HB region are also quite red-shifted.
The red shift is particularly strong for the water molecules which are directly interacting with
the Li and those which are simultaneously bound to the Li and to the nitrate O's (e.g. w3).

%
%In the frequency range 2800--3800 \centimeter, each water molecule has two vibrational bands. In addition to the free OH peak at 3700 \centimeter, we can see that the HB band is 
%characterized by quite strong red-shifted peaks around 3200 \centimeter.
%These peaks are associated to water molecules which are bound either to the cation or to both cation and anion and 
%are different with respect to the peaks associated to the water molecules which only bound to \nitrate in the simple 
%NO$_3^-$(H$_2$O)$_3$ cluster (3460 \centimeter, Fig. ~\ref{fig:vdos_Li_Na_K-NO3-3w_roman_font40}(b) ).
%Compared with the VDOS for NO$_3^-$(H$_2$O)$_3$, the bending modes of the water molecules in RNO$_3$(H$_2$O)$_3$ are redshifted slightly. The OH stretching modes of water molecules bounded to the alkali cation in RNO$_3$(H$_2$O)$_3$ are redshifted ($|\Delta\nu|>200$ \centimeter)(b).

We also calculated the effects of other alkali metal cations, namely Na$^+$ and K$^+$. 
The calculated VDOS for water molecules in clusters NaNO$_3$(H$_2$O)$_3$ and KNO$_3$(H$_2$O)$_3$ are shown in 
Fig.\thinspace\ref{fig:vdos_Li_Na_K-NO3-3w} b and c, respectively. As in the case of LiNO$_3$(H$_2$O)$_3$, the HB 
bands are also characterized by red-shifted peaks around 3200 \centimeter.
In addition, the peaks in the OH-stretching region are also compatible with infrared predissociation
(IRPD) spectra which have been recorded for the [Li$\cdot$(H$_2$O)$_{3-4}$Ar]$^+$
clusters \cite{rodriguez2011, Miller2008, Miller2008b}
and for [Na$\cdot$(H$_2$O)$_{4-7}$]$^+$ and [K$\cdot$(H$_2$O)$_{4-7}$]$^+$ clusters\cite{beck2011}, although there no
nitrate is present and only the effect of the cation was investigated.

To summarize, the vibrational spectra from the clusters clearly point to red-shifted peaks which are not 
recorded in the VSFG spectra at the solution/vapor interface for the \LiN solution. 
In other words, these clusters are not really representative of the solvation structures presents in the \LiN solution.
Therefore, these small clusters cannot be directly used to describe the topmost layer of the \LiN solution, 
and we need to build more realistic models to capture the main features of the interface. 
In particular, according to the cluster picture one would be tempted to rule out the possibility of a contact 
ion pair at the interface.

