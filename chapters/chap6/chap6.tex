\chapter{Hydrogen Bond Dynamics of Water/Vapor Interfaces }\label{CHAPTER_HB}
%There are two types of bonds in water: the stronger covalent bonds (molecular $\sigma$ bonds) within a single water molecule and 
%the much weaker H-bonds between water molecules.
H-bonds play a critical role in the behaviour of bulk water,\cite{Eisenberg1969,Luzar1996,Cabane2005} water near interfaces,\cite{Chowdhary2008} 
and aqueous solutions. \cite{Naslund2005} There are many methods to study the hydrogen bond (HB) dynamics in water, solutions or interfaces, 
such as molecular dynamics simulation,\cite{Tongraar2006,Chanda2006,Tongraar2010,Chowdhary2008,Banerjee2016} neutron scattering, 
Infrared (IR) spectroscopy,\cite{Werhahn2011,Fournier2016} etc.
In this chapter, we will introduce the general concepts and methods of HB Dynamics \cite{AL96,Luzar1996,DC87} used to analyze the structure 
and dynamic properties of bulk water and water-air interfaces. 

\section{Definitions of HB Population and Correlation Functions}
Luzar and Chandler \cite{AL96} have elucidated the HB dynamics of pure water, and
subsequently such analysis has been also extended to explore the HB dynamics
in complex situations, e.g., electrolytes, \cite{AC00} protein and  micellar surfaces. \cite{SP05}
There are temporal, geometric\cite{Kumar2007} and energetic criteria \cite{Sciortino1989}to define HB.
Here we use the geometric one.
Two water molecules are H-bonded only if their interoxygen distance between of specific tagged pair of water molecules 
is less than $r^{\text{c}}_{\text{OO}}$ and
the O-H$\cdots$O angle is less than $\phi^{\text{c}}$. \cite{AKS86,JT90,SB02} 
The value $r^{\text{c}}_{\text{OO}}$ corresponds to the first-minimum position of the O--O RDF of water. \cite{Sciortino1989}   
The choice for the cutoff angle $\phi^{\text{c}}$ for water-water molecules is obtained by studying the average number of H-bonds,
as a function of $\phi^{\text{c}}$. \cite{Luzar1993} We call this definition of HB the ADH criterion. 
In order to compare the impact of different HB definitions on HB dynamics, we will also use another definition of HB in our analysis. 
When the distance between the oxygen atoms of two water molecules is less than $r^{\text{c}}_{\text{OO}}$, 
and the oxygen-hydrogen-oxygen included angle is greater than a critical angle $\theta^{\text{c}}$, then we say that there is a HB between the two molecules. 
We denote this definition as the AHD definition of H bonds.
%The distance criteria of $R_{\text{OH}}$  was determined from the first minimum in the O--H RDF for SPC water. \cite{HJCB81}

% introduce h(t)
The configuration criterion above allows us to define a variable $h[r(t)] = h(t)$, HB population. 
The $h(t)$ has a value 1 when the particular tagged pair of molecules are bonded, and 0 otherwise. 
%=================
% added 2020-5-27: to show that h(t) is actually the fluctuation of itself (\delta h).
We know that the instantaneous fluctuation or deviation in a dynamical variable $A(t)$ from its time-independent equilibrium average $\langle A\rangle$ , 
is defined by \cite{DC87} 
$$
\delta A = A(t) - \langle A\rangle.
$$
For the $h(t)$, since the probability that a specific pair of molecules is bonded in a large system is extremely small, i.e., 
the time average of $h$ is zero, or  
$\langle h \rangle = 0$,
then
$$
\delta h(t) = h(t).
$$
Therefore, the $h(t)$ describe the instantaneous fluctuation $\delta h(t)$  of the HB population.  
%The behaviour of $r_{OO}(t)$  is depicted in Fig.\ref{fig:Ex30ps_hb}a. We find that in the equilibrium system, $r_{OO}(t)$ looks chaotic.
%=================

While the equilibrium average of the $\delta h(t)$ is zero, but we can obtain useful information by considering the equilibrium 
correlations between fluctuations at different times. The correlation between the $\delta h(t)$ and the $\delta h(0)$ can be written as 
$$
\langle \delta h(0) \delta h(t)\rangle = \langle h(0)h(t)\rangle-\langle h \rangle^2 = \langle h(0)h(t)\rangle,
$$
where the averaging $\langle\cdots\rangle$ is to be performed over the ensemble of initial conditions $(r^N, p^N)$.


In this paragraph, we will use three correlation functions to describe the HB dynamics of water/vapor interfaces of solutions,
the HB population correlation function \CHB, the survival probability \SHB and the reactive flux $k(t)$. \cite{Rapaport1983}

%\paragraph{Structure of HB Network}
%One of the importrant characteristics of HB network is the average number of H-bonds per molecule. 
%At room temperature and atmospheric pressure, this quantity is close to four or slightly higher. \cite{Malenkov2006} 
%
%Furthermore, one can consider a more detailed distribution function. A molecule can form multiple H-bonds with other molecules at the same time.
% Among these H-bonds, the molecule has $i$ times in the form of donors and $j$ times in the form of acceptors, 
%that is, the total number of H-bonds formed by the molecule at a certain time is $i+j$.
%Regarding the H-bonding of pure bulk water, people have obtained rich results with this analysis method and MD simulations. \cite{Malenkov1990,Malenkov2006}

\FloatBarrier
\paragraph{HB Population Auto-Correlation Function}
We use the correlation function \CHB to describe the structural relaxation of H-bonds: 
\begin{eqnarray}
C_{\text{HB}}(t)=\langle h(0)h(t) \rangle/\langle h\rangle
\label{eq:C_HB}.
\end{eqnarray}
With the aid of the ergodic principle, the ensemble average $\langle \cdots\rangle$ is implemented by time average.
The $\langle h\rangle$ is the probability that a pair of randomly chosen water molecules in the system is
H-bonded in a certain form at any time $t$. 
%Here is an explanation of the specific meaning of the word "form". 
Each water molecule has two H atoms and one O atom. Therefore, when a pair of water molecules $a$ and $b$ are bonded by a H-bond, 
the oxygen atom in each water molecule can act as both a donor and an acceptor. When water molecules $a$ and $b$ are used as donors, 
any one of its two H atoms can participate in the formation of H-bonds. Therefore, a pair of water molecules can form 4 different forms of H-bonds. 
In other words, if the role of H atoms between the pair of water molecules changes, but they still form H-bonds, 
we think that an old H-bond is broken and a new H-bond is formed.
 As examples, the dynamics of the interoxygen distance $r_{\text{OO}}(t)$, 
the cosine of H$-$O$\cdots$O angle cos$\phi(t)$  
and the $h(t)$ for a HB in a water cluster is displayed in Fig.\thinspace\ref{fig:Ex30ps_hb}, respectively.
%-------------------
\begin{figure}[hbtp]
\centering
\includegraphics [width=0.42\textwidth] {./diagrams/Ex30ps_hb}
\setlength{\abovecaptionskip}{0pt}
\caption{\label{fig:Ex30ps_hb}Dynamics of $r_{\text{OO}}(t)$ (top), cos$\phi(t)$ (middle), 
  and $h(t)$ (bottom) for a HB in a water cluster. The dashed lines show the interoxygen distance 
  boundary $r^{\text{c}}_{\text{OO}}$=3.5 \AA (top) and criterion of cosine of H$-$O$\cdots$O angle cos$\phi^{\text{c}}$ 
  with $\phi^{\text{c}}$=30$^{\circ}$, respectively.}
\end{figure} 

In a large system that consist of many water molecules, the probability that a specific pair of water molecules are H-bonded is extremely small. 
Therefore, the \CHB also relaxes to zero, when $t$ is large enough. 
The \CHB measures correlation in $h(t)$ independent of any possible bond breaking events. 
This function is similar to one of the intermittent HB correlation functions, introduced by Rapaport,\cite{Rapaport1983}
and can be studied by a continuous function, probability densities.
From the \CHB, the HB relaxation time can also be computed by
\begin{eqnarray}
  \tau_{\text{R}} &=& \frac{\int t C_{\text{HB}}(t)\text{d}t}{\int C_{\text{HB}}(t)\text{d}t}.
\label{eq:tau_relaxation}
\end{eqnarray}
The \CHB for the DFTMD simulated bulk water is shown in Fig.\thinspace\ref{fig:128w_c_itp_bk_ns40}.
We can obtain the relaxation time from Eq.\thinspace\ref{eq:tau_relaxation}: $\tau_R = 14.01$ ps for ADH definition of H-bonds, 
and $\tau_R = 14.16$ ps for AHD definition where $r^{\text{c}}_{\text{OO}}=3.5$ \AA and $\theta^{\text{c}}=120^{\circ}$.
%/home/gang/Data/bulk_pure/__bulk_pure/__hbacf/128w_c_bk_ns40.eps
\begin{figure}[hbtp]
\centering
\includegraphics [width=0.36\textwidth] {./diagrams/128w_c_bk_ns40}
\setlength{\abovecaptionskip}{0pt}
\caption{\label{fig:128w_c_itp_bk_ns40}Time dependence of \CHB ($c(t)$ for short)for the DFTMD simulated bulk water at 300 K with density $\rho =1.00$ g/cm$^3$.} 
%The length of the trajectory is 35 ps of physical time. Ref:\cite{Khaliullin2013}}
\end{figure} 
%
Because the thermal motion can cause distortions of H-bonds from the perfectly tetrahedral configuration,
water molecules show a librational motion on a time scale of $\sim$ 0.1 ps superimposed to rotational and diffusional motions ($> 1$ ps), 
which causes a time variation of interaction parameters.
A new HB population $h^{(d)}(t)$ was also defined to obviate the distortion of real HB dynamics
due to the above geometric definition. \cite{Sciortino1989,AC00}
The $h^{(d)}(t)$ is 1 when the interoxygen distance of a particular tagged pair of water molecules is less than $r^{\text{c}}_{\text{OO}}=3.5$ \AA at time $t$ and 0 otherwise. 
The difference between the operators $h^{(d)}(t)$ and $h(t)$ is that those molecular pairs that meet the condition of $h^{(d)}(t)=1$ may not meet the condition of $h(t)=1$.
That is, the H-bonds between the tagged molecular pairs that satisfy the condition $h^{(d)}(t)=1$ may have been broken, but they may more easily form H-bonds again.
The function 
\begin{eqnarray}
  C^{(d)}_{\text{HB}}(t)=\langle h(0)h^{(d)}(t) \rangle/\langle h\rangle
\label{eq:C_HB_d}
\end{eqnarray}
is the probability that the specific two water molecules are located in reformable region ($r_{\text{OO}} < r^{\text{c}}_{\text{OO}}$) at time $t$,
if they were H-bonded at time zero. 
The correlation function 
%
\begin{eqnarray}
n(t)=\langle h(0)[1-h(t)]h^{(d)}(t) \rangle/\langle h\rangle 
\label{eq:n_HB}
\end{eqnarray}
represents the probability at time $t$ 
that a tagged pair of initially H-bonded water molecules are unbonded but remain separated by less than $r_{\text{OO}}^{\text{c}}$.
In the above formula, $1-h(t)$ describes the breaking of a HB at time $t$ after its formation at time $t=0$.
%===============================
\FloatBarrier
\paragraph{Survival Probability}
%\paragraph{Probability of the first HB Breaking}
%===============================
Another scheme to describe HB dynamics is the survival probability \cite{AC00} for a newly generated HB.
%The probability densities
It is defined as
\begin{eqnarray}
S_{\text{HB}}(t)=\langle h(0)H(t) \rangle/\langle h\rangle 
\label{eq:S_HB},
\end{eqnarray}
where $H(t)=1$ if the tagged pair of molecules, remains \emph{continuously} H-bonded till time $t$ 
and 0 otherwise.  It describes the probability that an initially H-bonded molecular pair 
remains bonded at all times up to $t$. \cite{Chowdhuri2006}
The \SHB for the DFTMD simulated bulk water is shown in Fig.\thinspace\ref{fig:128w_s_itp_bk_ns40}.
%-------------------
\begin{figure}[hbtp]
\centering
\includegraphics [width=0.36\textwidth] {./diagrams/128w_s_bk_ns40}
\setlength{\abovecaptionskip}{0pt}
\caption{\label{fig:128w_s_itp_bk_ns40}Time dependence of \SHB ($s(t)$ for short) for the DFTMD simulated bulk water at 300 K with density $\rho =1.00$ g/cm$^3$.} 
%The length of the trajectory is 35 ps of physical time.
\end{figure} 

The average continuum HB lifetime $\langle \tau_{\mathrm{a}} \rangle$ is calculated by the integration of \SHB over $t$ (For detailed derivation, see Appendix \ref{diff_distr}.) :  
\begin{eqnarray}
  \langle\tau_{\mathrm{a}}\rangle = \int_0^\infty dt S_{\text{HB}}(t).
\label{eq:calculate_hb_lifetime_from_s}
\end{eqnarray}
%
The time derivative of \SHB
\begin{eqnarray}
P_a(t) = -\text{d}S_{\text{HB}}/\text{d}t
\label{eq:P_1}
\end{eqnarray}
represents the first passage time probability density of H bonds. $P_a(t)$ is also called probability distribution of HB lifetimes, \cite{Sciortino1990prl,Krausche1992,FWS99,Voloshin2009} or histogram of HB lifetimes.\cite{Geiger1984,Stanley2000}
It denotes the probability of the first HB breaking in time $t$ after it has been detected at $t=0$, i.e.,
\begin{eqnarray}
S_{\text{HB}}(t)= \int_t^\infty P_a(t')dt'.
\label{eq:P_2}
\end{eqnarray}
%
%In terms of $h$, the probability distribution can be expressed as
%\begin{eqnarray}
%P_a(t) = \frac{\langle [1-h(0)]\delta [t-\int_0^t h(t')dt'][1-h(t)]\rangle}{\langle 1-h(0)\rangle},
%\label{eq:P_3}
%\end{eqnarray}
%where $\langle \rangle$ denotes the average over all molecular pairs which are starting to H-bonded at time $t$.

%\paragraph{Average HB Lifetime $\tau_{\text{HB}}$} %\cite{HAK08}
%Like in water, librational motions of water molecules cause an rupturing and reforming of a H bond on a time scale of 60 fs.\cite{SHC86}
\FloatBarrier
\paragraph{Reactive Flux $k(t)$} 
Calculating the reactive flux HB correlation functions and determine the rate constant ($1/\tau_{\text{HB}}$),
is a more rigorous way to obtain the nature of H-bonds at water/vapor interfaces. \cite{AL00}
The rate of relaxation to equilibrium is characterized by the reactive flux correlation function, 
\begin{eqnarray}
k(t) = -\frac{\text{d}C_{\text{HB}}}{\text{d}t},
\label{eq:k}
\end{eqnarray}
i.e., $\langle j(0)[1-h(t)]\rangle/\langle h\rangle$,
where 
$j(0)=-\text{d}h/\text{d}t|_{t=0}$ 
is the integrated flux departing the HB configuration space at time $t=0$ (For detailed derivation, see Appendix \ref{calc_rf}.).
The reactive flux $k(t)$ quantifies the rate that an initially present HB breaks at time $t$, 
independent of possible breaking and reforming events in the interval from 0 to $t$.
Therefore, the $k(t)$ measures the effective decay rate of an 
initial set of H-bonds. \cite{DC87,FWS00}

For bulk neat water, there exists a $\sim 0.2$-ps transient period,
during which the $k(t)$ quickly changes from its initial value. \cite{FWS00}
However, at longer times, the $k(t)$ is independent of the HB definitions.
In order to verify this point of view and also to verify the reliability of our simulation method, 
we performed a DFTMD simulation of the bulk water system with a total time of 60 ps, 
and used the two different HB definitions --- ADH definition and AHD definition to calculate the $k(t)$. 
The calculation results in Fig.\thinspace\ref{fig:pure_bk_c_n_k}b show that when $t$ is large enough, 
the difference in $k(t)$ caused by different HB definitions is relatively small.
Therefore, the long time decay of the $k(t)$ reflects the general properties of H-bonds.

We assume that each HB acts independently of other H-bonds, \cite{AL96,AL00} 
and due to detailed balance condition, we can obtain 
\begin{eqnarray}
  \tau_{\text{HB}} &=& \frac{1- \langle h\rangle}{k},
\label{eq:rate}
\end{eqnarray}
where $k$ is the rate constant of breaking a HB (forward rate constant). \cite{Chandler1986,Chandler1978} 
For an aqueous interface, the probability of exactly a tagged molecule pair forming a HB is very low, that is, $\langle h\rangle \ll 1$. Therefore,
the $k$ is related to the average HB lifetime by $\tau_{\text{HB}}=1/k$.
We use $k'$ to represent the backward rate constant, that is, the rate constant from the HB \emph{on} state to the HB \emph{off} state for a tagged pair of molecules.
Therefore, the reaction time constant $\tau_\text{re}$ is 
\begin{eqnarray}
  \tau_\text{re} &=& \frac{1}{k+k'}.
\label{eq:reaction_rate_tau}
\end{eqnarray}
%
\FloatBarrier
\section{Dynamical Properties of H-Bonds in Water-Vapor Interfaces}
The pure water system and the interface between pure water and vacuum, i.e., the water/vapor interface, 
are ideal model systems for testing our algorithms.
For pure water systems, especially for bulk water, we can compare the results of the current method with the results of 
previous researchers.\cite{AL96,Kessler2015} On this basis, we will show in the next paragraphs about the pure water interface, 
the aqueous solution containing ions, and the HB dynamics in the interface corresponding to the aqueous solution.

\paragraph{Correlation Functions $c(t)$,$n(t)$ and $k(t)$}
The oxygen-oxygen radial distribution functions $g_\text{OO}(r)$ and $g_\text{OH}(r)$ for the bulk water system are 
shown in Fig.\thinspace\ref{fig:rdf_bk_pure_pbc}.
\begin{figure}[htb]
\centering                                          
%\includegraphics [width=0.6 \textwidth] {./diagrams/rdf_bk_pure_and_interf_pure_normed} 
\includegraphics [width=0.4 \textwidth] {./diagrams/rdf_bk_pure_pbc} 
\setlength{\abovecaptionskip}{0pt}
  \caption{\label{fig:rdf_bk_pure_pbc}Partial RDFs of liquid bulk water at ambient conditions.
The box size: 15.64 $\times$ 15.64 $\times$ 15.64 \A$^3$; $T = 300$ K.}
%(b) water/vapor interface (box size: 15.64 $\times$ 15.64 $\times$ 31.28 \A$^3$; $T = 300$ K).}
\end{figure}
\begin{figure}[htb]
\centering
\includegraphics [width=0.6 \textwidth] {./diagrams/pure_bk_c_n_k} 
\setlength{\abovecaptionskip}{0pt}
  \caption{\label{fig:pure_bk_c_n_k}Time dependence of the correlation functions (a) $n(t)$, $c(t)$ and (b) the rate function $k(t)$ 
of water--water H-bonds for \emph{bulk} water, calculated from the trajectory of a DFTMD simulation.
 The definition of $h(t)$ is based on specific H--O bond, instead of water-water pairs.
The simulation was for bulk water at $T=300$ K, and with a density of 1.00 g cm$^{-3}$. The length of the trajectory is 50 ps of physical time.}
\end{figure}
The correlation functions \CHB ($c(t)$ for short) from the trajectory of a DFTMD simulation with ADH (solid line) and AHD (dashed line) definition of H-bonds are 
shown in Fig.\thinspace\ref{fig:pure_bk_c_n_k}a. 
The length of the trajectory coincided with 60 ps of physical time. The simulation is for bulk water at the temperature $300$ K and with a density $1.00$ g/cm$^3$.
The reactive flux $k(t)$ (see Fig.\thinspace\ref{fig:pure_bk_c_n_k}b) we calculated here is very consistent with the result in \cite{AL96b}.

\begin{figure}[H] %htb
\centering
\includegraphics [width=0.6 \textwidth] {./diagrams/128w_itp_c_n_k} 
\setlength{\abovecaptionskip}{0pt}
  \caption{\label{fig:128w_itp_c_n_k}Time dependence of the correlation functions (a) $n(t)$, $c(t)$ and (b) the $k(t)$ 
of water--water H-bonds for water/vapor \emph{interface} at 300 K, calculated from the trajectory of a DFTMD simulation.
 The definition of $h(t)$ is based on specific H--O bond, instead of water-water pairs.
The length of the trajectory is 50 ps of physical time.}
\end{figure}
For the water/vapor interface of neat water, we reported the result of the correlation fucntion  $c(t)$, $n(t)$
in Fig.\thinspace\ref{fig:128w_itp_c_n_k}a and the reactive flux $k(t)$ in Fig.\thinspace\ref{fig:128w_itp_c_n_k}b.
 
%which had been used in the study of HB dynamics of liquid water. \cite{AL96,Khaliullin2013}
%The $k(t)$ calculated from the positional trajectory of water molecules in DFTMD simulations, is reported in Fig.\thinspace\ref{fig:128w_bk_2delta_t_60ps_k_log}. 
%
In both cases, the $k(t)$ quickly changes from its initial value on a time scale of less than 0.2 ps. 
This value can be roughly seen from Fig.\thinspace\ref{fig:pure_bk_and_itp_k}, which redraws the $k(t)$ in Fig.\thinspace\ref{fig:pure_bk_c_n_k} and 
Fig.\thinspace\ref{fig:128w_itp_c_n_k} in double logarithmic coordinates and compares them.
%
\begin{figure}[H]
\centering
\includegraphics [width=0.6 \textwidth] {./diagrams/pure_bk_and_itp_k} 
\setlength{\abovecaptionskip}{0pt}
  \caption{\label{fig:pure_bk_and_itp_k}Time dependence of the correlation functions $k(t)$ 
of water--water H-bonds for (a) bulk water and (b) water/vapor interface at 300 K.}
\end{figure}
%(see the inset of Fig.\space\ref{fig:121}). 
%For the times beyond the transient period, the $k(t)$ decays to zero monotonically, and the slop of the $\ln{k(t)}$ increases monotonically with $t$ (see Fig.\space\ref{fig:121_log_rf}). 
%These two properties were also found for bulk water using the SPC water model by Luzar and Chandler. \cite{AL96} 
%This log-log plot of the $k(t)$ shows that, as in the case of liquid water, this decay behaviour does not coincide with a power-law decay for water/vapor interface of neat water.
%This result is also the same as that of the classical molecular simulation of pure water. \cite{AL96b,Luzar1996}
%%
%\begin{figure}[htpb]
%\centering
%\includegraphics [width=0.5\textwidth] {./diagrams/121}
%\setlength{\abovecaptionskip}{0pt}
%  \caption{\label{fig:121}The time dependence of the $k(t)$ for the water/vapor interface of neat water, calculated by DFTMD simulations.
%  The inset shows the log-log plot of the $k(t)$.}
%\end{figure}
%
%\paragraph{Relation Between HB Dynamics and Anisotropy Decay}
%It is interesting to relate the HB kinetics with rotational dynamics (anisotropy decay) of single water molecules.\cite{HX01}

For the water/vapor interface of neat water, we focus on the reactive flux $k(t)$, 
which had been used in the study of HB dynamics of liquid water. \cite{AL96,Khaliullin2013}
The $k(t)$ calculated from the positional trajectory of water molecules in DFTMD simulations, is reported in Fig.\thinspace\ref{fig:121}. 
In the case of water/vapor interface, the $k(t)$ quickly changes from its initial value on a time scale of less than 0.2 ps 
(see the inset of Fig.\thinspace\ref{fig:121}). 
Beyond this transient period, the $k(t)$ decays to zero monotonically, and the slop of the $\ln{k(t)}$ increases monotonically with $t$ (see Fig.\thinspace\ref{fig:121}). 
These two properties have been found for bulk water using the SPC water model by Luzar and Chandler. \cite{AL96} 
This log-log plot of the $k(t)$ shows that, as in the case of liquid water, this decay behaviour does not coincide with a power-law decay for water/vapor interface of neat water.
This result is also the same as that of the classical molecular simulation of pure water. \cite{AL96b,Luzar1996}
%

It can be seen from Fig.\thinspace\ref{fig:128w_bk_itp_50ps_n_from_k_in_with_2_hb_def_type2} that the $n(t)$ of the water/vapor interface of neat water
is always greater than the value of $n(t)$ in the bulk water. This means that the HB between a pair of water molecules at time $t$ is broken
and the distance between them is less than $r^c_\text{OO}=3.5$ \AA in the water/vapor interface is more likely to occur than in bulk water. 
We interpret this result as the fact that at time $t$, there is a greater probability that the H-bonds on the interface are broken 
compared to the H-bonds in the bulk water.
\begin{figure}[htpb]
\centering
\includegraphics [width=0.42\textwidth] {./diagrams/121}
\setlength{\abovecaptionskip}{0pt}
  \caption{\label{fig:121}Time dependence of the $k(t)$ for the water/vapor interface of neat water, calculated by DFTMD simulations.
  The inset shows the log-log plot of the $k(t)$.}
\end{figure}
\begin{figure}[H]
\centering
\includegraphics [width=0.6\textwidth] {./diagrams/128w_bk_itp_50ps_n_from_k_in_with_2_hb_def_type2}
\setlength{\abovecaptionskip}{0pt}
\caption{\label{fig:128w_bk_itp_50ps_n_from_k_in_with_2_hb_def_type2} 
Time dependence of the population functions $n(t)$ for bulk water and the water/vapor interface at $T=300$ K from (a) ADH (b) AHD criteria.} 
\end{figure}

\paragraph{Reaction Constants $k$ and $k'$}
%\paragraph{$k$ and $k'$: Least Squares Fit}
%In order to show the effect of water molecule diffusion on the HB dynamics, we can calculate the sum of the functions $c(t)$ and $n(t)$, i.e., $c(t)+n(t)$.
\begin{figure}[H]
%Location: /home/gang/Github/hbacf/__hbacf_continuous/figures/plot_n_from_2_hb_def_bk_and_itp/
\centering
\includegraphics [width=0.6\textwidth] {./diagrams/128w_bk_itp_50ps_n_from_k_in_with_2_hb_def}
\setlength{\abovecaptionskip}{0pt}
\caption{\label{fig:128w_bk_itp_50ps_n_from_k_in_with_2_hb_def} 
Time dependence of the population functions $n(t)$ for (a) bulk water and (b) water/vapor interface, as computed from the ADH (solid line) and AHD (dashed line) 
criterion of H-bonds with the expression $n(t) = \int_0^t dt'k_{in}(t')$.} 
\end{figure}
The probability at time $t$ that a pair of water molecules bonded by H-bonds at the initial moment does not be bonded 
but the distance between their oxygen atoms is still less than $R_\text{OO}^c$ is 
\begin{eqnarray}
n(t) = \int_0^t dt'k_{in}(t'),
\label{eq:n_from_k_in}
\end{eqnarray}
where $k_{in}(t) = -\langle \dot h(0)[1-h(t)]h^d(t) \rangle/\langle h\rangle$ is the restricted rate function. 
Fig.\thinspace\ref{fig:128w_bk_itp_50ps_n_from_k_in_with_2_hb_def}a and Fig.\thinspace\ref{fig:128w_bk_itp_50ps_n_from_k_in_with_2_hb_def}b
show the function $n(t)$ in Eq. \ref{eq:n_from_k_in} for bulk water and water/vapor interface, respectively. 
In each figure, we have drawn the $n(t)$ function corresponding to the two different HB definitions. 
It can be found that the overall trend of n(t) does not depend on the choice of HB definition.
i.e., as $t$ increases, $n(t)$ increases rapidly from 0, and it reaches a maximum value at $t \approx 10$ ps, and then gradually decreases. %[EXPLAIN THE RESULTs]
Comparing the two figures \ref{fig:128w_bk_itp_50ps_n_from_k_in_with_2_hb_def}a and \ref{fig:128w_bk_itp_50ps_n_from_k_in_with_2_hb_def}b,
we find that the maximum value of $n(t)$ in the water/vapor interface is slightly higher than that in bulk water. 
This feature does not depend on the definitions of the HB we choose.

Khaliullin and K\"uhne have studied the H-bonding kinetics of pure water using AIMD simulations. \cite{Khaliullin2013}
Based on the concepts of $h(t)$, $h^{(d)}(t)$, $n(t)$ and $k(t)$, they have used the simulation data 
obtained by the AIMD simulation method to obtain the ratio $k/k'$ in the bulk water, and then the lifetime and relaxation time 
of the HB.  Here, we also use the AIMD simulation method to study the HB dynamics at the interface of aqueous 
solutions. We can obtain the optimal solution range of $k$ and $k'$ from the relationship 
between the reactive flux and the HB population correlation function $c(t)$ and $n(t)$, and the two rate constants $k$ and $k'$, i.e.,
\begin{eqnarray}
  k(t) = kc(t)-k'n(t).
\label{eq:fitting_k_rates}
\end{eqnarray}
%
%[Answer Q3]
We can find the optimal value of the rate constants, $k$ and $k'$, 
by a least squares fit of the calculated data $k(t)$, $c(t)$ and $n(t)$ beyond the transition phase.  
The functions $c(t)$ can be regarded as a $P$-dimensional column vector composed by $(c(1),c(2),\cdots,c(P))^T$, and denoted as ${\bf c}$,
with $c(i)$ representing the value of the correlation $c(t)$ at $t=i$.
Similarly, the functions $n(t)$ and $k(t)$ can also be viewed as $P$-dimensional column vectors and can be denotd as ${\bf n}$ and ${\bf k}$, respectively.
Therefore, the $k$ and $k'$ can be determined from the matrix ${\bf A} = \begin{bmatrix} {\bf c} & {\bf n} \end{bmatrix}$, i.e., 
\begin{equation}
\begin{bmatrix} k\\ -k' \end{bmatrix} = ({\bf A}^T {\bf A})^{-1} {\bf A}^T {\bf k}. 
\end{equation}
For bulk water and the water/vapor interface, the optimal $k$ and $k'$ are reported in Table 
\ref{tab:k_k_prime_128w_pure_1} and \ref{tab:k_k_prime_128w_pure_2}. 
% 
\begin{table}[htb]
\centering
\caption{\label{tab:k_k_prime_128w_pure_1} 
    The $k$ and $k'$ for the bulk water and the water/vapor interface. We carried on the short time region 0.2 ps $< t <$ 2 ps. 
    The unit for $k$ ($k'$) is ps$^{-1}$, and that for $\tau_{\text{HB}}$ ($=1/k$) is ps. (Same below.)
} 
\begin{tabular}{ccccccc}
 Criterion & $k$  (bulk) & $k'$ (bulk) & $\tau_{\text{HB}}$ (bulk) & $k$  (interf.) & $k'$ (interf.) & $\tau_{\text{HB}}$ (interf.)\\
\hline
  % With 4 digital!(Keep it)
  %ADH & 0.3345 & 0.8591 & 2.9895 & 0.3587 & 0.6730 & 2.7881  \\
  %ADH(from $k_{in}$) & 0.2959  & 0.9883 & 3.3795  & 0.3225 & 0.7652 & 3.1012 \\
  %AHD & 0.3334 & 1.0414 & 2.9991 & 0.3520  & 0.7847  &  2.8405\\
  %AHD(from $k_{in}$) & 0.2882 & 1.1490 & 3.4699 & 0.3140 & 0.8867 & 3.1836 \\
  ADH & 0.335 & 0.859 & 2.990 & 0.359 & 0.673 & 2.788  \\
  ADH(from $k_{in}$) & 0.296  & 0.988 & 3.380  & 0.323 & 0.765 & 3.101 \\
  AHD & 0.333 & 1.041 & 2.999 & 0.352  & 0.785  &  2.841\\
  AHD(from $k_{in}$) & 0.288 & 1.149 & 3.470 & 0.314 & 0.887 & 3.184 \\
\end{tabular}
\end{table}
%
\begin{table}[htb]
\centering
\caption{\label{tab:k_k_prime_128w_pure_2} 
    The $k$ and $k'$ for the bulk water and the water/vapor interface. We carried on the long time region 2 ps $< t <$ 12 ps.
} 
\begin{tabular}{ccccccc}
 Criterion & $k$  (bulk) & $k'$ (bulk) & $\tau_{\text{HB}}$ (bulk) & $k$  (interf.) & $k'$ (interf.) & $\tau_{\text{HB}}$ (interf.)\\
\hline
  %ADH & 0.1151 & 0.0311 & 8.6872 & 0.1593 & 0.0580 & 6.2786 \\
  %ADH(from $k_{in}$) & 0.1147  & 0.0391 & 8.7184 & 0.1569  & 0.0678 & 6.3723\\
  %AHD & 0.1071 & 0.0424 & 9.3450  & 0.1572 & 0.0763 & 6.3626 \\
  %AHD(from $k_{in}$) & 0.1053  & 0.0472 & 9.4963 & 0.1545  & 0.0884 & 6.4715 \\
  ADH & 0.115 & 0.031 & 8.687 & 0.159 & 0.058 & 6.279 \\
  ADH(from $k_{in}$) & 0.115  & 0.039 & 8.718 & 0.157  & 0.068 & 6.372\\
  AHD & 0.107 & 0.042 & 9.345  & 0.157 & 0.076 & 6.363 \\
  AHD(from $k_{in}$) & 0.105  & 0.047 & 9.496 & 0.155  & 0.088 & 6.472 \\
\end{tabular}
\end{table}
% 

To obtain the forward and backward rate constants ($k$ and $k'$),
here we performed the fitting in different time region $0.2 < t < 2$ ps and $2 < t < 12$ ps, respectively.
We note that in the larger time region, i.e., $2 < t < 12$ ps, the value of HB lifetime $\tau_\text{HB}$ is larger than that in shorter time region, $0.2 < t < 2$ ps,
no matter for the bulk water or for the interface. A larger $\tau_\text{re}$ value means that the distance between a water molecule and another water molecule 
stays within $r_\text{OO}^c= 3.5$ \AA for a longer time. 
For the long time region, these values of the $k$ are comparable in magnitude to that obtained by Ref.\thinspace{\cite{Khaliullin2013}} 

%For AHD definition:The bulk water: 14.1572 ps; the water/vapor iterface: 12.7806 ps.
%For the water/vapor interface of pure water, we also calculated the constants $k$ and $k'$ by least square fit. 

\section{Instantaneous Interfacial HB Dynamics}
The method of water-water pair based HB dynamics used in this section has been frequently used in previous literature.\cite{Luzar1994,AL96,AC00} 
The basis is the population operator $h(t)$ of the HB formed between two water molecules. 
We use the correlation function \CHB to describe the relaxation of H-bonds between two water molecules: 
\begin{eqnarray}
C_{\text{HB}}(t)=\langle h(0)h(t) \rangle/\langle h\rangle
\label{eq:C_HB}.
\end{eqnarray}
Similarly, with the aid of the ergodic principle, the ensemble average $\langle \cdots\rangle$ is implemented by time average.
The $\langle h\rangle$ is the probability that a pair of randomly chosen water molecules in the system is
H-bonded at any time $t$. 

The function \CHB is interpreted as the probability that the HB between a certain pair of water molecules is intact at time  $t$, 
if the pair of water molecules are H-bonded at time zero. 
In a large system that consist of many water molecules, the probability that a specific pair of water molecules are H-bonded is extremely small. 
Therefore, the \CHB relaxes to zero, when $t$ is large enough. 
The \CHB measures correlation in $h(t)$ independent of any possible bond breaking events. 
It is one of the intermittent HB correlation functions, introduced by Rapaport, \cite{Rapaport1983} 
and can be studied by a continuous function, probability densities.

Fig.\thinspace\ref{fig:128w_bk_2delta_t_60ps_water_pair_c_ns40} shows the correlation function $C_\text{HB}(t)$ 
for bulk water over time. %The result is calculated by DFTMD simulation, and the temperature is 300 K.
Comparing Fig.\thinspace\ref{fig:128w_c_itp_bk_ns40} and Fig.\thinspace\ref{fig:128w_bk_2delta_t_60ps_water_pair_c_ns40}, 
we find that although the trend of change of $C_{HB}$ is very similar: as time increases, it gradually decays from 1; 
but from a quantitative point of view, the latter decays more slowly. This difference comes from our definition 
of the HB population operator $h$, and does not depend on the HB criterion. For example, from Fig.\thinspace\ref{fig:128w_bk_2delta_t_60ps_water_pair_c_ns40}, 
we can see that the above conclusions are correct regardless of ADH or AHD criteria. It can be seen that our different definitions of the HB population operator 
will lead to different correlation functions related to it. To see the direct comparison of the $C_\text{HB}(t)$ in the two cases, 
refer to the Appendix \ref{DEF_POPULATION_OPERATOR}. 
In the following, we will use this HB population operator based on molecular pairs, which is based on water-water molecule pairs, 
or ion-water molecule pairs, unless otherwise specified.
\begin{figure}[H]
%Location: /home/gang/Github/water_pair_HB_dynamics/plot/plot_c/
\centering
\includegraphics [width=0.360\textwidth] {./diagrams/128w_bk_2delta_t_60ps_water_pair_c_ns40}
\setlength{\abovecaptionskip}{0pt}
\caption{\label{fig:128w_bk_2delta_t_60ps_water_pair_c_ns40} 
The $C_\text{HB}(t)$ for bulk water, based on water-water pair HB population operator $h(t)$, 
as computed from the ADH (solid line) and AHD (dashed line) criterion of H-bonds. Ref:\cite{Khaliullin2013}} %[DESCRIBE THE FIGURE][COMPARE THE RESULTs TO hbond-based CORRELATION FUNCTION]
\end{figure}

% One can uncomment if remove PERCENT
%====================================
%Based on the HB definition of water molecule pairs, we can also use least squares fitting to obtain the rate constant $k$, $k'$, 
%and the average lifetime $\tau_{HB}$ of the H-bonds. The results are shown in the Table \ref{tab:k_k_prime_128w_pure_2s}  to \ref{tab:k_k_prime_128w_pure_2u}.
%\begin{table}[htb]
%\centering
%\caption{\label{tab:k_k_prime_128w_pure_2s} 
%    The $k$ and $k'$ for the bulk water and the water/vapor interface. We carried on the long time region 0.2 ps $< t <$ 8 ps. 
%The unit for $k$ ($k'$) is ps$^{-1}$, and that for $\tau_{\text{HB}}$ ($=1/k$) is ps.} 
%\begin{tabular}{ccccccc}
% Criterion & $k$  (bulk) & $k'$ (bulk) & $\tau_{\text{HB}}$ (bulk) & $k$  (interf.) & $k'$ (interf.) & $\tau_{\text{HB}}$ (interf.)\\
%\hline
%  ADH & 0.14 & 0.28 & 7.16 & - & - & -  \\
%  AHD & 0.11 & 0.18 & 9.08 & - & -  &  -\\
%\end{tabular}
%\end{table}
%%
%\begin{table}[htb]
%\centering
%\caption{\label{tab:k_k_prime_128w_pure_2t} 
%    The $k$ and $k'$ for the bulk water and the water/vapor interface. We carried on the longer time region 0.2 ps $< t <$ 12 ps. 
%The unit for $k$ ($k'$) is ps$^{-1}$, and that for $\tau_{\text{HB}}$ ($=1/k$) is ps.} 
%\begin{tabular}{ccccccc}
% Criterion & $k$  (bulk) & $k'$ (bulk) & $\tau_{\text{HB}}$ (bulk) & $k$  (interf.) & $k'$ (interf.) & $\tau_{\text{HB}}$ (interf.)\\
%\hline
%  ADH & 0.10 & 0.17 & 9.59 & - & - & -  \\
%  AHD & 0.09 & 0.11 & 11.62 & - & -  &  -\\
%\end{tabular}
%\end{table}
%%
%\begin{table}[htb]
%\centering
%\caption{\label{tab:k_k_prime_128w_pure_2u} 
%    The $k$ and $k'$ for the bulk water and the water/vapor interface. We carried on the longer time region 1 ps $< t <$ 12 ps. 
%The unit for $k$ ($k'$) is ps$^{-1}$, and that for $\tau_{\text{HB}}$ ($=1/k$) is ps.} 
%\begin{tabular}{ccccccc}
% Criterion & $k$  (bulk) & $k'$ (bulk) & $\tau_{\text{HB}}$ (bulk) & $k$  (interf.) & $k'$ (interf.) & $\tau_{\text{HB}}$ (interf.)\\
%\hline
%  ADH & 0.06 & 0.06 & 17.96  & - & - & -  \\
%  AHD & 0.06 & 0.05 & 18.17 & - & -  & -\\
%\end{tabular}
%\end{table}

\FloatBarrier
\paragraph{Instantaneous Interfaces}
As Willard and Chandler mentioned, due to molecular motions, interfacial configurations
change with time, and the identity of molecules that lie at the interface also change with time, generally useful procedures for
identifying interfaces must accommodate these motions. \cite{Willard2010} 
To determine the instantaneous interface of the water/vapor system, we here adopted their proposed method based on spatial density.
The instantaneous coarse-grained density at space-time point $\mathbf{r},t$ can be expressed as polynmial
\begin{eqnarray}
\bar{\rho}(\mathbf{r}, t)=\sum_{i} \phi(|\mathbf{r}-\mathbf{r}_{i}(t)|; \xi) 
\end{eqnarray}
where ${\mathbf{r}}_i(t)$ is the position of the $i$th particle at time $t$ and the sum is over all such particles, and 
\begin{eqnarray}
\phi(\mathbf{r};\xi)=(2 \pi \xi^{2})^{-3/ 2} \exp (-r^{2} / 2 \xi^{2}) 
\label{eq:gaussian_coarse_graining}
\end{eqnarray} 
is a normalized Gaussian functions for a 3-dimensional system, where $r$ is the magnitude of ${\mathbf r}$, and $\xi$ is the coarse-graining length.
Equation \ref{eq:gaussian_coarse_graining} is introduced to improve the accurancy of the interface, such that we can extend the domain and make it a single unicom,
i.e., no cavity exists in the domain.
With the parameter $\xi$ set, the interfaces can be defined to be the 2-dimensional manifold ${\mathbf r} = {\mathbf s}$ such that
\begin{eqnarray}
\bar\rho(\mathbf{s};t)= \rho_c 
\label{eq:rho_c}
\end{eqnarray} 
where $\rho_c$ is a reference density. This instantaneous interface is a function of time as molecular configurations changes with time, that is 
${\mathbf s}(t) = {\mathbf s}(\{{\mathbf r}_i(t)\})$. 

%{Instantaneous Layering of the water/vapor interface} DELETED THE PARAGRAPH NAME
After the instantaneous surface is defined, we can define an interface layer for any non-uniform fluid system. 
Specifically, for the simulated water/vapor interface system in the cuboid simulation box, 
we can get another two-dimensional manifold ${\mathbf s}_0(t)$ by moving the instantaneous surface ${\mathbf s}(t)$ determined above 
along the system's normal coordinate to a certain distance $d$, which is another surface. We use these two surfaces 
as the two boundaries of the interface we will study. In other words, at any time point $t$, the volume between the two surfaces 
${\mathbf s}(t)$ and ${\mathbf s}_0(t)$ is defined as the instantaneous interface. 
Here, we use $d$ to denote the thickness of the instantaneous interface. As long as we change the value of $d$, we can get interfaces with different thicknesses. 
Different values of $d$ give us different layering strategies for the interface system. 
See Fig.\thinspace\ref{fig:128w_itp_add_z_d_trimed_with_inner_layers} as an example.
\begin{figure}
\centering
\includegraphics [width=0.32\textwidth] {./diagrams/128w_itp_add_z_d_trimed_with_inner_layers}
\setlength{\abovecaptionskip}{0pt}
\caption{\label{fig:128w_itp_add_z_d_trimed_with_inner_layers}
A slab of water (128 water molecules are included) with the instantaneous interface $\mathbf{s}$ represented as a blue mesh on the upper and lower phase boundary.
The normal is along the $z$-axis and the parameter $d$ is the thickness of the interfacial layer. 
The box dimensions are $15.64 \times 15.64 \times 31.28$ \AA, and the slab is periodically replicated in the $x$, $y$ and $z$ directions.} 
\end{figure}

Below we will combine the instantaneous interface and Luzar-Chandler's HB population operator \cite{AL96} to select the H-bonds 
in the interface. The dynamics of these H-bonds will vary with the thickness $d$ of the interface. By investigating the characteristics of HB dynamics
in these interfaces, we can obtain the dynamical characteristics of various solution interfaces. As we will see later, this method can be extended to HB dynamics 
in various environments, such as H-bonds around certain ions, in bulk water, etc., so that we can more easily select H-bonds in a certain environment. 
These different environments have a common feature: because the molecular configuration changes over time, the usual method first selects these molecules or molecular pairs, 
and then determines the H-bonds in this special environment based on a HB criterion, and finally calculate the HB lifetimes or autocorrelation functions of 
the HB population operators. %In the following, we call this method as Molecule Selection (MS). 
And here we are combining the general Luzar-Chandler HB population operator with the environment in which the HB is formed,
 that is, the space constraint satisfied by the configuration of the molecular pair. This combination can easily select those molecules and their H-bonds that meet arbitrary 
constraints when the molecular configuration changes.

\FloatBarrier
\paragraph{Interfacial Hydrogen Bond Population} \label{IHBP}
Once we have determined the instantaneous surface ${\mathbf s}(t)={\mathbf s}(\{{\mathbf r}_i(t)\})$, we can define interfacial H-bonds.
We use the parameter $d$ to denote the thickness of the instantaneous interface.
Now we define the interface HB population operator $h^{s}[{\mathbf r}(t)]$ as follows:
It has a value 1 when the particular tagged molecular pair are H-bonded \emph{and} both molecules are inside the instantaneous interface 
with a thickness $d$, and zero otherwise. 
The definition of  $h^{s}[{\mathbf r}(t)]$ is very critical to help us to efficiently obtain the dynamic characteristics of H-bonds in an
instantaneous interface system of any thickness. Note that the definition of HB here can be based on water molecule pairs or O-H pairs. 
In this paragraph, we only discuss H-bonds based on water molecule pairs. Starting from the H-bonds based on O-H pairs, the same analysis 
can also be done. 

Similar to the correlation function $C_\text{HB}(t)$ in Eq. \ref{eq:C_HB}, which describes the fluctuation of the general H-bonds,
we define the correlation function $C^s_\text{HB}(t)$ that describes the fluctuation of the interfacial H-bonds: 
\begin{eqnarray}
C^s_{\text{HB}}(t)=\langle h^s(0)h^s(t) \rangle/\langle h^s\rangle
\label{eq:C_s_HB}.
\end{eqnarray}
%
Similarly, we can define correlation functions 
\begin{eqnarray}
n^s(t)=\langle h^s(0)[1-h^s(t)]h^{(d),s} \rangle/\langle h^s\rangle
\label{eq:n_s_HB},
\end{eqnarray}
and 
\begin{eqnarray}
k^s(t)= -\frac{dC_{HB}^s}{dt}
\label{eq:k_s_HB}.
\end{eqnarray}
Therefore, using these new correlation functions, we can determine the reaction rate constant of breaking and reforming and the lifetimes of interfacial H-bonding.
%
\FloatBarrier
\paragraph{$d$-dependence of $C^s_\text{HB}(t)$}
\begin{figure}[htb]
\centering
\includegraphics [width=0.60\textwidth] {./diagrams/128w_itp_pure_water_pair_c_ihb}
\setlength{\abovecaptionskip}{0pt}
\caption{\label{fig:128w_itp_pure_water_pair_c_ihb} 
The $C^s_\text{HB}(t)$ for the instantaneous interfacial H-bonds with differnt thickness ($d$), based on water-water 
pair HB population operator $h^{s}(t)$, as computed from the (a) ADH and (b) AHD criteria of H-bonds.} 
\end{figure}
For the pure water interface, we used two geometric criteria of H-bonds to calculate the $h^s(t)$ and therefore correlation function $C^s_{HB}(t)$. 
The calculation results of the $C^s_{HB}(t)$ are shown in Fig.\thinspace\ref{fig:128w_itp_pure_water_pair_c_ihb}.
We find that the greater the thickness $d$ of the instantaneous interface is selected, 
the slower the relaxation of the interface H-bonds. When the thickness is greater than a certain thickness $d^c$ ( $\sim$ 3 \AA),
the relaxation of H-bonds at the interface hardly changes.
%

For comparison, we first calculate the HB dynamics of water molecules in the interface obtained by selecting molecules 
located in instantaneous interface. (See Appendix \ref{ihb_and_selection} for details.) 
In this algorithm, we first select the molecules in the interface at each moment and then make a statistical
average of the calculated correlation functions.
Specifically, to determine which water molecules are located in the instantaneous interface, we sample at regular intervals, and then calculate 
the correlation function $C_\text{HB}(t)$ for the water molecules located in the interface and their a statistical average.
As the thickness $d$ of the instantaneous interface changes, the $C_\text{HB}(t)$ in the interface will also change. 
Figure \ref{fig:128w_itp_pure_water_pair_c_ihb_scheme1} shows how the function $C_\text{HB}(t)$ changes with the thickness $d$.
The sub-figure (a) and (b) use HB definition criterion ADH, and AHD, respectively.
Comparing Fig.\thinspace\ref{fig:128w_itp_pure_water_pair_c_ihb} and Fig.\thinspace\ref{fig:128w_itp_pure_water_pair_c_ihb_scheme1}, we see that
when we use the method of selecting molecules in the interface, the dependence of the correlation function $C_\text{HB}(t)$  
on the interface thickness is very consistent with that of $C^s_{HB}(t)$. Moreover, regardless of the AHD definition 
or the ADH definition of the HB, this conclusion is basically valid. Beside the correlation functions $C_\text{HB}(t)$ 
or $C^s_\text{HB}(t)$ in the interface, we will further examine the correlation 
functions $C_\text{HB}(t)$, $n(t)$, $k(t)$ ($C^s_\text{HB}(t)$, $n^s(t)$, $k^s(t)$), and the rate constants $k$, $k'$ determined by them.
\begin{figure}[H]
\centering                                         
\includegraphics [width=0.6\textwidth] {./diagrams/128w_itp_pure_water_pair_c_ihb_scheme1}
\setlength{\abovecaptionskip}{0pt}
\caption{\label{fig:128w_itp_pure_water_pair_c_ihb_scheme1} 
The $C_\text{HB}(t)$ for the instantaneous interfacial H-bonds with different $d$, based on water-water pair HB population operator $h(t)$, 
as computed from the (a) ADH and (b) AHD criteria of H-bonds. These results are based on selecting the water molecules in the instantaneous interface and averaging 
the correlation functions of these water molecules. The sampling is performed every 4 ps. } 
\end{figure}

%[Plot the $k$ and $k'$ as functions of thickness $d$.]
\FloatBarrier
\paragraph{$d$-dependence of $k$ and $k'$} 
To find the reaction rate constants $k$ and $k'$, we can start from the correlation functions $C^s_\text{HB}(t)$, $n^s(t)$ and $k^s(t)$. 
We can also first select the water molecules in the instantaneous interface at each time point $t$, and start from the corresponding 
correlation functions $C_\text{HB}(t)$, $n(t)$ and $k(t)$ of the H-bonds of these selected water molecules.
Figure \ref{fig:128w_itp_pure_water_pair_k_k_prime_ihb_both_schemes} compares the rate constants ($k$ and $k'$) 
and the lifetime $\tau_\text{HB}$ obtained by the two different methods mentioned above, i.e., the Instantaneous Interfacial Hydrogen Bond (IHB) and Molecule Selection (MS) methods. 
We see that, whether it is $k$, $k'$ or $\tau_\text{HB}$, their changing \emph{trend} with the thickness $d$ of the 
instantaneous interface is only slightly affected by the calculation methods. 
To illustrate this point more clearly, we compare the $k$, $k'$ and $\tau_\text{HB}$ obtained under the two methods.
%
\begin{figure}[H]
\centering
\includegraphics [width=0.6\textwidth] {./diagrams/128w_itp_pure_water_pair_k_k_prime_ihb_both_schemes}
\setlength{\abovecaptionskip}{0pt}
\caption{\label{fig:128w_itp_pure_water_pair_k_k_prime_ihb_both_schemes} 
Dependence of (a) the reaction rate constants $k$ and $k'$ and (b) the HB lifetime $\tau_\text{HB}$ on the interface thickness,
obtained by the IHB and MS methods, respectively.
The corresponding $k$, $k'$ and $\tau_\text{HB}$ in the bulk water are also drawn with dashed lines as a reference.
In sub-figure a, the $k$ of bulk water is represented by a \emph{black dashed} line, and the $k'$ is represented by a \emph{blue dashed} line;
in sub-figure b, the $\tau_\text{HB}$ of bulk water is represented by a \emph{black dashed} line.
The ADH criterion of H-bonds is used and the least square fits are carried on the time 
region 0.2 ps $< t <$ 12 ps.}
\end{figure}
As we can see from Fig.\thinspace\ref{fig:128w_itp_pure_water_pair_k_k_prime_ihb_both_schemes}, 
when the thickness is large enough ($d_0 \sim 3$ \AA), these two constants agree well quantitatively. 
This result shows that the two extreme statistical methods (see Appendix \ref{ihb_and_selection}) 
for the HB dynamics of the interface did not produce much difference for the time scale (10$^2$ ps) 
and the scale ( 10$^2$ \AA ) of the simulation box we currently use.

We also found that when we focus on the molecules in the interface whose thickness is less than $d_0$, 
the values of the reaction rate constants depend on the method we use. 
That is, the $k$ obtained by the IHB method is relatively larger than by the MS method, and $k'$ is relatively smaller. 
Since $\tau_\text{HB} = 1/k$, this directly leads to a relatively shorter HB lifetime using the IHB method. 
This result is related to our definition of IHB, and it is the same as our expectations: The definition of interfacial H-bonds (or $h^s(t)$) makes the HB break rate 
on the interface artificially increased. At the same time, we know that the molecule selection method retains the original rate constant of H-bonds, 
but it may include the contribution of bulk water molecules to the rate constant. That is why the molecule selection method underestimate the $k$. 

In Fig.\thinspace\ref{fig:128w_itp_pure_water_pair_k_k_prime_ihb_both_schemes}, the $k$, $k'$ and $\tau_\text{HB}$ for the \emph{bulk} water 
are also drawn with dashed lines as a reference.
Comparing the above-mentioned physical quantities in the pure water interface and bulk water, we found that when the interface thickness $d>d_0$, 
no matter which statistical method is used, the value of the reaction rate constants of the interface water we get is \emph{greater} than that in the bulk water. 
Therefore, since the HB lifetime can be calculated by $\tau_\text{HB} = 1/k$, the value of $\tau_\text{HB}$ in interface water is smaller than that in bulk water.

Furthermore, we find from Fig.\thinspace\ref{fig:128w_itp_pure_water_pair_k_k_prime_ihb_both_schemes} that as the interface thickness $d$ increases, 
the values of $k$ and $k'$ also tend to the values of rates in the bulk water at the same condition.
These results are obtained by the least squares method in the same interval (0.2--12 ps). This verifies that the IHB method 
can get as good results as the method of molecule selection  when $d>d_0$. 
Because the IHB is easier to operate, this method can calculate the HB dynamics and thus HB lifetime on the interface 
when the $d>d_0$ (in this case, $d_0 \sim 3$ \A \ or the size of 2--3 water molecules).
We also noticed that the selection of water molecules and the statistical averaging depend on our sampling density on the trajectory of the simulated system, 
and the IHB method does not require such sampling. Therefore, the IHB method is more efficient method to determine the HBD of instantaneous interface.

The more realistic HB dynamical properties of interface molecules are between the results of the above two methods. 
Therefore, it is possible to approximate the true appearance of the HB dynamics of the interface molecules, 
by both the IHB method and the molecule selection method if the thickness of the interface is selected large enough. 

%In summary, if we study the dynamics of H-bonds in a very thin interface, we can use the method of molecular selection, 
%because the H-bonds obtained in this way are not artificially broken, and if the interface is thick enough  
%(see Fig.\thinspace\ref{fig:128w_itp_pure_water_pair_k_k_prime_ihb_both_schemes}a), then we can use the IHB method, because it can automatically define which H-bonds come 
%from the interface without the need to select the molecules in the interface layer.
To summarize, we have studied the HB dynamics for instantaneous interfaces using two different statistical methods,IHB and MS.
From the above results for water/vapor interface, we conclude that from the perspective of HB dynamics,
the thickness of the air-liquid interface of water is about 3 \A. This value is smaller than that obtained from the SFG spectra 
(Ref.\thinspace\ref{sfg_lino3_interface}), and this result may have reference significance for our study of the influence of ions on the H-bonds 
outside the solvation shell of ions. 
%
\begin{table}[htb]
\centering
\caption{\label{tab:k_k_prime_tau_128w_pure_ihb_ADH} 
    The $k$ and $k'$ for the interfacial HB dynamics of the water/vapor interface (by the method of IHB and of ADH criteria). 
We carried on the longer time region 0.2 ps $< t <$ 12 ps. 
}
%The unit for $k$ ($k'$) is ps$^{-1}$, and for $\tau_{\text{HB}}$ ($=1/k$) is ps. 
%The parameter values and units are the same below. 
\begin{tabular}{cccc}
 Thickness & $k$ & $k'$ & $\tau_{\text{HB}} (=1/k)$ \\
\hline
  1.0 & 0.653 & 0.080 & 1.53  \\
  2.0 & 0.261 & 0.133 & 3.83  \\
  3.0 & 0.168 & 0.104 & 5.94  \\
  4.0 & 0.148 & 0.092 & 6.76  \\
  5.0 & 0.147 & 0.087 & 6.81  \\
  6.0 & 0.139 & 0.087 & 7.17  \\
\end{tabular}
\end{table}
\begin{table}[htb]
\centering
\caption{\label{tab:k_k_prime_tau_128w_pure_ihb_AHD} 
    The $k$ and $k'$ for the interfacial HB dynamics of the water/vapor interface (by the method of IHB and of AHD criteria).} 
\begin{tabular}{cccc}
 Thickness & $k$ & $k'$ & $\tau_{\text{HB}} (=1/k)$ \\
\hline
  1.0 & 0.661 & 0.080 & 1.51  \\
  2.0 & 0.265 & 0.133 & 3.77  \\
  3.0 & 0.172 & 0.102 & 5.82  \\
  4.0 & 0.148 & 0.090 & 6.74  \\
  5.0 & 0.149 & 0.084 & 6.72  \\
  6.0 & 0.144 & 0.078 & 6.93  \\
\end{tabular}
\end{table}

\begin{table}[H]
\centering
\caption{\label{tab:k_k_prime_tau_128w_pure_ihb_scheme1_ADH} 
    The $k$ and $k'$ for the interfacial HB dynamics of the water/vapor interface (by the method of molecule selection and of ADH criteria).} 
\begin{tabular}{cccc}
 Thickness & $k$ & $k'$ & $\tau_{\text{HB}} (=1/k)$ \\
\hline
  1.0 & 0.526 & 0.072 & 1.90  \\
  2.0 & 0.246 & 0.158 & 4.07  \\
  3.0 & 0.160 & 0.114 & 6.26  \\
  4.0 & 0.140 & 0.097 & 7.15  \\
  5.0 & 0.138 & 0.090 & 7.24  \\
  6.0 & 0.133 & 0.085 & 7.49  \\
\end{tabular}
\end{table}
%  6.0 & 0.125 & 0.080 & 8.00  \\
%  7.0 & 0.133 & 0.085 & 7.49  \\
\begin{table}[H]
\centering
\caption{\label{tab:k_k_prime_tau_128w_pure_ihb_AHD} 
    The $k$ and $k'$ for the interfacial HB dynamics of the water/vapor interface (by the method of molecule selection and of AHD criteria).} 
\begin{tabular}{cccc}
 Thickness & $k$ & $k'$ & $\tau_{\text{HB}} (=1/k)$ \\
\hline
  1.0 & 0.610 & 0.083 & 1.64  \\
  2.0 & 0.235 & 0.142 & 4.62  \\
  3.0 & 0.138 & 0.102 & 7.22  \\
  4.0 & 0.141 & 0.098 & 7.07  \\
  5.0 & 0.120 & 0.078 & 8.40  \\
  6.0 & 0.119 & 0.071 & 8.39  \\
\end{tabular}
\end{table}
%  6.0 & 0.117 & 0.072 & 8.58  \\
%  7.0 & 0.119 & 0.071 & 8.39  \\

\FloatBarrier
\paragraph{Experiments on HB Dynamics}
%Experimentally Ref.  
An important  structural characteristic of the H-bonded network is the average number of H-bonds per molecule, $\langle h_{i,j}\rangle$. \cite{Chowdhary2008} 
For bulk water systems, we find that in the DFTMD simulations the average number of H-bonds in the bulk phase is $\sim$ 4.35 which is slightly
(higher) than the usual estimate of 3.4 (interface system) for SPC/E water.

%TODO
%For interfacial systems of neat water, we find the average number
%of hydrogen bonds is 3.XX which is slightly
%(lower/higher) than the usual estimate of 3.4 for SPC/E water. \cite{Chowdhary2008}

%%===============================================
%\section{Rotational Anisotropy Decay of Water at the Interface of Alkali-Iodine Solutions}\label{CHAPETR_AD}
%%===============================================
%Using the transition dipole auto-correlation function, 
%we determined the rotational anisotropy decay and therefore the OH-stretch relaxation at water/vapor interface of alkali iodide solutions.
%%The effects of ion environment on structure and dynamics of water are obtained by comparing the second-order Legendre polynomial, i.e.,  $P_2(x)=\frac{1}{2}(3x^2-1)$,  orientational correlation function of the transition dipole.
%The anisotropy decay can be determined from experimental signal in two different polarization configurations---parallel and perpendicular polarizations, by 
%\begin{equation}
%        R(t)=\frac{S_{\parallel}(t)-S_{\perp}(t)}{S_{\parallel}(t)+2S_{\perp}(t)}
%\label{eq:ad}
%\end{equation}
%where $t$ is the time between pump and probe laser pulses.  The anisotropy decay can also be obtained by simulations, 
%and calculated by the third-order response functions $R(t)$. \cite{Jansen10,Jansen06}
%%
%%In the first shell with a radius 3 \A, the entropy difference betweem the \Li shell and \Na shell,
%%$\Delta S=k_B\text{ln}\frac{\Omega_\text{Na}}{\Omega_\text{Li}}=k_B\text{ln}\frac{n_\text{Na}/V_\text{Na}}{n_\text{Li}/V_\text{Li}} =k_B\text{ln}1.05$.
%%
%%\paragraph{Probability Distribution of Ions}
%%The probability distribution, shown in Fig.~\ref{fig: prob_124_LiI_Sans_double_axis}, of the ions in the water/vapor interface of LiI and NaI solutions with repect to the depth of the ions in the solutions 
%%indicates that the \I ions prefer to staying at the topmost layer of surface of solutions.
%%(molar concentration: 0.9 M, temperature: 330 K) 
%%It shows that \I ions tend to the surface of the solutions, while \Na and \Li tend to stay in the bulk. This result is consistent with the calculations from Ishiyama and Morita\cite{TI07,TI14}.
%The orientational anisotropy $C_2(t)$ is given by the rotational time-correlation function 
%\begin{equation}
%C_2(t)=\langle P_2(\hat{u}(0)\cdot\hat{u}(t)) \rangle,
%\label{eq:tcf2}
%\end{equation}
%where $\hat{u}(t)$ is the time dependent unit vector of the transition dipole, $P_2(x)$ is the second Legendre polynomial, and $\langle \cdots \rangle$ indicate 
%equilibrium ensemble average. \cite{Corcelli05,LinYS2010} %\cite{2010Lin} % angular brackets
%
%The anisotropy decay $C_2(t)$ for the water/vapor interface of LiI solution is shown in Fig.\thinspace\ref{fig:c2_2LiI_16_inset}.
%This function decays faster than that of neat water, indicating that H-bonds
%at the interfaces of alkali-iodine solutions reorient faster than in neat water. The inset shows the first 0.4 ps of $C_2(t)$, from which we see a 
%quick change during the first $\sim 0.1$ ps primarily due to librations.
%%
%\begin{figure}[h]
%\centering
%\includegraphics [width=0.36\textwidth] {./diagrams/c2_2LiI_16_inset} 
%\setlength{\abovecaptionskip}{0pt}
%  \caption{\label{fig:c2_2LiI_16_inset} The time dependence of the $C_2(t)$ of OH bonds at the water/vapor interfaces of 0.9 M LiI solution 
%    and of neat water (dashed line) at 330 K, calculated by DFTMD simulations.} 
%    %The water/vapor interface of neat water is modeled 
%    %with a slab made of 121 water molecules in a simulation box of size $15.6 \times 15.6 \times 31.0$ \A$^3$.
%\end{figure}
%%
%We also calculated the $C_2(t)$ for the interface of other alkali-iodine solutions LiI and KI. 
%The results of $C_2(t)$ for the water/vapor interfaces of these solutions are shown in Fig.\thinspace\ref{fig:c2_2KI_2NaI_2LiI_16}.
%In all the cases $C_2(t)$ decays faster than in neat water, indicating that H-bonds
%at the interfaces of the three alkali-iodine solutions are orientated faster than that of neat water.
%They show that \I ions can accelerate the dynamics of molecular reorientation of water molecules at interfaces.   
%
%%
%\begin{figure}[htbp]
%\centering
%\includegraphics [width=0.36\textwidth] {./diagrams/c2_2KI_2NaI_2LiI_16} 
%\setlength{\abovecaptionskip}{0pt}
%  \caption{\label{fig:c2_2KI_2NaI_2LiI_16} The time dependence of the $C_2(t)$ of OH bonds in water molecules at the water/vapor 
%  interface of 0.9 M alkali-iodine solutions and of neat water (dashed line) at 330 K, calculated by DFTMD simulations.}
%\end{figure} 
%
%We have obtained non-single-exponential kinetics for the rotation of water molecules both at the surface 
%and in bulk water (Appendix \ref{single_exp}).
%%This result is true for water molecules bound to ions. 
%Therefore, the rotational motion of water molecules are not simply characterized by well-defined rate constants. 
%%Then the problem is to understand the kinetics.
%Similar non-single-exponential kinetics is also obtained in the HB kinetics
%in liquid water \cite{AL96,Dirama05} and in the time variation of the average frequency shifts of the 
%remaining modes after excitation in hole burning technique \cite{Rey2002,Moller2004} and using BLYP functional. \cite{Bankura2014}
%Luzar and Chandler interpreted 
%the non-single-exponential kinetics as the result of an interplay between 
%diffusion and HB dynamics. \cite{AL96} 
%We can understand the non-single-exponential kinetics of rotational 
%anisotropy decay by fitting the rotational anisotropy decay by a 
%biexponential function.
%
%To obtain the effects of diffusion and HB decay of water molecules
%in solutions respectively, we assume that there are two independent 
%decays in the process of an anisotropy decay. 
%Therefore, the $C_2(t)$ has the form \cite{TanHS05}
%\begin{equation}
%C_2(t)=A_1e^{-\kappa_1 t} +A_2e^{-\kappa_2 t},
%\label{eq:tcf3}
%\end{equation}
%where $A_i$ are constants and $\kappa_i$ are decay rates ($i=1, 2$). 
%The time constants and amplitudes of the biexponentials fits for 
%the $C_2(t)$ are listed in Table ~\ref{tab:2LiI_c2_biexp} and Table ~\ref{tab:2NaI_c2_biexp}.
%The biexponential fit is very close to the calculated $C_2(t)$, which can be seen in Fig.\thinspace\ref{fig:2LiI-124w_c2_fit_5ps_biexp} (compare Fig.\thinspace\ref{fig:2LiI-124w_c2_fit_5_single-exp}).
%%
%\begin{table}[hbt]
%\centering
%\caption{\label{tab:2LiI_c2_biexp}%
%	Biexponential fitting (5 ps) of the $C_2(t)$ for water molecules in 0.9 M LiI solution.}
%%\begin{ruledtabular}
%\begin{tabular}{lccccc}
%water molecules & $A_1$  & $\kappa_1$ (THz) & $A_2$ & $\kappa_2$ (THz) \\
%\hline
%I$^-$-shell & 0.44 & 0.25 & 0.39 & 0.26\\
%Li$^+$-shell & 0.88 & 0.07 & 0.07 & 8.24\\
%bulk & 0.84 & 0.11 & 0.09 & 4.88 \\
%surface & 0.73 & 0.27 & 0.22 & 13.47 \\
%\end{tabular}
%%\end{ruledtabular}
%\end{table}
%%--
%
%\begin{table}
%\centering
%  \caption{\label{tab:2NaI_c2_biexp}%
%	Biexponential fitting (5 ps) of the $C_2(t)$ for water molecules in 0.9 M NaI solution.}
%  \begin{tabular}{lccccc}
%  water molecules & $A_1$  & $\kappa_1$ (THz) & $A_2$ & $\kappa_2$ (THz) \\
%  \hline
%  I$^-$-shell & 0.86 & 0.14 & 0.08 &9.86 \\
%  Na$^+$-shell & 0.71 & 0.06 & 0.18 &0.79 \\
%  bulk & 0.81 & 0.06 & 0.10 & 1.25 \\
%  surface & 0.77 & 0.11 & 0.13 & 2.31 \\
%  \end{tabular}
%\end{table}
%%
%%图
%\begin{figure}[htbp]
%\centering
%\includegraphics [width=0.60\textwidth] {./diagrams/2LiI-124w_c2_fit_5_biexp} 
%  \caption{\label{fig:2LiI-124w_c2_fit_5ps_biexp} The time dependence of the $C_2(t)$ of OH bonds 
%  in water molecules at the water/vapor interface of LiI solution.}
%\end{figure} 
%%
%%[Notes: The 63-water-slab models is listed here as a reference. The number of water molecules is small; The data for KI/vapor and LiI/vapor interfaces come from  KI\_16 and LiI\_16 systems.  
%%Water(63) &0.831$\pm(1\times10^{-4})$ &  0.08760 $\pm(2\times 10^{-5})$&0.100$\pm(2\times10^{-4})$ & 1.029 $\pm(4\times10^{-3})$  \\ ]
%%
%%\begin{figure}[htbp]
%%\centering
%%\includegraphics [width=0.4 \textwidth] {./diagrams/c2_121-pure_2KI_2LiI_16_inset_fit_biexp} 
%%\setlength{\abovecaptionskip}{10pt}
%%\caption{\label{fig:c2_121-pure_2KI_2LiI_16_inset_fit_biexp} The fitted and calculated anisotropy decay of OH bonds in water molecules in LiI solution/vapor interface (red), LiI solution/vapor interface (blue) and neat water/vapor interface (black). The corresponding fitted functions are denoted by dashed lines. The concentration of LiI and KI solution is 0.9 M.}
%%\end{figure} 
%
%Then we considered the effect of ion species in solutions on the anisotropy decay of water molecules.
%From Table \ref{tab:2LiI_c2_biexp} and Table \ref{tab:2NaI_c2_biexp}, we find that 
%for both LiI and NaI solutions, there are two decay processes in the dynamics --- amplitude $\sim 1$,
%decay constant $\sim$ 0.1 THz, and for the other describe the initial fast decay 
%of the anisotropy, with amplitude $\sim 0.1$, decay constant $\sim$ (1--10) THz, 
%due to the inertial-librational motion preceding the orientational diffusion.
%That is, two decay processes exist in the dynamics of water molecules 
%at the water/vapor interfaces of alkali-iodine solutions. 
%%The one describe the initial fast decay of the anisotropy, 
%%with amplitude $\sim$ 0.1, decay constant $\sim$ (1--10) THz,
%%results from the inertial-librational motion preceding the orientational diffusion.
%%%
%%\begin{table}[H]
%%\centering
%%\caption{\label{tab:fitting_c2_for_each_type_of_water}%
%%  Biexponentially fitting (2 ps) of the $C_2(t)$ for different types of water molecules at the water/vapor interface of LiI solutions.}
%%\begin{tabular}{lccccc}
%%water molecules & $A_1$  & $\kappa_1$ (THz) & $A_2$ & $\kappa_2$ (THz) \\
%%\hline
%%$DDAA$ & 0.85 & 0.25   & 0.10 & 16.0\\
%%$DD'AA$ & 0.89 & 0.14  & 0.06 & 14.1 \\
%%$D'AA$ & 0.38 & 0.99 & 0.38 & 0.99 \\
%%\end{tabular}
%%\end{table}
%%%
%%\begin{table}[H] %[!hbtp]
%%\centering
%%\caption{\label{tab:table_CoordNo}%
%%The coordination number of the atoms in LiI (NaI) solutions.}
%%\begin{tabular}{lccc}
%%name & radius of the first shell (\AA) & coordination number \\
%%\hline
%%$n_\text{I-H}(\text{LiI})$ & 3.3 & 5.5 \\
%%$n_\text{I-H}(\text{NaI)}$ & 3.3 & 5.1 \\
%%$n_\text{I-O}(\text{LiI)}$ & 4.3 & 5.8 \\
%%$n_\text{I-O}(\text{NaI)}$ & 4.3 & 6.0 \\
%%$n_\text{Li-O}(\text{LiI)}$ & 3.0 & 4.0 \\
%%$n_\text{Na-O}(\text{NaI)}$ & 3.5 & 6.0 
%%\end{tabular}
%%\end{table}

