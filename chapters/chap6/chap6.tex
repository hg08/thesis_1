\chapter{Hydrogen bond dynamics at water/vapor interfaces}\label{CHAPTER_HB}
H-bonds play a critical role in the behaviour of bulk water,\cite{Eisenberg1969,Luzar1996,Cabane2005} 
water near interfaces,\cite{Chowdhary2008} and aqueous solutions. \cite{Naslund2005} 
There are many methods to study the hydrogen bond (HB) dynamics in water, solutions or interfaces, 
such as molecular dynamics simulation,\cite{Tongraar2006,Chanda2006,Tongraar2010,Chowdhary2008,Banerjee2016} neutron scattering, 
Infrared (IR) spectroscopy,\cite{Werhahn2011,Fournier2016} etc.
In this chapter, we will introduce the general concepts and methods of HB dynamics \cite{AL96,Luzar1996,DC87} used to analyze the structure 
and dynamic properties of bulk water and water-air interfaces. 

\section{Definitions of HB population and correlation functions}
Luzar and Chandler \cite{AL96} have pioneered the analysis of the HB dynamics of pure water, and
subsequently such analysis has been also extended to more complex systems, e.g., electrolytes, \cite{AC00} protein and  micellar surfaces. \cite{SP05}
There are temporal, geometric\cite{Kumar2007} and energetic criteria \cite{Sciortino1989}to define HB.
Here we use the geometric one.
Two water molecules are H-bonded if their interoxygen distance between of specific tagged pair of water molecules 
is less than cutoff radius $r^{\text{c}}_{\text{OO}}$ and
the O-H$\cdots$O angle is less than cutoff angle $\phi^{\text{c}}$. \cite{AKS86,JT90,SB02} 
The value $r^{\text{c}}_{\text{OO}}$ corresponds to the first-minimum position of the O--O Radial Distribution Function (RDF) of water. \cite{Sciortino1989}   
The choice for the cutoff angle $\phi^{\text{c}}$ for water-water molecules is obtained by studying the average number of H-bonds,
as a function of $\phi^{\text{c}}$. \cite{Luzar1993} We call this definition of HB the Acceptor-Donor-Hydrogen (ADH) criterion. 
To compare the impact of different HB definitions on HB dynamics, we also use another definition of HB in our analysis. 
When the distance between the oxygen atoms of two water molecules is less than $r^{\text{c}}_{\text{OO}}$, 
and the oxygen-hydrogen-oxygen included angle is greater than cutoff angle $\theta^{\text{c}}$, then we say that there is a HB between the two molecules. 
We denote this definition as the Acceptor-Hydrogen-Donor (AHD) criterion of H bonds.
In this thesis, we use $r^{\text{c}}_{\text{OO}}=3.5\AA$ both for ADH and AHD criteria, $\phi^{\text{c}}=30^{\circ}$ for ADH criterion, 
and $\theta^{\text{c}}=120^{\circ}$ for AHD criterion.

% introduce h(t)
The configuration criterion above allows us to define a variable $h[r(t)] = h(t)$, the HB population. 
Here an instantaneous configuration $r(t)$ denotes the positions of all the atoms in the system at time $t$.\cite{AL96}  
The $h(t)$ has a value 1 when the particular tagged pair of molecules are bonded, and 0 otherwise. 
%=================
% added 2020-5-27: to show that h(t) is actually the fluctuation of itself (\delta h).
We know that the fluctuation or deviation in a dynamical variable $A(t)$ from its time-independent equilibrium average $\langle A\rangle$ , 
is defined by \cite{DC87} 
$$
\delta A = A(t) - \langle A\rangle.
$$
For the $h(t)$, since the probability that a specific pair of molecules is bonded in a large system is extremely small, i.e., 
the time average of $h$ is zero, or  
$\langle h \rangle = 0$,
then
$$
\delta h(t) = h(t).
$$
Therefore, the $h(t)$ describe the fluctuation $\delta h(t)$  of the HB population.  
%=================



While the equilibrium average of the $\delta h(t)$ is zero, we can obtain useful information by considering the equilibrium 
correlations between fluctuations at different times. The correlation between the $\delta h(t)$ and the $\delta h(0)$ can be written as 
$$
\langle \delta h(0) \delta h(t)\rangle = \langle h(0)h(t)\rangle-\langle h \rangle^2 = \langle h(0)h(t)\rangle,
$$
where the averaging $\langle\cdots\rangle$ is to be performed over the ensemble of initial conditions.% $(r^N, p^N)$.


In this paragraph, we will use three different correlation functions to describe the HB dynamics of water/vapor interfaces of solutions:
the HB population auto-correlation function \CHB, the continuum HB population correlation function (survival probability) \SHB and the reactive flux $k(t)$. \cite{Rapaport1983}

\FloatBarrier
\paragraph{HB population auto-correlation function}
We use the auto-correlation function \CHB ($c(t)$ for short) of the HB population to describe the structural relaxation of H-bonds: 
\begin{eqnarray}
C_{\text{HB}}(t)=\langle h(0)h(t) \rangle/\langle h\rangle
\label{eq:C_HB}.
\end{eqnarray}
With the aid of the ergodic principle, the ensemble average $\langle \cdots\rangle$ is implemented by time average.
The $\langle h\rangle$ is the probability that a pair of randomly chosen water molecules in the system is
H-bonded at any time $t$. 
As examples, the dynamics of the interoxygen distance $r_{\text{OO}}(t)$, 
the cosine of H$-$O$\cdots$O angle cos$\phi(t)$  
and the $h(t)$ for a HB in a DFTMD simulated water cluster (H$_2$O)$_n$ (n=5) at 300 K is displayed in Fig.\thinspace\ref{fig:Ex30ps_hb}, respectively.
%-------------------
\begin{figure}[hbtp]
\centering
\includegraphics [width=0.42\textwidth] {./diagrams/Ex30ps_hb}
\setlength{\abovecaptionskip}{0pt}
\caption{\label{fig:Ex30ps_hb}Dynamics of $r_{\text{OO}}(t)$ (top), cos$\phi(t)$ (middle), 
  and $h(t)$ (bottom) for a HB in a water cluster. The dashed lines show the interoxygen distance 
  boundary $r^{\text{c}}_{\text{OO}}$=3.5 \AA (top) and criterion of cosine of H$-$O$\cdots$O angle cos$\phi^{\text{c}}$ 
  with $\phi^{\text{c}}$=30$^{\circ}$, respectively.}
\end{figure} 

In a large system that consist of many water molecules, the probability that a specific pair of water molecules are H-bonded is extremely small. 
Therefore, the \CHB also relaxes to zero, when $t$ is large enough. 
The \CHB measures correlation in $h(t)$ independent of any possible bond breaking events. 
This function is similar to one of the intermittent HB correlation functions, introduced by Rapaport,\cite{Rapaport1983}
and can be studied by a continuous function, probability densities.
From the \CHB, the HB relaxation time can also be computed by
\begin{eqnarray}
  \tau_{\text{R}} &=& \frac{\int t C_{\text{HB}}(t)\text{d}t}{\int C_{\text{HB}}(t)\text{d}t}.
\label{eq:tau_relaxation}
\end{eqnarray}
The \CHB for the DFTMD simulated bulk water is shown in Fig.\thinspace\ref{fig:128w_c_itp_bk_ns40}.
We can obtain the relaxation time from Eq.\thinspace\ref{eq:tau_relaxation}: $\tau_R = 14.01$ ps for ADH definition, 
and $\tau_R = 14.16$ ps for AHD definition. 
%/home/gang/Data/bulk_pure/__bulk_pure/__hbacf/128w_c_bk_ns40.eps
\begin{figure}[hbtp]
\centering
\includegraphics [width=0.36\textwidth] {./diagrams/128w_c_bk_ns40}
\setlength{\abovecaptionskip}{0pt}
\caption{\label{fig:128w_c_itp_bk_ns40}Time dependence of \CHB for the DFTMD simulated bulk water at 300 K with density $\rho =1.00$ g/cm$^3$.} 
%The length of the trajectory is 35 ps of physical time. Ref:\cite{Khaliullin2013}}
\end{figure} 
%

The method of water-water pair based HB dynamics used in this section has been frequently used in previous literature.\cite{Luzar1994,AL96,AC00} 
The basis is the population operator $h(t)$ of the HB formed between two water molecules. 
The function \CHB is interpreted as the probability that the HB between a certain pair of water molecules is intact at time  $t$, 
if the pair of water molecules are H-bonded at time zero. 
The \CHB measures correlation in $h(t)$ independent of any possible bond breaking events. 
It is one of the intermittent HB correlation functions, introduced by Rapaport.\cite{Rapaport1983} 

Here we also discuss the definition of another possible hydrogen bond population operator $\tilde{h}(t)$ besides $h(t)$ introduced in the text.
When a pair of water molecules $a$ and $b$ are H-bonded, 
the oxygen atom in each water molecule can act both as a donor and an acceptor. 
In particular, a pair of water molecules can form 4 different forms of H-bonds. 
In other words, if the role of H atoms between the pair of water molecules changes, but they still form H-bonds, 
we think that an old H-bond is broken and a new H-bond is formed.

The correlation functions of $h(t)$ ($\tilde{h}(t)$) of water molecules in bulk water is shown in Fig. \ref{fig:bk_water_c_two_population_operators_with_ADH}.
It shows that there are some differences in the correlation functions of the two definitions $h(t)$ and $\tilde{h}(t)$,
because hydrogen exchange is considered in the O--H pair-based HB population $\tilde{h}(t)$, but not in the water--water pair-based HB population $h(t)$.
%
\begin{figure} [htbp]
\centering
	\includegraphics [width=0.36\textwidth] {./diagrams/bk_water_c_two_population_operators_with_ADH}
\setlength{\abovecaptionskip}{0pt}
	\caption{\label{fig:bk_water_c_two_population_operators_with_ADH} The auto-correlation functions of $h(t)$ and $\tilde{h}(t)$ for water molecules in bulk water (ADH criterion). 
        The HB population is based on two different definitions, $h(t)$ is based on a pair of water molecules (solid line),\cite{Khaliullin2013} 
and $\tilde{h}(t)$ is based on O--H pairs between water molecules (dashed line).}
\end{figure} 

Fig.\thinspace\ref{fig:128w_bk_2delta_t_60ps_water_pair_c_ns40} shows the correlation function $C_\text{HB}(t)$ 
for bulk water over time. 
Comparing Fig.\thinspace\ref{fig:128w_c_itp_bk_ns40} and Fig.\thinspace\ref{fig:128w_bk_2delta_t_60ps_water_pair_c_ns40}, 
we find that although the trend of change of \CHB is the same: as time increases, it gradually decays from 1; 
but from a quantitative point of view, the latter decays more slowly. This difference comes from our definition 
of the HB population operator, and does not depend on the HB criterion. 
From Fig.\thinspace\ref{fig:128w_bk_2delta_t_60ps_water_pair_c_ns40}, 
we see that the above conclusions are correct regardless of ADH or AHD criteria. 
In the following, we will use this HB population operator based on molecular pairs, which is based on water-water molecule pairs, 
or ion-water molecule pairs, unless otherwise specified.
\begin{figure}[H]
%Location: /home/gang/Github/water_pair_HB_dynamics/plot/plot_c/
\centering
\includegraphics [width=0.360\textwidth] {./diagrams/128w_bk_2delta_t_60ps_water_pair_c_ns40}
\setlength{\abovecaptionskip}{0pt}
\caption{\label{fig:128w_bk_2delta_t_60ps_water_pair_c_ns40} 
The $C_\text{HB}(t)$ for bulk water, 
as computed from the ADH (solid line) and AHD (dashed line) criterion of H-bonds. Ref:\cite{Khaliullin2013}} 
\end{figure}
%

Because the thermal motion can cause distortions of H-bonds from the perfectly tetrahedral configuration,
water molecules show a librational motion on a time scale of $\sim$ 0.1 ps superimposed to rotational and diffusional motions ($> 1$ ps), 
which causes a time variation of interaction parameters.
A new HB population $h^{(d)}(t)$ was also defined to obviate the distortion of real HB dynamics
due to the above geometric definition. \cite{Sciortino1989,AC00}
The $h^{(d)}(t)$ is 1 when the interoxygen distance of a particular tagged pair of water molecules is less than $r^{\text{c}}_{\text{OO}}=3.5$ \AA at time $t$ and 0 otherwise. 
The difference between the operators $h^{(d)}(t)$ and $h(t)$ is that those molecular pairs that meet the condition of $h^{(d)}(t)=1$ may not meet the condition of $h(t)=1$.
That is, the H-bonds between the tagged molecular pairs that satisfy the condition $h^{(d)}(t)=1$ may have been broken, but they may more easily form H-bonds again.
The function 
\begin{eqnarray}
  C^{(d)}_{\text{HB}}(t)=\langle h(0)h^{(d)}(t) \rangle/\langle h\rangle
\label{eq:C_HB_d}
\end{eqnarray}
is the probability that the specific two water molecules are located in reformable region ($r_{\text{OO}} < r^{\text{c}}_{\text{OO}}$) at time $t$,
if they were H-bonded at time zero. 
The correlation function 
%
\begin{eqnarray}
n(t)=\langle h(0)[1-h(t)]h^{(d)}(t) \rangle/\langle h\rangle 
\label{eq:n_HB}
\end{eqnarray}
represents the probability at time $t$ 
that a tagged pair of initially H-bonded water molecules are unbonded but remain separated by less than $r_{\text{OO}}^{\text{c}}$.
In the above formula, $1-h(t)$ describes the breaking of a HB at time $t$ after its formation at time $t=0$.

The probability at time $t$ that a pair of water molecules bonded by H-bonds at the initial moment does not be bonded 
but the distance between their oxygen atoms is still less than $r_\text{OO}^c$ is calculated according to 
\begin{eqnarray}
n(t) = \int_0^t dt'k_\text{in}(t'),
\label{eq:n_from_k_in}
\end{eqnarray}
where $k_\text{in}(t) = -\langle \dot h(0)[1-h(t)]h^d(t) \rangle/\langle h\rangle$ is the restricted rate function. 
%===============================
\FloatBarrier
\paragraph{Survival Probability}
%===============================
Another scheme to describe the HB dynamics is the continuum HB population correlation function \SHB ($s(t)$ for short), or survival probability \cite{AC00} for a newly generated HB.
It is defined as
\begin{eqnarray}
S_{\text{HB}}(t)=\langle h(0)H(t) \rangle/\langle h\rangle 
\label{eq:S_HB},
\end{eqnarray}
where $H(t)=1$ if the tagged pair of molecules, remains \emph{continuously} H-bonded till time $t$ 
and 0 otherwise.  It describes the probability that an initially H-bonded molecular pair 
remains bonded at all times up to $t$. \cite{Chowdhuri2006}
The \SHB for the DFTMD simulated bulk water according to the formula \ref{eq:S_HB} is shown in Fig.\thinspace\ref{fig:128w_s_itp_bk_ns40}.
%
\begin{figure}[hbtp]
\centering
\includegraphics [width=0.36\textwidth] {./diagrams/128w_s_bk_ns40}
\setlength{\abovecaptionskip}{0pt}
\caption{\label{fig:128w_s_itp_bk_ns40}Time dependence of \SHB for the DFTMD simulated bulk water at 300 K with density $\rho =1.00$ g/cm$^3$.} 
%The length of the trajectory is 35 ps of physical time.
\end{figure} 

The average continuum HB lifetime $\langle \tau_{\mathrm{a}} \rangle$ is calculated by the integration of \SHB over $t$ (For detailed derivation, see Appendix \ref{diff_distr}.) :  
\begin{eqnarray}
  \langle\tau_{\mathrm{a}}\rangle = \int_0^\infty dt S_{\text{HB}}(t).
\label{eq:calculate_hb_lifetime_from_s}
\end{eqnarray}
%
The time derivative of \SHB
\begin{eqnarray}
P_a(t) = -\frac{\text{d}S_{\text{HB}}(t)}{\text{d}t}
\label{eq:P_1}
\end{eqnarray}
represents the first passage time probability density of H bonds. $P_a(t)$ is also called probability distribution of HB lifetimes, \cite{Sciortino1990prl,Krausche1992,FWS99,Voloshin2009} or histogram of HB lifetimes.\cite{Geiger1984,Stanley2000}
It denotes the probability of the first HB breaking in time $t$ after it has been detected at $t=0$, i.e.,
\begin{eqnarray}
S_{\text{HB}}(t)= \int_t^\infty P_a(t')dt'.
\label{eq:P_2}
\end{eqnarray}

\FloatBarrier
\paragraph{Reactive Flux $k(t)$} 
The rate of relaxation to equilibrium is characterized by the reactive flux correlation function, 
\begin{eqnarray}
k(t) = -\frac{\text{d}C_{\text{HB}}(t)}{\text{d}t},
\label{eq:k}
\end{eqnarray}
i.e., $\langle j(0)[1-h(t)]\rangle/\langle h\rangle$,
where 
$j(0)=-\text{d}h/\text{d}t|_{t=0}$ 
is the integrated flux departing the HB configuration space at time $t=0$ (For detailed derivation, see Appendix \ref{calc_rf}.).
The reactive flux $k(t)$ quantifies the rate that an initially present HB breaks at time $t$, 
independent of possible breaking and reforming events in the interval from 0 to $t$.
Therefore, $k(t)$ measures the effective decay rate of an 
initial set of H-bonds. \cite{DC87,FWS00}


We assume that each HB acts independently of other H-bonds, \cite{AL96,AL00} 
and due to detailed balance condition, we can obtain 
\begin{eqnarray}
  \tau_{\text{HB}} &=& \frac{1- \langle h\rangle}{k},
\label{eq:rate}
\end{eqnarray}
where $k$ is the rate constant of breaking a HB (forward rate constant). \cite{Chandler1986,Chandler1978} 
For an aqueous interface, the probability of exactly a tagged molecule pair forming a HB is very low, that is, $\langle h\rangle \ll 1$. Therefore,
the $k$ is related to the average HB lifetime by $\tau_{\text{HB}}=1/k$.
We use $k'$ to represent the backward rate constant, that is, the rate constant from the HB \emph{on} state to the HB \emph{off} state for a tagged pair of molecules.
Therefore, the reaction time constant $\tau_\text{re}$ is 
\begin{eqnarray}
  \tau_\text{re} &=& \frac{1}{k+k'}.
\label{eq:reaction_rate_tau}
\end{eqnarray}

%
\FloatBarrier
\section{Dynamical properties of H-bonds in bulk water and at the water-vapor interfaces}
The bulk water system and the interface between pure water and vacuum, i.e., the water-vapor interface, 
are ideal model systems for testing our algorithms.
For bulk water, we can compare the results of the current method with the results of 
previous works.\cite{AL96,Kessler2015} After the validation for bulk water, we will show in this paragraphs the results of the HB dynamics of the air-water interface.

%[all the data on the simulation]
All simulations in this chapter were performed at 300 K within the canonical NVT ensemble.
The length of the trajectory is 60 ps of physical time.
%The definition of $h(t)$ is based on specific H--O bond, instead of water-water pairs.
The simulated bulk water consisted of 128 water molecules in a periodic cubic box of length L = 15.64 \A, which corresponds to a density of 1.00 g cm$^{-3}$.
The simulated water-vapor interface consisted of 128 water molecules in a periodic box with size 15.64 $\times$ 15.64 $\times$ 31.28 \A$^3$.

\paragraph{Correlation functions $c(t)$,$n(t)$ and $k(t)$}
The RDFs $g_\text{OO}(r)$ and $g_\text{OH}(r)$ for the bulk water system are 
shown in Fig.\thinspace\ref{fig:rdf_bk_pure_pbc}.
\begin{figure}[htb]
\centering                                          
%\includegraphics [width=0.6 \textwidth] {./diagrams/rdf_bk_pure_and_interf_pure_normed} 
\includegraphics [width=0.4 \textwidth] {./diagrams/rdf_bk_pure_pbc} 
\setlength{\abovecaptionskip}{0pt}
  \caption{\label{fig:rdf_bk_pure_pbc}Partial RDFs of the simulated bulk water.}
\end{figure}
\begin{figure}[htb]
\centering
\includegraphics [width=0.6 \textwidth] {./diagrams/pure_bk_c_n_k} 
\setlength{\abovecaptionskip}{0pt}
  \caption{\label{fig:pure_bk_c_n_k}Time dependence of (a) $n(t)$, $c(t)$ and (b) $k(t)$ 
of water--water H-bonds for \emph{bulk} water.}
\end{figure}
The correlation functions \CHB from the trajectory of a DFTMD simulation calculated according to formula \ref{eq:C_HB} with ADH (solid line) and AHD (dashed line) definition of H-bonds are 
shown in Fig.\thinspace\ref{fig:pure_bk_c_n_k} a. 
The reactive flux $k(t)$ calculated according to formula \ref{eq:k} (see Fig.\thinspace\ref{fig:pure_bk_c_n_k}b) is very consistent with the result in \cite{AL96b}.
For bulk water, there exists a $\sim 0.2$-ps transient period,
during which $k(t)$ quickly changes from its initial value. \cite{FWS00}
However, at longer times, the $k(t)$ is independent of the HB definitions.
%[we performed a DFTMD simulation of the bulk water system with a total time of 60 ps, and used the two different HB definitions (ADH and AHD) to calculate $k(t)$. ]
The calculation results in Fig.\thinspace\ref{fig:pure_bk_c_n_k} b show that when $t$ is large enough, 
the difference in $k(t)$ caused by different HB definitions is relatively small.
Therefore, the long time decay of $k(t)$ reflects the general properties of H-bonds.
Calculating the reactive flux HB correlation functions is a more rigorous way to obtain the nature of H-bonds. \cite{AL00}

\begin{figure}[H] %htb
\centering
\includegraphics [width=0.6 \textwidth] {./diagrams/128w_itp_c_n_k} 
\setlength{\abovecaptionskip}{0pt}
  \caption{\label{fig:128w_itp_c_n_k}Time dependence of (a) $n(t)$, $c(t)$ and (b) $k(t)$ 
of water--water H-bonds for water/vapor \emph{interface}.}
\end{figure}

Let's now discuss the result for the water-vapor interface.
For the water-vapor interface, we reported the result of the correlation function  $c(t)$, $n(t)$
in Fig.\thinspace\ref{fig:128w_itp_c_n_k} a and the reactive flux $k(t)$ in Fig.\thinspace\ref{fig:128w_itp_c_n_k} b.
%
In both bulk and interface cases, the $k(t)$ quickly changes from its initial value on a time scale of less than 0.2 ps. 
This value can be roughly seen from Fig.\thinspace\ref{fig:pure_bk_and_itp_k}, which redraws the $k(t)$ in Fig.\thinspace\ref{fig:pure_bk_c_n_k} and 
Fig.\thinspace\ref{fig:128w_itp_c_n_k} in double logarithmic coordinates and compares them.
%
\begin{figure}[H]
\centering
\includegraphics [width=0.6 \textwidth] {./diagrams/pure_bk_and_itp_k} 
\setlength{\abovecaptionskip}{0pt}
  \caption{\label{fig:pure_bk_and_itp_k}Time dependence of $k(t)$ 
of water--water H-bonds for (a) bulk water and (b) water-vapor interface.}
\end{figure}
%
For the water/vapor interface, we focus on the reactive flux $k(t)$, 
which had been used in the study of HB dynamics of liquid water. \cite{AL96,Khaliullin2013}
The $k(t)$ calculated from the trajectory of water molecules in simulations, is reported in Fig.\thinspace\ref{fig:121}. 
In the case of water/vapor interface, $k(t)$ quickly changes from its initial value on a time scale of less than 0.2 ps 
(see the inset of Fig.\thinspace\ref{fig:121}). 
Beyond this transient period, $k(t)$ decays to zero monotonically, and the slope of the $\ln{k(t)}$ increases monotonically with $t$ (see Fig.\thinspace\ref{fig:121}). 
These two properties have been found for bulk water using the SPC water model by Luzar and Chandler. \cite{AL96} 
This log-log plot of the $k(t)$ shows that, as in the case of liquid water, this decay behaviour does not coincide with a power-law decay for water/vapor interface of neat water.
This result is also the same as that of the classical molecular simulation of pure water. \cite{AL96b,Luzar1996}
%

The functions $n(t)$ calculated according to Eq.\thinspace\ref{eq:n_from_k_in} for bulk water and water/vapor interface are shown in 
Fig.\thinspace\ref{fig:128w_bk_itp_50ps_n_from_k_in_with_2_hb_def_type2}. 
It shows that the overall trend of n(t) does not depend on the choice of HB definition.
i.e., as $t$ increases, $n(t)$ increases rapidly from 0, and it reaches a maximum value at $t \approx 10$ ps, and then gradually decreases. %[EXPLAIN THE RESULTs]
We also find that the maximum value of $n(t)$ at the water/vapor interface is slightly higher than that in bulk water.
It can be seen that $n(t)$ of the water/vapor interface
is always greater than that in the bulk water, whether we take the definition of ADH or AHD.
We interpret this result as the fact that at time $t$, there is a greater probability that the H-bonds on the interface are broken 
compared to the H-bonds in the bulk water.
\begin{figure}[htpb]
\centering
\includegraphics [width=0.42\textwidth] {./diagrams/121}
\setlength{\abovecaptionskip}{0pt}
  \caption{\label{fig:121}Time dependence of $k(t)$ for the interface of neat water, according to Eq.\thinspace\ref{eq:k}.
  The inset shows the log-log plot of $k(t)$.}
\end{figure}
\begin{figure}[H]
\centering
\includegraphics [width=0.6\textwidth] {./diagrams/128w_bk_itp_50ps_n_from_k_in_with_2_hb_def_type2}
\setlength{\abovecaptionskip}{0pt}
\caption{\label{fig:128w_bk_itp_50ps_n_from_k_in_with_2_hb_def_type2} 
Time dependence of $n(t)$ for bulk water and the water/vapor interface from (a) ADH (b) AHD criteria.} 
\end{figure}

\paragraph{Reaction rate constants $k$ and $k'$}
Khaliullin and K\"uhne have studied the H-bonding kinetics of pure water using the AIMD simulations.\cite{Khaliullin2013}
Based on the HB population operators $h(t)$ and $h^{(d)}(t)$, and the correlation functions $n(t)$ and $k(t)$, they have used the simulation data 
to obtain the ratio $k/k'$ in the bulk water, and then the lifetime and relaxation time 
of the HB.  Here, we use the DFTMD simulations to study the HB dynamics at the air-water interfaces.
We can obtain the optimal solution range of $k$ and $k'$ from the relationship between the reactive flux 
and the HB population correlation function $c(t)$ and $n(t)$, and the two rate constants $k$ and $k'$, i.e.,
\begin{eqnarray}
  k(t) = kc(t)-k'n(t).
\label{eq:fitting_k_rates}
\end{eqnarray}
%[Answer Q3]
We can find the optimal value of the rate constants, $k$ and $k'$, 
by a least squares fit of the calculated data $k(t)$, $c(t)$ and $n(t)$ beyond the transition phase.  
The functions $c(t)$ can be regarded as a $P$-dimensional column vector composed by $(c(1),c(2),\cdots,c(P))^T$, and denoted as ${\bf c}$,
with $c(i)$ representing the value of the correlation $c(t)$ at $t=i$.
Similarly, the functions $n(t)$ and $k(t)$ can also be viewed as $P$-dimensional column vectors and can be denoted as ${\bf n}$ and ${\bf k}$, respectively.
Therefore, the $k$ and $k'$ can be determined from the matrix ${\bf A} = \begin{bmatrix} {\bf c} & {\bf n} \end{bmatrix}$, i.e., 
\begin{equation}
\begin{bmatrix} k\\ -k' \end{bmatrix} = ({\bf A}^T {\bf A})^{-1} {\bf A}^T {\bf k}. 
\end{equation}
For bulk water and the water/vapor interface, the optimal $k$ and $k'$ are reported in Table 
\ref{tab:k_k_prime_128w_pure_1} and \ref{tab:k_k_prime_128w_pure_2}. 
% 
\begin{table}[htb]
\centering
\caption{\label{tab:k_k_prime_128w_pure_1} 
    The $k$ and $k'$ for the bulk water (bulk) and the water/vapor interface (intrf.). We carried on the short time region 0.2 ps $< t <$ 2 ps. 
    The unit for $k$ ($k'$) is ps$^{-1}$, and that for $\tau_{\text{HB}}$ ($=1/k$) is ps. (Same below.)
} 
\begin{tabular}{ccccccc}
 Criterion & $k$  (bulk) & $k'$ (bulk) & $\tau_{\text{HB}}$ (bulk) & $k$  (intrf.) & $k'$ (intrf.) & $\tau_{\text{HB}}$ (intrf.)\\
\hline
  % With 4 digital!(Keep it)
  %ADH & 0.3345 & 0.8591 & 2.9895 & 0.3587 & 0.6730 & 2.7881  \\
  %ADH(from $k_{in}$) & 0.2959  & 0.9883 & 3.3795  & 0.3225 & 0.7652 & 3.1012 \\
  %AHD & 0.3334 & 1.0414 & 2.9991 & 0.3520  & 0.7847  &  2.8405\\
  %AHD(from $k_{in}$) & 0.2882 & 1.1490 & 3.4699 & 0.3140 & 0.8867 & 3.1836 \\
  ADH & 0.30  & 0.99 & 3.38  & 0.32 & 0.77 & 3.10 \\
  AHD & 0.29 & 1.15 & 3.47 & 0.31 & 0.89 & 3.18 \\
\end{tabular}
\end{table}
%
\begin{table}[htb]
\centering
\caption{\label{tab:k_k_prime_128w_pure_2} 
    The $k$ and $k'$ for the bulk water (bulk) and the water/vapor interface (intrf.). We carried on the long time region 2 ps $< t <$ 12 ps.
} 
\begin{tabular}{ccccccc}
 Criterion & $k$  (bulk) & $k'$ (bulk) & $\tau_{\text{HB}}$ (bulk) & $k$  (intrf.) & $k'$ (intrf.) & $\tau_{\text{HB}}$ (intrf.)\\
\hline
  %ADH & 0.1151 & 0.0311 & 8.6872 & 0.1593 & 0.0580 & 6.2786 \\
  %ADH(from $k_{in}$) & 0.1147  & 0.0391 & 8.7184 & 0.1569  & 0.0678 & 6.3723\\
  %AHD & 0.1071 & 0.0424 & 9.3450  & 0.1572 & 0.0763 & 6.3626 \\
  %AHD(from $k_{in}$) & 0.1053  & 0.0472 & 9.4963 & 0.1545  & 0.0884 & 6.4715 \\
  ADH & 0.12  & 0.04 & 8.72 & 0.16  & 0.07 & 6.37\\
  AHD & 0.11  & 0.05 & 9.50 & 0.16  & 0.09 & 6.47 \\
\end{tabular}
\end{table}
% 

To obtain the forward and backward rate constants ($k$ and $k'$),
here we performed the fitting in different time region $0.2 < t < 2$ ps and $2 < t < 12$ ps, respectively.
We note that in the larger time region, i.e., $2 < t < 12$ ps, the value of HB lifetime $\tau_\text{HB}$ is larger than that in shorter time region, $0.2 < t < 2$ ps,
no matter for the bulk water or for the interface. A larger $\tau_\text{re}$ value means that the distance between a water molecule and another water molecule 
stays within $r_\text{OO}^c= 3.5$ \AA for a longer time. 
For the long time region, these values of the $k$ are comparable in magnitude to that obtained by Ref.\thinspace{\cite{Khaliullin2013}} 

%For AHD definition:The bulk water: 14.1572 ps; the water/vapor iterface: 12.7806 ps.
%For the water/vapor interface of pure water, we also calculated the constants $k$ and $k'$ by least square fit. 

\section{Instantaneous interfacial HB dynamics}

To study the interface HB dynamics, we first determine the instantaneous interface and then define the interfacial HB population operator. 
Based on these two definitions, we can derive the correlation functions and reaction rate constants, in each interface layer. 
Using these quantities we can discuss the change in the HB dynamics at the interface with the interface layer's thickness.

% One can uncomment if remove PERCENT
%====================================
%Based on the HB definition of water molecule pairs, we can also use least squares fitting to obtain the rate constant $k$, $k'$, 
%and the average lifetime $\tau_{HB}$ of the H-bonds. The results are shown in the Table \ref{tab:k_k_prime_128w_pure_2s}  to \ref{tab:k_k_prime_128w_pure_2u}.
%\begin{table}[htb]
%\centering
%\caption{\label{tab:k_k_prime_128w_pure_2s} 
%    The $k$ and $k'$ for the bulk water and the water/vapor interface. We carried on the long time region 0.2 ps $< t <$ 8 ps. 
%The unit for $k$ ($k'$) is ps$^{-1}$, and that for $\tau_{\text{HB}}$ ($=1/k$) is ps.} 
%\begin{tabular}{ccccccc}
% Criterion & $k$  (bulk) & $k'$ (bulk) & $\tau_{\text{HB}}$ (bulk) & $k$  (interf.) & $k'$ (interf.) & $\tau_{\text{HB}}$ (interf.)\\
%\hline
%  ADH & 0.14 & 0.28 & 7.16 & - & - & -  \\
%  AHD & 0.11 & 0.18 & 9.08 & - & -  &  -\\
%\end{tabular}
%\end{table}
%%
%\begin{table}[htb]
%\centering
%\caption{\label{tab:k_k_prime_128w_pure_2t} 
%    The $k$ and $k'$ for the bulk water and the water/vapor interface. We carried on the longer time region 0.2 ps $< t <$ 12 ps. 
%The unit for $k$ ($k'$) is ps$^{-1}$, and that for $\tau_{\text{HB}}$ ($=1/k$) is ps.} 
%\begin{tabular}{ccccccc}
% Criterion & $k$  (bulk) & $k'$ (bulk) & $\tau_{\text{HB}}$ (bulk) & $k$  (interf.) & $k'$ (interf.) & $\tau_{\text{HB}}$ (interf.)\\
%\hline
%  ADH & 0.10 & 0.17 & 9.59 & - & - & -  \\
%  AHD & 0.09 & 0.11 & 11.62 & - & -  &  -\\
%\end{tabular}
%\end{table}
%%
%\begin{table}[htb]
%\centering
%\caption{\label{tab:k_k_prime_128w_pure_2u} 
%    The $k$ and $k'$ for the bulk water and the water/vapor interface. We carried on the longer time region 1 ps $< t <$ 12 ps. 
%The unit for $k$ ($k'$) is ps$^{-1}$, and that for $\tau_{\text{HB}}$ ($=1/k$) is ps.} 
%\begin{tabular}{ccccccc}
% Criterion & $k$  (bulk) & $k'$ (bulk) & $\tau_{\text{HB}}$ (bulk) & $k$  (interf.) & $k'$ (interf.) & $\tau_{\text{HB}}$ (interf.)\\
%\hline
%  ADH & 0.06 & 0.06 & 17.96  & - & - & -  \\
%  AHD & 0.06 & 0.05 & 18.17 & - & -  & -\\
%\end{tabular}
%\end{table}

\FloatBarrier
\paragraph{Instantaneous Interfaces}
As Willard and Chandler mentioned, due to molecular motions, interfacial configurations
change with time, and the identity of molecules that lie at the interface also change with time, generally useful procedures for
identifying interfaces must accommodate these motions. \cite{Willard2010} 
To determine the instantaneous interface of the water/vapor system, we here adopted their proposed method based on spatial density.
The coarse-grained density at space-time point $\mathbf{r},t$ can be expressed as polynomial
\begin{eqnarray}
\bar{\rho}(\mathbf{r}, t)=\sum_{i} \phi(|\mathbf{r}-\mathbf{r}_{i}(t)|; \xi) 
\end{eqnarray}
where ${\mathbf{r}}_i(t)$ is the position of the $i$th particle at time $t$ and the sum is over all such particles, and 
\begin{eqnarray}
\phi(\mathbf{r};\xi)=(2 \pi \xi^{2})^{-3/ 2} \exp (-r^{2} / 2 \xi^{2}) 
\label{eq:gaussian_coarse_graining}
\end{eqnarray} 
is a normalized Gaussian functions for a 3-dimensional system, where $r$ is the magnitude of ${\mathbf r}$, and $\xi$ is the coarse-graining length.
Equation \ref{eq:gaussian_coarse_graining} is introduced to improve the accuracy of the interface, such that we can extend the domain and make it a single unicom,
i.e., no cavity exists in the domain.
With the parameter $\xi$ set, the interfaces can be defined to be the 2-dimensional manifold ${\mathbf r} = {\mathbf s}$ such that
\begin{eqnarray}
\bar\rho(\mathbf{s};t)= \rho_c 
\label{eq:rho_c}
\end{eqnarray} 
where $\rho_c$ is a reference density. This interface is a function of time as molecular configurations changes with time, that is 
${\mathbf s}(t) = {\mathbf s}(\{{\mathbf r}_i(t)\})$. 

%{Instantaneous Layering of the water/vapor interface} DELETED THE PARAGRAPH NAME
After the instantaneous surface is defined, we can define an interface layer for any non-uniform fluid system. 
Specifically, for the simulated water/vapor interface system in the cuboid simulation box, 
we can get another two-dimensional manifold ${\mathbf s}_0(t)$ by moving the surface ${\mathbf s}(t)$ determined above 
along the system's normal coordinate to a certain distance $d$ (two grey surfaces are shown in Fig.\thinspace\ref{fig:128w_itp_add_z_d_trimed_with_inner_layers}).
We use these two surfaces as the two boundaries of the interface. In other words, at any time point $t$, the volume between the two surfaces 
${\mathbf s}(t)$ and ${\mathbf s}_0(t)$ is defined as the instantaneous interface. 
Here, we use $d$ to denote the thickness of the interface. As we change the value of $d$, we can get interfaces with different thicknesses. 
Different values of $d$ give us different layering strategies for the interface system. 
See Fig.\thinspace\ref{fig:128w_itp_add_z_d_trimed_with_inner_layers} as an example.
\begin{figure}
\centering
\includegraphics [width=0.32\textwidth] {./diagrams/128w_itp_add_z_d_trimed_with_inner_layers}
\setlength{\abovecaptionskip}{0pt}
\caption{\label{fig:128w_itp_add_z_d_trimed_with_inner_layers}
A slab of water (128 water molecules are included) with the instantaneous interface represented as a blue mesh on the upper and lower phase boundary.
The normal is along the $z$-axis and the parameter $d$ is the thickness of the interfacial layer.
The grey surfaces are obtained by translating the interfaces to the inside of the system along the $z$-axis (or the opposite direction of the $z$-axis) by $d$.} 
%The box dimensions are $15.64 \times 15.64 \times 31.28$ \AA$^3$, and the slab is periodically replicated in the $x$, $y$ and $z$ directions. 
\end{figure}

Below we will combine the instantaneous interface and Luzar-Chandler's HB population operator \cite{AL96} to select the H-bonds 
in the interface. The dynamics of these H-bonds will vary with the thickness $d$ of the interface. By investigating the characteristics of HB dynamics
in these interfaces, we can obtain the dynamical characteristics of various solution interfaces. 
%As we will see later, this method can be extended to HB dynamics 
%in various environments, such as H-bonds around certain ions, in bulk water, etc.
%These different environments have a common feature: because the molecular configuration changes over time, the usual method first selects these molecules or molecular pairs, 
%and then determines the H-bonds in this special environment based on a HB criterion, and finally calculate the HB lifetimes or autocorrelation functions of 
%the HB population operators. 

\FloatBarrier
\paragraph{Interfacial hydrogen bond population} \label{IHBP}
Once we have determined the surface ${\mathbf s}(t)={\mathbf s}(\{{r}_i(t)\})$, we can define interfacial H-bonds.
Now we define the interface HB population operator $h^{(\text{s})}[{r}(t)]$ as follows:
It has a value 1 when the particular tagged molecular pair are H-bonded, \emph{and} both molecules are inside the instantaneous interface 
with a thickness $d$, and zero otherwise. 
The definition of  $h^{(\text{s})}[{r}(t)]$ is critical to help us to efficiently obtain the H-bonds' dynamic characteristics of 
the interfacial layer with a given thickness $d$. %Note that the definition of HB here is based on water molecule pairs or O-H pairs. 
In this paragraph, we only discuss H-bonds based on water molecule pairs. Starting from the H-bonds based on O-H pairs, the same analysis 
can also be done. 

Similar to the correlation function \CHB in Eq. \ref{eq:C_HB}, which describes the fluctuation of the general H-bonds,
we define the correlation function \CSHB that describes the fluctuation of the interfacial H-bonds: 
\begin{eqnarray}
C^\text{(s)}_{\text{HB}}(t)=\langle h^\text{(s)}(0)h^\text{(s)}(t) \rangle/\langle h^\text{(s)}\rangle
\label{eq:C_s_HB}.
\end{eqnarray}
%
Similarly, we define correlation functions 
\begin{eqnarray}
n^\text{(s)}(t)=\langle h^\text{(s)}(0)[1-h^s(t)]h^{\text{(d,s)}} \rangle/\langle h^\text{(s)}\rangle
\label{eq:n_s_HB},
\end{eqnarray}
and 
\begin{eqnarray}
k^\text{(s)}(t)= -\frac{dC^\text{(s)}_\text{HB}}{dt}
\label{eq:k_s_HB}.
\end{eqnarray}
Therefore, using these new correlation functions, we can determine the reaction rate constant of breaking and reforming and the lifetimes of interfacial H-bonding.
We will discuss the dependence of the correlation functions \CHB, \CSHB, and the reaction rates $k$ and $k'$ on the interface thickness $d$ in the next two paragraphs.
%
\FloatBarrier
\paragraph{$d$-dependence of \CSHB}
\begin{figure}[htb]
\centering
\includegraphics [width=0.60\textwidth] {./diagrams/128w_itp_pure_water_pair_c_ihb}
\setlength{\abovecaptionskip}{0pt}
\caption{\label{fig:128w_itp_pure_water_pair_c_ihb} 
The \CSHB for the instantaneous interfacial H-bonds with different thickness ($d$), 
as computed from the (a) ADH and (b) AHD criteria of H-bonds.} 
\end{figure}
For the water/vapor interface, we used two geometric criteria of H-bonds to calculate the \hbos and therefore correlation function \CSHB. 
The calculation results of the \CSHB are shown in Fig.\thinspace\ref{fig:128w_itp_pure_water_pair_c_ihb}.
%We find that the greater the thickness $d$ of the instantaneous interface is selected, 
%the slower the relaxation of the interface H-bonds. 
%When the thickness is greater than a certain thickness $d^c$ ( $\sim$ 3 \AA),
%the relaxation of H-bonds at the interface hardly changes.
We find that the HB dynamics is faster at the water/vapor interface when compared to bulk.
As $d$ increases, we get slower and slower HB dynamics. 
%When $d$ exceeds 3 \A, the interface's HB dynamics no longer changes with $d$. 
This feature means that as $d$ increases, the interface's HB dynamics will converge and eventually tend to the bulk water's hydrogen bond dynamics.
This behavior is independent of the HB definition as shown by the comparison of the results in panel a and b of Fig.\thinspace\ref{fig:128w_itp_pure_water_pair_c_ihb}.
%

For comparison, we also calculate the HB dynamics of water molecules in the interface obtained by selecting molecules 
located in the interface. We call this method Molecule Sampling (MS) (See Appendix \ref{ihb_and_selection} for details). 
In this method, we first select the molecules at the interface at each moment and then make a statistical
average of the calculated correlation functions.
Specifically, to determine which water molecules are located at the instantaneous interface, 
we sample at regular intervals, and then calculate the correlation function \CHB for the water molecules located in the interface and their a statistical average.
As the thickness $d$ changes, the \CHB in the interface will also change. 
Figure \ref{fig:128w_itp_pure_water_pair_c_ihb_scheme1} shows how the function \CHB changes with the thickness $d$.
The panel a and b use HB definition criterion ADH, and AHD, respectively.
Comparing Fig.\thinspace\ref{fig:128w_itp_pure_water_pair_c_ihb} and Fig.\thinspace\ref{fig:128w_itp_pure_water_pair_c_ihb_scheme1}, we see that
when we use the method of MS at the interface, the dependence of the correlation function \CHB
on the interface thickness is consistent with that of \CSHB for large $d$. 
Moreover, regardless of the AHD definition or the ADH definition of the HB, this conclusion is basically valid.
 
Beside the correlation functions \CHB or \CSHB in the interface, we will further examine the correlation 
functions \CHB, $n(t)$, $k(t)$ (\CSHB, $n^\text{(s)}(t)$, $k^\text{(s)}(t)$), and the rate constants $k$, $k'$ determined by them.
\begin{figure}[H]
\centering                                         
\includegraphics [width=0.6\textwidth] {./diagrams/128w_itp_pure_water_pair_c_ihb_scheme1}
\setlength{\abovecaptionskip}{0pt}
\caption{\label{fig:128w_itp_pure_water_pair_c_ihb_scheme1} 
The \CHB for the instantaneous interfacial H-bonds with different thickness $d$,  
as computed from the (a) ADH and (b) AHD criteria of H-bonds. 
These results are based on MS method. The sampling is performed every 4 ps.} 
%These results are based on selecting the water molecules in the instantaneous interface and averaging 
%the correlation functions of these water molecules. The sampling is performed every 4 ps.
\end{figure}

%[Plot the $k$ and $k'$ as functions of thickness $d$.]
\FloatBarrier
\paragraph{$d$-dependence of $k$ and $k'$} 
To find the reaction rate constants $k$ and $k'$, we also have two choices. we can start from the correlation functions \CSHB, $n^\text{(s)}(t)$ and $k^\text{(s)}(t)$;
We can also first select the water molecules at the instantaneous interface at each time point $t$, and start from the corresponding 
correlation functions \CHB, $n(t)$ and $k(t)$ of the H-bonds of these selected water molecules.
Figure \ref{fig:128w_itp_pure_water_pair_k_k_prime_ihb_both_schemes} compares the rate constants ($k$ and $k'$) 
and the lifetime $\tau_\text{HB}$ obtained by the two different methods, i.e., the Instantaneous Interfacial Hydrogen Bond (IHB) and MS methods. 
We see that, whether it is $k$, $k'$ or $\tau_\text{HB}$, their changing \emph{trend} with the thickness $d$ of the interface is only slightly affected by the calculation methods. 
To illustrate this point more clearly, we compare the $k$, $k'$ and $\tau_\text{HB}$ obtained under the two methods.
We listed more detailed data in Table\thinspace\ref{tab:k_k_prime_tau_128w_pure_ihb_ADH} to \ref{tab:k_k_prime_tau_128w_pure_ihb_AHD}.
%

As we can see from Fig.\thinspace\ref{fig:128w_itp_pure_water_pair_k_k_prime_ihb_both_schemes}, 
when the thickness is large enough ($d_0 \sim 4$ \AA), these two constants agree well quantitatively. 
This result shows that the two extreme statistical methods (see Appendix \ref{ihb_and_selection}) 
for the HB dynamics of the interface did not produce much difference for the time scale (10$^2$ ps) 
and the scale ( 10$^2$ \AA ) of the simulation box.
\begin{figure}[H]
\centering
\includegraphics [width=0.6\textwidth] {./diagrams/128w_itp_pure_water_pair_k_k_prime_ihb_both_schemes}
\setlength{\abovecaptionskip}{0pt}
\caption{\label{fig:128w_itp_pure_water_pair_k_k_prime_ihb_both_schemes}Dependence of (a) the reaction rate constants $k$ and $k'$ 
and (b) the HB lifetime $\tau_\text{HB}$ on the interface thickness $d$, obtained by the IHB and MS methods, respectively.
The corresponding $k$, $k'$ and $\tau_\text{HB}$ in the bulk water are also drawn with dashed lines as a reference.
In sub-figure a, the $k$ of bulk water is represented by a \emph{black dashed} line, and the $k'$ is represented by a \emph{blue dashed} line;
in sub-figure b, the $\tau_\text{HB}$ of bulk water is represented by a \emph{black dashed} line.
The ADH criterion of H-bonds is used and the least square fits are carried on the time 
region 0.2 ps $< t <$ 12 ps.}
\end{figure}

We also found that when we focus on the molecules in the interface whose thickness is less than $d_0$, 
the values of the reaction rate constants slightly depend on the method we use. 
That is, the $k$ obtained by the IHB method is relatively larger than by the MS method, and $k'$ is smaller. 
Since $\tau_\text{HB} = 1/k$, large $k$ directly leads to a relatively shorter HB lifetime using the IHB method. 
This result is related to our definition of IHB, and it is the same as our expectations: 
The definition of interfacial H-bonds (or \hbos) makes the HB break rate 
on the interface artificially increased. At the same time, the MS method retains the original rate constant of H-bonds, 
but it may include the contribution of bulk water molecules to the rate constant. 
That is why the MS method slightly underestimate the $k$. 

In Fig.\thinspace\ref{fig:128w_itp_pure_water_pair_k_k_prime_ihb_both_schemes}, the $k$, $k'$ and $\tau_\text{HB}$ for the \emph{bulk} water 
are also drawn with dashed lines as a reference.
Comparing the above-mentioned physical quantities in the pure water interface and bulk water, 
we found that when the interface thickness $d>d_0$, 
no matter which statistical method is used, the value of the calculated reaction rate constants of the interface water is \emph{greater} than that in the bulk water. 
Therefore, since the HB lifetime can be calculated by $\tau_\text{HB} = 1/k$, the value of $\tau_\text{HB}$ in interface water is smaller than that in bulk water.

Furthermore, we find from Fig.\thinspace\ref{fig:128w_itp_pure_water_pair_k_k_prime_ihb_both_schemes} that as the interface thickness $d$ increases, 
the values of $k$ and $k'$ also tend to the values of rates in the bulk water at the same condition.
These results are obtained by the least squares method in the same interval (0.2--12 ps). This verifies that the IHB method 
can get as good results as the method of molecule selection  when $d>d_0$. 
Because the IHB is easier to operate, this method can calculate the HB dynamics and thus HB lifetime on the interface 
when the $d>d_0$ (in this case, $d_0 \sim 4$ \A \ or the size of 2--3 water molecules).
We also noticed that the selection of water molecules and the statistical averaging depend on our sampling rate on the trajectory of the simulated system, 
and the IHB method does not require such sampling. Therefore, the IHB method is more convenient method to determine the HBD of instantaneous interface.

Finally, because the real HB dynamical properties of interface molecules are between the results of the above two methods, 
we can approximate the interfacial HB dynamics, by either the IHB and the MS method if the thickness of the interface is large enough. 

%In summary, if we study the dynamics of H-bonds in a very thin interface, we can use the method of molecular selection, 
%because the H-bonds obtained in this way are not artificially broken, and if the interface is thick enough  
%(see Fig.\thinspace\ref{fig:128w_itp_pure_water_pair_k_k_prime_ihb_both_schemes}a), then we can use the IHB method, because it can automatically define which H-bonds come 
%from the interface without the need to select the molecules in the interface layer.

%
\begin{table}[H]%[htb]
\centering
\caption{\label{tab:k_k_prime_tau_128w_pure_ihb_ADH} 
    The $k$ and $k'$ for the interfacial HB dynamics of the water/vapor interface (by the method of IHB, with ADH criteria). 
We carried on the longer time region 0.2 ps $< t <$ 12 ps. 
}
%The unit for $k$ ($k'$) is ps$^{-1}$, and for $\tau_{\text{HB}}$ ($=1/k$) is ps. 
%The parameter values and units are the same below. 
\begin{tabular}{cccc}
 Thickness & $k$ & $k'$ & $\tau_{\text{HB}} (=1/k)$ \\
\hline
  1.0 & 0.653 & 0.080 & 1.53  \\
  2.0 & 0.261 & 0.133 & 3.83  \\
  3.0 & 0.168 & 0.104 & 5.94  \\
  4.0 & 0.148 & 0.092 & 6.76  \\
  5.0 & 0.147 & 0.087 & 6.81  \\
  6.0 & 0.139 & 0.087 & 7.17  \\
\end{tabular}
\end{table}
\begin{table}[htb]
\centering
\caption{\label{tab:k_k_prime_tau_128w_pure_ihb_AHD} 
    The $k$ and $k'$ for the interfacial HB dynamics of the water/vapor interface (by the method of IHB, with AHD criteria).} 
\begin{tabular}{cccc}
 Thickness & $k$ & $k'$ & $\tau_{\text{HB}} (=1/k)$ \\
\hline
  1.0 & 0.661 & 0.080 & 1.51  \\
  2.0 & 0.265 & 0.133 & 3.77  \\
  3.0 & 0.172 & 0.102 & 5.82  \\
  4.0 & 0.148 & 0.090 & 6.74  \\
  5.0 & 0.149 & 0.084 & 6.72  \\
  6.0 & 0.144 & 0.078 & 6.93  \\
\end{tabular}
\end{table}

\begin{table}[H]
\centering
\caption{\label{tab:k_k_prime_tau_128w_pure_ihb_scheme1_ADH} 
    The $k$ and $k'$ for the interfacial HB dynamics of the water/vapor interface (by the method of molecule selection, with ADH criteria).} 
\begin{tabular}{cccc}
 Thickness & $k$ & $k'$ & $\tau_{\text{HB}} (=1/k)$ \\
\hline
  1.0 & 0.526 & 0.072 & 1.90  \\
  2.0 & 0.246 & 0.158 & 4.07  \\
  3.0 & 0.160 & 0.114 & 6.26  \\
  4.0 & 0.140 & 0.097 & 7.15  \\
  5.0 & 0.138 & 0.090 & 7.24  \\
  6.0 & 0.133 & 0.085 & 7.49  \\
\end{tabular}
\end{table}
%  6.0 & 0.125 & 0.080 & 8.00  \\
%  7.0 & 0.133 & 0.085 & 7.49  \\
\begin{table}[H]
\centering
\caption{\label{tab:k_k_prime_tau_128w_pure_ihb_AHD} 
    The $k$ and $k'$ for the interfacial HB dynamics of the water/vapor interface (by the method of molecule selection, with AHD criteria).} 
\begin{tabular}{cccc}
 Thickness & $k$ & $k'$ & $\tau_{\text{HB}} (=1/k)$ \\
\hline
  1.0 & 0.610 & 0.083 & 1.64  \\
  2.0 & 0.235 & 0.142 & 4.62  \\
  3.0 & 0.138 & 0.102 & 7.22  \\
  4.0 & 0.141 & 0.098 & 7.07  \\
  5.0 & 0.120 & 0.078 & 8.40  \\
  6.0 & 0.119 & 0.071 & 8.39  \\
\end{tabular}
\end{table}
%  6.0 & 0.117 & 0.072 & 8.58  \\
%  7.0 & 0.119 & 0.071 & 8.39  \\

%\paragraph{Experiments on HB dynamics}
%An important  structural characteristic of the H-bonded network is the average number of H-bonds per molecule, $\langle h_{i,j}\rangle$. \cite{Chowdhary2008} 
%For bulk water systems, we find that in the DFTMD simulations the average number of H-bonds in the bulk phase is $\sim$ 4.35 which is slightly
%(higher) than the usual estimate of 3.4 (interface system) for SPC/E water.

%TODO
%For interfacial systems of neat water, we find the average number
%of hydrogen bonds is 3.XX which is slightly
%(lower/higher) than the usual estimate of 3.4 for SPC/E water. \cite{Chowdhary2008}

%%===============================================
%\section{Rotational Anisotropy Decay of Water at the Interface of Alkali-Iodine Solutions}\label{CHAPETR_AD}
%%===============================================
%Using the transition dipole auto-correlation function, 
%we determined the rotational anisotropy decay and therefore the OH-stretch relaxation at water/vapor interface of alkali iodide solutions.
%%The effects of ion environment on structure and dynamics of water are obtained by comparing the second-order Legendre polynomial, i.e.,  $P_2(x)=\frac{1}{2}(3x^2-1)$,  orientational correlation function of the transition dipole.
%The anisotropy decay can be determined from experimental signal in two different polarization configurations---parallel and perpendicular polarizations, by 
%\begin{equation}
%        R(t)=\frac{S_{\parallel}(t)-S_{\perp}(t)}{S_{\parallel}(t)+2S_{\perp}(t)}
%\label{eq:ad}
%\end{equation}
%where $t$ is the time between pump and probe laser pulses.  The anisotropy decay can also be obtained by simulations, 
%and calculated by the third-order response functions $R(t)$. \cite{Jansen10,Jansen06}
%%
%%In the first shell with a radius 3 \A, the entropy difference betweem the \Li shell and \Na shell,
%%$\Delta S=k_B\text{ln}\frac{\Omega_\text{Na}}{\Omega_\text{Li}}=k_B\text{ln}\frac{n_\text{Na}/V_\text{Na}}{n_\text{Li}/V_\text{Li}} =k_B\text{ln}1.05$.
%%
%%\paragraph{Probability Distribution of Ions}
%%The probability distribution, shown in Fig.~\ref{fig: prob_124_LiI_Sans_double_axis}, of the ions in the water/vapor interface of LiI and NaI solutions with repect to the depth of the ions in the solutions 
%%indicates that the \I ions prefer to staying at the topmost layer of surface of solutions.
%%(molar concentration: 0.9 M, temperature: 330 K) 
%%It shows that \I ions tend to the surface of the solutions, while \Na and \Li tend to stay in the bulk. This result is consistent with the calculations from Ishiyama and Morita\cite{TI07,TI14}.
%The orientational anisotropy $C_2(t)$ is given by the rotational time-correlation function 
%\begin{equation}
%C_2(t)=\langle P_2(\hat{u}(0)\cdot\hat{u}(t)) \rangle,
%\label{eq:tcf2}
%\end{equation}
%where $\hat{u}(t)$ is the time dependent unit vector of the transition dipole, $P_2(x)$ is the second Legendre polynomial, and $\langle \cdots \rangle$ indicate 
%equilibrium ensemble average. \cite{Corcelli05,LinYS2010} %\cite{2010Lin} % angular brackets
%
%The anisotropy decay $C_2(t)$ for the water/vapor interface of LiI solution is shown in Fig.\thinspace\ref{fig:c2_2LiI_16_inset}.
%This function decays faster than that of neat water, indicating that H-bonds
%at the interfaces of alkali-iodine solutions reorient faster than in neat water. The inset shows the first 0.4 ps of $C_2(t)$, from which we see a 
%quick change during the first $\sim 0.1$ ps primarily due to librations.
%%
%\begin{figure}[h]
%\centering
%\includegraphics [width=0.36\textwidth] {./diagrams/c2_2LiI_16_inset} 
%\setlength{\abovecaptionskip}{0pt}
%  \caption{\label{fig:c2_2LiI_16_inset} The time dependence of the $C_2(t)$ of OH bonds at the water/vapor interfaces of 0.9 M LiI solution 
%    and of neat water (dashed line) at 330 K, calculated by DFTMD simulations.} 
%    %The water/vapor interface of neat water is modeled 
%    %with a slab made of 121 water molecules in a simulation box of size $15.6 \times 15.6 \times 31.0$ \A$^3$.
%\end{figure}
%%
%We also calculated the $C_2(t)$ for the interface of other alkali-iodine solutions LiI and KI. 
%The results of $C_2(t)$ for the water/vapor interfaces of these solutions are shown in Fig.\thinspace\ref{fig:c2_2KI_2NaI_2LiI_16}.
%In all the cases $C_2(t)$ decays faster than in neat water, indicating that H-bonds
%at the interfaces of the three alkali-iodine solutions are orientated faster than that of neat water.
%They show that \I ions can accelerate the dynamics of molecular reorientation of water molecules at interfaces.   
%
%%
%\begin{figure}[htbp]
%\centering
%\includegraphics [width=0.36\textwidth] {./diagrams/c2_2KI_2NaI_2LiI_16} 
%\setlength{\abovecaptionskip}{0pt}
%  \caption{\label{fig:c2_2KI_2NaI_2LiI_16} The time dependence of the $C_2(t)$ of OH bonds in water molecules at the water/vapor 
%  interface of 0.9 M alkali-iodine solutions and of neat water (dashed line) at 330 K, calculated by DFTMD simulations.}
%\end{figure} 
%
%We have obtained non-single-exponential kinetics for the rotation of water molecules both at the surface 
%and in bulk water (Appendix \ref{single_exp}).
%%This result is true for water molecules bound to ions. 
%Therefore, the rotational motion of water molecules are not simply characterized by well-defined rate constants. 
%%Then the problem is to understand the kinetics.
%Similar non-single-exponential kinetics is also obtained in the HB kinetics
%in liquid water \cite{AL96,Dirama05} and in the time variation of the average frequency shifts of the 
%remaining modes after excitation in hole burning technique \cite{Rey2002,Moller2004} and using BLYP functional. \cite{Bankura2014}
%Luzar and Chandler interpreted 
%the non-single-exponential kinetics as the result of an interplay between 
%diffusion and HB dynamics. \cite{AL96} 
%We can understand the non-single-exponential kinetics of rotational 
%anisotropy decay by fitting the rotational anisotropy decay by a 
%biexponential function.
%
%To obtain the effects of diffusion and HB decay of water molecules
%in solutions respectively, we assume that there are two independent 
%decays in the process of an anisotropy decay. 
%Therefore, the $C_2(t)$ has the form \cite{TanHS05}
%\begin{equation}
%C_2(t)=A_1e^{-\kappa_1 t} +A_2e^{-\kappa_2 t},
%\label{eq:tcf3}
%\end{equation}
%where $A_i$ are constants and $\kappa_i$ are decay rates ($i=1, 2$). 
%The time constants and amplitudes of the biexponentials fits for 
%the $C_2(t)$ are listed in Table ~\ref{tab:2LiI_c2_biexp} and Table ~\ref{tab:2NaI_c2_biexp}.
%The biexponential fit is very close to the calculated $C_2(t)$, which can be seen in Fig.\thinspace\ref{fig:2LiI-124w_c2_fit_5ps_biexp} (compare Fig.\thinspace\ref{fig:2LiI-124w_c2_fit_5_single-exp}).
%%
%\begin{table}[hbt]
%\centering
%\caption{\label{tab:2LiI_c2_biexp}%
%	Biexponential fitting (5 ps) of the $C_2(t)$ for water molecules in 0.9 M LiI solution.}
%%\begin{ruledtabular}
%\begin{tabular}{lccccc}
%water molecules & $A_1$  & $\kappa_1$ (THz) & $A_2$ & $\kappa_2$ (THz) \\
%\hline
%I$^-$-shell & 0.44 & 0.25 & 0.39 & 0.26\\
%Li$^+$-shell & 0.88 & 0.07 & 0.07 & 8.24\\
%bulk & 0.84 & 0.11 & 0.09 & 4.88 \\
%surface & 0.73 & 0.27 & 0.22 & 13.47 \\
%\end{tabular}
%%\end{ruledtabular}
%\end{table}
%%--
%
%\begin{table}
%\centering
%  \caption{\label{tab:2NaI_c2_biexp}%
%	Biexponential fitting (5 ps) of the $C_2(t)$ for water molecules in 0.9 M NaI solution.}
%  \begin{tabular}{lccccc}
%  water molecules & $A_1$  & $\kappa_1$ (THz) & $A_2$ & $\kappa_2$ (THz) \\
%  \hline
%  I$^-$-shell & 0.86 & 0.14 & 0.08 &9.86 \\
%  Na$^+$-shell & 0.71 & 0.06 & 0.18 &0.79 \\
%  bulk & 0.81 & 0.06 & 0.10 & 1.25 \\
%  surface & 0.77 & 0.11 & 0.13 & 2.31 \\
%  \end{tabular}
%\end{table}
%%
%%图
%\begin{figure}[htbp]
%\centering
%\includegraphics [width=0.60\textwidth] {./diagrams/2LiI-124w_c2_fit_5_biexp} 
%  \caption{\label{fig:2LiI-124w_c2_fit_5ps_biexp} The time dependence of the $C_2(t)$ of OH bonds 
%  in water molecules at the water/vapor interface of LiI solution.}
%\end{figure} 
%%
%%[Notes: The 63-water-slab models is listed here as a reference. The number of water molecules is small; The data for KI/vapor and LiI/vapor interfaces come from  KI\_16 and LiI\_16 systems.  
%%Water(63) &0.831$\pm(1\times10^{-4})$ &  0.08760 $\pm(2\times 10^{-5})$&0.100$\pm(2\times10^{-4})$ & 1.029 $\pm(4\times10^{-3})$  \\ ]
%%
%%\begin{figure}[htbp]
%%\centering
%%\includegraphics [width=0.4 \textwidth] {./diagrams/c2_121-pure_2KI_2LiI_16_inset_fit_biexp} 
%%\setlength{\abovecaptionskip}{10pt}
%%\caption{\label{fig:c2_121-pure_2KI_2LiI_16_inset_fit_biexp} The fitted and calculated anisotropy decay of OH bonds in water molecules in LiI solution/vapor interface (red), LiI solution/vapor interface (blue) and neat water/vapor interface (black). The corresponding fitted functions are denoted by dashed lines. The concentration of LiI and KI solution is 0.9 M.}
%%\end{figure} 
%
%Then we considered the effect of ion species in solutions on the anisotropy decay of water molecules.
%From Table \ref{tab:2LiI_c2_biexp} and Table \ref{tab:2NaI_c2_biexp}, we find that 
%for both LiI and NaI solutions, there are two decay processes in the dynamics --- amplitude $\sim 1$,
%decay constant $\sim$ 0.1 THz, and for the other describe the initial fast decay 
%of the anisotropy, with amplitude $\sim 0.1$, decay constant $\sim$ (1--10) THz, 
%due to the inertial-librational motion preceding the orientational diffusion.
%That is, two decay processes exist in the dynamics of water molecules 
%at the water/vapor interfaces of alkali-iodine solutions. 
%%The one describe the initial fast decay of the anisotropy, 
%%with amplitude $\sim$ 0.1, decay constant $\sim$ (1--10) THz,
%%results from the inertial-librational motion preceding the orientational diffusion.
%%%
%%\begin{table}[H]
%%\centering
%%\caption{\label{tab:fitting_c2_for_each_type_of_water}%
%%  Biexponentially fitting (2 ps) of the $C_2(t)$ for different types of water molecules at the water/vapor interface of LiI solutions.}
%%\begin{tabular}{lccccc}
%%water molecules & $A_1$  & $\kappa_1$ (THz) & $A_2$ & $\kappa_2$ (THz) \\
%%\hline
%%$DDAA$ & 0.85 & 0.25   & 0.10 & 16.0\\
%%$DD'AA$ & 0.89 & 0.14  & 0.06 & 14.1 \\
%%$D'AA$ & 0.38 & 0.99 & 0.38 & 0.99 \\
%%\end{tabular}
%%\end{table}
%%%
%%\begin{table}[H] %[!hbtp]
%%\centering
%%\caption{\label{tab:table_CoordNo}%
%%The coordination number of the atoms in LiI (NaI) solutions.}
%%\begin{tabular}{lccc}
%%name & radius of the first shell (\AA) & coordination number \\
%%\hline
%%$n_\text{I-H}(\text{LiI})$ & 3.3 & 5.5 \\
%%$n_\text{I-H}(\text{NaI)}$ & 3.3 & 5.1 \\
%%$n_\text{I-O}(\text{LiI)}$ & 4.3 & 5.8 \\
%%$n_\text{I-O}(\text{NaI)}$ & 4.3 & 6.0 \\
%%$n_\text{Li-O}(\text{LiI)}$ & 3.0 & 4.0 \\
%%$n_\text{Na-O}(\text{NaI)}$ & 3.5 & 6.0 
%%\end{tabular}
%%\end{table}

\section{Summary}
In this chapter, we have introduced the HB population operator and various correlation functions \CHB, \SHB, $n(t)$, and $k(t)$ based on this operator. 
The HB dynamics calculations for the bulk water and the interface are based on these correlation functions. 
The \CHB describes the relaxation of the H-bonds. The \SHB gives the average lifetime $\langle \tau_a \rangle$ of the continuous H-bonds. 
The $n(r)$ describes the probability that the HB is broken but still separated by cutoff radius $r_{OO}^\text{c}$. 
Based on DFTMD simulations, starting from the functions \CHB, $n(t)$, and $k(t)$, 
we have calculated the reaction rate constants $k$ and $k'$ for HB rupture and regeneration for bulk water and the water/vapor interface.

We have studied the HB dynamics for instantaneous interfaces using two different statistical methods, the IHB and MS.
We find that as the interface thickness increases, 
the HB reaction rate constants tend to the rate constants in the bulk water.
From the above results for water/vapor interface, we conclude that from the perspective of HB dynamics,
the thickness of the air-water interface of water is about 3 \A. This value is smaller than that obtained from the SFG spectra 
(Ref. paragraph \thinspace\ref{sfg_lino3_interface}), and this result has reference significance for our study of the influence of ions on the H-bonds 
outside the solvation shell of ions. 
As the presence of ions in the solution affects the HB network, we will analyze these effects in the next chapter.

