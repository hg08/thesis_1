\chapter{Hydrogen Bond Dynamics of Water/Vapor Interfaces }\label{CHAPTER_HB}
%The influence of ions propensity for the aqueous surface on the water's HB network are also of special interest to the atmospheric chemistry community.
In this chapter, we explore the effects of nitrate ions, iodide ions and alkali metal cations 
on the HB dynamics at the water/vapor interface of alkali nitrate solutions and alkali
iodine solutions, by DFTMD 
and we provide a microscopic interpretation of recent experimental results. [\cite{HuaWei2014}]
%\cite{TianCS09, TianCS11,DV13}

\section{Definitions of Hydrogen Bond Population and Correlation Functions}
Luzar and Chandler [\cite{AL96}] have elucidated the HB dynamics of pure water, and
subsequently such analysis has been also extended to explore the HB dynamics
in complex situations, e.g., electrolytes, [\cite{AC00}] protein and  micellar surfaces. [\cite{SP05}]
There are temporal, geometric and energetic criteria to define HB. Here we use the geometric one.
Two water molecules are H-bonded only if their interoxygen distance between of specific tagged pair of water molecules 
is less than 3.5 \A, OH intermolecular distance less than $R_{\text{OH}}=2.45$ \AA and 
the O-H$\cdots$O angle is less than $\pi/6$. [\cite{AKS86,JT90,SB02}] 
The value 3.5 \AA is the first-minimum position of the O--O RDF of water. [\cite{Sciortino1989}]   
The distance criteria of $R_{\text{OH}}$  was determined from the first minimum in the O--H RDF for SPC water. [\cite{HJCB81}]
We will use three correlation functions to describe the HB dynamics of water/vapor interfaces of solutions,
the HB population correlation function \CHB, the survival probability \SHB and the reactive flux $k(t)$. [\cite{DCR83}]

\paragraph{HB Population Auto-Correlation Function}
We use the correlation function \CHB to describe the structural relaxation of H-bonds: \cite{AL96,AC00}
\begin{eqnarray}
C_{\text{HB}}(t)=\langle h(0)h(t) \rangle/\langle h\rangle
\label{eq:C_HB},
\end{eqnarray}
where $\langle \cdots\rangle$ denotes an average over all molecular pairs.
The value of $h$ is 1 when the specific tagged pair of molecules is H-bonded, and is 0 otherwise.
The $\langle h\rangle$ is the probability that a pair of randomly chosen water molecules in the system is
H-bonded at any time $t$. As examples, the dynamics of the interoxygen distance $r_{\text{OO}}(t)$, 
the cosine of O$-$H$\cdots$O angle cos$\phi(t)$  
and the $h(t)$ for a HB in a water cluster is displayed in Fig.\space\ref{fig:Ex30ps_hb},respectively.
%-------------------
\begin{figure}[hbtp]
\centering
\includegraphics [width=0.6\textwidth] {./diagrams/Ex30ps_hb}
\setlength{\abovecaptionskip}{0pt}
\caption{\label{fig:Ex30ps_hb}The dynamics of $r_{\text{OO}}(t)$ (top), cos$\phi(t)$ (middle), 
  and $h(t)$ (bottom) for a HB in a water cluster. The dashed lines show the interoxygen distance 
  boundary $r^{\text{c}}_{\text{OO}}$=3.5 \AA (top)  and criterion of cosine of O$-$H$\cdots$O angle cos$\phi^{\text{c}}$ 
  with $\phi^{\text{c}}$=$\pi/6$, respectively.}
\end{figure} 
%-------------------

The function \CHB is interpreted as the probability that the specific HB is intact at time  $t$, 
if it was intact at time zero. 
In a large system that consist of many water molecules, the probability that a specific pair of water molecules are H-bonded is extremely small. 
Therefore, the \CHB relaxes to zero, when $t$ is large enough. 
The \CHB measures correlation in $h(t)$ independent of any possible bond breaking events. 
It is one of the intermittent HB correlation functions, introduced by Rapaport, [\cite{DCR83}] 
and can be studied by a continuous function, probability densities.

%
A new HB population $h^{(d)}(t)$ was also defined to obviate the distortion of real HB dynamics
due to the above geometric definition. [\cite{Sciortino1989,AC00}]
The $h^{(d)}(t)$ is 1 when the interoxygen distance of a particular tagged pair of water molecules is less than 3.5 \AA at time $t$ and 0 otherwise. 
The function 
\begin{eqnarray}
  C^{(d)}_{\text{HB}}(t)=\langle h(0)h^{(d)}(t) \rangle/\langle h\rangle
\label{eq:C_HB_d},
\end{eqnarray}
is the probability that the specific two water molecules are located in reformable region ($r_{\text{OO}} < 3.5$ \AA) at time $t$,
if they were H-Bonded at time zero. 

%===============================
\paragraph{Survival Probability}
%===============================
Another scheme to describe HB dynamics is the survival probability [\cite{AC00}] for a newly generated HB.
%The probability densities
It is defined as
\begin{eqnarray}
S_{\text{HB}}(t)=\langle h(0)H(t) \rangle/\langle h\rangle 
\label{eq:S_HB},
\end{eqnarray}
where $H(t)=1$ if the tagged pair of molecules, remains continuously H-bonded till time $t$ 
and 0 otherwise.  It describes the probability that an initially H-bonded molecular pair 
remains bonded at all times up to $t$. [\cite{Chowdhuri2006}]

The average lifetime of H-bonds $\tau_{\text{HB}}$ is calculated by the integration of \SHB over $t$). In practice, we have 
\begin{eqnarray}
  \tau_{\text{HB}} &=& \int_0^{t_{\text{max}}} dt S_{\text{HB}}(t),
\label{eq:calculate_hb_lifetime}
\end{eqnarray}
where $t_{\text{max}}$ is the maximum of the time interval, i.e., $S_{\text{HB}}(t)\approx 0$, for $t > t_{\text{max}}$.
%
The time derivative of \SHB
\begin{eqnarray}
P(t) &=& -\text{d}S_{\text{HB}}/\text{d}t
\label{eq:P_of_t}
\end{eqnarray}
is called probability distribution of HB lifetime. [\cite{Sciortino1990prl,FWS99}]
In terms of $h$, the probability distribution can be expressed as
\begin{eqnarray}
P(t) &=& \frac{\langle [1-h(0)]\delta [t-\int_0^t h(t')dt'][1-h(t)]\rangle}{\langle 1-h(0)\rangle},
\label{eq:P_of_t}
\end{eqnarray}
where $\langle \rangle$ denotes the average over all molecular pairs which are starting to H-bonded at time $t$.
%\paragraph{Average HB Lifetime $\tau_{\text{HB}}$} %\cite{HAK08}
%Like in water, librational motions of water molecules cause an rupturing and reforming of a H bond on a time scale of 60 fs.\cite{SHC86}

%\paragraph{Relation Between HB Dynamics and Anisotropy Decay}
%It is interesting to relate the HB kinetics with rotational dynamics (anisotropy decay) of single water molecules.\cite{HX01}
%
\paragraph{Reactive Flux $k(t)$} 
Calculating the reactive flux HB correlation functions and determine the rate constant ($1/\tau_{\text{HB}}$), is a more rigorous way to obtain the nature of H-bonds at water/vapor interfaces. [\cite{AL00}]
The rate of relaxation to equilibrium is characterized by the reactive flux correlation function, 
\begin{eqnarray}
k(t) = -\text{d}C_{\text{HB}}/\text{d}t=\langle j(0)[1-h(t)]\rangle/\langle h\rangle,
\label{eq:k}
\end{eqnarray}
where 
$j(0)=-\text{d}h/\text{d}t|_{t=0}$ 
is the integrated flux departing the HB configuration space at time $t=0$, and
$1-h(t)$ describes the breaking of a HB at time $t$ after its formation at time $t=0$.
The $k(t)$ quantifies the rate that an initially present HB breaks at time $t$, 
independent of possible breaking and reforming events in the interval from 0 to $t$.
Therefore, the reactive flux $k(t)$ measures the effective decay rate of an 
initial set of H-bonds. [\cite{FWS00, DC87}]

For bulk neat water, there exists a $\sim 0.2$-ps transient period,
during which the $k(t)$ quickly changes from its initial value. [\cite{FWS00}]
However, at longer times, the $k(t)$ is independent of the HB definitions.
Therefore, the long time decay reflects the general properties of water solutions.
We assume that each HB acts independently of other H-bonds, [\cite{AL96,AL00}] 
and due to detailed balance condition, we can obtain 
\begin{eqnarray}
  \tau_{\text{HB}} &=& \frac{1- \langle h\rangle}{k},
\label{eq:rate}
\end{eqnarray}
where $k$ is the rate constant of breaking an H bond (forward rate constant). [\cite{Chandler1978,Chandler1986}] 
The $k$ is related to the average HB lifetime by $\tau_{\text{HB}}=1/k$.

% This paragraph is just an approximation:
%For dynamical trajectories which are longer than the initial time decay ($t > t_{\text{trans}}\sim 0.2$ ps),
%we have
%\begin{eqnarray}
%k(t)&=&\langle j(0)[1-h(t)]\rangle/\langle h\rangle \sim ke^{-k t}.
%\label{eq:k_2}
%\end{eqnarray}

%\paragraph{Mean First Passage Time Probability Densities}
%From Fig.\space\ref{fig:124_2NaI_hbacf_testing}, we can see that the calculation of \SHB converges very well in the time scale of 5 ps. 
%\begin{figure}[H]
%\centering
%\includegraphics [width=0.5\textwidth] {./diagrams/124_2NaI_hbacf_testing}
%\setlength{\abovecaptionskip}{20pt}
%\caption{\label{fig:124_2NaI_hbacf_testing} The function \SHB of water--water H-bonds at interface of 0.9 M NaI solution at 330 K, for different pieces of 5-ps trajectories. It shows that the results are consistent to each other pretty well. }
%\end{figure}
%\begin{figure}[H]
%\centering
%\includegraphics [width=0.5\textwidth] {./diagrams/LiI_hbacf_s_m_330K} %fig.5.7
%\setlength{\abovecaptionskip}{20pt}
%\caption{\label{fig:LiI_hbacf_s_m_330K}The functions \SHB of water--water H-bonds at interface of LiI solution at 330 K. 
%It showns that the lifetime of H-bonds decrease as the concentration of LiI solution increase.
%Inset represents the short time behavior of the logarithm of the correlation function.}
%\end{figure}
\FloatBarrier
\section{Environment Effects on Hydrogen Bond Dynamics}
\paragraph{Effects of Water/Vapor Interface}
For the water/vapor interface of neat water, we focus on the reactive flux HB correlation function $k(t)$, 
 which had been used in the study of HB dynamics of liquid water. [\cite{AL96}]
The $k(t)$ calculated from the positional trajectory of water molecules in DFTMD simulations, is reported in Fig.\space\ref{fig:121}. 
In the case of water/vapor interface, the $k(t)$ quickly changes from its initial value also on a time scale of less than 0.2 ps. 
Beyond this transient period, the $k(t)$ decays to zero monotonically, and the slop of ln$k(t)$ increases monotonically with $t$. 
These two properties were also found for bulk water using the SPC water model by Luzar and Chandler. [\cite{AL96}]
This log-log plot of the $k(t)$ shows that, as in the case of liquid water, this decay behaviour does not coincide with a power-law decay for water/vapor interface of neat water.
This result is also the same as that of the classical molecular simulation of pure water. [\cite{AL96b,Luzar1996}]
%
\begin{figure}[htpb]
\centering
\includegraphics [width=0.5\textwidth] {./diagrams/121}
\setlength{\abovecaptionskip}{0pt}
  \caption{\label{fig:121}The time dependence of the $k(t)$ for the water/vapor interface of neat water, calculated by DFTMD simulations.
  The inset shows the log-log plot of the $k(t)$.}
\end{figure}
\begin{figure}[hbtp]
\centering
\includegraphics [width=0.5\textwidth] {./diagrams/118_2lii_log} 
\setlength{\abovecaptionskip}{0pt}
\caption{\label{fig:118_2lii_log}The log-log plot of the $k(t)$ for water/vapor interface of 0.9 M LiI solution.}
\end{figure}
%121_neat_rfhb-log.eps
\begin{figure}[H]
\centering
\includegraphics [width=1.0\textwidth] {./diagrams/hbrf_4pl}
\setlength{\abovecaptionskip}{0pt}
  \caption{\label{fig:hbrf_4pl}The time dependence of the $k(t)$ for the water/vapor interfaces 
  with different thickness $d$ ($d$= 2, 3, 4 and 5 \AA) of four interface models calculated by DFTMD simulations. 
  The value of $h(t)$ is calculated every 0.1 ps. 
  (a): neat water; (b): 0.9 M LiI solution; (c): 0.9 M NaI solution; (d): 0.9 M KI solution.} 
\end{figure} 
For the water/vapor interface of alkali-iodine solutions, the $k(t)$ is also calculated. 
The result for the interface of 0.9 M LiI solution is shown in Fig.\thinspace\ref{fig:hbrf_4pl} (b). 
The log-log plot of $k(t)$ is not a straight line, indicating that,
for water/vapor interface of the LiI solution, this decay does not coincide with a power-law decay,
neither.

{\color{red}[Q3: A conclusion is missing. what is the diff between neat water interface and interface with ions?]}
{\color{blue} We focus on the value of $k(t)$ when $t$ is sufficiently large ($t>0.2$ ps). 
The $k(t)$ in Fig. \ref{fig:hbrf_4pl}(a) and Fig. \ref{fig:hbrf_4pl}(b) are calculated by the interface 
under the same conditions. The difference is that the former is the result of the interface of pure water,
and the latter is the result of the interface of the solution containing LiI ions. 
Both interfaces are thick enough that they both have a convergent $k(t)$. 
It can be seen from Fig. \ref{fig:hbrf_4pl} (b) that, as the thickness increases, $k(t)$ converges into a curve.}

{\color{blue} As can be seen from Fig. \ref{fig:hbrf_4pl} (b), the value of $k (t)$ for $d = 2$ \A (blue solid line) is significantly larger 
than that of $k(t)$ for $d \ge 3$. This shows that $k(t)$ at the interface is greater than $k(t)$ of bulk water of the solution. 
This means that, compared to bulk phase, the ratio of H-bonds that initially exist at the interface and are broken at time $t$ is larger.
{\color{red}[Q2: it would make sense here to compare your results to those obtained for BLYP water (bulk and water surface). Is there any paper about BLYP or PBE water dynamics? please comment. Answer2: For the same calculation, I have not find much result from ab inito MD simulation, except 
two consistent experimental/first-principles calculation results.]}
This conclusion is consistent with the two conclusions we obtained earlier (see Section \ref{sfg_alkali_iodide_interface}): 
(1) \I is a strong structure-breaking anion; [\cite{Trevani2000}] 
(2) compared to pure water, the OH stretching peak at the interface of a solution containing iodide ions will blue shift. [\cite{Tongraar2010}] 

Comparing these black solid curves, we can see that the interface of the solution containing ions has lower $k(t)$.
In other words, compared to the pure water interface, 
the ratio of H-bonds that were initially bonded at the solution interface and broken at time $t$ is lower.
Because the effect of iodide ions is to increase the $k(t)$ of the interface, the decrease of $k (t)$ of the interface with a larger thickness
may only be due to the contribution of cations located under the first layer of water molecules at the interface. 
Therefore, although the iodide ion increases the HB rupture rate at the top layer of the interface, 
in general, the HB rupture rate of the entire solution interface is reduced due to the presence of cations under the first layer of water molecules. 
To verify this conclusion, we calculated the $k(t)$ at the interface of NaI (Fig. \ref{fig:hbrf_4pl} (c)) and KI (Fig. \ref{fig:hbrf_4pl} (d)) aqueous solution. 
The results for both interface systems support our conclusions above.}

{\color{red}[Q1: what is the diff between bulk and interface? please comment]}
{\color{blue} Let us examine the difference in $k(t)$ between interface water and bulk water. 
No matter from pure water (Fig. \ref{fig:hbrf_4pl} (a)) 
or solution (Fig. \ref{fig:hbrf_4pl} (b), (c) or (d)), we find that when the interface thickness is thin, the fluctuation of $k(t)$ is larger.
Because the thinner the interface, the fewer pairs of water molecules that can form hydrogen bonds. 
In our calculations, the fewer samples are used to average, so the fluctuation of $k (t)$ is greater. 
We can find that at the interface of pure water, when $t> 0.2$ ps, the $k(t)$ value of the interface with different thickness is almost equal 
at any time period $\Delta t$. For example, $\Delta t$ is selected as $\sim$ 2 ps, 
and its average value is shown in Table \ref{tab:hbrf_neat}. In each time period of 2 ps, the values of $k(t)$ for different layers are approximately equal
($\pm 0.004$ ps$^{-1}$). Therefore, as far as the nature of HB reaction flux is concerned, the difference between interface and bulk phase of neat water is not obvious. 
}
\begin{table}[htbp]
\centering
\caption{\label{tab:hbrf_neat} 
   {\color{blue}The average value of $k(t)$ (unit: ps$^{-1}$) over different time periods of 2 ps for layers of the water/vapor interface of neat water.  
   $\overline{k(t)}_{\Delta t}$ denotes the average value of $k(t)$ over the time period $\Delta t$.} }
\begin{tabular}{cccc}
  Thickness($d$) & $\overline{k(t)}_{\text{2--4 ps}}$ & $\overline{k(t)}_{\text{4--6 ps}}$ & $\overline{k(t)}_{\text{6--8 ps}}$\\
\hline
  2\AA & 0.027 & 0.019 & 0.019\\
  3\AA & 0.028 & 0.021 & 0.020 \\
  4\AA & 0.031 & 0.025 & 0.022 \\
  5\AA & 0.028 & 0.019 & 0.014 
\end{tabular}
\end{table}

As for the effect of water/vapor interface on the HB dynamics in alkali-iodine solutions,
we also calculate the survival probability for interfaces with different sizes of thickness. 
The result for the interface of the LiI solution exhibits that H-bonds at water/vapor 
interface decay faster than that in bulk water.
The result for the logarithm of \SHB is displayed in Fig.\space\ref{fig:2LiI-124w_S_layers} in Appendix \ref{thickness_interface}, 
in which the thickness of the alkali-iodine solutions can be determined.
{\color{red}[Q4: what do we learn here?]}
{\color{blue}Therefore, as the interface thickness increases, the \SHB converges to a curve, 
which characterizes the HB dynamics of bulk solutions. 
In particular, it gives the average lifetime of H-bonds in bulk solutions.}
\FloatBarrier
\paragraph{Effects of the Ions Concentration}
Effects of ions' concentration on HB dynamics have been studied extensively by Chandra. [\cite{AC00}]
%Pal and coworkers provided details on the structure of water around the micellar surface.\cite{SP05} 
We calculated the \CHB for the water/vapor interfaces of the alkali-iodine solutions, 
and the relaxation time $\tau_{\text{R}}$ for each of them can be determined by 
\begin{eqnarray}
        C_{\text{HB}}(\tau_\text{{R}})=1/e. \nonumber
\label{eq:relaxation_time}
\end{eqnarray}
Here, the \emph{interface} means \emph{all} the water molecules in each model. 
The $\tau_{\text{R}}$ for the water/vapor interfaces of the LiI (NaI) solutions are displayed in 
Table \ref{tab:tau_hb}. Generally, they are in the range 1--10 ps. 
The values of $\tau_{\text{R}}$ decrease as the concentration of the solutions increases.
\begin{table}[htbp]
\centering
\caption{\label{tab:tau_hb} 
  The relaxation time $\tau_{\text{R}}$ (unit: ps) of the correlation function \CHB  for the water/vapor interface of the LiI (NaI) solutions, calculated by DFTMD simulations.}
\begin{tabular}{ccc}
  concentration  & $\tau_{\text{R}}$ (LiI) & $\tau_{\text{R}}$ (NaI) \\
\hline
  0 & 11.50 & 11.50 \\
  0.9 M & 7.04 & 10.60 \\
  1.8 M & 4.40 & 1.96 
\end{tabular}
\end{table}

The concentration dependence of the HB dynamics can be also found in the \SHB. 
Fig.\space\ref{fig:124_2LiI-2NaI_hbacf_S}(a) gives the \SHB 
for the water/vapor interfaces of 0.9 M and 1.8 M LiI solutions.
The same quantity for NaI solutions is displayed in Fig.\space\ref{fig:124_2LiI-2NaI_hbacf_S}(b).
This result indicates that, for the interface of alkali-iodine solution, the lifetime of H-bonds 
decrease as the concentration of LiI (or NaI) solution increase.
\begin{figure}
\centering
\includegraphics [width=1.1\textwidth, center] {./diagrams/124_2LiI-2NaI_hbacf_S} 
\setlength{\abovecaptionskip}{0pt}
  \caption{\label{fig:124_2LiI-2NaI_hbacf_S} The time dependence of the \SHB  of 
  H-bonds at the water/vapor interfaces of (a) LiI and (b) NaI solutions at 330 K.
	The insets show the plots of ln$S_{\text{HB}}(t)$.} 
\end{figure}
%%
\FloatBarrier
\paragraph{Effect of Nitrate ions}
%I simulate the alkali nitrate solution/vapor interface to find how the nitrate affect the structure of the interface.
\begin{figure}[htbp] % or \begin{SCfigure}
\centering
\includegraphics [width=0.6\textwidth] {./diagrams/256_LiNO3_hbacf_sh_no3} %fig.5.10
\setlength{\abovecaptionskip}{0pt}
\caption{\label{fig:256_LiNO3_hbacf_sh_no3} The \SHB of water--water (W--W) and nitrate--water (N--W) H-bonds at the water/vapor
  interface of the \LiN solution. The inset is the plot of ln\SHB. 
  These results are calculated for the temporal resolution $t_t=1$ fs.}
\end{figure}
%
%\begin{figure}[H]
%\centering
%\includegraphics [width=0.4\textwidth] {./diagrams/256_LiNO3_hbacf_hh_all_traj_sh_no3}
%\setlength{\abovecaptionskip}{20pt}
%\caption{\label{fig:256_LiNO3_hbacf_hh_all_traj_sh_no3}The functions ln\SHB of water--water H-bonds (black) and Nitrate -water H-bonds (red) in the the \LiN solution-vapor interface at 300 K. The lifetime of H-bonds $\tau_{\text{HB}}$ is calculated by the integration of \SHB over t$\in$(0,$\infty$), which give 0.42  and 0.20 ps, for water--water H-bonds and Nitrate -water H-bonds, respectively.}
%\end{figure}
%The density profile is a indicator of a table interfacial system (see Fig.\space\ref{fig:density_4MPlus_alkali-I}).
%\begin{figure}[htbp]
%\centering
% \includegraphics [width=0.6\textwidth] {./diagrams/density_4MPlus_alkali-I} %fig5.11
%\setlength{\abovecaptionskip}{20pt}
%\caption{\label{fig:density_4MPlus_alkali-I}The density as a function of the slab coordinate \Z. The result is calculated by MD with SPC water model.}
%\end{figure}

As shown in Fig.\space\ref{fig:vdos_LiNO3-256w_w_near_nitrate} in chapter ~\ref{CHAPTER_SFG_Calculation}, 
from the VDOS, the water molecules bound to \nitrate have higher OH stretching frequency (55 cm$^{-1}$ larger) 
than those H-bonded to other water molecules. 

Now, the difference between nitrate--water and water--water H-bonds 
can be also analyzed in terms of the survival probability $S_{\text{HB}}(t)$,
\cite{AKS86,JT90,AL96} 
reported in Fig.\space\ref {fig:256_LiNO3_hbacf_sh_no3}.
The integration of \SHB from 0 to $t_{\max}=5.0$ ps, [\cite{Steinel2004}] gives the relaxation time $\tau_\text{HB}$, which can be interpreted as 
the average lifetime of the HBs. [\cite{SC02}] 
The values of $\tau_{\text{HB}}$ is dependent on a temporal resolution $t_t$, during which the H-bonds that break and reform are treated as intact. [\cite{AL00}] Here, 
we choose the temporal resolution as $t_t=1$ fs. 
Then, Fig.\space\ref {fig:256_LiNO3_hbacf_sh_no3} gives $\tau_\text{HB}=0.20$ ps for nitrate--water H-bonds at interfaces, and $\tau_\text{HB}=0.42$ ps for water--water H-bonds.
This result of $\tau_\text{HB}$ is consistent with the experimental result of Kropman and Bakker ($\tau_\text{HB}=0.5\pm0.2$ ps). [\cite{MFK01}]
The smaller value of $\tau_\text{HB}$ for nitrate--water H-bonds implies that the nitrate--water H-bonds are weaker than bonds between water molecules. 
This is also consistent with the VDOS analysis and the blue-shifted frequency of the OH stretching in the nitrate-water HB.
%[DELETED From both the VDOS and HB dynamics calculations, we conclude that it is the weak HBs between nitrate and water make the higher surface propensity 
%of nitrate anions, and then induce the depletion of SFG intensity at 3200 \cm for the alkali nitrate salty interfaces.]

%Fig. ~\ref{fig:256_LiNO3_hbacf_Nitrate_effect} shows that the nitrate ions accelerate the HB dynamics at the vapor/water interface of alkali nitrate solution.
%\begin{figure}[H]
%\centering 
% \includegraphics [width=0.6\textwidth] {./diagrams/256_LiNO3_hbacf_Nitrate_effect} %fig5.12
%\setlength{\abovecaptionskip}{20pt}
%\caption{\label{fig:256_LiNO3_hbacf_Nitrate_effect}The functions \CHB of bulk water--water H-bonds (W-W (Bulk)) and nitrate--water H-bonds (N-W) 
%at interfaces of alkali nitrate solution  (LiNO$_3$(H$_2$O$_{256}$)  at 300 K. }
%\end{figure} 
%NOT CLEAR, TO EXPLAIN BETTER The HB relaxation time is about $2.5$ ps, which is the same as that
%for nitrate--water H-bonds at interfaces of alkali nitrate solution.
%[NOT CLEAR: For bulk water, the HB relaxation time $\tau$ is $3.7$ ps. The difference between the HB dynamics of H-bonds outside the first shell of \Li and HB dynamics for nitrate--water H-bonds at interfaces
%is not visible from the values of the HB relaxation time. They reflect the difference between HB
%dynamics between bulk water and water/vapor interfaces.]
\FloatBarrier
\paragraph{Effects of Alkali Metal Ions and \I on HB Dynamics}
\begin{figure}[!ht]
\centering
\includegraphics [width=\textwidth] {./diagrams/C_S_HB_124_2LiI-2NaI-2KI} %fig5.15
\setlength{\abovecaptionskip}{0pt}
  \caption{\label{fig:C_S_HB_124_2LiI-2NaI-2KI} The time dependence of functions (a) \CHB and (b) \SHB of water--water H-bonds at water/vapor interfaces of 0.9 M alkali-iodine solutions.} 
\end{figure}
% Notice that if I use the data from 123_2NaI_16, etc, the result is different! what the reason.
%
\begin{figure}[!h]
\centering
\includegraphics [width=0.6\textwidth] {./diagrams/hb_lifetime_124_2LiI-2NaI-2KI} %fig5.16
\setlength{\abovecaptionskip}{0pt}
  \caption{\label{fig:hb_lifetime_124_2LiI-2NaI-2KI} The resolution dependence of the lifetime $\tau_{\text{HB}}$ of water--water H-bonds at interfaces of 
  different 0.9 M alkali-iodine solutions at 330 K, calculated for six temporal resolutions ($t_t$). [\cite{Ferrario1990,Mountain1995,Root1997}] 
  For $t_t = 5$ fs, the calculated HB lifetime is 0.30 ps, 0.31 ps and 0.23 ps, for the interface of LiI, NaI and KI solution, respectively.}
\end{figure}
%
\begin{table}[hbtp]
\centering
\caption{\label{tab:tau_hb_alkali_iodine} 
The lifetime $\tau_{\text{HB}}$ (unit: ps) of H-bonds in the first hydration shell of I$^-$ ion and of alkali metal ion at the water/vapor interface of 0.9 M LiI (NaI, KI) solution.}
\begin{tabular}{cccc}
  &\I-shell &cation-shell& interface \\
\hline
 LiI & 0.22 & 0.24 & 0.23\\
 NaI & 0.24 & 0.28 & 0.26\\
 KI  & 0.20 & 0.23 & 0.20\\
\end{tabular}
\end{table} 
%Water/Vapor & -&-&
Table \ref{tab:tau_hb_alkali_iodine} shows that, for all three alkali-iodine solutions, the lifetime $\tau_{\text{HB}}$ of H-bonds in the 
solvation shell of alkali metal (iodine) ions is larger (smaller) than 
that of H-bonds at the water/vapor interfaces of the same solutions, 
respectively. For LiI solution, the water molecules bound to the cation ion
\Li, on average, have a HB relaxation time $\tau_{\text{R}} \sim 0.24$ ps. This
relaxation time is longer than that of molecules bound to \I or at the interface of the LiI solution. 
%
\begin{figure}[htbp]
\centering
\includegraphics [width=\textwidth] {./diagrams/hbacf_C_sh2_2p}
\setlength{\abovecaptionskip}{0pt}
\caption{\label{fig:hbacf_C_sh2_2p}The \CHB of water--water H-bonds in the {\color{blue}ion solvation shell} 
  of (a) cations and (b) I$^-$ at the interfaces of 0.9 M LiI, NaI and KI solutions, respectively.
  The {\color{blue} dashed line} shows \CHB for the {\color{blue} interface (the thickness $d = 8$ \A)} of the LiI solution.{\color{red}[ Not clear what the difference between continous and dashed black line is.]} 
  This interface contains H-bonds between water molecules similar to those in pure bulk water, that is,
  water molecules participating in these H-bonds are not in the solvation shell of ions.} 
\end{figure}
%Fecko and co-workers' study of liquid D$_2$O by IR spectroscopy reveals that the vibrational dynamics observed are dominated by underdamped displacement of the hydrogen-bond coordinate at very short times ( less than 200 fs).\cite{CJF03,CJF05} 
Fig.\space\ref{fig:hbacf_C_sh2_2p} (a) and (b) show that the \CHB of H-bonds within the alkali cations and \I decay faster than those in bulk water and at the surface of LiI solution.
From Fig.\space\ref{fig:hbacf_C_sh2_2p}(b), we find that, for all three alkali-iodine solutions, the \CHB for hydration shell water molecules of \I decays faster than that for molecules at the water/vapor interface.
%The simulation produces similar result as Omta and coworker's experiments of femtosecond pump-probe spectroscopy, which demonstrate that anions ( $\text{SO}^{2-}_4$, $\text{ClO}^-_4$, etc) have no influence on the dynamics of bulk water, even at high concentration up to 6 M.\cite{AWO03} 
%Here, we find that the cations \Li and \Na does not alter the H-bonding network outside the first hydration shell of cations. It is concluded that no long-range structural-changing effects for alkali metal cations.
The radii of hydration shells are 5.0 \AA for \li, 5.38 \AA for \na,
5.70 \AA for \pot, and 6.0 \AA for \I ions, which are obtained from the RDFs.
The RDFs $g_{\text{ion-O}}$ (ion=\li, \na) for the interfaces 
of LiI (NaI) solutions are shown in Fig.\space\ref{fig:124_2NaI-2LiI_gdr_Li-O_Na-O_1501}(a),
and the coordination numbers are in Fig.\space\ref{fig:124_2NaI-2LiI_gdr_Li-O_Na-O_1501}(b).
\begin{figure}[!htbp]
\centering
\includegraphics [width=0.5 \textwidth]{./diagrams/124_2NaI-2LiI_gdr_Li-O_Na-O_1501}%fig.6.1 
\setlength{\abovecaptionskip}{0pt}
\caption{\label{fig:124_2NaI-2LiI_gdr_Li-O_Na-O_1501}
  (a) The RDF $g_{\text{ion-O}}(r)$(ion=\li, \na) and (b) the coordination number of \Li (\na) ions at the interfaces of LiI (NaI) solution. 
	For \Na, the coordination number $n_\text{Na}$=5; while for \Li, $n_\text{Li}$=4.} 
\end{figure} % There is a first shell exist for both \Li and \Na cations.
%\section{Hydrogen Bond Dynamics by Classical Molecular Dynamics Simulations}
%\begin{figure}[H]
%\centering
% \includegraphics [width=0.5\textwidth] {./diagrams/4MPlus-alkali-I_hbacf_C1603}
%\setlength{\abovecaptionskip}{20pt}
%\caption{\label{fig:4MPlus-alkali-I_hbacf_C1603}The function \CHB of water--water H-bonds at interfaces with different alkali metal ions in 4.0 M water solution at 300 K.}
%\end{figure}
%The HB dynamics obtained from classical MD simulations can not catch the fast HB relaxation, and it give a totally different HB dynamics for the water molecules in these alkali halide solution/vapor interfaces.
\FloatBarrier
\section{Rotational Anisotropy Decay of Water at the Interface of Alkali-Iodine Solutions}\label{CHAPETR_AD}
Using the transition dipole auto-correlation function, 
we determined the rotational anisotropy decay and therefore the OH-stretch relaxation at water/vapor interface of alkali iodide solutions.
%The effects of ion environment on structure and dynamics of water are obtained by comparing the second-order Legendre polynomial, i.e.,  $P_2(x)=\frac{1}{2}(3x^2-1)$,  orientational correlation function of the transition dipole.
The anisotropy decay can be determined from experimental signal in two different polarization configurations---parallel and perpendicular polarizations, by 
\begin{equation}
        R(t)=\frac{S_{\parallel}(t)-S_{\perp}(t)}{S_{\parallel}(t)+2S_{\perp}(t)}
\label{eq:ad}
\end{equation}
where $t$ is the time between pump and probe laser pulses.  The anisotropy decay can also be obtained by simulations, and calculated by the third-order response functions $R(t)$. [\cite{Jansen10,Jansen06}]
%
%In the first shell with a radius 3 \A, the entropy difference betweem the \Li shell and \Na shell,
%$\Delta S=k_B\text{ln}\frac{\Omega_\text{Na}}{\Omega_\text{Li}}=k_B\text{ln}\frac{n_\text{Na}/V_\text{Na}}{n_\text{Li}/V_\text{Li}} =k_B\text{ln}1.05$.
%
%\paragraph{Probability Distribution of Ions}
%The probability distribution, shown in Fig.~\ref{fig: prob_124_LiI_Sans_double_axis}, of the ions in the water/vapor interface of LiI and NaI solutions with repect to the depth of the ions in the solutions 
%indicates that the \I ions prefer to staying at the topmost layer of surface of solutions.
%(molar concentration: 0.9 M, temperature: 330 K) 
%It shows that \I ions tend to the surface of the solutions, while \Na and \Li tend to stay in the bulk. This result is consistent with the calculations from Ishiyama and Morita\cite{TI07,TI14}.
The orientational anisotropy $C_2(t)$ is given by the rotational time-correlation function 
\begin{equation}
C_2(t)=\langle P_2(\hat{u}(0)\cdot\hat{u}(t)) \rangle,
\label{eq:tcf2}
\end{equation}
where $\hat{u}(t)$ is the time dependent unit vector of the transition dipole, $P_2(x)$ is the second Legendre polynomial, and $\langle \rangle$ indicate 
equilibrium ensemble average. [\cite{Corcelli05,LinYS2010}] %\cite{2010Lin} % angular brackets

The anisotropy decay $C_2(t)$ for the water/vapor interface of LiI solution is shown in Fig.\space\ref{fig:c2_2LiI_16_inset}.
This function decays faster than that of neat water, indicating that H-bonds
at the interfaces of alkali-iodine solutions reorient faster than in neat water. The inset shows the first 0.4 ps of $C_2(t)$, from which we see a 
quick change during the first $\sim 0.1$ ps primarily due to librations.
%
\begin{figure}[h]
\centering
\includegraphics [width=0.6\textwidth] {./diagrams/c2_2LiI_16_inset} 
\setlength{\abovecaptionskip}{0pt}
  \caption{\label{fig:c2_2LiI_16_inset} The time dependence of the $C_2(t)$ of OH bonds at the water/vapor interfaces of 0.9 M LiI solution and of neat water (dashed line) at 330 K, calculated by DFTMD simulations. The water/vapor interface of neat water is modeled with a slab made of 121 water molecules in a simulation box of size $15.6 \times 15.6 \times 31.0$ \A$^3$.}
\end{figure}
%
We also calculated the $C_2(t)$ for the interface of other alkali-iodine solutions LiI and KI. 
The results of $C_2(t)$ for the water/vapor interfaces of these solutions are shown in Fig.\space\ref{fig:c2_2KI_2NaI_2LiI_16}.
In all the cases $C_2(t)$ decays faster than in neat water, indicating that H-bonds
at the interfaces of the three alkali-iodine solutions are orientated faster than that of neat water.
[{\color{red} This is not clear.}]
{\color{blue}They show that \I ions can accelerate the dynamics of molecular reorientation of water molecules at interfaces.}   

%
\begin{figure}[htbp]
\centering
\includegraphics [width=0.6 \textwidth] {./diagrams/c2_2KI_2NaI_2LiI_16} 
\setlength{\abovecaptionskip}{0pt}
  \caption{\label{fig:c2_2KI_2NaI_2LiI_16} The time dependence of the $C_2(t)$ of OH bonds in water molecules at the water/vapor 
  interface of 0.9 M alkali-iodine solutions and of neat water (dashed line) at 330 K, calculated by DFTMD simulations.}
\end{figure} 

We have obtained non-single-exponential kinetics for the rotation of water molecules both at the surface 
and in bulk water (Appendix \ref{single_exp}).
%This result is true for water molecules bound to ions. 
Therefore, the rotational motion of water molecules are not simply characterized by well-defined rate constants. 
%Then the problem is to understand the kinetics.
[{\color{red} Please compare to more recent results on ab initio water (BLYP or PBE).}]{\color{blue}Similar non-single-exponential kinetics is also obtained in the HB kinetics
in liquid water [\cite{AL96,Dirama05}] and in the time variation of the average frequency shifts of the 
remaining modes after excitation in hole burning technique [\cite{Rey2002,Moller2004}] and using BLYP functional.} [\cite{Bankura2014}]
Luzar and Chandler interpreted 
the non-single-exponential kinetics as the result of an interplay between 
diffusion and HB dynamics. [\cite{AL96}] 
Here, we can understand the non-single-exponential kinetics of rotational 
anisotropy decay by fitting the rotational anisotropy decay by a 
biexponential function.

To obtain the effects of diffusion and HB decay of water molecules
in solutions respectively, we assume that there are two independent 
decays in the process of an anisotropy decay. 
Therefore, the $C_2(t)$ has the form [\cite{TanHS05}]
\begin{equation}
C_2(t)=A_1e^{-\kappa_1 t} +A_2e^{-\kappa_2 t},
\label{eq:tcf3}
\end{equation}
where $A_i$ are constants and $\kappa_i$ are decay rates ($i=1,2$). 
The time constants and amplitudes of the biexponentials fits for 
the $C_2(t)$ are listed in Table ~\ref{tab:2LiI_c2_biexp} and Table ~\ref{tab:2NaI_c2_biexp}.
The biexponential fit is very close to the calculated $C_2(t)$, which can be seen in Fig.\space\ref{fig:2LiI-124w_c2_fit_5ps_biexp} (compare Fig.\space\ref{fig:2LiI-124w_c2_fit_5_single-exp}).
%
\begin{table}[hbt]
\centering
\caption{\label{tab:2LiI_c2_biexp}%
	Biexponential fitting (5 ps) of the $C_2(t)$ for water molecules in 0.9 M LiI solution.}
%\begin{ruledtabular}
\begin{tabular}{lccccc}
water molecules & $A_1$  & $\kappa_1$ (THz) & $A_2$ & $\kappa_2$ (THz) \\
\hline
I$^-$-shell & 0.44 & 0.25 & 0.39 & 0.26\\
Li$^+$-shell & 0.88 & 0.07 & 0.07 & 8.24\\
bulk & 0.84 & 0.11 & 0.09 & 4.88 \\
surface & 0.73 & 0.27 & 0.22 & 13.47 \\
\end{tabular}
%\end{ruledtabular}
\end{table}
%--

\begin{table}
\centering
  \caption{\label{tab:2NaI_c2_biexp}%
	Biexponential fitting (5 ps) of the $C_2(t)$ for water molecules in 0.9 M NaI solution.}
  \begin{tabular}{lccccc}
  water molecules & $A_1$  & $\kappa_1$ (THz) & $A_2$ & $\kappa_2$ (THz) \\
  \hline
  I$^-$-shell & 0.86 & 0.14 & 0.08 &9.86 \\
  Na$^+$-shell & 0.71 & 0.06 & 0.18 &0.79 \\
  bulk & 0.81 & 0.06 & 0.10 & 1.25 \\
  surface & 0.77 & 0.11 & 0.13 & 2.31 \\
  \end{tabular}
\end{table}
%
%图
\begin{figure}[htbp]
\centering
\includegraphics [width= \textwidth] {./diagrams/2LiI-124w_c2_fit_5_biexp} 
  \caption{\label{fig:2LiI-124w_c2_fit_5ps_biexp} The time dependence of the $C_2(t)$ of OH bonds 
  in water molecules at the water/vapor interface of LiI solution.}
\end{figure} 
%=================
%\begin{table}[htbp]
%\centering
%\caption{\label{tab:table_2KI_2LiI_anisotropy_decay_with_variatio}%
%The fitted parameters of anisotropy decay of water molecules in LiI (KI) solution/vapor interfaces and neat water/vapor interface. The constants and amplitudes comes from fitting (0--20 ps).} 
%\begin{tabular}{lccccc}
%Interfaces & $A_1$  & $\kappa_1$ (THz) & $A_2$ & $\kappa_2$ (THz) \\
%\hline
%water molecules & 0.595$\pm(1\times10^{-3})$ & 0.07434 $\pm(9\times 10^{-5})$&0.310$\pm(9\times10^{-4})$ & 0.290 $\pm(8\times10^{-4})$  \\
%KI & 0.797$\pm(4\times10^{-4})$ & 0.16530 $\pm(7\times 10^{-5})$ &0.122$\pm(4\times10^{-4})$ & 0.884 $\pm(6\times10^{-3})$ \\ 
%LiI & 0.836$\pm(2\times10^{-4})$ & 0.20463$\pm(4\times 10^{-5})$ &0.091$\pm(2\times10^{-4})$ & 1.752 $\pm(9\times10^{-3})$ \\
%\end{tabular}
%\end{table}
%
%[Notes: The 63-water-slab models is listed here as a reference. The number of water molecules is small; The data for KI/vapor and LiI/vapor interfaces come from  KI\_16 and LiI\_16 systems.  
%Water(63) &0.831$\pm(1\times10^{-4})$ &  0.08760 $\pm(2\times 10^{-5})$&0.100$\pm(2\times10^{-4})$ & 1.029 $\pm(4\times10^{-3})$  \\ ]
%
%\begin{figure}[htbp]
%\centering
%\includegraphics [width=0.4 \textwidth] {./diagrams/c2_121-pure_2KI_2LiI_16_inset_fit_biexp} 
%\setlength{\abovecaptionskip}{10pt}
%\caption{\label{fig:c2_121-pure_2KI_2LiI_16_inset_fit_biexp} The fitted and calculated anisotropy decay of OH bonds in water molecules in LiI solution/vapor interface (red), LiI solution/vapor interface (blue) and neat water/vapor interface (black). The corresponding fitted functions are denoted by dashed lines. The concentration of LiI and KI solution is 0.9 M.}
%\end{figure} 

Then we considered the effect of ion species in solutions on the anisotropy decay of water molecules.
From Table \ref{tab:2LiI_c2_biexp} and Table \ref{tab:2NaI_c2_biexp}, we find that 
for both LiI and NaI solutions, there are two decay processes in the dynamics --- amplitude $\sim 1$,
decay constant $\sim$ 0.1 THz, and for the other describe the initial fast decay 
of the anisotropy, with amplitude $\sim 0.1$, decay constant $\sim$ (1--10) THz, 
due to the inertial-librational motion preceding the orientational diffusion.
That is, two decay processes exist in the dynamics of water molecules 
at the water/vapor interfaces of alkali-iodine solutions. 
%The one describe the initial fast decay of the anisotropy, 
%with amplitude $\sim$ 0.1, decay constant $\sim$ (1--10) THz,
%results from the inertial-librational motion preceding the orientational diffusion.
%
\begin{table}[H]
\centering
\caption{\label{tab:fitting_c2_for_each_type_of_water}%
  Biexponentially fitting (2 ps) of the $C_2(t)$ for different types of water molecules at the water/vapor interface of LiI solutions.}
\begin{tabular}{lccccc}
water molecules & $A_1$  & $\kappa_1$ (THz) & $A_2$ & $\kappa_2$ (THz) \\
\hline
$DDAA$ & 0.85 & 0.25   & 0.10 & 16.0\\
$DD'AA$ & 0.89 & 0.14  & 0.06 & 14.1 \\
$D'AA$ & 0.38 & 0.99 & 0.38 & 0.99 \\
\end{tabular}
\end{table}
%
\begin{table}[H] %[!hbtp]
\centering
\caption{\label{tab:table_CoordNo}%
The coordination number of the atoms in LiI (NaI) solutions.}
\begin{tabular}{lccc}
name & radius of the first shell (\AA) & coordination number \\
\hline
$n_\text{I-H}(\text{LiI})$ & 3.3 & 5.5 \\
$n_\text{I-H}(\text{NaI)}$ & 3.3 & 5.1 \\
$n_\text{I-O}(\text{LiI)}$ & 4.3 & 5.8 \\
$n_\text{I-O}(\text{NaI)}$ & 4.3 & 6.0 \\
$n_\text{Li-O}(\text{LiI)}$ & 3.0 & 4.0 \\
$n_\text{Na-O}(\text{NaI)}$ & 3.5 & 6.0 
\end{tabular}
\end{table}
%In the first shell with a radius 3 \A, the entropy difference between the \Li shell and \Na shell,
%$\Delta S=k_B\text{ln}\frac{\Omega_\text{Na}}{\Omega_\text{Li}}=k_B\text{ln}\frac{n_\text{Na}/V_\text{Na}}{n_\text{Li}/V_\text{Li}} =k_B\text{ln}1.05$.

%\paragraph{The anisotropy decay}
%

\FloatBarrier
\paragraph{Classification of Water Molecules Based on H-Bonds}
We also studied the relation between the anisotropy decay of water molecules and their environment. 
Following the definition used in Ref.[\cite{TianCS08}], we use the following labels to denote water molecules in solution: 
{\color{blue} $D$ denotes that the water molecule donates a HB, $D'$ donates that the water donates a H-I bond, and $A$ donates that the water accepts a HB.} %\cite{2008NJ} 
$DDAA$ represents a water molecule with two H-Bonds donated to water molecules and two H-Bonds accepted from water molecules (see Fig.\space\ref{fig:Multiple_figs}(a));
$DD'AA$ represents a water molecule with two HBs donated to a water molecule and \I, and with two H-Bonds accepted from other water molecules (see Fig.\space\ref{fig:Multiple_figs}(c)), 
$D'AA$ represents a water molecule bonded to \I at the water/vapor interface and other H-Bonds to water molecules (see Fig.\space\ref{fig:Multiple_figs}(d)).
Clearly, we can see that $D'AA$ molecules are of free OH stretching during the dynamics. All four types of water molecules are displayed in Fig.\space\ref{fig:Multiple_figs}. 
% 
\begin{figure}[ht]%[!htbp]
\centering
\includegraphics [width=0.4 \textwidth] {./diagrams/Multiple_figs} 
\caption{\label{fig:Multiple_figs} Four types of water molecules at the water/vapor interfaces of LiI solution, regarding the HB environments: (a) $DDAA$; (b) $DDA$; (c) $DD'AA$; (d) $D'AA$. The cyan balls denote \I ions. }
\end{figure} 

It is evident, from our calculations (Fig.\space\ref{fig:2LiI-124w_c2_fit_biexp_7wat_2ps_class_150324}), that the $C_2(t)$ for $DDAA$ and $DD'AA$ molecules do not decay exponentially (Table \ref{tab:fitting_c2_for_each_type_of_water}).
%[BUT Table \ref{tab:fitting_c2_for_each_type_of_water} CAN NOT GIVE THE EVIDENCE. STH. IS MISSING!] 
This result is similar to the reactive flux HB correlation function $k(t)$, i.e., 
the escaping rate kinetics of H-bonds in bulk water. [\cite{Luzar1996}] 
{\color{blue}The relaxation of H-bonds in water appears complicated, with no simple characterization in terms of a few relaxation rate constants. 
Most of the authors believe that the cooperativity between neighbouring H-bonds, [\cite{Sciortino1989, Ohmine1995}] or 
self evident coupling between translational diffusion and HB dynamics is the source of the complexity. [\cite{Luzar1996}}] 
\st{Since the source of the non-exponential dynamics of $k(t)$ is due to the correlations 
between neighbouring H-bonds} {\color{red}[Are you sure about this statement? DOUBLE CHECK]}, \st{the non-exponential anisotropy decay 
for $DDAA$ and $DD'AA$ molecules is also due to the same reason.} 
However, for $D'AA$ molecules at the interface of the LiI solution,
the $C_2(t)$ decays exponentially, i.e.
\begin{eqnarray}
  C_2(t) &=& C e^{-{\kappa}t},
\label{eq:C_2_D_prime_AA}
\end{eqnarray}
where the amplitude is $C=0.76$, and the reorientation rate is $\kappa = 0.99$ ps$^{-1}$.
The single exponential decay of $C_2(t)$ for $D'AA$ molecules, indicates that each $D'AA$  molecule reorientate independently to each other. 

Furthermore, the $C_2(t)$ for $D'AA$ molecules decays much faster than that for $DDAA$ or $DD'AA$ molecules.
From the definitions, the $D'AA$ water molecule owns only three H-bonds, while both $DDAA$ and $DD'AA$ water molecules own four H-bonds.
Therefore, the correlation between H-bonds around the $D'AA$ molecule is weaker than those around the $DDAA$ or $DD'AA$ molecule. 
Faster decay of $C_2(t)$ for $D'AA$ molecules shows that the reorientation process of $D'AA$
molecules is much smaller than those water molecules in bulk phase, e.g., the $DDAA$, and $DD'AA$ molecules.

Finally, for $D'AA$ molecules, the inertial-librational motion can not be seen (Fig.\space\ref{fig:2LiI-124w_c2_fit_biexp_7wat_2ps_class_150324}). 
This result implies that the rotational anisotropy decay of $D'AA$ molecules
are of the same time scale of the inertial libration, i.e., $\sim$ 0.2 ps.

Rotational anisotropy decay of water molecules is found at the interface of LiI solution. 
The result comes from a different HB types from the usual $DDAA$ HB type in pure bulk water.
The faster anisotropy decay for $D'AA$ molecules reflects the less correlation between different H-bonds for $D'AA$ molecules, which comes from Hydrogen--Iodide bond at the interfaces, the existence of free OH stretching.
From Fig.\space\ref{fig:prob_124_LiI_double_axis}, we have known that in the LiI solution, 
\I ions prefer to locate at the water/vapor interface.  
Therefore, we infer that the reduction of the inter-correlations between H-bonds occurs at the water/vapor interfaces. 

%
\st{In conclusion, the ultra-fast anisotropy decay, is dominated by population transfer.}{\color{red}[What do you mean?]}
{\color{blue}In conclusion, single exponential type rotational anisotropy decay exists for water molecules at the water/vapor interface of the alkali-iodine solutions,
and this faster anisotropy decay of water molecules at the water/vapor interface is the effects of Hydrogen--Iodide (H--I) bond at the interface.} 
Since the iodide's surface propensity is high, this difference of HB structure 
from neat water/vapor interface is the source of 
the HB dynamics as well as the Im$\chi^{(2)}$ spectrum of the interface of alkali-iodine solutions.  
\st{The effects of H--I bond on the HB dynamics at the interfaces, and the relation between the interfacial HB
dynamics and rotational anisotropy decay can also be studied in the future.}{\color{red}[Question: This i don't understand ... is not what you have discussed so far?? Answer: It was a plan.]}
%图
\begin{figure}[H] %[!htbp]
\centering
\includegraphics [width=0.6 \textwidth] {./diagrams/2LiI-124w_c2_fit_biexp_7wat_2ps_class_150324} 
\caption{\label{fig:2LiI-124w_c2_fit_biexp_7wat_2ps_class_150324} The time dependence of the $C_2(t)$ for water molecules in different HB environments at the water/vapor interface of LiI solution.}
\end{figure}  

%\subsection{\LiN Solution/vapor Interface}
%The anisotropy decay of OH bonds in water molecules in 0.4 M LiNO3 solution/vapor interface is shown in Fig.\space\ref{fig:c2_LiNO3_inset}.  In the model of the interface, there is one \Li and one \nitrate in the 15.6 \AA$\times$15.6 \AA$\times$31.0 \AA simulation box. 
%The larger decay rate consistent to the conclusion infered from the VDOS for the interfaces, although the concentration of \LiN is lower. This result obtained from another DFTMD trajectory consistent with the previous one, and it reflects that the \nitrate on the surface of the alkali nitrate solution weaken the H-bonds and  accelerate the anisotropy decay of water molecules at the interfaces.
%\begin{figure}[htbp]
%\centering
%\includegraphics [width=0.4\textwidth] {./diagrams/c2_LiNO3_inset} 
%\setlength{\abovecaptionskip}{10pt}
%\caption{\label{fig:c2_LiNO3_inset} The anisotropy decay of OH chromophores in water molecules in LiNO3 solution/vapor interface.}
%\end{figure} 
