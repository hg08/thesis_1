\chapter{Summary}\label{CHAPTER_Summary}
Using DFTMD simulations, I have analyzed the interfacial structure and dynamics of electrolyte solutions containing alkali nitrates.
In particular I have presented a detailed analysis of the HB structure at the interface and I have calculated the interface vibrational spectra
in order to provide a molecular interpretation of available experimental data. 

As a first system I have analyzed the behaviour of a salty water/vapor interface containing LiNO$_3$.
Both the measured and calculated SFG spectra shows a reduced intensity of the lower frequency portion region, 
when compared to the pure water/vapor interface. 
This reduction is attributed to the H-bonds established between the \nitrate and the surrounding water molecules at the interface.
This effects is only related to the presence of \nitrate at the water surface and is not affected by the presence of alkali metal ions.
Indeed I have shown that although the \Li can reside relative close to the water surface, also forming a water mediated
ion pair with \nit, its effect on the SFG spectrum is not visible. The water molecule which mediate the interaction 
between the \nitrate and the \Li would produce a red-shifted peak in small water cluster, but its influence is not visible 
neither in the SFG spectra. 

We have also shown that the use of simple models, such as small cluster is not suitable to reproduce the experimental spectra and cannot provide a microscopic interpretation of the spectra. 
Realistic models of the interface are required to address the perturbation of the ion on the water surface.
The elucidated mechanism is possibly more general to anions which have high propensity for the water surface,
as for example other molecular ions.

%As for the HB dynamics of water/vapor interfaces of alkaline iodine solutions, the HB relaxation time $\tau \approx 2.5$ ps (see Table \ref{}), which is the same as that for nitrate--water H-bonds at interfaces of alkali nitrate solution. 
%For bulk water, the HB relaxation time $\tau \approx 3.7$ ps. 
The difference between the HB dynamics of H-bonds outside the first shell of the \Li and that of nitrate-water H-bonds 
at interfaces is not visible from the values of the HB relaxation time. They reflect the difference between HB dynamics in bulk water and that at the water/vapor interfaces.
For the water/vapor interface of alkaline iodine solutions, I find that the cations does not alter the H-bonding network outside the first hydration shell of cations. 
It is concluded that no long-range structural-changing effects for alkali metal cations.

From these results of nonlinear susceptibilities, which shows bonded OH-stretching peaks with higher frequencies, 
I conclude that these water molecules at the water/vapor interfaces of LiI, NaI, and KI solutions are participating 
in weaker H-bonds, compared with those at the pure water surface. 
%This conclusion is based on the DFTMD simulations with a simulation box with a length scale $\sim$ 10\A. 
The origin of the characteristics may come from a unique distribution of \I ions and alkali metal cations, 
which form a double layer [\cite{Shultz2010}] over the thickness on the order of 5--10 \A\ (see Appendix \ref{thickness_interface}).

Finally, calculating the rotational anisotropy decay, I obtain non-single-exponential dynamics 
for the rotation of water molecules both at the surface and in bulk phase for alkaline 
iodine solutions.
Therefore, the rotational motion of water molecules are not simply characterized by well-defined rate constants. 
Faster rotational anisotropy decay exists for water molecules at the interface of aqueous alkali-iodine solutions, which is the result of a different HB types ($D'AA$) 
from the usual HB type ($DDAA$) in pure bulk water. This effect on anisotropy decay is due to the H--I bond at the interface. 
Since the iodide's surface propensity is high, this difference of HB structure from pure water/vapor interface changed the Im$\chi^{(2),\text{R}}$ spectrum 
and the HB dynamics of the interface of alkali-iodine solution. 
%\paragraph{Discussion} We use DFT based molecular dynamics simulations to model alkali nitrate and alkaline iodine solutions, and calculate the SFG spectra, HB dynamics and 
%anisotropy decay of water molecules of these interfaces. The effects of alkali cations, 
%nitrate anions, and different bonding environments on these properties.

%The DFT calculations (despite taking electronic correlation into account) are not expensive,their cost is comparable with that of the Hartree–Fock method. Therefore, the same computer power allows us to explore much larger molecules than with other post-Hartree–Fock
%(correlation) methods.\cite{Piela07}

%DFT transforms the many-body problem of interacting
%electrons and nuclei into a coupled set of one-particle equations, which are
%computationally much more manageable.\cite{RMN02} 
%First-principles calculations based on the KS scheme of DFT have successfully predicted and explained a wide range of solid-state properties. However, it is true only for cohesive and structural properties. Systematicaly constructing functionals that are universally applicable is still a hard problem.
%Some examples of the failures of DFT are as follows.
%The band gaps of materials\cite{ASeidl}, the barriers of chemical reactions, 
%the energies of dissociating molecular ions, and charge transfer excitation energies are underestimated\cite{Kuehne12}. 
%The binding energies of charge transfer complexes and the response to an electric field in molecules and materials are overestimated. 
%Actually, all of these diverse issues are induced by the delocalization error of approximate functionals, due to the dominating Coulomb term that pushes electrons apart.\cite{Cohen08,Sanchez08,Cohen08b} 
%Furthermore, typical DFT calculations fail to describe degenerate or near-degenerate states, such as arise in transition metal systems, the breaking of chemical bonds, and strongly correlated materials. These problems come from another error--the static correlation error of approximate functionals, because it is difficult to describe the interaction of degenerate states by using the electron density.

%The delocalization error and static correlation error of commonly used approximations \cite{Cohen08} can be understood through the perspective of fractional charges and fractional spins and reducing these errors will provide wider applications of DFT.
