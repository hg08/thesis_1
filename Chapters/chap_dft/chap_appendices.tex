\chapter{}\label{CHAPETR_A}
\section{Proof of Hohenberg-Kohn Theorem} \label{proof_HK}
The Hohenberg-Kohn theorem: The density $n({\bf r})$ of a non-degenerate ground state uniquely determines the external potential $V({\bf r})$ up to an arbitrary constant.

Proof:
Consider the subset $\mathscr{\nu}\subset L^{\frac{3}{2}}+L^{\infty}$ of potential that yields a normalizable non-degenerate ground state. The solution of the Schrodinger equation provides us with a mapping from the external potential to the ground state wave-function
\begin{equation}
 v(\vec {\bf r})\rightarrow |\Psi[v]\rangle.
\end{equation}
Since non-degenerate ground state, $|\Psi[v]\rangle$ is uniquely determined apart from a trivial phase factor.
Thus we have established a map $C: \mathscr{\nu}\rightarrow \Phi$, where $\Phi$ is the set of ground states.
(1) We can prove that $C$ is inevitable. Suppose that $|\Psi_1\rangle$ and $|\Psi_2\rangle$ (They are functions in Sobolev space, i.e.,  $|\Psi_1\rangle, |\Psi_2\rangle \in \mathscr{H}^1(R^{3N})$.) correspond to external potential $v_1$ and $v_2 \in L^{\infty}+L^{\frac{3}{2}}$. Here, $v_1 \neq v_2+\text{C}$. If we assume that $|\Psi_1\rangle=|\Psi_2\rangle=|\Psi\rangle$, by subtraction of Hamiltonian for  $|\Psi_1\rangle$ and $|\Psi_2\rangle$, we find
\begin{equation}
 (v_1-v_2)|\Psi\rangle = (E_1-E_2)|\Psi\rangle
\label{subtraction}
 \end{equation}
If $v_1-v_2$ is  not constant in some region then $\Psi$ must vanish in this region for the above equation(~\ref{subtraction})
to be true. However, if $v_1, v_2 \in L^{\infty}+L^{\frac{3}{2}}$, then $|\Psi\rangle$ cannot vanish on a open set by the unique continuation theorem. Therefore, we obtain a contradiction and hence the assumption we made is wrong. Therefore, $\Psi_1 \neq \Psi_2$ and we obtain the result that different potentials (more than a constant) give different wave-functions.  So we find that the map $C: \mathscr{\nu}\rightarrow \Phi$ is inevitable.

Define the set $A$ as the set of densities which come from a non-degenerate ground state (we only consider the ground 
state densities from potentials in the set $L^{\infty}+L^{\frac{3}{2}}$). The set $A$ is obviously a subset of $S$:
\begin{equation}
S=\{n|n({\bf r})\geq0, \sqrt{n}\in H^1(R^3), \int d^3rn({\bf r})=N\} .
\end{equation}

The electron density $n({\bf r})$ is obtained from the many body rarefaction, which is usually the ground state wave-function of Hamiltonian $H_v={\hat T}+\hat V+ \hat W$, by
\begin{equation}
 n({\bf r_1})=\langle\Phi |\hat n({\bf r_1})|\Phi\rangle = N\sum_{\sigma_1...\sigma_N}\int d^3_{r_1}...d^3_{r_N}|\Phi({\bf r}_1\sigma_1,...,{\bf r}_N\sigma_N)|.
\label{density}
\end{equation}
According to Eq.(~\ref{density}),from a given wave-function in the set of ground states $\Phi$,  we can calculate the density.
This provides us a second map $D:\Phi \rightarrow A$ from ground state wave-functions to ground state densities. To show this we calculate 
\begin{eqnarray}
E[v_1] &=& \langle\Psi[v_1]|\hat T + \hat V_1 +\hat W|\Psi[v_1]\rangle \nonumber\\  
           &<& \langle\Psi[v_2]|\hat T + \hat V_1 +\hat W|\Psi[v_2]\rangle \nonumber \\
           &=&  \langle\Psi[v_2]|\hat T + \hat V_2 +\hat W|\Psi[v_2]\rangle +\int n_2({\bf r})(v_1({\bf r})-v_2({\bf r}))d{\bf r} \nonumber\\
           &=&  E[v_2] +\int n_2({\bf r})(v_1({\bf r})-v_2({\bf r}))d{\bf r} \label{Ev1}.
\end{eqnarray}
Likewise, we find
\begin{equation}
E[v_2]<  E[v_1] +\int n_1({\bf r})(v_2({\bf r})-v_1({\bf r}))d{\bf r} \label{Ev2}.
\end{equation}
 Using (~\ref{Ev1})+(~\ref{Ev2}), we have
 \begin{equation}
  \int (n_2({\bf r})-n_1({\bf r}))(v_1({\bf r})-v_2({\bf r}))d{\bf r} < 0.
 \end{equation}
If we assume $n_2=n_1$, then obtain contradiction $0<0$, and we conclude that: different ground states must yield
different densities, i.e., $D$ is also inevitable. Therefore, the map $DC: \nu \rightarrow A$ is also inevitable and the density therefore uniquely determines the external potential. The HK theorem is proved.

%Decomment the following line if I need gray background for lst
%\lstset{backgroundcolor=\color{lightgray}}

\section{The Computational Details of the DFTMD Simulations}\label{computational_detail}
In the thesis, the DFT calculations were done with the CP2K program, which incorporates the Gaussian and plane waves (GPW) method\cite{GL99}. 
Core electrons were described by Goedecker-Teter-Hutter pseudopotentials\cite{MK05}, while the valence electrons were expanded as a double-zeta Gaussian basis set.
The keyword ''CUTOFF'' defines the plane wave cutoff (default unit is in Ry) for the finest level of the multi-grid. 
The higher the plane wave cutoff, the finer the grid.
Having constructed the multi-grid, QUICKSTEP then map the Gaussians onto the grids. The keyword ''REL$\_$CUTOFF'' controls which product Gaussians 
are mapped onto which level of the multi-grid. CP2K tries to map each Gaussian onto a grid such that the number of grid points 
covered by the Gaussian---no matter how wide or narrow---are roughly the same. ''REL$\_$CUTOFF'' defines the plane wave cutoff of a reference grid 
covered by a Gaussian with unit standard deviation $e^{|{\bf r}|^2}$.\cite{CP2K}

Therefore, the two most important keywords effecting the integration grid and the accuracy of a calculation are ''CUTOFF'' and ''REL$\_$CUTFF''.
If ''CUTOFF'' is too low, then all grids will be coarse and the calculation may become inaccurate; and if REL$\_$CUTOFF is too low, then even if you have a high ''CUTOFF'', 
all Gaussians will be mapped onto the coarsest level of the multi-grid, and thus the effective integration grid for the calculation may still be too coarse.

The CP2K input file for the system of water/vapor interface is as follows. 
% backgroundcolor=\color{lightgray},
\begin{lstlisting}[language=C]                  ]
&FORCE_EVAL
  METHOD QS
  &DFT
    BASIS_SET_FILE_NAME ./BASIS_MOLOPT   
    POTENTIAL_FILE_NAME ./GTH_POTENTIALS
    &MGRID
      CUTOFF 280
    &END MGRID
    &QS
      EPS_DEFAULT 1.0E-12
      WF_INTERPOLATION PS 
      EXTRAPOLATION_ORDER 3
    &END QS
    &SCF

      &OT ON
      #MINIMIZER DIIS
      &END OT
     SCF_GUESS        RESTART
     #SCF_GUESS ATOMIC
     EPS_SCF     1.0E-6
     MAX_SCF 300

    &END SCF
    &XC
      &XC_FUNCTIONAL BLYP  
      &END XC_FUNCTIONAL
      &vdW_POTENTIAL
         DISPERSION_FUNCTIONAL PAIR_POTENTIAL
         &PAIR_POTENTIAL
            TYPE DFTD3
            CALCULATE_C9_TERM .TRUE.
            REFERENCE_C9_TERM .TRUE.
            LONG_RANGE_CORRECTION .TRUE.
            PARAMETER_FILE_NAME ./dftd3.dat
            REFERENCE_FUNCTIONAL BLYP 
            R_CUTOFF 8.
            EPS_CN 0.01
         &END PAIR_POTENTIAL
      &END vdW_POTENTIAL
    &END XC
  &END DFT
  &SUBSYS
    &CELL
      ABC 10.00 10.00 40.0
    &END CELL
    #input coords from an equilibrated structure 
    &COORD
      @INCLUDE 'pos.inc'  #pos.inc is the coordinates of the system
    &END COORD
    &KIND H
      BASIS_SET DZVP-MOLOPT-SR-GTH
      POTENTIAL GTH-BLYP-q1
    &END KIND
    &KIND O
      BASIS_SET DZVP-MOLOPT-SR-GTH
      POTENTIAL GTH-BLYP-q6
    &END KIND
    &KIND I 
      BASIS_SET DZVP-MOLOPT-SR-GTH
      POTENTIAL GTH-BLYP-q7
    &END KIND
    &KIND Na 
      BASIS_SET DZVP-MOLOPT-SR-GTH
      POTENTIAL GTH-BLYP-q9                        
    &END KIND
  &END SUBSYS
&END FORCE_EVAL
&GLOBAL
  PROJECT 63wat
  RUN_TYPE MD        
  PRINT_LEVEL LOW
&END GLOBAL

&MOTION
  &GEO_OPT
    TYPE minimization
    OPTIMIZER BFGS
    MAX_ITER 20
  &END GEO_OPT

  &MD
    &THERMOSTAT
      &NOSE
        LENGTH 3
        YOSHIDA 3
        TIMECON 1000.
        MTS 2
      &END NOSE
    &END THERMOSTAT
    ENSEMBLE NVT
    STEPS 1000000
    TIMESTEP 0.5
    TEMPERATURE 300.0
    TEMP_TOL 60
  &END MD

  &PRINT
   &TRAJECTORY
     &EACH
       MD 1
     &END EACH
   &END TRAJECTORY
   &VELOCITIES ON
     &EACH
       MD 1
     &END EACH
   &END VELOCITIES
   &FORCES ON
     &EACH
       MD 1
     &END EACH
   &END FORCES

   &RESTART_HISTORY
     &EACH
       MD 1000
     &END EACH
   &END RESTART_HISTORY
   &RESTART
     BACKUP_COPIES 1
     &EACH
       MD 1
     &END EACH
   &END RESTART
  &END PRINT
&END MOTION
 \end{lstlisting}

 \section{Structural Characterization of Water Clusters}\label{structure_of_cluster}
 The structural parameters of the considered water clusters are shown in this section.
Table ~\ref{tab:3_nitrate_bond} gives the calculated lengths of H-bonds in the cluster NO$_3^-$(H$_2$O)$_3$. $r_a$ denotes the average distances (with the standard deviation) of the length of H bonds. 
 In Table ~\ref{tab:table_geo_opt} (Table ~\ref{tab:3w_nitrate}) selected distances characterizing 
 the RNO$_3$(H$_2$O)$_3$ (NO$_3^-$(H$_2$O)$_3$) clusters are reported.
 In Table ~\ref{tab:table_rnitrate_3w} selected geometrical parameters are reported for the RNO$_3$   
 (H$_2$O)$_3$ (R=Li, Na, K) clusters as averaged over 2 ps DFTMD trajectory at 300 K.
% 
\begin{table}[!h]
\centering
\caption{\label{tab:3_nitrate_bond}%
  Lengths of H-bonds ($r_a$) in each water molecule of the cluster NO$_3^-$(H$_2$O)$_3$.} 
\begin{tabular}{cc} \\\toprule
 water & \multicolumn{1}{c}{ $r_a$}(\A)\\
\hline
 w1 &2.40$\pm$0.52; 3.02$\pm$0.72 \\
 w2 &2.56$\pm$0.48; 3.20$\pm$0.41 \\
 w3 &2.29$\pm$0.47; 3.11$\pm$0.72
\end{tabular}
\end{table}
%
\begin{table}[!htbp]
\centering
\caption{\label{tab:table_geo_opt}%
The structural parameters of RNO$_3$(H$_2$O)$_3$ from geometry optimization. 
The unit for length and angle are \AA and degree ($^\circ$), respectively.}
\begin{tabular}{l*{4}ccc}
Parameters  & LiNO$_3$(H$_2$O)$_3$& NaNO$_3$(H$_2$O)$_3$ & KNO$_3$(H$_2$O)$_3$\\
\hline
$r_\text{HB1}$& 1.67 & 1.71 & 1.82 \\
$r_\text{HB2}$& 1.91 & 1.78 & 1.92\\
$r_\text{HB3}$& 1.82 & 1.69 & 1.94\\
$r_\text{R-O(w1)}$ & 1.91 & 2.31 & 2.70\\
$r_\text{R-O(w2)}$ & 1.90 & 2.26 & 2.70\\
$r_\text{R-O(\nitrate)}$ & 1.84 & 2.29 & 2.69 \\
$\angle$HOH(w1)& 109 & 106 &107 \\
$\angle$HOH(w2)& 106 & 105&105 \\
$\angle$HOH(w3)& 108 & 107 &106
\end{tabular}
\end{table}
%
\begin{table}[!htbp]
\centering
\caption{\label{tab:3w_nitrate}%
The parameters of water molecules and HBs in NO$_3^-$(H$_2$O)$_3$ at 300 K.}
\begin{tabular}{lccc}
water &$R_\text{OH}$ &$\angle$HOH ($^\circ$) & $r_\text{OH}$ \\
\hline
w1 &0.98$\pm$0.02 &101$\pm$4 & 2.40$\pm$0.52, 3.02$\pm$0.72 \\
w2 &0.98$\pm$0.02 &101$\pm$5 & 2.56$\pm$0.48, 3.20$\pm$0.41 \\
w3 &0.98$\pm$0.02 &101$\pm$4 & 2.29$\pm$0.47, 3.11$\pm$0.72
\end{tabular}
\end{table}
%
\begin{table}[!htbp]
\centering
\caption{\label{tab:table_rnitrate_3w}%
The parameters of RNO$_3$(H$_2$O)$_3$. For RNO$_3$(H$_2$O)$_3$, $R_\text{OH}$ and $R'_\text{OH}$ 
  denote the lengths of O-H bonds in which H atoms is H-bonded and is free, respectively.
  The unit for length and angle are \AA and degree ($^\circ$), respectively.
  }
\begin{tabular}{l*{4}ccc}
Parameters & LiNO$_3$(H$_2$O)$_3$& NaNO$_3$(H$_2$O)$_3$ & KNO$_3$(H$_2$O)$_3$\\
\hline
$r_\text{HB1}$ & $1.83\pm0.14$ & $1.78\pm0.09$ & $1.82\pm0.13$\\
$r_\text{HB2}$ & $2.00\pm0.25$ & $1.91\pm0.24$ & $1.80\pm0.12$\\
$r_\text{HB3}$ &$1.79\pm0.16$ & $1.76\pm0.11$ & $1.89\pm0.18$\\
$R_\text{OH}$(w1) &$0.97\pm0.01$ &$0.98\pm0.04$ &$0.97\pm0.03$ \\
$R'_\text{OH}$(w1) &$1.00\pm0.02$ &$1.00\pm0.02$ & $1.00\pm0.03$ \\
$R_\text{OH} $(w2) &$0.97\pm0.01$ &$0.98\pm0.02$ &$0.97\pm0.02$ \\ 
$R'_\text{OH}$(w2) &$0.99\pm0.01$ &$1.00\pm0.02$ & $1.00\pm0.03$ \\
$R_\text{OH}$(w3) &$0.97\pm0.01$ & $0.97\pm0.02$&$0.97\pm0.03$ \\
$R'_\text{OH}$(w3) &$1.00\pm0.02$ &$1.00\pm0.02$ & $1.00\pm0.03$ \\
$r_\text{R-O(w1)}$ & $1.95\pm0.09$ & $2.34\pm0.08$ & $2.76\pm0.11$\\
$r_\text{R-O(w3)}$ & $1.92\pm0.07$ & $2.32\pm0.11$ & $2.74\pm0.13$\\
$r_\text{R-O(\nitrate)}$ & $1.91\pm0.08$ & $2.31\pm0.09$ & $2.74\pm0.12$ \\
$\angle$HOH (w1) &$107\pm4$ & $106\pm4$ &$105\pm5$ \\
$\angle$HOH (w2) &$106\pm6$ & $105\pm4$ &$106\pm4$ \\
$\angle$HOH (w3) &$108\pm5$ & $106\pm3$ &$106\pm3$ 
\end{tabular}
\end{table}
%--

\section{Calculation of Molecular Dipole Moment Derivative and Dipole Polarizibility Derivative} \label{CHAPETR_A_derivatives} 
We describe the main procedure to implement the parametrization of the molecular dipole moment 
derivative $\frac{\partial \mu_\kappa}{\partial r_z}$ and dipole polarizibility derivatives $\frac{\partial\alpha_{\eta\xi}}{\partial r_z}$. 

Given a DFT MD trajectory of total length $\sim 10^2$ ps for bulk water, sampled in a frequency $\sim 1$ ps$^{-1}$.
For the $j$-th water molecule in the $n$-th snapshot of the trajectory, 
we denote the two OH bonds as H$^{n,j,\epsilon=1}$ and H$^{n,j,\epsilon=-1}$. We want to calculate statistical average of the
derivatives over all time steps and all OH bonds, therefore, 
we just denote the corresponding OH bonds by ${\epsilon=1}$ and ${\epsilon=-1}$, respectively. 
The H atoms in a water molecule are denoted by H$^{\epsilon=1}$ and H$^{\epsilon=-1}$ respectively, and the O atom by O$^{0}$.
%For the bond OH$_{\epsilon}$, we calculate vector OH$_{\epsilon}$, and $|\text{OH}_{\epsilon}|$ in the lab framework,
%then the direction cosines ($\cos\alpha_{\epsilon}$, $\cos\beta_{\epsilon}$, $\cos\gamma_{\epsilon}$) of the vector OH$_{\epsilon}$ in the molecular frame MF$_{n,j}$.
%The size of the array to store the direction cosines is $ 40\times 32\times 3 \times 3$.
%Calculate the direction cosine matrix between bond framework and molecular framework. 

There are two direction cosine matrices, we name them as ${\bf D}^{\text{b},\epsilon=-1}$ and  ${\bf D}^{\text{b},\epsilon=1}$. \cite{Khatib2017},
or ${\bf D}^{\text{b},-1}$ and  ${\bf D}^{\text{b},1}$ for short, where ${\bf D}^{\text b}$ denotes the direction cosine matrix between 
bond framework and molecular framework.
Then the direction matrix ${\bf D}^{\text{b},\epsilon} $ can be represented by direction cosines between $({\bf x}^{\text{b},\epsilon}$ and $({\bf x}^m)$, 
where $\epsilon=\pm 1$ and $\theta$ denotes the H-O-H angle in the $j$-th water molecule for the $n$-step:
\[\begin{sistema}
  \hat x^{\text{b},\epsilon}_1 =\ \ \epsilon \cos\frac{\theta}{2} \ \hat x^m_1 + 0\ \hat x^m_2 + \sin\frac{\theta}{2} \ \hat x^m_3 \\
  \hat x^{\text{b},\epsilon}_2 =\ \ \ \ \ \ \ \ 0\ \hat x^m_1   +  \ \epsilon\ \hat x^m_2 \ + \ \ \ 0 \ \hat x^m_3 \\
  \hat x^{\text{b},\epsilon}_3 = -\epsilon \sin\frac{\theta}{2} \ \hat x^m_1 + 0 \ \hat x^m_2 + \cos\frac{\theta}{2} \ \hat x^m_3
\end{sistema}\]
%\begin{subequations}
%\begin{align}
%  &(\hat x^{\text{b},\epsilon}_1) = \epsilon \cos\frac{\theta}{2} (\hat x^m_1) + 0 (\hat x^m_2) + \sin\frac{\theta}{2} (\hat x^m_3) \\
%  &(\hat x^{\text{b},\epsilon}_2) = 0 (\hat x^m_1) + \epsilon (\hat x^m_2) + 0 (\hat x^m_3) \\
%  &(\hat x^{\text{b},\epsilon}_3) = -\epsilon \sin\frac{\theta}{2} (\hat x^m_1) + 0 (\hat x^m_2) + \cos\frac{\theta}{2} (\hat x^m_3)
%\end{align}
%\end{subequations}
i.e., the cosine matrices are
\begin{equation}
  {\bf D}^{\text{b},\epsilon} =\left(
  \begin{matrix}
    \epsilon\text{cos}\frac{\theta}{2} &  0  & \text{sin}\frac{\theta}{2}\\
    0 & \epsilon & 0\\
    -\epsilon\text{sin}\frac{\theta}{2} & 0 & \text{cos}\frac{\theta}{2}
  \end{matrix}
  \right).
\end{equation}
  The molecular framework is given by the direction cosine matrix ${\bf D}^\text{m,l}$ (or ${\bf D}^\text{m}$) between molecular framework  and the lab framework
  %How to determine  $({\bf D}^\text{m,l})$?  
%Since we can know the coordinates of $(\hat x^m_3)$ in the lab framework: 
%\begin{equation}
%(\hat x^m_3) = (a_{31}) \hat x^l_1 + (a_{32}) \hat x^l_2 + (a_{31})_{nj} \hat x^l_3,
%\end{equation}
%Therefore, the relation between the unit vector  $(\hat x^m_1)$ and $(\hat x^m_2)$ are
%
\begin{subequations}
\begin{align}
  \hat {\bf x}^m ={\bf D}^\text{m} \hat{\bf x}^l.
\end{align}
\end{subequations}
%i.e., we obtain  $({\bf D}^\text{m})$.

% 2a1c.
%\begin{subequations}
%  \begin{align}\left(
%  \begin{matrix}
%  \{\bf D} x^{\epsilon} \\
%  \Delta y^{\epsilon} \\
%  \Delta z^{\epsilon} 
%  \end{matrix}
%    \right)
%    =\Delta r \left(
%  \begin{matrix}
%  \cos\alpha^{\epsilon}\\
%  \cos\beta^{\epsilon}\\
%  \cos\gamma^{\epsilon}
%  \end{matrix}
%    \right)
%\end{align}
%\end{subequations}
%i.e., $\Delta {\bf r}^{\epsilon} =(\Delta x^{\epsilon}, \Delta y^{\epsilon} ,\Delta z^{\epsilon} )^\text{T}$.
%        
%2a2.
%\begin{subequations}
%\begin{align}
%  &   (x'_H)_{\epsilon} -(x_O)_{0} = (x_H)_{\epsilon} -(x_O)_{0} + \Delta  x_{\epsilon} \\
%  &   (y'_H)_{\epsilon} -(y_O)_{0} = (y_H)_{\epsilon} -(y_O)_{0} + \Delta  y_{\epsilon} \\
%  &   (z'_H)_{\epsilon} -(z_O)_{0} = (z_H)_{\epsilon} -(z_O)_{0} + \Delta  z_{\epsilon}
%\end{align}
%\end{subequations}
%
%2a4.We need to calculate the dipole moment of each molecule for $\Delta r \in [-0.05,0.05]$. 
The dipole moment of each OH bond with different length is required to determine the dipole moment derivative. 
    %(For convenient, we have defined $ \text{inc} =\Delta r/|\Delta r|$.)
    %We elongate (reduce) the bond OH$_{n, j,\epsilon}$, and keep other bonds still, then we obtain a updated coordinate file ${\text{pos}}nj\epsilon\text{inc}.\text{xyz}$, where $(n, j,\epsilon)$ is the index of the H in OH$_{\epsilon}$.
Therefore, we elongate (reduce) one bond ${\epsilon}$ by $\Delta r \in \{-0.05 $\AA, $0.05 $\AA $\}$, and keep other 
bonds in the total system still, then we obtain a updated coordinate.
Then the MLWF centers for the system can be calculated from the updated coordinate, using force and energy calculation at the DFT level.
%which is represented by the coordinate file ${\text{pos}}nj\epsilon\text{inc}.\text{xyz}$.
    %2a5a: before next step, we generated a template input file 1.inp. 
    %2a5b: From the template input file 1.inp, we generate an input file for the disturbed position represented by ${\text{pos}}nj\epsilon\text{inc}.\text{xyz}$ with the elongated bond OH$_{\epsilon}$: $n j \epsilon\text{inc}.\text{inp}$
    %2a5c: Using DFTMD, we use the input file $nj\epsilon\text{inc}.\text{inp}$ to calculate and write out the Wannier centers for the system  ${\text{pos}}n j \epsilon\text{inc}.\text{xyz}$. The output: ${\text{HOMO}}n j \epsilon \text{inc}.\text{xyz}$.
    %2a6: From the Wannier centers of the water molecule (can be found in the output ${\text{HOMO}}nj \epsilon \text{inc}.\text{xyz}$), we can calculate the dipole moment for the elongated (reduced) bond OH$_{\epsilon}$.  The $k$ component of the dipole moment: $(\mu^{b,r+\Delta r}_k)_{\epsilon}$.
    From the Wannier centers of the $j$-th water molecule, we can calculate the dipole moment for the elongated (reduced) bond ${\epsilon}$. 
    %Therefore, the $k$ component of the dipole moment $(\mu^{b,r+\Delta r}_k)_{\epsilon}$ can be obtained.
    In the bond frame, $|\mu^b| =|\mu^b_z|$. 
    Therefore, we can calculate the $k=z$ component of the dipole moment for ${\epsilon}$ in water molecule $j$ from the MLWF centers.\cite{Silvestrelli1999} 
      The MLWF centers are computed\cite{Silvestrelli1999} and the partial dipole moment for a given molecular species $I$ is defined as\cite{Salanne08}
      %
      \begin{equation}
        \mu^I = \sum_{i\in I}(Z_i {\bf R}_i - 2\sum_{n\in i} {\bf r}^w_n).
      \end{equation}
       %
In particular, here it is expressed as 
%\begin{equation}
%  \mu^{\text{b},r+\Delta r,\epsilon} = \frac{1}{2}Z_O {\bf R}^{0} +Z_H{\bf R}^{\epsilon} +1\times (-2) \times r^{w,\epsilon} +\frac{1}{2}\times 2\times (-2)\times r^{w,0},
%\end{equation}
\begin{equation}
  \mu^{\text{b},r+\Delta r,\epsilon} = \frac{1}{2}Z_O {\bf R}^{0} +Z_H{\bf R}^{\epsilon} -2r^{w,\epsilon} -2r^{w,0},
\end{equation}
where $\epsilon=\pm 1$.
%3c: 
In the $\epsilon$ frame, the $k=z$ component dipole moment derivative for bond $\epsilon$ in water molecule $j$ is
        %
        \begin{equation}
          \frac{\partial \mu^{\text{b},\epsilon}}{\partial r} = (\mu^{\text{b},r+|\Delta r|,\epsilon}-\mu^{\text{b},r-|\Delta r|,\epsilon})/2|\Delta r|
        \end{equation}
        %3d
        Using ${\bf D}^{\text{b},\epsilon}$, we can finally represent this vector as $( 0, 0, \frac{\partial \mu^{\text{b},\epsilon}}{\partial r} )^\text{T}$.
        Since
        \begin{subequations}
          \begin{align}
            \left(
            \begin{matrix}
              (\frac{\partial \mu^\text{m}}{\partial r})_1\\
              (\frac{\partial \mu^\text{m}}{\partial r})_2\\
              (\frac{\partial \mu^\text{m}}{\partial r})_3
            \end{matrix}
            \right)
            = {\bf D}^{\text{b},\epsilon}
            \left(
            \begin{matrix}
              0\\
              0\\
              \frac{\partial \mu^{\text{b},\epsilon}}{\partial r}
            \end{matrix}
            \right).
            \end{align}
        \end{subequations}
    % END of r-loop 
    Then, to calculate the individual polarizability for a OH bond from Wannier centers, 
    calculations involving finite electric fields of 0.0001 au intensity were performed independently 
    along the $x$, $y$, and $z$ directions at each sampled time step.\cite{sulpizi2013}
        %4a: 
    For the electric field $\varepsilon \in \{\varepsilon_x,\varepsilon_y, \varepsilon_z\}$,
            %similar to 2a5a, but add an electric filed $ \varepsilon$. %The template input file is: 2.inp.
            %4a1: from the template input file, generate a input file $nj\epsilon\varepsilon.\text{inp}$, where if extra electric filed $\varepsilon=\varepsilon_x$, that means we add a electric field along $x$ axis. Similar for $\varepsilon_y$ and $\varepsilon_z$.
            %4a2: similar to 2a5c, for the position file ${\text{pos}}nj\epsilon.\text{xyz}$, the output Wannier centers are in   ${\text{HOMO}}nj\epsilon\varepsilon.\text{xyz}$.
        Similar to the case of no external electric field, the MLWF centers are calculated. %in ${\text{HOMO}}nj\epsilon\varepsilon.\text{xyz}$.
 For $\Delta r \in \{-0.05 $\AA, $0.05 $\AA $\}$, the dipole moment is given by 
%\begin{equation}
%  \mu^{b,r+\Delta r,\epsilon,\varepsilon} = Z_H{\bf R}^{\epsilon,\varepsilon} + 
%  \frac{1}{2}Z^0 {\bf R}^{0,\varepsilon} + 
%  1\times (-2) \times {\bf r}^{\text{w},\epsilon,\varepsilon} +\frac{1}{2}\times 2\times (-2)\times ({\bf r}^\text{w})^{0,\varepsilon}
%\end{equation}
\begin{equation}
  \mu^{b,r+\Delta r,\epsilon,\varepsilon} = Z_H{\bf R}^{\epsilon,\varepsilon} + 
  \frac{1}{2}Z^0 {\bf R}^{0,\varepsilon} -2 {\bf r}^{\text{w},\epsilon,\varepsilon} -2{\bf r}^{\text{w},0,\varepsilon}.
\end{equation}
        %END $ \Delta r$-loop  
%4a4 the polarizability tensors $\alpha^{b,r+|\Delta r|}$ and $\alpha^{b,|r-\Delta r|}$. 
From the relation
\[\begin{sistema}
\ \ \ \ 0\ \ = (\alpha^{b,r+\Delta r})_{11}\varepsilon_{1} + (\alpha^{b,r+\Delta r})_{12}\varepsilon_{2} + (\alpha^{b,r+\Delta r})_{13}\varepsilon_{3}\\
\ \ \ \ 0\ \ = (\alpha^{b,r+\Delta r})_{21}\varepsilon_{1} + (\alpha^{b,r+\Delta r})_{22}\varepsilon_{2} + (\alpha^{b,r+\Delta r})_{23}\varepsilon_{3}\\
(\delta \mu^{b,r+\Delta r})_{3} = (\alpha^{b,r+\Delta r})_{31}\varepsilon_{1} + (\alpha^{b,r+\Delta r})_{32}\varepsilon_{2} + (\alpha^{b,r+\Delta r})_{33}\varepsilon_{3}.
\end{sistema}\]
%      \begin{subequations}
%          \begin{align}
%            & 0 = (\alpha^{b,r+\Delta r})_{11}\varepsilon_{1} + (\alpha^{b,r+\Delta r})_{12}\varepsilon_{2} + (\alpha^{b,r+\Delta r})_{13}\varepsilon_{3}\\
%            & 0 = (\alpha^{b,r+\Delta r})_{21}\varepsilon_{1} + (\alpha^{b,r+\Delta r})_{22}\varepsilon_{2} + (\alpha^{b,r+\Delta r})_{23}\varepsilon_{3}\\
%            & (\delta \mu^{b,r+\Delta r})_{3} = (\alpha^{b,r+\Delta r})_{31}\varepsilon_{1} + (\alpha^{b,r+\Delta r})_{32}\varepsilon_{2} + (\alpha^{b,r+\Delta r})_{33}\varepsilon_{3}.
%          \end{align}
%      \end{subequations}
%
where
\begin{equation}
  (\delta \mu^{b,r+\Delta r})_{3} = \mu^{b,\epsilon,r+\Delta r,\varepsilon_x}_{3} - \mu^{b,\epsilon,r+\Delta r}_{3}.
\end{equation}
%
and the expressions of the electric filed ${\varepsilon^\text{b}}$ in a OH framework for the 3 cases of external electric field 
which is along $x$, $y$ and $z$ in lab framework, respectively, we obtain 9 equations.
%
For
\begin{equation}
  \varepsilon^{\text{l}} = (\varepsilon_0, 0,0)^T \nonumber
\end{equation}  
where $\varepsilon_0 = 0.0001$ au, we can obtain the intensity of the external electric filed in the molecular framework  
\begin{subequations}
  \begin{align}
    &\varepsilon^\text{m}_1 = {\bf D}^{\text{m}}_{12} \varepsilon^\text{l}_2\\
    &\varepsilon^\text{m}_2 = {\bf D}^{\text{m}}_{22} \varepsilon^\text{l}_2\\
    &\varepsilon^\text{m}_3 = {\bf D}^{\text{m}}_{32} \varepsilon^\text{l}_2,
    \end{align}
\end{subequations}
where ${\bf D}^{\text{m}}_{pq}$ is the $pq$-component of ${\bf D}^\text{m}$.
%
In OH bond framework,  
\begin{equation}
  \varepsilon^{\text b} = {\bf D}^{\text b} \varepsilon^{\text{m}}.
\end{equation}
Similarly, we obtain similar but different expansions of the intensity of the electric field for the other two cases: 
when $\varepsilon^{\text{l}} = (0,\varepsilon_0, 0)^\text{T}$ and when $\varepsilon^{\text{l}} = (0,0,\varepsilon_0)^\text{T}$, respectively.
Here, $\varepsilon_x$ is the electric field along $x$-axis in the lab frame. 
    %4a5: 
    The dipole polarizability for the bond ${\epsilon}$ is as follows:
\begin{subequations}
  \begin{align}
    &\frac{\partial \alpha^{b,\epsilon}_{31}}{\partial r} = (\alpha^{b,\epsilon,r+|\Delta r|}_{31} -\alpha^{b,\epsilon,r-|\Delta r|}_{31})/2|\Delta r|\\
    &\frac{\partial \alpha^{b,\epsilon}_{32}}{\partial r} = (\alpha^{b,\epsilon,r+|\Delta r|}_{32} -\alpha^{b,\epsilon,r-|\Delta r|}_{32})/2|\Delta r|\\
    &\frac{\partial \alpha^{b,\epsilon}_{33}}{\partial r} = (\alpha^{b,\epsilon,r+|\Delta r|}_{33} -\alpha^{b,\epsilon,r-|\Delta r|}_{33})/2|\Delta r|.
  \end{align}
\end{subequations}

At last, the average for $(\frac{\partial \mu^{\text{b},\epsilon}}{\partial r})_\kappa$ and $(\frac{\partial \alpha^{\text{b},\epsilon}}{\partial r})_{\eta\xi}$ 
(The subscripts $\kappa,\eta,\xi = x^m, y^m, z^m$, or 1,2,3) over all OH bonds gives the molecular dipole and polarizability derivatives. 

\section{Thickness of the Interface of the Alkali Solutions} \label{thickness_interface}
To check the scale of the interface of alkali solutions, we chose several different thickness values 
of slab of the interface and calculate the corresponding susceptibility for these slabs, respectively.
Take water/vapor interface of LiNO$_3$ solution as an example. We chose 7 different thickness 
values---$d=2,3,4,5,6,7,8$ \A, and calculate VSFG intensities $I_{SSP} \propto |\chi^{(2),R}_{SSP}|^2$ 
for the water/vapor interface with a thickness of each of these values. 
The result is given in Fig.~\ref{fig:117_2LiNO3_30ps_2-6A_150_Im_150720} (a) and (b).
It shows that $|\chi^{(2),R}_{SSP}|^2$ converges as the thickness increase to 8 \A. 
%
\begin{figure}[!h]
\centering
\includegraphics [width= \textwidth] {./diagrams/117_2LiNO3_30ps_2-6A_150_Im_150720}
\setlength{\abovecaptionskip}{0pt}
\caption{\label{fig:117_2LiNO3_30ps_2-6A_150_Im_150720} The calculated $|\chi^{(2)}_{SSP}|^2$, 
of water molecules at aqueous/vapor interfaces with different thickness. This calculation is done for a model 
 for water/vapor interface where a slab of 117 water molecules containing one \Li and one \nitrate is included 
 in a period simulation box of 15.6 \AA$\times$15.6 \AA$\times$31.0 \AA at 300 K.}
\end{figure}
%
%As the limitation of the box scale, we chose 4 different thickness values---$d=2,3,4,5$ \A, for checking. The VSFG intensities $I_{SSP} \propto |\chi^{(2),R}_{SSP}|^2$ for the water/vapor interface of the LiI solution are given in Fig.~\ref{fig:124_2LiI_30ps_2-5A_150_Im_150720} (a) and (b). The results show that $|\chi^{(2),R}_{SSP}|^2$ increases as the thickness increase from 2 \AA to 5 \A. 
%\begin{figure}[!h]
%\centering
%  \includegraphics [width=0.8 \textwidth] {./diagrams/124_2LiI_30ps_2-5A_150_Im_150720}
%\setlength{\abovecaptionskip}{0pt}
%\caption{\label{fig:124_2LiI_30ps_2-5A_150_Im_150720} The calculated $|\chi^{(2),R}_{SSP}|^2$, 
%  of water molecules at water/vapor interfaces with different thickness. This calculation is done for a model 
%  for water/vapor interface where a slab of 118 water molecules containing one \Li and one \I is included 
%  in a period simulation box of 15.6 \AA $\times$ 15.6 \AA $\times$ 15.6 \AA.}
%\end{figure}
%%

The thickness of the the water/vapor interface of solution can also be seen from the calculated RDF. 
For example, the RDFs of interface of the NaI solution is shown in Fig.\space\ref{fig:2NaI-124w_gdr_OH_s2_150122}.
It indicates that the thickness of the water/vapor interface is about 8 \A.

\begin{figure}[!h]
\centering
\includegraphics [width=0.75\textwidth] {./diagrams/2NaI-124w_gdr_OH_s2_150122}
\setlength{\abovecaptionskip}{0pt}
\caption{\label{fig:2NaI-124w_gdr_OH_s2_150122}The RDF $g_{\text{OH}}(r)$  in the NaI solution/vapor interface. 
The simulation system includes one \Na ion, one \I ion, and 124 water molecules in 15.6 \AA $\times$ 15.6 \AA $\times$ 31.0 \AA 
  simulation box. The solid and dashed lines are corresponding to one of the two interfaces in our simulation, respectively. 
  The simulation time is 22.5 ps.}
\end{figure}
%

%
Furthermore, we can also estimate the thickness of a water/vapor interface by calculating the HB dynamics for it. 
Fig.\thinspace\ref{fig:2LiI-124w_S_layers} shows the dependence of the logarithm of survival probability on the thickness of the water/vapor interface 
of the LiI solution. We see that, when $d = 8$ \AA the $\ln S_{\text{HB}}$ of the interface does not change  
as the thickness increase, indicating the thickness of the interface. 
\begin{figure}[htpb]
\centering
\includegraphics [width=0.7\textwidth] {./diagrams/2LiI-124w_S_layers} %fig.5.9
\setlength{\abovecaptionskip}{0pt}
\caption{\label{fig:2LiI-124w_S_layers}The logarithms of the function \SHB for water-water H-bonds at interfaces with different thickness
  in 0.9 M LiI solution. $\kappa$=0.16. }
\end{figure}

%To make sure we start from the right configuration in the liquid/vapor interface system and to explain 
%the calcualted VDOS of the water molecules in the interface,we study the interface from the view of free energy.
%
%For a NVT ensemble, the most stable state is the one with the lowest free energy. We will find out 
%different configurations which have minimum free energies.

\section{Dipole Orientation Distribution at Water/Vapor Interfaces}
\paragraph{Testing the Program for Calculating Dipole Orientation Distribution}
To test the program, we use the trajectory of the water molecules calculated by MD to calculate the dipole tilt angle.
%
\begin{figure}
\centering
\includegraphics [width=0.6\textwidth]{./diagrams/dipole_orien_128w}   % Here is how to import EPS art
\setlength{\abovecaptionskip}{10pt}
\caption{\label{fig:bulk} The dipole moment orientation for water molecules in bulk water. The number of bins $n=72$.}
\end{figure}
%
Fig.\space\ref{fig:bulk} shows the dipole moment orientation distribution of all water molecules in the model.  
It is a sine-like function.
We assume that the system is isotropic, then the dipole moment will be distributed uniformly in any direction ($\theta,\phi$). The probability density
$P(\theta,\phi)=1/{4\pi}$.The probability that the dipole moment is point
to the area element $\text{d}\theta\text{sin}\theta\text{d}\phi$ in the direction ($\theta,\phi$) is 
\begin{equation}
P(\theta,\phi)\text{d}\theta\text{sin}\theta\text{d}\phi\nonumber,
\label{eq:6}
\end{equation}
thus the probability that the dipole moment is pointing to the direction in the internal ($\theta,\theta+\text{d}\theta$) is
\begin{equation}
P(\theta)\text{d}\theta=\frac{1}{4\pi}\int_0^{2\pi}\text{d}\phi\text{d}\theta\text{sin}\theta,\nonumber
\label{eq:7}
\end{equation}
i.e., the probability density is $P(\theta)=\frac{1}{2}\text{sin}\theta$. From Fig.\space\ref{fig:bulk}, we find that the resulting $P(\theta)$ is indeed a sine function. If we have chosen $n$ bins to plot the histogram, then the width of each bin is ${2\pi}/n$, and the maximum will be
\begin{equation}
P(\pi/2)=1\times\frac{2\pi}{n}.
\label{eq:8}
\end{equation}
For example, if $n=72$, the maximum probability should be 
$P(\pi/2)=0.087$ (Fig. ~\ref{fig:bulk}).
Our simulation fits this result well. So our program for calculating the dipole moment orientation works well.

For a longer MD trajectory, we repeat the calculation. Thus we have the comparison of the dipole orientation distribution between 
the long and the short MD trajectories.

Now we calculate the OH group orientation distribution of the water molecules with similar method.
This is a natural result of fluctuations. The main reason for no probability in small $\theta$'s is
that here we did not consider the integration of the angle $\phi$ and that the sample points is small, i.e., less than 100 values.

\section{HB Population Operator Auto-Correlation Function}
%!function: hbacf2.f95
\begin{lstlisting}[language=fortran]                  ]
      PROGRAM hbacf
      IMPLICIT NONE 

      !Purpose: For any pair of water molecules, calculate functions S_HB(t), S_d_HB(t), C_HB(t) and C_d_HB(t)
      
      !=====================
      !parameters dictionary
      !=====================
      CHARACTER(LEN=30) :: filename,pos_filename,list_filename 
      INTEGER,PARAMETER :: RK=4              
      INTEGER,PARAMETER :: NMAX=2001 ! max number of atoms
      INTEGER,PARAMETER :: NMO_MAX=300000 ! max number of movie
      REAL(KIND=RK),PARAMETER :: RATE=0.50 ! condition for cutting off auto-correlation functions
      REAL,PARAMETER :: ROOC=29.16 ! cutoff distance of rOO (5.4**2 )
      REAL,PARAMETER :: ROHC=12.25 ! rOH (3.5**2)
      REAL,PARAMETER :: COSPHIC=0.866 ! 1.732/2; phiC=pi/6.
      REAL(KIND=RK),PARAMETER :: HB_MIN=0.0000001 ! condition for the existence of h-bond
      REAL(KIND=RK),PARAMETER :: DELTA_T=0.01 ! ps
      REAL(KIND=RK) :: r12,r13,r23,cosphi,pm,qj,qj_d,tot_hb
      INTEGER :: i,j,k,jj,nmo,nat,iatom,imovie,np,m1,m2,m3,mt 
      REAL(KIND=RK),ALLOCATABLE,DIMENSION (:) :: h,h_d
      REAL(KIND=RK),ALLOCATABLE,DIMENSION (:) :: hb,hb_d
      REAL,ALLOCATABLE,DIMENSION (:,:) :: x,y,z
      CHARACTER(LEN=3) :: atom_type  
      INTEGER,ALLOCATABLE,DIMENSION(:) :: ndx_1, ndx_2, ndx_3
      REAL(KIND=RK),ALLOCATABLE,DIMENSION (:) :: corr_h, corr_h_d
      REAL(KIND=RK) :: scalar, scalar_d 
      
      !==========
      !read data
      !==========
      WRITE(6,*) 'The name of the system:'
      READ(5,*) fiename
      WRITE(6,*) 'The name of the trajecotry file:'
      READ(5,*) pos_filename     
      WRITE(6,*) 'The name of the input file:'
      READ(5,*)  list_filename    
      WRITE(6,*) 'The total steps of the trajecotry:'
      READ(5,*) nmo !number of movie steps
      WRITE(6,*) 'The total number of atoms in the system:'
      READ(5,*) nat !number of atoms per mole.
      WRITE(6,*) 'The total number of water pairs:'
      READ(5,*) np !number of pairs   

      ALLOCATE(ndx_1(np))          
      ALLOCATE(ndx_2(np))          
      ALLOCATE(ndx_3(np))          

      IF (nmo) >= nmo_max .OR. nat >= nmax) THEN
        WRITE(6,*)'!! nmo > nmo_max or nat > nmax !!'
        STOP
      ENDIF
      
      list_filename=trim(list_filename)
      OPEN(10,file=list_filename)     
      DO k = 1,np
          READ(10,*)ndx_1(k),ndx_2(k),ndx_3(k)
      ENDDO
      CLOSE(10)

      ALLOCATE(x(nat,nmo))
      ALLOCATE(y(nat,nmo))
      ALLOCATE(z(nat,nmo))
      ALLOCATE(h(nmo))
      ALLOCATE(h_d(nmo))
      ALLOCATE(hb(np))  !Average H-bonded population 
      ALLOCATE(hb_d(np))
      
      !=======================
      !read in trajectory file 
      !=======================
      OPEN(10,file=trim(pos_filename))     
      DO imovie = 1,nmo
         READ(10,*)  !Neglect data of this line
         READ(10,*)                 
         DO iatom = 1,nat
             READ (10,*)atom_type,x(iatom,imovie),y(iatom,imovie),&
                        z(iatom,imovie)
         ENDDO
      ENDDO
      CLOSE(10)
      ALLOCATE(corr_h(nmo))
      ALLOCATE(corr_h_d(nmo))

      !==================================
      !Calculate autocorrelation function
      !C_HB(t) and C^d_HB(t)  
      !==================================
      
      tot_hb = 0.0
      corr_h = 0.0, corr_h_d = 0.0
      hb = 0.0, hb_d = 0.0

      !--------------
      !Calculate h(j)
      !--------------
      DO k = 1,np
        qj = 0
        qj_d = 0
        m1 = ndx_1(k)
        m2 = ndx_2(k)
        m3 = ndx_3(k)
        
        DO jj = 1, nmo
          h(jj) = 0
          h_d(jj) = 0 
          r13 = (x(m1,jj)-x(m3,jj))**2+       &
                    (y(m1,jj)-y(m3,jj))**2+  &
                    (z(m1,jj)-z(m3,jj))**2  !r: square of distances
          r12 = (x(m1,jj)-x(m2,jj))**2+       &
                    (y(m1,jj)-y(m2,jj))**2+  &
                    (z(m1,jj)-z(m2,jj))**2
          r23 = (x(m2,jj)-x(m3,jj))**2+       &
                    (y(m2,jj)-y(m3,jj))**2+  &
                    (z(m2,jj)-z(m3,jj))**2
          pm = (x(m3,jj)-x(m2,jj))*           &
                   (x(m1,jj)-x(m2,jj))+      & 
                   (y(m3,jj)-y(m2,jj))*      & 
                   (y(m1,jj)-y(m2,jj))+      & 
                   (z(m3,jj)-z(m2,jj))*      &
                   (z(m1,jj)-z(m2,jj)) 
          cosphi = pm/(sqrt(r23*r12)) !pm: point multiplication.
          IF (r13 < ROHC .AND. r12 < ROOC   & 
             .AND. cosphi > COSPHIC) THEN    
              h(jj) = 1.0 
              qj = qj + h(jj)                          
          ENDIF
          IF (r13 < ROHC ) THEN                           
              h_d(jj) = 1.0
              qj_d = qj_d + h_d(jj)                          
          ENDIF
        ENDDO   
        qj = qj/nmo ! ave of hb for each pair 
        qj_d = qj_d/nmo
        hb(k) = qj
        hb_d(k) = qj_d
        tot_hb = tot_hb + hb(k)

        DO mt = 0,nmo-1     ! time interval
            IF (hb(k)>hb_min) THEN 
                scalar = 0.d0
                DO j = 1, nmo-mt-1
                    scalar = scalar + h(j)*h(j+mt)  ! 1: the first pair of water molecules
                ENDDO
                scalar=scalar/(nmo-mt-1)  ! C_k(t)
                corr_h(mt+1) = corr_h(mt+1) + scalar  ! sum_C_k(t)
            ENDIF
        ENDDO

        DO mt = 0,nmo-1 ! time interval
          IF (hb_d(k)>hb_min) THEN
                scalar_d = 0.d0
                DO j = 1, nmo-mt-1
                    scalar_d = scalar_d + h(j)*h_d(j+mt) ! 1: the first pair of water molecules
                ENDDO
                scalar_d = scalar_d/(nmo-mt-1) ! C_k(t)
                corr_h_d(mt+1) = corr_h_d(mt+1) + scalar_d  ! sum_C_k(t)
            ENDIF
        ENDDO
      ENDDO ! k loop 
      tot_hb = tot_hb/np

      DO mt = 0,nmo-1 ! time interval
          corr_h(mt+1) = corr_h(mt+1)/(np*TOT_HB)  
          corr_h_d(mt+1) = corr_h_d(mt+1)/(np*TOT_HB)  
      ENDDO
      
      !---------------------
      !Release the variables
      !---------------------
      DEALLOCATE(x,y,z,ndx_1,ndx_2,ndx_3)          

      END PROGRAM
\end{lstlisting}
%
For a given system, the input file gives all pairs of molecules which may H-bonded. Each line in the file 
include three index of O or H atom in a pair of molecules, which may H-bonded. It provides all pairs of
molecules that may H-bonded. For example, for cluster LiNO$_3$(H$_2$O)$_3$ (Fig.\space\ref{fig:clusters_4})(b), there are three 
water molecules and three nitrate O atoms, i.e., $N_{\text{water}}=3$ and $N_{\text {nitrate O}}=3$.  The number of possible H-bonds $N_{\text HB}$ for the cluster is
\begin{equation}
N_{\text{HB}}=2N_{\text{water}} [N_{\text {nitrate O}}+ (N_{\text{water}}-1)],
\label{N_HB}
\end{equation}
where the factor 2 stems from the fact that there are two H atoms in a water molecule.
Thus, for the cluster LiNO$_3$(H$_2$O)$_3$, $N_{\text{HB}}$=30. The input file generated for the cluster is as follows.
% backgroundcolor=\color{lightgray},
\begin{lstlisting}[language=fortran]                  
! filename: LiNO3-3w_input.dat
! 1st column: O (nitrate O, or water O)
! 2nd column: O (water O)
! 3rd column: H (water H)
           2           5           6
           3           5           6
           4           5           6
           8           5           6
          11           5           6
           2           5           7
           3           5           7
           4           5           7
           8           5           7
          11           5           7
           2           8           9
           3           8           9
           4           8           9
           5           8           9
          11           8           9
           2           8          10
           3           8          10
           4           8          10
           5           8          10
          11           8          10
           2          11          12
           3          11          12
           4          11          12
           5          11          12
           8          11          12
           2          11          13
           3          11          13
           4          11          13
           5          11          13
           8          11          13
\end{lstlisting}
%
%\section{Instantaneous Interfaces of  Vapor/Aqueous Interfaces}
%We have tried a different layering method for the solution/vapor interfaces: plane layering and instantaneous layering.
%Based on the density distribution, we determined the surface of the solutions. 
%In the plane layering method, we assume that the surface is a plane, and we calculate the Im$\chi^{(2)}$ spectra for layers with different thicknesses. This is what we did in the above sections.
%This section, we use the second layering method to study Vapor/Aqueous interfaces.
%\begin{figure}[H]
%\centering
%\includegraphics [width=0.6 \textwidth] {./diagrams/interface_chandler}
%\setlength{\abovecaptionskip}{10pt}
%\caption{\label{fig:interface_chandler} The liquid/vapor interfaces of  \LiN solution and pure water.
% The alkali cation and the nitrate ion are in the top interface (salty interface).
%Four water molecules are directly bonded to \Li (green) and there are  more than 6 water molecules bonded to \nitrate.}
%\end{figure}

%\section{Input File 2}\label{append_eg_list}
%\lstset{language=Fortran}
%\begin{lstlisting}
%!filename: 256_LiNO3_NEAR_Li_input.dat
%! 1st col.: O (water O in the first shell of Li atom)
%! 2nd col.: O (water O)
%! 3rd col.: H (water H in the same water molecle as O in the 2nd column)
%         156         139         528
%         156         139         529
%         156         152         554
%         156         152         555
%         156         209         668
%         156         209         669
%         156         222         694
%         156         222         695
%         209         156         562
%         209         156         563
%         209         222         694
%         209         222         695
%         209         232         714
%         209         232         715
%         209         234         718
%         209         234         719
%         209         235         720
%         209         235         721
%         222         156         562
%         222         156         563
%         222         209         668
%         222         209         669
%         222         219         688
%         222         219         689
%         222         234         718
%         222         234         719
%         222         241         732
%         222         241         733
%         234          25         300
%         234          25         301
%         234          80         410
%         234          80         411
%         234         209         668
%         234         209         669
%         234         222         694
%         234         222         695
%\end{lstlisting}
%In above list, we see that for each O atom in water molecules, if they located in the first solution shell of \Li at the initial time,
%we will consider the possible H-bonds formed by it and other water molecules nearby. For example, for O(156), there are 4 other water 
%molecules can be H-bonded to it. The number of nearby water molecules depends on the distance criteria we choose, but the result 
%will not change heavily. Since each water molecules owns two H atoms, the number of the possible H-Bonds formed by O(156) is 8.  
%Therefore, there are 36 possible bonds to be considered. 

%\section{Graphs}
%\begin{figure}[h]
%\centering
%\includegraphics [width=0.5\textwidth] {./diagrams/FT_all_w_in_interf_s6}
%\setlength{\abovecaptionskip}{20pt}
%\caption{\label{fig:FT_all_w_in_interf} The VDOS of all water molecules in aqueous/vapor interface LiNO$_3$(H$_2$O)$_{256}$. Both VDOS for the first 10 ps and the latter 10 ps have the same spectral shape. This calculation ensure convergence of dynamics of the system in 20 ps. }
%\end{figure}


\section{Details of DFT Simulations for Alkali Iodide Solutions}
This paragraph gives the simulated SFG spectrum of the aqueous/vapor interface of LiI solutions at 330 K.
%The spectrum are calculated from the average of two LiI solution/vapor interfaces, which are under the same condition.
For LiI and NaI solutions, we use the following procedures to calculate the molar concentration of ions in the solutions we study:
%&V_{\text{liquid}}=3.796\times10^{-33}\text{ m}^3 \nonumber
$n_j=N_j\times[1/(6.02\times10^{23})] {\text{ mol}}$, and
$V_{\text{liquid}}=15.6\times15.6\times15.6 \text{ \AA}^3$, 
where $n_j$, $N_j$ and $V_{\text{liquid}}$ is the amount of substance $j$, the number of substance $j$, and the 
volume of the liquid part of the aqueous/vapor interface.  
For the LiI solution/vapor interface system. 
The simulation box is with the size of 31.0 \AA$ \times$15.6 \AA$ \times$15.6 \AA. 
Half of the volume of the simulation box is vacuum. In the liquid part of the simulation box, there are two \Li cations and two \I anions.
Therefore, the molar concentration of the solution LiI is $c_{\text{LiI}}={n_{\text{LiI}}}/{V_\text{liquid}}=0.9\times10^3  \text{ mol}/\text{m}^3$.

We calculated the RDF for Ow-Hw which are located at the top surface. The result of NaI solution is shown in fig.~\ref{fig:2NaI-124w_gdr_OH_s2_150122}.
The results  for two interfaces consistent to each other when the interface is large enough, i.e., the thickness is $\sim 6$ \A.

\section{Calculation of Nonlinear Optical Susceptibilities}\label{calculation_of_chi}

%Here are some useful identities:\cite{Neipert06}
%\begin{equation}
%i\int_0^\infty dt e^{-it(a-ib)}i\int_0^\infty dt' e^{-it'(a-ib)}=\frac{1}{(a-bi)^2}.
%\label{integral_identity2}
%\end{equation}
%
0. The Proof of Eq.\space\ref{integral_identity1a}:
Since 
\begin{equation}
  i\int_0^\infty dt e^{-it(a-ib)}=\frac{1}{a-bi},
  \label{integral_identity1}
\end{equation}
set $a = (\omega_{v'} - \omega_{v}) - \omega$, and $b = \gamma_{v'v}$,
then we have
 \begin{align}
 \int_0^\infty dt e^{-it((\omega_{v'}-\omega_v)-\omega-i\gamma_{v'v})}=\frac{-i}{(\omega_{v'} -\omega_v)-\omega-i\gamma_{v'v}},\nonumber
 \end{align}
which is Eq. \ref{integral_identity1a}.

2. The Derivation of Eq.\space\ref{eq:chi}:
\begin{align}
   \chi^{\text{(2),R}}_{\eta\xi\kappa}&=\frac{-i}{k_{\text{B}}T \omega} \int_0^\infty dt e^{i \omega t}\left\langle \dot{\bf A}_{\eta\xi}(t) \dot{\bf M}_{\kappa}(0)\right\rangle
\end{align}

Derivation: 
The resonant term $\chi^{(2),R}_{\eta\xi\kappa}$ is given by \cite{Morita2008}
\begin{align}
  \chi^{\text{(2),R}}_{\eta\xi\kappa}&=\frac{i\omega}{k_{\text{B}}T} \int_0^\infty dt e^{i \omega t}\left\langle {\bf A}_{\eta\xi}(t) {\bf M}_{\kappa}(0)\right\rangle,
\end{align}


Some definitions and relations used in calculating SFG spectra:

1. Definition of double product of a $m$-order tensor $A$ and a $n$-order tensor $B$ is a tensor with order $m+n-2$.
\begin{equation}
    A:B=A_{ij}B_{lm}\delta_{jl}\delta_{im}.
\label{tensor_double_product}
\end{equation}

2. The components of product of $AB$ is defined by 
\begin{equation}
    (AB)_{ijlm}=A_{ij}B_{lm}.
\label{tensor_product}
\end{equation}

3. For vectors $\bf{a}$, $\bf{b}$, $\bf{c}$ and $\bf{d}$, $\bf{ab}:\bf{cd}=(\bf{a}\cdot\bf{d})(\bf{b}\cdot\bf{c})$.

Proof:
\begin{align}
    {\bf{ab}:\bf{cd}}&=({\bf{ab}})_{ij}({\bf{cd}})_{lm}\delta_{jl}\delta_{im} \nonumber \\
    &=({\bf{ab}})_{ij}({\bf{cd}})_{ji} \nonumber\\
    &=(a_i b_j)(c_j d_i) \nonumber\\
    &=(\bf{b}\cdot\bf{c})(\bf{a}\cdot\bf{d})
\end{align}

\section{Fitting Anisotropy Decay with a Single Exponential Function}\label{single_exp}
We assume that the anisotropy decay is a single exponential given by 
\begin{equation}
C_2(t)=e^{-\kappa t},
\label{eq:tcf2}
\end{equation}
where $\kappa$ is a rate constant for the anisotropy decay.
%The time correlation functions for water molecules bonded to specific ions, for selected frequency windows up to 1.5 ps, are shown in Fig.\space\ref{fig:2NaI-124w_0-25ps_c2_150224}.

\begin{figure} %[htbp]
\centering
\includegraphics [width=0.4 \textwidth] {./diagrams/2NaI-124w_c2_fit_150223} 
\setlength{\abovecaptionskip}{10pt}
\caption{\label{fig:2NaI-124w_c2_fit_150223} The anisotropy decay of OH bonds in water molecules in NaI water solutions. Water molecules bonded to \I anions decay fastest,
while water molecules bonded to \Na slowest. (0--10 ps).}
\end{figure} 
\begin{figure}
\centering
\includegraphics [width=0.6 \textwidth] {./diagrams/2LiI-124w_0-25ps_c2_150222b} 
\setlength{\abovecaptionskip}{10pt}
\caption{\label{fig:2LiI-124w_0-25ps_c2_150222b} The anisotropy decay of OH bonds in water molecules at the water/vapor interface of NaI and LiI solutions.}
\end{figure} 
%--
\begin{table}
\centering
\caption{\label{tab:table_alkali_halide_1}%
The fitted parameters---the relaxation period $\tau$ and the decay rate $\kappa$, of anisotropy decay of water molecules 
  at the water/vapor interface of LiI solutions.} 
\begin{tabular}{lccc}
central ions&  $T$ (ps) & $\kappa$ (THz) \\
\hline
bound to \I  & 3.5 & 0.29 \\
bound to \Li & 12.3 & 0.08 \\
bulk & 6.9 & 0.14 \\
surface & 2.3 & 0.44 \\
\end{tabular}
\end{table}
%--
\begin{table}
\centering
\caption{\label{tab:table_alkali_halide_2}%
  The fitted parameters of anisotropy decay of water molecules at the water/vapor interface of NaI solution.}
\begin{tabular}{lccc}
water molecules& $T$ (ps) &  $\kappa$ (THz) \\
\hline
bound to \I &10.7 & 0.09 \\
bound to \Na & 14.8 & 0.07 \\
bulk & 11.8 & 0.08 \\
surface & 5.6 & 0.18 \\
\end{tabular}
\end{table}
%--
\begin{table}
\centering
\caption{\label{tab:table_expfit}%
  The fitted parameters of anisotropy decay of water molecules in LiI (NaI) solutions.}
\begin{tabular}{lccc}
water molecules & $c$  &  $\kappa$ (THz) &$T=1/\kappa$ (ps) \\
\hline
bonded to \I (LiI) & 0.82 & 0.24 & 4.2 \\
bonded to \Li (LiI) & 0.88 & 0.07 & 14.3 \\
bulk (LiI) & 0.85 & 0.12 & 8.4\\
surface (LiI) & 0.81 & 0.35 & 2.9  \\
bonded to \I (NaI) & 0.86 & 0.14 & 7.1 \\
bonded to \Na (NaI) & 0.79 & 0.07 & 14.3 \\
bulk (NaI) & 0.83 & 0.06 & 16.7 \\
surface (NaI) & 0.78 & 0.12 & 8.4 \\
\end{tabular}
\end{table}
\section{Calculating Free Energy by Blue-Moon Ensemble Method} 
From the Blue-Moon ensemble method, we obtain the constraint force acting on the atoms (the unit of which is a.u.force). Suppose the unit of the distance between atoms is \A,
and distance is chosen as the reaction coordinate. We will give the formula to calculate the free energy whose unit is kcal/mol.
There are three formula of the unit transform:
\begin{equation}
1 \text{a.u.force} =8.2387\times 10^{-8}\text{N}\nonumber;
\label{eq:au2n}
\end{equation}
\begin{equation}
1\AA = 10^{-10} \text{m}\nonumber;
\label{eq:aa}
\end{equation}
\begin{equation}
1\text{J} = 1.44\times 10^{20} \text{ kcal/mol}.
\label{eq:j2kcpm}
\end{equation}
we obtain the relative free energy
\begin{equation}
F = \sum_{i}^{N}f_i{\Delta{r}} \text{ a.u.energy},
\label{eq:f-e}
\end{equation}
where $i$ denote a point on the 1D reaction coordinate, $N$ is the number of the sampling points of the reaction coordinates,
and $f_i$ denotes the average force on atoms over the trajectory when $i$ is fixed. Furthermore,
\begin{equation}
\begin{split}
&F = \sum_{i}^{N}f_i{\Delta{r}}\times 8.2387\times10^{-18}\text{ J}\\
&= \sum_{i}^{N}f_i{\Delta{r}}\times 8.2387\times10^{-18}\times1.44\times10^{20}\text{ kcal/mol}.
\end{split}
\label{eq:f-e2}
\end{equation}
If we choose $\Delta{r}= 0.2$\AA, then
\begin{equation}
F = \sum_{i}^{N}f_i{\Delta{r}}\times237\text{ kcal/mol}.
\label{eq:f-e}
\end{equation}

Now we estimate the error of the free energy from the summation approximation. It reads
\begin{equation}
\delta{F} = \frac{1}{N}\sum_{i}^{N}\delta{f_i}{\Delta{r}}\times 8.2387\times10^{-18}\times1.44\times10^{20}\text{ kcal/mol}.
\label{eq:dleta_f}
\end{equation}
Usually, $\delta{f_i}\approx\delta{f}$, thus
\begin{equation}
\delta{F} = \frac{1}{N}\delta{f}\sum_{i}^{N}{\Delta{r}}\times 8.2387\times10^{-18}\times1.44\times10^{20}\text{ kcal/mol}.
\label{eq:dleta_f-2}
\end{equation}
For example, when $\Delta{r}= 0.2$\AA, $\delta{f}=0.0075 \text{ a.u.force}$, we get
\begin{equation}
\begin{split}
&\delta{F} = \frac{0.0075\times\sum_{i}^{N}237}{N}\text{ kcal/mol}\\
&          = 0.0075\times237\text{ kcal/mol}\\
&          \approx 1.78\text{ kcal/mol}.
\end{split}
\label{eq:dleta_f-3}
\end{equation}
