\chapter{Experimental SFG spectra of salty interfaces}\label{CHAPTER_SFG_Exp}
In this chapter, we will give the experimental results obtained on salty solutions containing alkali cations and nitrate (iodide) anions. [\cite{PS03,AJ12,HuaWei2014}] 

%We will choose the interface of lithium nitrate solution as our research object, which is a typical example of the interface of alkali metal nitrate solution.
From the experimental data of surface tension dependence on solute concentration $\text{d}\gamma/\text{d}m_2$ 
at low electrolyte concentrations ($\leq$1.5 M ), [\cite{Weissenborn95,Hey81,Jarvis68,Jarvis72}] 
%and the assumption that \Na is the most excluded cation
the relation of the surface/bulk molar concentration ratio $K_{\text{p}}$ [\cite{Pegram06}] among \li, \Na and \K is: 
\begin{equation}
0=K_{\text{p,Na}^+}< K_{\text{p,K}^+}< K_{\text{p,Li}^+}.
\label{eq:bscr}
\end{equation}
i.e., \Na is the most surface-excluded in the water solution RNO$_3$, \K is less excluded, 
and \Li is the least excluded cation.
In modeling the alkali nitrate solution at the interface we decided to start with \Li ions, which is the least excluded of the vapor-liquid interface 
among the alkali metal ions as the cation in the model. 
%
%The aim of our work here is to provide a molecular picture 
%for the water/vapor interface of the \ch{LiNO3}-containing solution 
%to interpret the experimental spectra. In particular a question raises if simple models 
%for the solvated ion clusters can already provide information
%on the ions-water complexes formed at the interface. 
%It is known that nitrate (\nitrate) is a naturally occurring ion which is part of the nitrogen cycle, 
%and it can reach both surface water and groundwater as a consequence of agricultural activity. 
%Nitrate also has effects on human health,  because of the toxicity caused by its reduction to nitrite (NO$_2^-$). \cite{WHO11}
%The biological effect of nitrite in humans is its involvement in the oxidation of
%normal Hb to metHb, which is unable to transport oxygen to the tissues.
%For example, infants who drink water with high levels of nitrate (more than 10 mg/L) may suffer 
%a serious health condition due to blue baby syndrome.\cite{Knobeloch00}
%

%The goal of this chapter is to find the origin of the main characteristics of the VSFG spectra of the \LiN solution,
%and provide a molecular picture to interpret the recorded spectra.
%In order to achieve this goal, we simulate water/vapor interface including \Li and \nitrate, 
%as shown in Fig.\thinspace\ref{fig:interface_chandler},
%and extract the vibrational spectroscopic properties of the water/vapor interface of LiNO$_3$ solution.
%=========
%
%\begin{figure}[htbp]
%\centering
%\includegraphics [width=0.5 \textwidth] {./diagrams/interface_chandler}
%\setlength{\abovecaptionskip}{0pt}
%\caption{\label{fig:interface_chandler} The water/vapor interfaces of \LiN solution and pure water. 
%The right panel shows that the \Li and the \nitrate ions are separated by a water molecule at the salty interface.}
%\end{figure}
%The water molecules below 4 \AA  of the instantaneous interfaces are shown opaquely. 

%
Hua \etal [\cite{HuaWei2014}] have recently measured the VSFG spectra of water/vapor interface of \LiN salt solutions in the OH stretching region
(3000--3800 \centimeter) using Heterodyne Detected VSFG spectroscopy. [\cite{HuaWei2011,HuaWei2011b,ChenXiangKe2010}] 
The experimental result of the VSFG intensity of the alkali nitrate interfaces is given by in Fig.\space\ref{fig:Allen12}. 
At a difference with the spectra for the water interface, in the spectra of 
\LiN solutions, a depletion of the 3200 \cm peak is observed, with an 
enhancement of the 3400 \cm peak.
A similar behaviour had been observed for the interface of NaNO$_3$ and 
Mg(NO$_3$)$_2$ solutions. [\cite{AJ12,HuaWei2014}] It has been 
suggested that this depletion of the 3200 \cm peak, and in some cases 
the enhancement of the 3400 \cm peak, is an indication that nitrate 
ions reside at the interface. On the other hand the small 
cations should have little surface propensity. 
It has also been argued that the positive electric field found at the interface of NaCl, NaI and 
NaNO$_3$ salt solutions is due to the formation of an ionic double layer 
between anions located near the surface and their counter-cations (e.g.
Na$^+$) located further below. In Phase-Sensitive (PS) VSFG experiments the 
magnitude of the induced change in the Im$\chi^{(2)}$ spectra comparatively
to that of the neat water suggested that \nitrate has a surface propensity 
just in between I$^-$ and Cl$^-$. [\cite{Verreault2013,Verreault2009}] 
% exp. results.
\begin{figure}[htbp]
\centering
  \includegraphics [width=0.6 \textwidth] {./diagrams/vsfg_alkali_nitrate}
\setlength{\abovecaptionskip}{0pt}
  \caption{\label{fig:Allen12}Experimental VSFG intensity of \LiN solutions, compared with that of neat water. [\cite{HuaWei2014}]}
\end{figure}

%\begin{figure}[htbp]
%    \centering
%    \begin{tikzpicture}
%        \begin{axis}[
%        scale=0.8, % scale the figure but the labels
%        /pgf/number format/.cd,
%        %use comma,
%        1000 sep={}, % separator of thousands
%        legend style={draw=none},
%        legend pos=north west,
%        %title=VSFG Intensity,
%        xlabel={Wave Number (cm$^{-1}$)},
%        every axis x label/.style={
%        at={(rel axis cs: 0.5, -0.20)},
%        anchor=center}, 
%        ylabel={Intensity (Arb. Unit)},
%        every axis y label/.style={    
%        at={(rel axis cs:-0.18, 0.5)},rotate=90,
%        anchor=center}, 
%        ymin=0,
%        ymax=0.3,
%        minor y tick num=1,
%        ]
%        \addplot[mark=x, black,domain=3000:3800,very thick]table{./chapters/chap4/data/Net_water.dat};
%        \addlegendentry{Neat Water}
%        \addplot[mark=*, blue,domain=3000:3800,very thick]table{./chapters/chap4/data/LiNO3.dat};
%        \addlegendentry{1M  \LiN}
%      \end{axis}
%   \end{tikzpicture}
% \setlength{\abovecaptionskip}{0pt}
%  \caption{\label{fig:Allen12}Experimental VSFG intensity of \LiN solutions, compared with that of neat water.\cite{HuaWei2014}}
% \end{figure}
