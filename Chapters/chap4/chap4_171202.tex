\chapter{Alkali Nitrate Clusters}\label{CHAPTER_results_clusters}
In the first two sections of this chapter, we will consider two clusters, NO$_3^-$(H$_2$O)$_3$ cluster and \li(H$_2$O)$_4$ Cluster to study the the structure and dynamical properties of nitrate ions and alkali cations, seperately. Finally, in section 3.3, VDOS of clusters of alkali nitrate and water is given and the effects of alkali metal cations and the nitrate anion are discussed. 

\section{Cluster of Nitrate and Water}\label{section_3w_nitrate}
For the cluster NO$_3^-$(H$_2$O)$_3$, the symmetric isomer of  it is obtained by geometry optimization, implemented in the package CP2K. Fig. ~\ref{fig:3_NO3_small} shows the geometry 
\begin{wrapfigure}{l}[0.05cm]{8.0cm}
\centering
\includegraphics[width=0.3\textwidth]{./diagrams/3_NO3_small}
\setlength{\abovecaptionskip}{10pt}
\caption{The stable geometry optimized structure of the cluster NO$_3^-$(H$_2$O)$_3$. The red dotted lines denote the H-bonds.}\label{fig:3_NO3_small}
\end{wrapfigure}
optimized symmetric isomer structure of the cluster NO$_3^-$(H$_2$O)$_3$, and the geometrical 
parameters, i.e., the length $R_\text{OH}$ of covalent bonds , the H--O--H angle $\angle$HOH , and the water H--nitrate O bond length $r_\text{OH}$, of the cluster are listed in Table ~\ref{tab:3w_nitrate}. 
Due to the definition of the hydrogen bond (HB)\cite{JT90,SB02}, there are three Hydrogen bonds (H bonds) in this structure of the cluster. 
The three water molecules in the cluster NO$_3^-$(H$_2$O)$_3$ have similar dynamical properties. But in each water molecule, only one of the two OH group is hydrogen bonded to \nitrate. 
Therefore, the two OH groups in each water molecule are vibrating in different modes. Fig. ~\ref{fig:vdos_NO3-3w_2_H6H7} gives the VDOS of OH groups in one of these water molecules in the symmetric cluster NO$_3^-$(H$_2$O)$_3$.
It shows that both H atoms in the water molecule have two vibrations modes. 
The two H atoms in a water molecule in the cluster prefer to vibrateing in different modes at the same time. One H atom is vibrating in frequencies 3680--3700 cm$^{-1}$, while the other in 3380--3440 cm$^{-1}$. The average difference of frequencies between the vibrational modes is $\Delta\nu=$ 255 \centmeter. 
\begin{figure}[htbp]
\centering
\includegraphics [width=0.5\textwidth] {./diagrams/vdos_NO3-3w_2_H6H7}%
\setlength{\abovecaptionskip}{20pt}
\caption{\label{fig:vdos_NO3-3w_2_H6H7}The VDOS of two hydrogen atoms in the same water molecule (Fig. ~\ref{fig:3_NO3_small}), compared with the VDOS of the whole water molecule in NO$_3^-$(H$_2$O)$_3$ at 300 K. The DFTMD simulation time is 5 ps.}
\end{figure}  %(Calculated from the function vdos3.f and ft$\_$5s.sh)
This difference reflects the symmetry breaking of the density distribution of the nitrate ion in the cluster. 
And the symmetry breaking results from the interaction between the 
three water molecules and the nitrate ion. 
This symmetry breaking is also shown in the number of H bonds in the cluster. 
There are totally three H bonds and three quasi-hydrogen bonds in the cluster.  
A quasi-hydrogen bond is formed if the O--H distance $r_\text{OH}$ satisfies the condition $r_\text{OH}<3.5$ \AA, but not that the O--H$\cdots$O angle is less than $\frac{\pi}{6}$. \cite{JT90} 
We label the three water molecules as water 1, 2 and 3, respectively. 
In water molecule 1, 2 and 3, the differences $\Delta{d}$ between the HB and the quasi-hydrogen bond are 0.62 \AA, 0.64 \AA  and 0.82 \AA, respectively (Table ~\ref{tab:3w_nitrate}) .
%likely a finite temperature effect, since both differences $\Delta\nu$ and $\Delta{d}$ decrease, as the temperature decrease.
\begin{figure}[htbp]
\centering
\includegraphics [width=0.5\textwidth] {./diagrams/gdr_ON-wat--3_NO3_Sans} 
\setlength{\abovecaptionskip}{20pt}
\caption{\label{gdr_ON-wat--3_NO3_Sans}The nitrate O--water O  and nitrate O--water H RDFs for the cluster NO$_3^-$(H$_2$O)$_3$. Nitrate O is denoted by $\text{O}_\text{n}$ and the water O (water hydrogen) by $\text{O}_\text{w}$ ($\text{H}_\text{w}$). 
The peaks for the former are 1.93, 2.95 and 3.95 \A, and for the later are 2.95 and 4.80 \A.}
\end{figure} 

\paragraph{Lengths of Hydrogen Bonds}
Table ~\ref{tab:3_nitrate_bond} shows the lengths of H-bonds in the cluster NO$_3^-$(H$_2$O)$_3$. $r_a$ and $\delta$ denote the average distances and the standard deviations of the length of H bonds, respectively. 
%From the definition of HB\cite{AL96,AL96b}, there are only three H-bonds in this cluster. 
%This symmetry breaking comes from
% (1) the effect of the finite temperature and (2) the interaction between the three water molecules and the nitrate ion.
%Thermostat changes the velocity of oxygen atoms and nitrogen atom and breaks the symmetry.
%The balance between H bonds (potential energy) and the kinetic energy (proportional to the temperature \T) of the H and O atoms leads to this symmmetry breaking in the cluster.
\begin{table}
\centering
\caption{\label{tab:3_nitrate_bond}%
The lengths of H-bonds in the cluster NO$_3^-$(H$_2$O)$_3$. The indices of H atoms: H6, H7 in w1;H9,H10 in w2 and H12, H13 in w3.} 
\begin{tabular}{ccc} \\\toprule
 HBs& $r_a\pm\delta$ (100 K)(\A) & \multicolumn{1}{c}{ $r_a\pm\delta$ (300 K)}(\A)\\
\hline
 H6-O2 &2.75$\pm$0.62& 2.40$\pm$0.52 \\
 H7-O4 &2.79$\pm$0.58& 3.02$\pm$0.72 \\
 H9-O3 &2.89$\pm$0.60 &2.56$\pm$0.48 \\
 H10-O4 &2.74$\pm$0.49&3.20$\pm$0.41 \\
 H12-O3 &2.46$\pm$0.45&2.29$\pm$0.47 \\
 H13-O2 &2.75$\pm$0.59 &3.11$\pm$0.72
\end{tabular}
\end{table}
From Table ~\ref{tab:3_nitrate_bond}, we obtain that when $T=300$ K, the difference $r_a$ between different hydrogen atoms in one water molecule is
$\Delta{r_a}=0.69$ \AA, while $\Delta{r_a}=0.13$ \AA for $T=100$ K.
%$\Delta{r_a}=(0.62+0.64+0.82)/3=0.69$ \AA, 
To see what will happen at lower temperature, 
we calculated the VDOS of water moleucles in the clsuter NO$_3^-$(H$_2$O)$_3$ at 100 K (Fig. ~\ref{fig:vdos_LiNO3-3w_100K_w1-2-3_font35}). 
It shows that the vibrational peaks for the three water molecules are very close to each other. 
These peaks are almost the same, in both OH stretching and bending modes.  
At the lower temperature $T=100$ K, the effect of Temperature is not as obvious as the higher $T=300$ K. 
The three water molecules are more symmetric distributed bonded to the center nitrate.
Therefore, the difference between H-bonds and quasi- H bonds in the symmetric isomer of NO$_3^-$(H$_2$O)$_3$ is likely a finite temperature effect, 
which is verified by the calculation of VDOS.

%\paragraph{VDOS of Oxygen and Hydrogen Atoms in Water}
%In the range of the higher freqencies ($>$ 1000 \cm), the applitude of the vibration of hydrogen atoms is much larger than that of oxygen atoms. 
%%Fig. ~\ref{fig:vdos_O_NO3-3w_2_s} shows that the oxygen atom in a water molecule is vibrating at the very two modes as that of the hydrogen atoms in the same water molecule. 
%%As an example, here is shown the VDOS of water molecule in the cluster NO$_3^-$(H$_2$O)$_3$.
%%This is consistent with the fact that there are covalence bonds between the oxygen atom and the hydrogen atoms in a water molecule. 
%%The oxygen atom vibrates in two different modes, because the two hydrogen atoms interacting with it are in different environment and therefore are vibrating in different modes.
%%\begin{figure}[htbp]
%%\centering
%%\includegraphics [width=0.5\textwidth]{./diagrams/vdos_O_NO3-3w_2_s}   
%%\setlength{\abovecaptionskip}{20pt}
%%\caption{\label{fig:vdos_O_NO3-3w_2_s}The VDOS of the O atom (O$_{\text{w}}$) and H atoms (H$_{\text{w}}$) in a water molecule in the cluster NO$_3^-$(H$_2$O)$_3$. 
%%}
%%\end{figure}
%However, the $g(\nu)$[H]--$g(\nu)$[O] diagram (Fig.~\ref{fig:vdos-lr_H6H7-O5}) also shows the correlation between the vibrations of O and H atoms in a water molecule. 
%In 3650--3750 \cm region, during which both VDOSs reaches their peak values, 
%the VDOS of O atoms are linearly correlated to VDOS of H atoms in the same water molecule in the cluster NO$_3^-$(H$_2$O)$_3$. 
%For the frequency interval 3400--3500 \centmeter, there exists an almost linear relation between  $g(\nu)$[H] and $g(\nu)$[O]. 
%The nonlinear relation reflects the nonresonance between vibrations of H atoms and O atoms. 
%To eliminate the noise, we have neglected the smaller VDOS values for both $g(\nu)$[H] and $g(\nu)$[O]. 
%As we know before from VDOS of water molecules in the symmetric cluster  NO$_3^-$(H$_2$O)$_3$, 
%3400--3500 \cm and 3650--3750 \cm regions are H-bonded OH stretching mode and free OH stretching mode. 
%\begin{figure}[htbp]
%\centering
%\includegraphics [width=0.5\textwidth]{./diagrams/vdos-lr_H6H7-O5}  % location of this figure: ~/VDOS_NO3-3w_2S 
%\setlength{\abovecaptionskip}{20pt}
%\caption{\label{fig:vdos-lr_H6H7-O5} The $g(\nu)$[O]--$g(\nu)$[H] diagram (the H and O atoms in the same water molecule) in the cluster NO$_3^-$(H$_2$O)$_3$ for two frequency intervals 3400--3500 \cm and 3650--3750 \centmeter.}
%\end{figure}
%For the free OH stretch, the VDOS of O atoms and H atoms in the same water molecule peaks at the same frequency.
%But for the H-bonded OH stretch with lower freqencies, the O atoms and H atoms are vibrating in different ones. 
%in both free OH stretching modes and H-bonded stretching modes.
\begin{figure}[htbp]
\centering
\includegraphics [width=0.4 \textwidth] {./diagrams/vdos_LiNO3-3w_100K_w1-2-3_font35} % _ocation: ~/530--NO3-3w_100K/ 
\setlength{\abovecaptionskip}{20pt}
\caption{\label{fig:vdos_LiNO3-3w_100K_w1-2-3_font35} The VDOS $g(\nu)$ of water 
molecules in the cluster NO$_3^-$(H$_2$O)$_3$ ($T=100$ K) shows that 
the vibrational peaks for the three water molecules are very close to each other ($\Delta\nu <$ 10 \cm), 
for both vibrational and bending modes, and the difference between H-bonds and quasi- H bonds diminishs.}
\end{figure}

%============
%section 4_Li
%============
\section{Cluster of Alkali Metal Cation and Water}
\begin{wrapfigure}{l}[0.05cm]{6.5cm}
\centering
\includegraphics[width=0.25\textwidth]{./diagrams/4_Li}
\setlength{\abovecaptionskip}{10pt}
\caption{\label{fig:4_Li}The geometrical structure of the cluster Li$^+$(H$_2$O)$_4$.}
\end{wrapfigure}
In the LiNO$_3$ solution, there are four water molecules in the solvation shell of \li. For comparison, we calculate the Li-O$_{\text{w}}$ and Li-H$_{\text{w}}$ RDFs, VDOS of water molecules for a cluster \li(\wat)$_4$ (Fig. ~\ref{fig:4_Li}). 
The peak for Li-O$_{\text{w}}$ RDF is at 2.02 \AA, and for Li-HW is 2.69 \AA (Fig. ~\ref{gdr_4_Li-wat}), which shows that the solvation shell of \Li is very solid.
\begin{figure}[htbp]
\centering
\includegraphics[width=0.5 \textwidth] {./diagrams/gdr_4_Li-wat}
\setlength{\abovecaptionskip}{20pt}
\caption{\label{gdr_4_Li-wat}The Li-O$_{\text{w}}$ and Li-H$_{\text{w}}$  RDF for the cluster \li(H$_2$O)$_4$.} 
\end{figure}
\begin{figure}[htbp]
\centering
\includegraphics [width=0.75 \textwidth] {./diagrams/vdos_4_Li-wat_2p} 
\setlength{\abovecaptionskip}{20pt}
\caption{\label{vdos_4_Li-wat_2p} The VDOS $g(\nu)$ of water molecules in the cluster \li(\wat)$_4$ at $T=300$ K, calcualted from a longer trajectory, during which the H-bonded water molecules is generated. 
(a) The VDOS for all water molecules (black line), Li-bonded (red) and Hydrogen bonded water molecules (blue).
(b) The VDOS for H-bonded water molecules (blue), and for both Li-bonded and H-bonded water molecules (orange). }
\end{figure}
The VDOS (Fig. ~\ref{vdos_4_Li-wat_2p} (a))
shows that there are two types of OH stretching modes in \li(\wat)$_4$ cluster:
free OH stretch which peaks at 3700 \cm and bonded OH stretch at 3300 \centmeter. 
The water molecules just bonded to \Li has two degenerate free O-H stetching modes, 
while water molecules bonded to other water molecules can have bonded OH stretch mode.
Fig. ~\ref{vdos_4_Li-wat_2p} (b) gives that if a H-bonded water molecule is in the solvation shell of \Li, 
the bonded OH streching frequency is 100 \cm higher than that of the only-H-bonded OH strech. 
\begin{figure}[htbp]
\centering
\includegraphics [width=0.5 \textwidth] {./diagrams/vdos_4_Li-wat_w1_5ps} 
\setlength{\abovecaptionskip}{20pt}
\caption{\label{vdos_4_Li-wat_w1_5ps}The VDOS $g(\nu)$ of water molecules in the cluster \li(\wat)$_4$ at $T=300$ K, calcualted from a 5-ps trajectory, during which no structural transformation occures (there is no H-bonded \water).}
\end{figure}
The VDOS of one of the water molecules bonded to \Li (Fig. ~\ref{vdos_4_Li-wat_w1_5ps}) 
shows that the Li-bonded water molecules have free OH stretch since there is only one stretching mode at 3700 \centmeter.
%
\section{Clusters of Alkali Nitrate and Water}
\begin{wrapfigure}{r}[0.05cm]{7.5cm}
\centering
\includegraphics[width=0.3\textwidth]{./diagrams/3_RNO3}
\setlength{\abovecaptionskip}{10pt}
\caption{\label{fig:3_RNO3}The most stable structure of the cluster RNO$_3$(H$_2$O)$_3$ (R=Li, Na or K).}
\end{wrapfigure}
In our simulations, the cluster is geometry optimized and the most stable 
configuration is shown in Fig. ~\ref{fig:3_RNO3}.
We found that for the cluster containg LiN$O_3$ and three water molecules, 
a contact ion pair (\li--\nitrate pair) is maintained during the simulation trajectories where a direct interaction involves the cation and one of the nitrate oxygens.
In the stable stucture LiNO$_3$(H$_2$O)$_3$, 
there are three H bonds and three Li$\cdot\cdot\cdot$O bonds. The average lengths and standard deviations of them are list in Table ~\ref{tab:table_lino3}. 
Both the average length and the standard deviation of HB1 and HB3 are very close to each other and both of them are smaller than those of HB2. 
Since both water molecule 1 (w1) and 2 (w2) are bonded to \li, we calculate an average value $\bar{d}_{\text{HB}}=1.81$ \AA of the lengths of HB1 and HB3.
The difference between length of HB2 and $\bar{d}_{\text{HB}}$ is $\delta d_{\text{HB}}=2.00-1.81=0.19$ \AA.
By testing the difference of environment of each H bonds,  we obtain that $\delta d_{\text{HB}}$ comes from the difference between Li$\cdot\cdot\cdot$O  and O$\cdot\cdot\cdot$H  bonds in the nitrate ion.
\begin{table}[htbp]
\centering
\caption{\label{tab:table_lino3}%
The lengths $r_a\pm\delta$ of O$\cdot\cdot\cdot$ H bonds and Li$\cdot\cdot\cdot$O bonds in the cluster LiNO$_3$(H$_2$O)$_3$.}
\begin{tabular}{ccc}
Bonds& $r_a$( \AA) &$\delta$ ( \AA)\\
\hline
HB1 &1.83& 0.14\\
HB2 &2.00& 0.25 \\
HB3 &1.79&0.16 \\
O(w1)-\Li &1.95& 0.09\\
O(w3)-\Li &1.92 &0.07\\
O(\nitrate)-\Li &1.91 &0.08
\end{tabular}
\end{table}

 
\subsection{Structural Characterization}
 In Table ~\ref{tab:table_geo_opt} (Table ~\ref{tab:3w_nitrate}) selected distances characterizing the RNO$_3$(H$_2$O)$_3$ (NO$_3^-$(H$_2$O)$_3$) clusters are reported.
 In Table ~\ref{tab:table_rnitrate_3w} selected geoemtrical paramteres are reported for the RNO$_3$(H$_2$O)$_3$ (R=Li, Na, K) clusters
 as averaged over 2 ps BOMD trajectory at 300 K.  
 
 The Radial Distribution Function (RDF) between the alkali (\li, \na or \pot) 
 and the water oxygen  (panel (a)) and the nitrate oxygen - water  hydrogen (panel (b)) are also reported. 
 The sharp peaks in the RDF (Fig. ~\ref{fig:gdr_OH_OR_RNO3-3w_300K} (b)) 
 shows that \nitrate is solvated and in particular stronger H-bond is formed in the presence of the cation. 

%======
We obtain the effects of nitrate ions and alkali metal ions on the structure and dynamical properties of the RNO$_3$(H$_2$O)$_n$ cluster, by using DFT-based molecular dynamics simulation.
In the cluters RNO$_3$(H$_2$O)$_3$ (R=Li, Na ,K), \Li and nitrate induced redshift is most obvious. 
Therefore, we consider the most stable configuration of LiNO$_3$(H$_2$O)$_n$ ($n$=4 and 5)
(Fig. ~\ref{fig:4_LiNO3_5_LiNO3}).
The VDOS of water molecules in them is in Fig. ~\ref{fig:broken_LiNO3-4w-5w}. 
We obtain the redshift induced by \Li and \nitrate (3120 \cm for 
LiNO$_3$(H$_2$O)$_4$, and 3163 \cm for LiNO$_3$(H$_2$O)$_5$) (Fig.  ~\ref{fig:broken_LiNO3-4w-5w} (c)).
\Li and nitrate oxygen- bonded OH stretching frequency (Fig. ~\ref{fig:broken_LiNO3-4w-5w}(c))
is lower than nitrate oxygen and water H-bonded OH stretching frequency
(Fig. ~\ref{fig:broken_LiNO3-4w-5w}(b)).
The former is also lower than water oxygen and water H-bonded OH stretching frequency (Fig. ~\ref{fig:broken_LiNO3-4w-5w} (d)). 
Therefore, Li-water oxygen bond is stronger than HB, and thus \Li is an surface-excluded cation in \LiN solution. 
Fig. ~\ref{fig:broken_LiNO3-4w-5w} (e) 
shows the VDOS of the water molecule (w5) with two free OHs. 
Only the bending mode and the free OH stretching mode exist in the water molecule.
%
By using DFTMD simulation and analysing the VDOS of water molecules in alkali-nitrate-water clusters, 
redshift of H-bonded OH stretching band is induced by alkali cations and nitrate anions in these clusters,
compared with the cluster including only water molecules and nitrate anions.

\subsection{Effects of Alkali Metal Cations}
We use the clusters NO$_3^-$(H$_2$O)$_3$ (Fig. ~\ref{fig:3_NO3_small}) and RNO$_3$(H$_2$O)$_3$ (Fig. ~\ref{fig:3_RNO3}) to explore the effect of the alkali metal cations R$^{+}$ and nitrate on the vibrations of water molecules. 
The structural parameters of the former are obtained from geometry optimization.  
For RNO$_3$(H$_2$O)$_3$, the stable configurations are consistent with the results from ground state calculation using higher level exchange-correlation functional\cite{AC}.
This result confirms the reliability of the functional BLYP, which is used in the current calculation. 
Two water molecules interact with R$^+$ and both water molecules have a  H atom which is H-bonded to an O atom. 
The structural characters are as follows: 
(1) R-O bond exists  and does not change in these clusters RNO$_3$(H$_2$O)$_3$, during the simulation process.
(2) Only single HBs (SHBs) exist in RNO$_3$(H$_2$O)$_3$\cite{Pathak08}, i.e., each water molecule can be bonded simultaneously to only one O atom of \nitrate anion. 
The structural parameters are in Table ~\ref{tab:table_rnitrate_3w}. 
%=========
 The VDOS of water molecules in the cluster NO$_3^-$(H$_2$O)$_3$ peak at 3460 \centmeter and 3695 \cm(Fig. ~\ref{fig:vdos_Li_Na_K-NO3-3w_roman_font40}), which are assigned to OH stretch.\cite{JRS74} 
 The water molecules bounded to R$^+$ and \nitrate vibrates at the lowest frequencies among the three water molecules in each
 cluster RNO$_3$(H$_2$O)$_3$, while the bonded OH stretch in other water molecules are with higher frequencies 3200--3400 \centmeter.
 Comparing with the HB regions of clusters NO$_3^-$(H$_2$O)$_3$ and RNO$_3$(H$_2$O)$_3$ in Fig. ~\ref{fig:vdos_Li_Na_K-NO3-3w_roman_font40} (3460 \cm line), we conclude that this redshift comes from the interaction between water molecules and \nit, and the interaction between the water molecules and the alkali metal ions, which induce more stable R-water O and R-nitrate oxygen bonds.
%============
\begin{table}
\centering
\caption{\label{tab:table_rnitrate_3w}%
The parameters of RNO$_3$(H$_2$O)$_3$ at 300 K. Note: For RNO$_3$(H$_2$O)$_3$, $R_\text{OH}$ and $R'_\text{OH}$ denote the lengths of O-H bonds in which H atoms is H-bonded and is free, respectively; The unit for distance is \AA, and the unit for angle is degree ($^\circ$).}
%\begin{ruledtabular}
\begin{tabular}{l*{4}ccc}
Parameters & LiNO$_3$(H$_2$O)$_3$& NaNO$_3$(H$_2$O)$_3$ & KNO$_3$(H$_2$O)$_3$\\
\hline
$r_\text{HB1}$ & $1.83\pm0.14$ & $1.78\pm0.09$ & $1.82\pm0.13$\\
$r_\text{HB2}$ & $2.00\pm0.25$ & $1.91\pm0.24$ & $1.80\pm0.12$\\
$r_\text{HB3}$ &$1.79\pm0.16$ & $1.76\pm0.11$ & $1.89\pm0.18$\\
$R_\text{OH}$(w1) &$0.97\pm0.01$ &$0.98\pm0.04$ &$0.97\pm0.03$ \\
$R'_\text{OH}$(w1) &$1.00\pm0.02$ &$1.00\pm0.02$ & $1.00\pm0.03$ \\
$R_\text{OH} $(w2) &$0.97\pm0.01$ &$0.98\pm0.02$ &$0.97\pm0.02$ \\ 
$R'_\text{OH}$(w2) &$0.99\pm0.01$ &$1.00\pm0.02$ & $1.00\pm0.03$ \\
$R_\text{OH}$(w3) &$0.97\pm0.01$ & $0.97\pm0.02$&$0.97\pm0.03$ \\
$R'_\text{OH}$(w3) &$1.00\pm0.02$ &$1.00\pm0.02$ & $1.00\pm0.03$ \\
$r_\text{R-O(w1)}$ & $1.95\pm0.09$ & $2.34\pm0.08$ & $2.76\pm0.11$\\
$r_\text{R-O(w3)}$ & $1.92\pm0.07$ & $2.32\pm0.11$ & $2.74\pm0.13$\\
$r_\text{R-O(\nitrate)}$ & $1.91\pm0.08$ & $2.31\pm0.09$ & $2.74\pm0.12$ \\
$\angle$HOH (w1) &$107\pm4$ & $106\pm4$ &$105\pm5$ \\
$\angle$HOH (w2) &$106\pm6$ & $105\pm4$ &$106\pm4$ \\
$\angle$HOH (w3) &$108\pm5$ & $106\pm3$ &$106\pm3$ 
\end{tabular}
%\end{ruledtabular}
\end{table}
%--
\begin{table}
\centering
\caption{\label{tab:table_geo_opt}%
The structual parameters of RNO$_3$(H$_2$O)$_3$ from geometry optimization. Note: The units of distance and angles are Angstrom (\AA) and degree ($^\circ$), respectively.}
%\begin{ruledtabular}
\begin{tabular}{l*{4}ccc}
Parameters  & LiNO$_3$(H$_2$O)$_3$& NaNO$_3$(H$_2$O)$_3$ & KNO$_3$(H$_2$O)$_3$\\
%\mbox{Three}&\mbox{Four}&\mbox{Five}\\
\hline
$r_\text{HB1}$& 1.67 & 1.71 & 1.82 \\
$r_\text{HB2}$& 1.91 & 1.78 & 1.92\\
$r_\text{HB3}$& 1.82 & 1.69 & 1.94\\
$r_\text{R-O(w1)}$ & 1.91 & 2.31 & 2.70\\
$r_\text{R-O(w2)}$ & 1.90 & 2.26 & 2.70\\
$r_\text{R-O(\nitrate)}$ & 1.84 & 2.29 & 2.69 \\
$\angle$HOH(w1)& 109 & 106 &107 \\
$\angle$HOH(w2)& 106 & 105&105 \\
$\angle$HOH(w3)& 108 & 107 &106
\end{tabular}
%\end{ruledtabular}
\end{table}
%
\begin{table}
\centering
\caption{\label{tab:3w_nitrate}%
The parameters of the water molecules and HBs in NO$_3^-$(H$_2$O)$_3$ at 300 K.}
\begin{tabular}{lccc}
water &$R_\text{OH}$ &$\angle$HOH ($^\circ$) & $r_\text{OH}$ \\
\hline
w1 &0.98$\pm$0.02 &101$\pm$4 & 2.40$\pm$0.52, 3.02$\pm$0.72 \\
w2 &0.98$\pm$0.02 &101$\pm$5 & 2.56$\pm$0.48, 3.20$\pm$0.41 \\
w3 &0.98$\pm$0.02 &101$\pm$4 & 2.29$\pm$0.47, 3.11$\pm$0.72
\end{tabular}
\end{table}
%
\begin{figure}[htbp]
\centering
\includegraphics [width=0.5 \textwidth] {./diagrams/gdr_OH_OR_RNO3-3w_300K}
\setlength{\abovecaptionskip}{20pt}
\caption{\label{fig:gdr_OH_OR_RNO3-3w_300K} (a) The RDF $g_{\text{R-O}}$ for clusters RNO$_3$(H$_2$O)$_3$ (R=Li, Na or K) at 300 K; (b): the RDF $g_{\text{O-H}}$ for clusters RNO$_3$(H$_2$O)$_3$ and NO$_3^-$(H$_2$O)$_3$ at 300 K.}
\end{figure}
The VDOS of H atoms and water molecules in NO$_3^-$(H$_2$O)$_3$ (Fig. ~\ref{fig:vdos_NO3-3w_2_H-wat}) shows that H's contribution dominates that of the water molecule. 
%The VDOS of water molecules in NO$_3^-$(H$_2$O)$_3$ peak at 3430 \cm and 3685 cm$^{-1}$, which are assigned to OH stretch.\cite{Scherer74} 
\begin{figure}[htbp]
\centering
\includegraphics [width=0.5 \textwidth] {./diagrams/broken_LiNO3-3w-vs-KNO3-3w}
\setlength{\abovecaptionskip}{20pt}
\caption{\label{fig:broken_LiNO3-3w-vs-KNO3-3w} VDOS of water molecules in RNO$_3$(H$_2$O)$_4$ at 300 K.
(a): w1 bonded to R$^+$ and water oxygen; 
(b): w2 bonded to nitrate oxygen and water hydrogen; (c): w3 bonded to R$^+$ and nitrate oxygen.}
\end{figure} 
The VDOS of water molecules in RNO$_3$(H$_2$O)$_3$ is shown in Fig. ~\ref{fig:broken_LiNO3-3w-vs-KNO3-3w}.  
The bending modes of VDOS bands of water molecules in
RNO$_3$(H$_2$O)$_3$ are redshifted than that of NO$_3^-$(H$_2$O)$_3$. 
%We find that there are three double HBs (DHBs)\cite{PAK}.
Differences among VDOS of water molecules in each cluster RNO$_3$(H$_2$O)$_3$ can be found in Fig. ~\ref{fig:broken_LiNO3-3w-vs-KNO3-3w}. 
In (1400, 3800) \centmeter, each water molecules has three vibrational bands. 
The bending band peaks at about 1600 \centmeter, the H-bonded OH stretching band is in (3000, 3400) \centmeter, and the free OH stretching band is at 3700 \centmeter. 
But the H-bonded OH stretching band of RNO$_3$(H$_2$O)$_3$ are 
broader than that in NO$_3^-$(H$_2$O)$_3$ and are of lower frequencies (redshifted), compared to that in NO$_3^-$(H$_2$O)$_3$ 
(3432 \centmeter, see Fig. ~\ref{fig:broken_LiNO3-3w-vs-KNO3-3w}).
The water molecules bonded to R$^+$ and \nitrate vibrates at the lowest frequencies 3000--3180 \cm among the three water molecules, while the bonded OH stretch in other water molecules are with higher frequencies 3180--3380 \centmeter.

%\begin{figure}[htbp]
%\centering
%\includegraphics [width=0.65\textwidth]{./diagrams/nitrate_alkali_cluster_2}   % Here is how to import EPS art
%\setlength{\abovecaptionskip}{0pt}
%\caption{\label{fig:nitrate_alkali_cluster_2} The cluster (a) LiNO$_3$(H$_2$O)$_4$ and (b) LiNO$_3$(H$_2$O)$_5$. \nit: N, O2, O3 and O4; {H$_2$O(1) (w1)}: O5, H6 and H7; {H$_2$O(2) (w2)}: O8, H9 and H10; {H$_2$O(3) (w3)}: O11, H12 and H13; 
%{H$_2$O(4) (w4)}: O14, H15 and H16; {H$_2$O(5) (w5)}: O17, H18 and H19 (for LiNO$_3$(H$_2$O)$_5$);
%alkali cation: Li$^+$.}
%\end{figure}

The VDOS of water molecules in clusters NaNO$_3$(H$_2$O)$_3$ and KNO$_3$(H$_2$O)$_3$ are also studied. The results are shown in Fig. ~\ref{fig:vdos_Na_K-NO3-3w_font35} (a) and (b), respectively. 
Like in the case of LiNO$_3$(H$_2$O)$_3$, the HB bands are also characterized by red-shifeted peaks around 3200 \centmeter.
\begin{figure}[htbp]
 \centering
 \includegraphics [width=0.99 \textwidth] {./diagrams/vdos_Na_K-NO3-3w_font35}
 \setlength{\abovecaptionskip}{10pt}
 \caption{\label{fig:vdos_Na_K-NO3-3w_font35} VDOS of water molecules in clusters (a) NaNO$_3$(H$_2$O)$_3$ and (b) KNO$_3$(H$_2$O)$_3$.  w1 denotes \water bonded to R$^+$ and water, w2 denotes H$_2$O bonded to \nitrate and \water, w3 denotes \water bonded to R$^+$ and \nit.}
 \end{figure}
%
\begin{figure}[htbp]
\centering
\includegraphics [width=0.8 \textwidth]{./diagrams/vdos_Li_Na_K-NO3-3w_roman_font40} 
\setlength{\abovecaptionskip}{10pt}
\caption{\label{fig:vdos_Li_Na_K-NO3-3w_roman_font40}The VDOS of water moleculs in clusters NO$_3^-$(H$_2$O)$_3$ (denoted by -) and of water molecules bounded to the alkali cation in RNO$_3$(H$_2$O)$_3$ (denoted by Li, Na and K, respectively).  The VDOS of water molecules in the cluster NO$_3^-$(H$_2$O)$_3$ at 300 K. Compared with that of  NO$_3^-$(H$_2$O)$_3$, the bending modes of the water molecules in RNO$_3$(H$_2$O)$_3$ are redshifted slightly (a); the OH stretching modes of water molecules bounded to the alkali cation in RNO$_3$(H$_2$O)$_3$ are redshifted ($|\Delta\nu|>200$ \centmeter)(b).}
\end{figure}

Additionally, we consider the most stable configuration of LiNO$_3$(H$_2$O)$_n$ ($n$=4 and 5). 
The structures of the clusters are shown in Fig. ~\ref{fig:4_LiNO3_5_LiNO3} (a) anb (b).
From the VDOS of water molecules in them (Fig. ~\ref{fig:broken_LiNO3-4w-5w}, 
we obtain the redshift induced by \Li and \nitrate (3120 \cm for 
LiNO$_3$(H$_2$O)$_4$, and 3163 \cm for LiNO$_3$(H$_2$O)$_5$) (Fig. ~\ref{fig:broken_LiNO3-4w-5w}(c)).
\Li and nitrate O- bonded OH stretching frequency (Fig. ~\ref{fig:broken_LiNO3-4w-5w}(c)) is lower than nitrate O and water H-bonded OH stretching frequency (Fig. ~\ref{fig:broken_LiNO3-4w-5w}(b)).
The former is also lower than water O and water H-bonded OH stretching frequency (Fig. ~\ref{fig:broken_LiNO3-4w-5w}(d)). 
Therefore, Li-water O bond is stronger than HB, and thus \Li is an surface-excluded cation in \LiN solution. 
Fig. ~\ref{fig:broken_LiNO3-4w-5w}(e) shows the VDOS of the water molecule (w5) with two free OHs. 
Only the bending mode and the free OH stretching mode exist in the water molecule.
\begin{figure}[htbp]
\centering
\includegraphics [width=0.6 \textwidth] {./diagrams/4_LiNO3_5_LiNO3}%
\setlength{\abovecaptionskip}{10pt}
\caption{\label{fig:4_LiNO3_5_LiNO3} The clusters LiNO$_3$(H$_2$O)$_4$ and LiNO$_3$(H$_2$O)$_5$. 
}
\end{figure} 

%[VDOS is used to extract the vibrational signatures for the water molecules in these systems.]
For all the clusters containg 3 to 5 water molecules, a contact ion pair
is maintained during the 300 K simulation trajectories where a direct interaction involves the cation and 
one of the nitrate oxygens.
The most stable isomer of the RNO$_3$(H$_2$O)$_3$ complex (R=Li, Na, K) is shown in Fig. ~\ref{fig:3_RNO3}
and compared to the the symmetric solvation of the simple NO$_3^-$(H$_2$O)$_3$.
Independently of the alkali (Li, Na, K) the cluster most stable structure is the similar.
%--
The vibrational features associated to one of the small clusters are calculated from the VDOS and  
reported in Fig. ~\ref{fig:vdos_LiNO3-3w_w1-2-3_gauss150_roman_font35}.
    \begin{figure}[htbp]
    \centering
    \includegraphics [width=0.5 \textwidth] {./diagrams/vdos_LiNO3-3w_w1-2-3_gauss150_roman_font35}%
\setlength{\abovecaptionskip}{10pt}
\caption{\label{fig:vdos_LiNO3-3w_w1-2-3_gauss150_roman_font35}  VDOS of each water molecule in RNO$_3$(H$_2$O)$_3$ (R=Li). w1 denotes the water molecule bonded to \Li and \water, w2 the water molecule bonded to \nitrate and \wat, and w3 the water molecule bonded to \Li and \nit.}
\end{figure}

In the frequency range 2800--3800 \centmeter, each water molecule has two vibrational bands. 
In addition to the free OH peak at 3700 \centmeter, we can see that the HB band is 
characterized by quite strong red-shifeted peaks around 3200 \centmeter.
These red-shifted peaks are associated to water molecules which are bound either to the cation or to both cation and anion and 
are different with respect to the peaks associated to the water molecules which only bound to \nitrate in the simple 
NO$_3^-$(H$_2$O)$_3$ cluster (3460 \centmeter, Fig. ~\ref{fig:vdos_Li_Na_K-NO3-3w_roman_font40}(b) ).

Among the different alkali the strongest red-shift is found for the LiNO$_3$(H$_2$O)$_3$ cluster.
The peaks in the OH-strechting region are also compatible with infrared predissociation (IRPD) spectra 
which have been recorded for the Li(H$_2$O)$_{3-4}$Ar clusters\cite{OR11,DJM10a,DJM10b} and
for Na(K)(H$_2$O)$_{4-7}$ clusters\cite{OR11},
although there no nitrate is present and only the effect of the cation was investigated.
%Also the peaks associated to the bending modes of the water molecules in
%RNO$_3$(H$_2$O)$_3$ are red-shifted than that of NO$_3^-$(H$_2$O)$_3$ 

To explore the effect of adding some additional water molecules to the cluster
and we considered the clusters LiNO$_3$(H$_2$O)$_n$ ($n$=4 and 5). 
The most stable configurations are shown in Fig. ~\ref{fig:4_LiNO3_5_LiNO3}
and the corresponding VDOS for the water molecules are shown in Fig. ~\ref{fig:vdos_LiNO3-4w_5w_roman_font40}. 
Also in the vibrational spectra of the LiNO$_3$(H$_2$O)$_n$ ($n$=4 and 5) we find that 
the OH stretching peaks in the HB region are quite red-shifted.
The red-shift is particularly strong for the water molecules which are directly interacting with the 
\Li cation and those which are simultaneously bound to the \Li and to the \nitrate oxygens (e.g. w3).

The vibrational spectra from the clusters clearly point to red-shifted peaks which are not 
recorded in the Vibrational \sfg spectra at the aqueous/vapor interface for the \LiN solutions. 
Clearly these small clusters cannot be directly used to describe the topmost layer
of the \LiN solution, and we need to build more realistic models which may better capture what is going
on at the interface. In particular according to the cluster picture one would be tempted of ruling out 
the possibility of a contact ion pair at the interface.
%
\begin{figure}[htbp]
\centering
\includegraphics [width=0.8\textwidth] {./diagrams/vdos_LiNO3-4w_5w_roman_font40}% Here is how to import EPS art
\setlength{\abovecaptionskip}{20pt}
\caption{\label{fig:vdos_LiNO3-4w_5w_roman_font40}  VDOS of water molecules in (a) LiNO$_3$(H$_2$O)$_4$ and (b) LiNO$_3$(H$_2$O)$_5$.  
w1 denotes \water bonded to \Li and water, w2 H$_2$O bonded to \nitrate and water, w3 \water bonded to \Li and \nit, and w4 \water bonded to water. VDOS of w5 in LiNO$_3$(H$_2$O)$_5$, is not shown, since w5 is only bonded to \li and only has a free OH streching mode. }
\end{figure} 

\begin{figure}[htbp]
\centering
\includegraphics [width=0.45\textwidth] {./diagrams/broken_LiNO3-4w-5w} % Here is how to import EPS art
\setlength{\abovecaptionskip}{20pt}
\caption{\label{fig:broken_LiNO3-4w-5w}  VDOS of water molecules in LiNO$_3$(H$_2$O)$_4$ and LiNO$_3$(H$_2$O)$_5$. 
(a): {w1} bonded to \Li and water O; 
(b): {w2} bonded to nitrate O and water H; (c): {w3} bonded to \li and nitrate O; (d): {w4} bonded to water H and water O; 
(e): {w5} bonded to \li.}
\end{figure} 
%Comparing with the VDOS of clusters NO$_3^-$(H$_2$O)$_3$ and RNO$_3$(H$_2$O)$_3$ in Fig. ~\ref{fig:vdos_NO3-3w_2_H6H7},
%Fig. ~\ref{fig:vdos_NO3-3w_2_H-wat} and Fig. ~\ref{fig:broken_LiNO3-3w-vs-KNO3-3w}(c) (3142 \cm line), 
%we conclude that this redshift comes from the interaction between water molecules and \nit, and the interaction
%between the water molecules and the alkali metal ions, which induce more stable R-water O and R-nitrate oxygen bonds.
%===============
%VDOS of H and O
%===============
\begin{figure}[htbp]
\centering
\includegraphics [width=0.5\textwidth] {./diagrams/vdos_NO3-3w_2_H-wat}%
\setlength{\abovecaptionskip}{20pt}
\caption{\label{fig:vdos_NO3-3w_2_H-wat}The comparison between VDOS of H, and VDOS of a whole water molecule, for a water molecule ({w1}, Fig. ~\ref{fig:3_NO3_small}), 
in NO$_3^-$(H$_2$O)$_3$ at 300 K.}
\end{figure}  %(Calculated from the function vdos3.f and ft$\_$5s.sh)
%Fig. ~\ref{fig:vdos_NO3-3w_2_H-wat} gives the VDOS of H atoms in the water molecule {w1} (Fig. ~\ref{fig:3_RNO3}), and the VDOS of O and H atoms in the water molecule w1 in NO$_3^-$(H$_2$O)$_3$ at 300 K. It shows that in a water molecules, the H atoms contribute much more than the O atom to the VDOS.
%This figure shows that H atoms contribute more than O atoms into the VDOS.

%\paragraph{VDOS}
%Now we compare the VDOS of H atoms in each water molecule for the cluster LiNO$_3$(H$_2$O)$_3$. 
%The VDOS of water molecules in the cluster is much differnt from that of NO$_3^-$(H$_2$O)$_3$ (Fig. ~\ref{fig:3_NO3_small}). 
%In LiNO$_3$(H$_2$O)$_3$, one of the H atoms does not vibrate at frequency near 3370 \cm 
%and the other does not vibrate at any frequencies in the frequency interval 3200--3500 \cm.
%In the symmetric isomer NO$_3^-$(H$_2$O)$_3$, the two hydrogens in a water molecule is symmetric for time period much longer than 1 ps. 
%At a time point, they have different distance with the oxygen atom and then vibrate at different frequncies. 
%They exchange roles even in the time period as short as 5 ps, but broken symmetry still exists. 
%In long time period longer than 10$^2$ ps, they paly symmetric role in the NO$_3^-$(H$_2$O)$_3$ cluster.

%
%Fig. ~\ref{fig:LiNO3-3w_b} shows that at the range of intermediate freqency, both H atoms in w2 vibrate at 1608cm$^{-1}$. 
%That means this vibrational mode for a hydrogen atom is independent of the condition that whether it is hydrogen-bonded or not.
%To find the reason of this vibrational mode, we compare the environments of the three water molecules and look for relations between them. ]

%==================
%\begin{figure}[htbp]
%\centering
%\includegraphics [width=0.5 \textwidth] {./diagrams/gdr_ON-wat--3_NO3_100K_300K}
%\setlength{\abovecaptionskip}{20pt}
%\caption{\label{gdr_test} The comparion of the nitrete O--water H RDFs bwtween \nit(\wat)$_3$ at 100 K and 300 K.
%The distribution of H atoms at 300 K is broader than that at 100 K. 
%} 
%\end{figure}
