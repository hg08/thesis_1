\chapter{Free Energy Difference between the Water Separated and the Contact Ion Pair}\label{calculate_free_energy} 
From the blue-moon ensemble method, we obtain the constraint force (unit: a.u.force) acting on the atoms. 
The distance (unit: \A) is chosen as the reaction coordinate. 
The formula for calculating the free energy (unit: kcal/mol) is given as follows.
First, note that
\begin{equation}
  1\ \text{a.u.force} =8.2387\times 10^{-8}\ \text{N}\nonumber;
\label{eq:au2n}
\end{equation}
\begin{equation}
  1\ \text{\A} = 10^{-10}\ \text{m} \nonumber;
\label{eq:aa}
\end{equation}
\begin{equation}
  1\ \text{J} = 1.44\times 10^{20}\ \text{ kcal/mol}.\nonumber
\label{eq:j2kcpm}
\end{equation}
We can obtain the relative free energy by
%\begin{equation}
%  F = \sum_{i}^{N}f_i{\Delta{r}} \ \text{ a.u.energy},
%  \label{eq:f-e}
%\end{equation}
\begin{equation}
  F = \sum_{i}^{N}f_i{\Delta{r}},\nonumber
  \label{eq:f-e}
\end{equation}
where $i$ denote a point on the 1D reaction coordinate, 
$N$ is the number of the sampling points of the reaction coordinates,
and $f_i$ denotes the average force on atoms over the trajectory when $i$ is fixed. 
%Furthermore,
%\begin{equation}
%  \begin{split}
%  &F = \sum_{i}^{N}f_i{\Delta{r}}\times 8.2387\times10^{-18}\text{ J}\\
%  &= \sum_{i}^{N}f_i{\Delta{r}}\times 8.2387\times10^{-18}\times1.44\times10^{20}\text{ kcal/mol}.
%  \end{split}
%\label{eq:f-e2}
%\end{equation}

Now we estimate the error of the free energy $\delta{F}$ from the summation approximation. It reads
\begin{equation}
  \delta{F} = \frac{1}{N}\sum_{i}^{N}\delta{f_i}{\Delta{r}}.
  \label{eq:dleta_f}
\end{equation}
Usually, $\delta{f_i}\approx\delta{f}$, thus
\begin{equation}
  \delta{F} = \frac{1}{N}\delta{f}\sum_{i}^{N}{\Delta{r}}.
\label{eq:dleta_f-2}
\end{equation}
Particularly, if $\Delta{r}= 0.2$ \AA, $\delta{f}=0.0075\ \text{a.u.force}$, we get 
%\begin{equation}
%\begin{split}
%  &\delta{F} = \frac{0.0075\times\sum_{i}^{N}237}{N}\ \text{kcal/mol}\\
%  &          \approx 1.78\ \text{kcal/mol}.
%\end{split}
%\label{eq:dleta_f-3}
%\end{equation}
\begin{equation}
\begin{split}
  &\delta{F} \approx 1.78\ \text{kcal/mol}.\nonumber
\end{split}
\label{eq:dleta_f-3}
\end{equation}
%  &= 0.0075\times237\ \text{kcal/mol}\\

